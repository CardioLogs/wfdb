\documentclass[twoside]{article}
\oddsidemargin 0.1in
\evensidemargin -0.1in
\def\textwidth{6.375 in}

\title{Installing the WFDB Software Package}
\author{George B. Moody\\
Harvard-MIT Division of Health Sciences and Technology, Cambridge, MA, USA}
\date{}

\begin{document}
\setcounter{page}{109}

\maketitle

\section*{Summary}
This appendix briefly describes how to install the WFDB Software Package on a
new system.  The package includes C-language sources for the WFDB library and
for the applications described in this manual; \LaTeX{}, {\tt texinfo}, and
{\tt troff} sources for this and other manuals; and a one-minute sample record
({\tt 100s}).

These notes are included here for those who may not have ready access to the
World Wide Web.  For those who do, please visit PhysioNet, where quick-start
guides including installation notes for popular operating systems are available
(see {\tt http://www.physio\-net.org/physio\-tools/wfdb.shtml}

\section*{Obtaining the WFDB Software Package}
The latest version of the WFDB Software Package can always be downloaded in
source form from PhysioNet and its mirrors (see
{\tt http://www.physio\-net.org/physio\-tools/wfdb.shtml}); binaries for
popular operating systems and development snapshots are also usually available
for downloading there.  For information on obtaining the latest version of the
WFDB Software Package on CD-ROM, {\tt Sources}.

The WFDB Software Package is recommended to all who are able to use it, since
it is freely available in source form and is actively maintained.  The
predecessor of the WFDB Software Package (the DB Software Package) was included
in source form in the second and third editions of the MIT-BIH Arrhythmia
Database CD-ROM, and parts of it have been included on several other CD-ROMs
that contain databases of ECGs and other signals, including the first edition
of the MIT-BIH Arrhythmia Database CD-ROM, both editions of the European ST-T
Database CD-ROM and the MIT-BIH Polysomnographic Database CD-ROM, and on the
MGH/Marquette Foundation Waveform Database CD-ROMs ({\tt Sources}).  If you
have one of these CD-ROMs and you cannot upgrade to the WFDB Software Package,
see the files {\tt readme.doc} (in the root directory of the CD-ROM) and
{\tt lib.doc} (in the {\tt lib} directory), or {\tt README.TXT} (in the
{\tt software} directory).

\subsection*{UNIX, Linux, and similar operating systems}

Before beginning the installation of the WFDB Software Package, obtain and
install the {\tt libwww} package from {\tt http://www.w3.org/Library/} or
from {\tt http://\-www.physio\-net.org/physio\-tools/libwww/}.  This package is
provided with most current versions of Linux.  (If you have a program called
{\tt libwww-config}, then {\tt libwww} is installed already.)  You may
omit this step if you do not wish to have NETFILES support.

Also download and install the {\tt XView} software packages from {\tt
http://www.\-physio\-net.org/\-physio\-tools/xview/} if you wish to use WAVE.
Sources are available, as are binaries for several versions of Linux.  If you
are using SunOS or Solaris, XView binaries are available in the Open Look
Software Development package and may be installed already.  (If you have a
program called {\tt textedit}, then {\tt XView} is installed already.) If you
are able to use an existing set of binaries, these are recommended, since the
sources may take a {\em long} time to compile.  Be sure that the directory
containing {\tt textedit}, usually {\tt /usr/openwin/bin}, is in your
{\tt PATH}.  You may omit this step if you do not wish to use {\em WAVE}.

Select an existing directory in a writable file system for the WFDB Software
Package installation; {\tt /usr/local/src} is a good choice in most cases.
Make sure that at least 6 megabytes are available (most of this space can be
recovered after the installation is complete).

If you have downloaded the software from PhysioNet or another source, you
will have a {\tt gzip}-compressed {\tt tar} archive.  Unpack it using
the commands:

\begin{verbatim}
gzip -d wfdb.tar.gz
tar xfv wfdb.tar
\end{verbatim}

(If you have GNU {\tt tar}, as on Linux, you can combine these into a
single command: {\tt tar xfvz wfdb.tar.gz}.)

This will create a directory with a name of the form {\tt wfdb-}{\em m.n.r},
where {\em m.n.r} is the version number of the included WFDB library (e.g.,
{\tt 10.2.0}).  Enter this directory.

If you are installing the software from a CD-ROM, copy the contents of the
{\tt src/wfdb} directory to your writable file system and then enter that
directory.  One way to do this is to change to the {\tt src} directory on the
CD-ROM and then to type:

\begin{verbatim}
tar cfv - wfdb | ( cd /usr/local/src; tar xfv - )
cd wfdb
\end{verbatim}

In either case, you should now be ready to configure, compile, and install
the software, using the commands:

\begin{verbatim}
./configure
make install
\end{verbatim}

The {\tt make} command requires root permissions, and installs the package
in subdirectories of {\tt /usr}.  If you do not have root permissions,
you may install the package in any writable directory by adding an
appropriate option to the {\tt make} command above:

\begin{verbatim}
make install WFDBROOT={\em /path/to/another/directory}
\end{verbatim}

Note that in this case you will need to add {\em WFDBROOT}{\tt /bin}
to your {\tt PATH}, and {\em WFDBROOT}{\tt /lib} to your
{\tt LD\_LIBRARY\_PATH}.

Depending on the speed of your system and of your C compiler, {\tt make} will
generally require between 1 and 10 minutes.

\subsection*{MS-Windows}

If you have not already done so, install the Cygwin development environment
(freely available from {\tt http://sources.redhat.com/cygwin/}).  This includes
{\tt gcc} (the GNU C/C++ compiler) as well as a comprehensive assortment of
other Unix utilities ported to MS-Windows.  Accept the defaults suggested by
the installer.

{\emph Important:} Although you may be able to compile the WFDB software
package using a proprietary compiler, this is {\emph not supported}.  The
{\tt Makefile.dos} files in several of the subdirectories of the package's
source tree can be used with the {\tt make} utilities provided with most
commercial C compilers, although you will need to customize them for your
compiler. Your feedback is appreciated.

Before beginning the installation of the WFDB Software Package, obtain and
install the {\tt libwww} package from {\tt http://www.w3.org/Library/} or
from {\tt http://\-www.physio\-net.org/physio\-tools/libwww/}.  You may
omit this step if you do not wish to have NETFILES support.

Download
{\tt http://www.physio\-net.org/physio\-tools/binaries/win\-dows/bin/which.exe}
and put it into a directory in your PATH.  (This utility is needed by
{\tt configure} in a later step.  The sources for {\tt which.exe} are available
within {\tt http://www.physio\-net.org/physio\-tools/util\-ities/}.)

Open a Cygwin terminal window (the Cygwin installer will have added this to
your MS-Windows start menu).  Perform the remaining steps by typing the
commands given below into the terminal window.

Check that {\tt which} and {\tt gcc} are accessible by typing the command:

\begin{verbatim}
which gcc
\end{verbatim}

The output of this command should be:

\begin{verbatim}
/usr/bin/gcc
\end{verbatim}

If you don't see this output, repeat steps 1 and 2 above as necessary to
correct the problem before continuing.

If you have downloaded the WFDB software package from PhysioNet or another
source, you will have a {\tt gzip}-compressed {\tt tar} archive.  Unpack it
using the {\tt tar} command included with the Cygwin package:

\begin{verbatim}
tar xfvz wfdb.tar.gz
\end{verbatim}

If your browser decompressed the file during the download, use this command
instead:

\begin{verbatim}
tar xfv wfdb.tar
\end{verbatim}

This will create a directory with a name of the form {\tt wfdb-}{\em m.n.r},
where {\em m.n.r} is the version number of the included WFDB library (e.g.,
{\tt 10.2.0}).  Enter this directory.

If you are installing the software from a CD-ROM, copy the contents of the
{\tt src/wfdb} directory to your writable file system and then enter that
directory.  One way to do this is to change to the {\tt src} directory on the
CD-ROM and then to type:

\begin{verbatim}
tar cfv - wfdb | ( cd c:/usr/local/src; tar xfv - )
cd wfdb
\end{verbatim}

In either case, you should now be ready to configure, compile, and install
the software, using the commands:

\begin{verbatim}
./configure
make install
\end{verbatim}

If you have Microsoft or Turbo C or C++, and a Microstar Laboratories DAP 1200-
or 2400-series analog interface board, you can compile {\tt sample} (a program
for creating database records from analog signals, and for replaying them in
analog form).  To do so successfully, you must first have installed the
Microstar {\tt \#include} files and DAP interface library on your system.
Specifically, files {\tt c\_lib.c}, {\tt clock.h}, and {\tt ioutil.h} must be
installed in your {\tt include} directory, and the version of the file
{\tt cdapl.lib} that is compatible with your compiler must be installed in a
directory in which libraries are found by your linker.  Read and customize
{\tt lib/Makefile.dos} and {\tt app/Makefile.dos} as appropriate for your
compiler, and use your compiler's {\tt make} utility to generate {\tt wfdb.lib}
and then {\tt sample.exe}.

\subsection*{Other systems}

If you are installing the software from a CD-ROM, copy the contents of the
{\tt wfdb} directory to your hard disk.  Note that the text files are in
UNIX format (i.e., lines are terminated by ASCII line-feed characters only).
If your system expects text files in MS-DOS format (with both a carriage return
and a line-feed at the end of each line; VMS is one such system), use
{\tt u2d.exe} (available from PhysioNet or on the CD-ROM) to reformat the
text files under MS-DOS.  If your system is a Macintosh (which expects that
lines are terminated by carriage returns only), you will have to reformat the
text files yourself, which may be done under MS-DOS a PC using {\tt u2m.exe},
or on a Macintosh using third-party software.  Additional notes for Macintosh
users may be found in file {\tt MAC} (within the {\tt wfdb} directory).

The WFDB Software Package is written in highly portable C, and (with the
exception of a few MS-DOS or UNIX-specific display or data-acquisition
programs) should be easy to compile with any K\&R or ANSI C compiler.
The UNIX and MS-DOS {\tt make} description files ({\tt makefile.unx}
and {\tt makefile.dos} in {\tt wfdb} and in each of its subdirectories)
should get you started.
\end{document}
