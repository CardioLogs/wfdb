\input texinfo	@c -*-texinfo-*-
@comment %**start of header (This is for running Texinfo on a region.)
@setfilename wpg
@settitle WFDB Programmer's Guide
@setchapternewpage odd
@dircategory Libraries
@direntry
* WFDB library: (wpg).		The Waveform Database library.
@end direntry
@comment %**end of header (This is for running Texinfo on a region.)

@titlepage
@sp 5
@center @titlefont{WFDB Programmer's Guide}
@sp 4
@center Tenth Edition (revised and with additions for WFDB library version VERSION)
@center LONGDATE
@sp 5
@center George B. Moody
@sp 1
@center Harvard-MIT Division of Health Sciences and Technology
@page
@vskip 0pt plus 1filll
Copyright @copyright{} 1980 -- 2010 George B. Moody
@sp 2
The most recent versions of the software described in this guide may be
downloaded from @uref{http://www.physionet.org/}.  For further
information, write to:

@display
George B. Moody
Massachusetts Institute of Technology
77 Massachusetts Avenue, Room E25-505A
Cambridge, MA 02139
USA
@end display

@ifnothtml
See @uref{http://www.physio@-net.org/physio@-tools/@-wpg/}
for an HTML version of this guide.
@end ifnothtml

@sp 2
Permission is granted to make and distribute verbatim copies of this
guide provided that the copyright notice and this permission notice are
preserved on all copies.

Permission is granted to copy and distribute modified versions of this
guide under the conditions for verbatim copying and under the conditions
that follow in this paragraph.  Each copy of the resulting derived work
must contain a notice that it is a modified version of this guide.  The
notice must state which edition of this guide was the source for the derived
work, and it must credit the authors of this guide and of the
modifications.  The entire resulting derived work must be distributed
under the terms of a permission notice identical to this one.

Permission is granted to copy and distribute translations of this guide
into another language, under the above conditions for modified versions.

The author would appreciate receiving copies of any modified or
translated versions of this guide for reference purposes.
@end titlepage

@contents

@node Top, Overview, (dir), (dir)

@ifinfo
This guide documents the Waveform Database interface library (the WFDB
library).  This file contains the text of the Tenth Edition of
the @cite{WFDB Programmer's Guide} (LONGDATE), with
revisions for release VERSION of the WFDB library.
@end ifinfo

@menu
* Overview::			What is the WFDB library?
* Usage::			How to compile and run a program that uses
				the WFDB library.
* Functions::			Call and return syntax of each function, with
				descriptions and program examples.
* Data Types::		        Simple and compound types, including annotator,
				calibration, and signal information structures,
				and annotation structures.
* Annotation Codes::    	Table of codes, descriptions of mapping macros.
* Database Files::		A description of the standard file types, and
				notes about reading from nonstandard sources.
* Examples::			Annotated example programs.
* Exercises::			Test yourself.
* Glossary::			So you think you know what `time' is, eh?
				A guide for the perplexed.
* Installation::		Notes on installing the WFDB software package.
* WFDB Applications::		Brief descriptions of the WFDB application
				programs provided with the WFDB software package.
* Extensions::			Notes on extending the capabilities of the
				WFDB library.
* Sources::			Where to get the WFDB software package, databases
				of ECGs and other signals, and related items.
* Recent changes::              New material, not included in the most
                                recent printed edition of this guide.
Indices

* Concept Index::		An item for each concept.
* Function and Macro Index::	An item for each WFDB library function and macro.

* Copying::                     You can make copies of this guide.  Here are
                                the terms for doing so.
@end menu

@node     Overview, Usage, Top, Top
@comment  node-name,  next,  previous,  up
@unnumbered Preface

@cindex WFDB library
@cindex MIT DB
@cindex AHA DB

This guide documents the Waveform Database interface library (the
@dfn{WFDB library}), a package of C-callable functions that provide
clean and uniform access to digitized, annotated signals stored in a
variety of formats.  These functions were originally designed for use
with databases of electrocardiograms, including the MIT-BIH Arrhythmia
Database (@dfn{MIT DB}) and the AHA Database for the Evaluation of
Ventricular Arrhythmia Detectors (@dfn{AHA DB}).  In February 1990, the
predefined annotation set was expanded to accommodate the needs of the
European ST-T Database (@dfn{ESC DB}).  The WFDB library is sufficiently
general, however, to be useful for dealing with any similar collection
of digitized signals, which may or may not be annotated.  The WFDB
library has evolved to support the development of numerous other
databases that include signals such as blood pressure, respiration,
oxygen saturation, EEG, as well as ECGs.  Among these multi-parameter
databases are the MIT-BIH Polysomnographic Database, the MGH/Marquette
Foundation Waveform Database, and the MIMIC Database.  Thus the WFDB
library is considerably more than an @emph{ECG} database interface.

This guide describes how to write C-language programs that use databases
of ECGs and other signals.  A standard set of such programs is included
in the WFDB Software Package, and is described in the @cite{WFDB
Applications Guide}; other documents describe the databases themselves,
and existing programs that use them (@pxref{Sources}, for information
about obtaining these and related items).

There are a few important concepts that should be well understood before
going further.  These concepts include @dfn{records}; @dfn{signals},
@dfn{samples}, and @dfn{time}; and @dfn{annotations}.

@menu
* Concepts 1::			Records (``tapes'') and record names.
* Concepts 2::			Signals, samples, and time.
* Concepts 3::			Annotations and annotation files.

@emph{If this is your first exposure to the WFDB library,
study the three nodes above before going on.}

* Applications::		Examples of programs based on the WFDB library.
* Guide::			What's in this guide, and where.
                		What you need to know to get started.
				Acknowledgments, where to send your
				comments, and how to get your very own
				printed copy of this guide.
@end menu

@node     Concepts 1, Concepts 2, Overview, Overview
@unnumberedsec Records
@html
<!-- link: records.htm -->
@end html
@cindex record
@cindex record name
@cindex tape

The databases for which the WFDB library was designed consist of a
small number of @dfn{records}, each of which is quite large (typically
a megabyte or more).  Before 1990, database records usually originated
as multi-channel analog tape recordings that had been digitized and
stored as disk files.  For this historical reason, they are sometimes
referred to as @dfn{tapes}, although most newly created records are
digitally recorded onto disk media.  Each record contains a continuous
recording from a single subject.  A typical application program
accesses only a single record, and most (if not all) of the access
within the record is sequential.  Much less frequently, it may be of
interest to compare the contents of several records, or to select sets
of records.  These databases are therefore qualitatively different
from those for which conventional database management software is
written.

@cindex file names
Records are identified by @dfn{record names} of up to 20 characters
(the limit is @code{MAXRNL}, defined in @file{<wfdb/wfdb.h>}).  For
example, record names in the MIT DB are three-digit numbers, those in
the AHA DB are four-digit numbers, and those in the ESC DB are
four-digit numbers prefixed by the letter @samp{e}.  You may create
database records with names containing letters, digits, and
underscores.  Case is significant in record names that contain
letters, even in environments such as MS-Windows for which case
translation is normally performed by the operating system on file
names; thus @samp{e0104} is the name of a record found in the ESC DB,
whereas @samp{E0104} is not.  A record is comprised of several files,
which contain signals, annotations, and specifications of signal
attributes; each file belonging to a given record normally includes
the record name as the first part of its name.  A record is an
extensible collection of files, which need not all be located in the
same directory, or even on the same physical device.  Thus it is
possible, for example, to create a local disk file of your own
annotations for a record read from a web server or a CDROM, and to
treat your file as part of the record.

@node     Concepts 2, Concepts 3, Concepts 1, Overview
@unnumberedsec Signals, Samples, and Time
@cindex adu
@cindex gain
@cindex physical units
@cindex sample
@cindex sampling frequency
@cindex sample interval
@cindex scales (time and amplitude)
@cindex signal
@cindex units (ADC)
@cindex units (physical)

Signals are commonly understood to be functions of time obtained by
observation of physical variables.  In this guide, a @dfn{signal} is
defined more restrictively as a finite sequence of integer
@dfn{samples}, usually obtained by digitizing a continuous observed
function of time at a fixed @dfn{sampling frequency} expressed in Hz
(samples per second).  The time interval between any pair of adjacent
samples in a given signal is a @dfn{sample interval}; all sample
intervals for a given signal are equal.  The integer value of each
sample is usually interpreted as a voltage, and the units are called
analog-to-digital converter units, or @dfn{adu}.  The @dfn{gain} defined
for each signal specifies how many adus correspond to one @dfn{physical
unit} (usually one millivolt, the nominal amplitude of a normal QRS
complex on a body-surface ECG lead roughly parallel to the mean cardiac
electrical axis).  All signals in a given record are usually sampled at
the same frequency, but not necessarily at the same gain
(@pxref{Multi-Frequency Records}, for exceptions to
this rule).  MIT DB records are sampled at 360 Hz; AHA and ESC DB
records are sampled at 250 Hz.

@cindex sample number
@cindex time
The @dfn{sample number} is an attribute of a sample, defined as the
number of samples of the same signal that precede it;  thus the sample
number of the first sample in each signal is zero.  Within this guide,
the units of @dfn{time} are sample intervals;  hence the ``time'' of a
sample is synonymous with its sample number.

Samples having the same sample number in different signals of the same
record are treated as simultaneous.  In truth, they are usually not
@emph{precisely} simultaneous, since most multi-channel digitizers
sample signals in ``round-robin'' fashion.  If this subtlety makes a
difference to you, you should be prepared to compensate for inter-signal
sampling skew in your programs.

@node     Concepts 3, Applications, Concepts 2, Overview
@unnumberedsec Annotations
@cindex annotation
@cindex annotator
@cindex @code{atr}
@cindex beat label
@cindex label (beat)
@cindex QRS label
@cindex reference annotations

MIT DB records are each 30 minutes in duration, and are @dfn{annotated}
throughout; by this we mean that each beat (QRS complex) is described by
a label called an @dfn{annotation}.  Typically an @dfn{annotation file}
for an MIT DB record contains about 2000 beat annotations, and smaller
numbers of rhythm and signal quality annotations.  AHA DB records are
either 35 minutes or 3 hours in duration, and only the last 30 minutes
of each record are annotated.  ESC DB records are each 2 hours long, and
are annotated throughout.  The ``time'' of an annotation is simply the
sample number of the sample with which the annotation is associated.
Annotations may be associated with a single signal, if desired.  Like
samples in signals, annotations are kept in time and signal order in
annotation files (but @pxref{Annotation Order}, for exceptions to this
rule).  No more than one annotation in a given annotation file may be
associated with any given sample of any given signal.  There may be many
annotation files associated with the same record, however; they are
distinguished by @dfn{annotator names}.  The annotator name @file{atr}
is reserved to identify @dfn{reference annotation files} supplied by the
developers of the databases to document correct beat labels.  You may
use other annotator names (which may contain letters, digits and
underscores, as for record names) to identify annotation files that you
create.  You may wish to adopt the convention that the annotator name is
the name of the file's creator (a program or a person).

Annotations are visible to the WFDB library user as C structures, the
fields of which specify time, beat type, and several user-definable
variables.  The WFDB library performs efficient conversions between these
structures and a compact bit-packed representation used for storage of
annotations in annotation files.

@node     Applications, Guide, Concepts 3, Overview
@unnumberedsec Applications

Some typical uses of the WFDB library are these:

@cindex annotation editor
@cindex waveform editor
@itemize @bullet
@item
A @emph{waveform editor}, such as @code{wave} (see the
@uref{http://@-www.@-physionet.@-org/@-physio@-tools/@-wug/,
@cite{WAVE User's Guide}}), reads the
digitized signals of a database record and displays them with
annotations superimposed on the waveforms.  Such a program allows the
user to select any portion of the signals for display at various scales,
and to add, delete, or correct annotations.

@cindex digital filter
@cindex filter (digital)
@item
@emph{Signal processing programs} (e.g., @pxref{Example 7}) apply digital
filters to the signals of a database record and then record the filtered
signals as a new record.  Similar programs perform sampling frequency
conversion.

@item
@emph{Analysis programs} (e.g., @pxref{Example 10}) read the digitized
signals, analyze them, and then record their own annotations.

@cindex annotation comparator
@cindex comparator (annotation)
@item
An @emph{annotation comparator}, such as @code{bxb}
(@pxref{WFDB Applications}), reads two or more sets of annotations
corresponding to a given record, and tabulates discrepancies between
them.  If the reference annotations supplied with the database are
compared in this way with annotations produced using an analysis
program, this comparison is a means of establishing the accuracy of the
analysis program's output.
@end itemize

The WFDB library provides the means for programs such as those described
above to select a database record, read and write signals, read and
write annotations, jump to arbitrary points in the record, and determine
attributes of the signals such as the sampling frequency.  The library
also provides a variety of other more specialized services for programs
that need them.  The library defines an interface between programs and
the database that is sufficiently powerful, general, and efficient to
eliminate the need for @emph{ad hoc} user-written database I/O.

@node     Guide, , Applications, Overview
@unnumberedsec About this Guide

You should have a good grasp of the C language in order to make the best
use of this guide.  If ANSI C prototypes, used here to document the WFDB
library functions, are unfamiliar to you, see pp. 217--218 in the second
edition of @cite{The C Programming Language} by Kernighan and Ritchie,
Prentice Hall, 1988.  (This is the famous @cite{K&R}; all @cite{K&R}
references in this guide include page numbers for the second edition.
Newcomers to C should have a copy for ready reference while reading this
guide.)  It may also be helpful to have a copy of a database directory,
such as the
@uref{http://@-www.@-physionet.@-org/@-physio@-bank/@-data@-base/@-html/@-mitdbdir/,
@cite{MIT-BIH Arrhythmia Database Directory}}.  The 
@uref{http://@-www.@-physionet.@-org/@-physio@-tools/@-wag/,
@cite{WFDB Applications Guide}} will be
useful as a reference for existing WFDB library-based applications
(available from PhysioNet, @uref{http://@-www.@-physio@-net.@-org/}).

You should have access to a computer that has the WFDB library and at
least one or two database records on-line, or access to the World Wide Web,
where database records can be obtained from PhysioNet and other sources.  (If
you are installing the WFDB library on a new computer for the first time,
please read the installation notes supplied with the WFDB library first, or
@pxref{Installation, , Installing the WFDB Software Package}, then return
here.)  You should know how to create a C source file using your
favorite editor, and you should know how to compile it and how to run
the resulting executable program.

Resist all temptation to plunge into the esoteric details of file
formats.  (Those who find such details irresistible will find them in
Section 5 of the @cite{WFDB Applications Guide}; note, however, that
support for new file formats is added to the WFDB library from time to
time, so that the information you find there may be incomplete.)  The
WFDB library provides an efficient means of reading and writing files in
many formats; it is not a trivial task to duplicate it, and time spent
doing so is time that could be spent doing something useful, enjoyable,
or possibly both.  If you really think you need to understand the file
formats in order to translate them into whatever the ECGWhizz Model 666
needs, consider instead writing a format translator using the WFDB
library to read the files; then you will at least have a program that
requires only recompilation with a new version of the WFDB library when
file formats change.  @emph{In extremis}, use @file{rdann} and
@file{rdsamp} --- available from PhysioNet in source and ready-to run
formats --- to translate files into text format.

Chapter 1 of this guide begins with a simple example program that reads a few
samples from a database record.  The C version of this program is also
translated into a variety of other languages supported by the WFDB library
itself or by separately available bindings.  This example should help you
understand the mechanics of compiling and using a program that does something
with an ECG database.  Chapter 2 introduces the library functions themselves,
with a number of brief examples; you may wish to skim through this material on
a first reading to get acquainted with what is available, and then refer to it
as needed while writing your programs.  Data structures for annotations and for
signal and annotator attributes are described in chapter 3.  Chapter 4 contains
a table of annotation types and descriptions of several annotation-mapping
macros.  Database files and related topics are discussed in chapter 5, which
can be skipped on a first reading.  Chapter 6 contains additional example
programs that illuminate a few of the darker corners of the WFDB library.  The
glossary defines the ordinary-sounding words such as @dfn{signal} that have
specialized meanings in this guide; such words are @dfn{emphasized} in their
first appearances in order to warn you that you should look them up in the
glossary on a first reading (@pxref{Glossary}).

If the WFDB library has not yet been installed on your system,
@pxref{Installation, , Installing the WFDB Software Package}.
Another appendix (@pxref{WFDB Applications}) includes brief
descriptions of the application programs that are distributed with the
WFDB library as part of the WFDB software package.

@cindex operating systems (supported)
Another appendix discusses porting the WFDB library to new machines or
operating systems, and includes notes on adding support for new file
formats, annotation codes, and other enhancements (@pxref{Extensions}).
The WFDB library has been written with portability in mind.  It runs on
a wide variety of machines and operating systems, including Unix (BSD
4.x, System V, SunOS, Solaris, HP-UX, OSF/1, Version 7, XENIX, VENIX,
ULTRIX, GNU/Linux, FreeBSD, OpenBSD, IRIX, AIX, AUX, Darwin, Mac OS X, SCO,
Coherent, and more), MS-DOS, MS-Windows, VMS, and classic Mac OS.
This guide was written for Unix users (with notes for MS-Windows and
MS-DOS users where differences exist), but others should find only minor
differences.

At the end of the guide is a list of sources for databases and other
materials that may be useful to readers (@pxref{Sources}), and a log
of changes made to this library since 1999 (@pxref{Recent changes}).

Many friends have contributed to the development of the WFDB library.
Thanks to Paul Albrecht, Ted Baker, Phil Devlin, Scott Greenwald,
Thomas Heldt, Isaac Henry, David Israel, Roger Mark, Joe Mietus,
Warren Muldrow, Wei Zong, and especially to Paul Schluter, whose
elegant 8080 assembly language functions inspired these (long live
@code{getann}!).  Pat Hamilton and Bob Farrell contributed ports, to
classic Mac OS and the MS 32-bit Windows environments, respectively.
Jose Garcia Moros and Salvador Olmos contributed Matlab/Octave
reimplementations of a useful subset of the WFDB library.  Jonas
Carlson wrote, documented, and contributed a set of Matlab wrappers
for the WFDB library, and Michael Craig created and contributed
the WFDB Toolbox for Matlab.  Isaac Henry contributed SWIG wrappers that
provide interfaces for a variety of scripting languages, as well as a
set of translations of the C example programs in this guide into the
languages supported by the SWIG wrappers.  Mike Dakin provided the
first implementation of NETFILES (HTTP client support in the WFDB
library) based on the W3C's @code{libwww}.  Following the W3C's
decision to end development of @code{libwww}, Benjamin Moody
reimplemented NETFILES using @code{libcurl}, and also implemented
variable-layout records.  Thanks also to the many readers of earlier
versions of this guide; if this edition answers your questions, it is
because someone else has already asked them, and hounded the author
until he produced comprehensible answers.

Before May, 1999, and the release of version 10.0.0 of the library, the
WFDB library was known as the DB library, and this guide was the @cite{ECG
Database Programmer's Guide}.  The name of the library was changed because
of confusion caused by the proliferation of another library with
the same name (a reimplementation of the Berkeley Unix DBM library).
The names of this guide, and of the @cite{WFDB Applications Guide} (formerly
the @cite{ECG Database Applications Guide}), have been changed in view of
the increasingly broad range of applications in which the library is being
used.

The first edition of this guide was written as a tutorial for MIT
students using the ECG databases for a variety of signal-processing and
analysis projects.  The guide, and the WFDB library itself, have been
extensively revised since they first appeared in 1981.  Your comments
and suggestions are welcome.  Please send them to:

@display
George B. Moody  <@email{george@@mit.edu}>
MIT Room E25-505A
Cambridge, MA 02139
USA

@end display

@ifnotinfo
@cindex Emacs Info
@cindex GNU emacs
@cindex Info (GNU emacs)
If you use the GNU @code{emacs} editor, you can peruse a hypertext
version of this guide using @code{info} if it has been installed on your
system; among its many other features, @code{emacs} makes it easy to
copy code from the examples into your own programs.  Installation
instructions are included in the WFDB Software Package; type @kbd{C-h i}
within GNU @code{emacs} to start up @code{info} (@pxref{Sources}, for
information about obtaining GNU @code{emacs}).
@end ifnotinfo

@ifnothtml
An HTML version of this guide, suitable for viewing using any web browser,
is included with the WFDB Software Package.  The latest version may always be
viewed at @uref{http://@-www.@-physio@-net.@-org/@-physio@-tools/@-wpg/} using your web
browser.
@end ifnothtml

You can format and print copies of this guide using TeX if you have it (see
@file{makefile} in the @file{doc} directory of the library distribution for
instructions on doing so).  You may obtain preformatted versions in PDF and
PostScript formats from @uref{http://www.physionet.org/}.


@node     Usage, Functions, Overview, Top
@chapter Using the WFDB Library

This chapter gives a brief overview of the steps needed to compile,
load, and run a program that uses the WFDB library.  It assumes that you
are able to log onto a Unix-based computer on which the WFDB Software
Package has been installed
(@pxref{Installation, , Installing the WFDB Software Package}),
and that you know how to create a source file using a text editor such
as @code{emacs} or @code{vi}.  If you are using an MS-DOS system, there
are a few differences noted below.


@menu
* print samples::		A trivial example program.
* compiling::			Compiling a C program with the WFDB library.
* other languages::             Using the WFDB library with C++, Fortran,
                                  Java, Perl, Python, C#, or Matlab.
* WFDB path::			WFDB library environment variables.
* running example::		Running the example program.
* name restrictions::		Names to avoid when writing your programs.
* WFDB path syntax::		Customizing the WFDB path.
* exercises 1::			A few questions.
@end menu

@node     print samples, compiling, Usage, Usage
@section A Trivial Example Program in C

Suppose we wish to print the first ten samples of record @file{100s}.
(Record @file{100s} is the first minute of MIT-BIH Arrhythmia Database
record @file{100}, supplied as a sample in the @code{data} directory of all
source distributions of the WFDB Software Package.)  We might begin by
creating a source file called @file{psamples.c} that contains:

@example
#include <stdio.h>
#include <wfdb/wfdb.h>

main()
@{
    int i;
    WFDB_Sample v[2];
    WFDB_Siginfo s[2];

    if (isigopen("100s", s, 2) < 2)
        exit(1);
    for (i = 0; i < 10; i++) @{
        if (getvec(v) < 0)
            break;
        printf("%d\t%d\n", v[0], v[1]);
    @}
    exit(0);
@}
@end example

@noindent
(See @uref{http://www.physionet.org/physiotools/wfdb/examples/psamples.c}
for a copy of this program.)

All programs that use the WFDB library @emph{must} include the
statement

@example
#include <wfdb/wfdb.h>
@end example

@noindent
which defines function interfaces, data types (such as the
@code{WFDB_Sample} and @code{WFDB_Siginfo} types used in this example), and
a few useful constants.  (Most MS-DOS C compilers accept @samp{/} as a
directory separator.  If you prefer to use the non-portable @samp{\}
under MS-DOS, remember to quote it: @samp{#include <wfdb\\wfdb.h>}.)

The functions used in the example are described in detail in the next
chapter, and the data types are described in the following chapter
(@pxref{Data Types}).  For now, note that @code{isigopen} prepares a
record to be read by @code{getvec}, which reads a sample from each of
the two signals each time it is called.  

Note that in some cases it may be important to insure that all memory
allocated by the WFDB library is freed before the program exits;  in
the example program, this can be done by adding the line

@example
wfdbquit();
@end example

@noindent
just above @code{exit(0);}
@ifhtml
(@pxref{wfdbquit, , @code{wfdbquit}}).
@end ifhtml
@ifinfo
(@pxref{wfdbquit}).
@end ifinfo
@iftex
(@pxref{wfdbquit, , @code{wfdbquit}}).
@end iftex

@node     compiling, other languages, print samples, Usage
@comment  node-name,  next,  previous,  up
@section Compiling a Program with the WFDB Library
@html
<!-- link: compiling.htm -->
@end html
@cindex WFDB library (compiling with)
@cindex compiling
@cindex loader options

The WFDB library is developed and tested using @code{gcc}, the GNU C/C++
compiler, but careful attention has been given to making it usable with any
ANSI/ISO C compiler.  Since @code{gcc} is free, high quality, and supported, it
is highly recommended that you use it for compiling your WFDB applications.

To compile the example program using @code{gcc}, we can say:

@example
gcc -o psamples psamples.c -lwfdb
@end example

@noindent
to produce an executable program called @code{psamples}.  (Your C
compiler may be named @file{cc}, @file{acc}, @file{CC}, or something
else, rather than @file{gcc}.)  You may use any other compiler
options you choose, but the @samp{-lwfdb} option must appear in the
@code{cc} command line following any and all source (@file{*.c}) and
object (@file{*.o}) file names, in order to instruct the loader to
search the WFDB library for any functions that the program needs (in
this case, @code{isigopen} and @code{getvec}).  Some programs will need
additional libraries, and the corresponding @samp{-l} options can
usually be given before or after the @samp{-lwfdb} option.

@cindex NETFILES
@cindex curl
@cindex libcurl
@cindex libwww
@cindex W3C libwww

If the WFDB library was installed with NETFILES support, it will make use of
functions contained in the @code{libcurl} or @code{libwww} libraries.  If you
have dynamically linkable versions of the @code{libcurl} or @code{libwww}
libraries, as under GNU/Linux, these will be loaded automatically when you run
@code{psamples}.  If you have only static versions of these libraries, however,
it is necessary to provide additional arguments in the @code{cc} command line
in order to compile successfully.  One way to do this is to follow the model
used to compile the standard WFDB applications supplied with the WFDB library;
see @file{Makefile} in the @file{app} directory of the WFDB software package
source tree.

If you are using WFDB version 10.2.6 or a later version and @file{gcc}
or a compatible compiler, the @file{wfdb-config} utility is available
to help construct commands for compiling programs that use the WFDB
library (and the @code{libcurl} or @code{libwww} libraries, if available).
Use it like this:

@example
gcc `wfdb-config --cflags` -o psamples psamples.c `wfdb-config --libs`
@end example

@noindent
Note that this command contains backticks (`), not apostrophes (').
@file{wfdb-config} is particularly useful if the WFDB library or
its @file{*.h} files are installed in non-standard locations, or if
you have only static @code{libcurl} or @code{libwww} libraries.

Under MS-Windows, @file{gcc} is included in the freely available Cygwin
software development system (@uref{http://@-www@-.cygwin@-.com/}), and also in
the freely available MinGW package (@uref{http://@-www@-.mingw@-.org/}).  An
MS-DOS version of @file{gcc} is available in the free djgpp package
(@uref{http://@-www@-.delorie@-.com/@-djgpp/}). These are used within a Cygwin
terminal emulator window or an MS-DOS box in exactly the same way as described
above for C compilers on all other platforms.  For most purposes, Cygwin is
recommended, since it provides a Unix-compatible standard C library
(@code{cygwin1.dll}), so that applications behave exactly as they do on all
other platforms.  WAVE can only be built under Windows in this way.  When
building WFDB-based plugins for use with .NET applications or others such as
Matlab that rely on the native Windows C library, however, the WFDB library
must be recompiled to use the native library.  This can be done using either
MinGW gcc, or Cygwin gcc with its @code{-mno-cygwin} option.

If you choose to use an incompatible proprietary compiler, you are on
your own!  You may be able to create a linkable version of the WFDB
library from the sources in the @file{lib} directory of the WFDB source
tree using a proprietary compiler, but doing so is unsupported (see your
compiler's documentation).  If you are not able to build the WFDB
library using your compiler, you can compile the library sources
together with the source file(s) for your application.  It may be
easiest to copy the library sources (both the @file{*.c} and the
@file{*.h} files) into the same directory as the application sources.
If you follow this approach, find the directory that contains @file{stdio.h}
on your system and make a @file{wfdb} subdirectory within that directory,
then copy the WFDB library's @file{*.h} files into the @file{wfdb}
subdirectory (this is necessary so that statements of the form
@samp{#include <wfdb/wfdb.h>} will be handled properly by your compiler).
For example, to compile @file{psamples.c} with Microsoft C/C++, set up
the WFDB library source files as just described, then use this command:

@example
cl psamples.c wfdbio.c signal.c annot.c calib.c wfdbinit.c
@end example

@noindent
With Borland C/C++ or Turbo C or C++, substitute @samp{bcc} or
@samp{tcc}, respectively, for @samp{cl} in the command above.  You will
find that some WFDB applications do not need to be compiled with all of
the WFDB library sources (for example, @file{psamples} needs only
@file{wfdbio.c} and @file{signal.c}); in such cases, you may omit the
unneeded sources for faster compilation and smaller executable binaries.


@node     other languages, WFDB path, compiling, Usage
@comment  node-name,  next,  previous,  up
@section Using the WFDB library with other languages

Bindings and wrappers are available so that programs written in a
number of other programming languages can make use of the WFDB library.

@unnumberedsubsec C++ bindings
@cindex C++ bindings

If you prefer to write your applications in C++, you may do so, but note
that the WFDB library is written in C.  (Most C++ compilers can be run
in ANSI/ISO C compatibility mode in order to compile the WFDB library
itself.)  Each C++ source file that uses WFDB library functions must
include @file{<wfdb/wfdb.h>}, in order to instruct your compiler to use
C conventions for argument passing and to use unmangled names for the
WFDB library functions.  In order for this to work, your C++ compiler
should predefine @samp{__cplusplus} or @samp{c_plusplus}; if it
predefines neither of these symbols, modify @file{<wfdb/wfdb.h>} so that
the symbols @samp{wfdb_CPP} and @samp{wfdb_PROTO} are defined at the top
of the file, or define @samp{__cplusplus} in each of your source files
before including @file{<wfdb/wfdb.h>}.  Compile and link your program
using whatever standard methods are supported by your compiler for
linking C++ programs with C libraries.  See your compiler manual for
further information.

@unnumberedsubsec Fortran wrappers
@cindex Fortran bindings
@cindex wrappers for Fortran
A set of wrapper functions is also available for those who wish to use the WFDB
library together with applications written in Fortran.  These functions,
defined in @uref{http://www.physionet.org/physiotools/wfdb/fortran/wfdbf.c,
@file{wfdbf.c}}, provide a thin `wrapper' around the WFDB library functions, by
accepting Fortran-compatible arguments (there are no structures, and all
arguments are passed by reference rather than by value).  For example, here is
the Fortran equivalent of the example program in the previous section:

@example
        integer i, v(2), g

        i = isigopen("100s", 2)
        do i = 1, 10
         g = getvec(v)
         write (6,3) v(1), v(2)
 3      format("v(1) = ", i4, "    v(2) = ", i4)
        end do
        end
@end example

@noindent
(See @uref{http://www.physionet.org/physiotools/wfdb/fortran/fsamples.f}
for a copy of this program;  an extensively commented version of this program
is also available, at
@uref{http://www.physionet.org/physiotools/wfdb/fortran/example.f}.)

@noindent
To compile this program using @code{g77} (the GNU Fortran compiler), save it as
@file{fsamples.f} in the current directory, copy @file{wfdbf.c} (which can be
found in the same directory as @file{wfdb.h}, usually @file{/usr/include/wfdb})
to the current directory, then type:

@example
g77 -o fsamples -DFIXSTRINGS fsamples.f wfdbf.c -lwfdb
@end example

@noindent
The Fortran wrapper functions are not discussed in this guide; for further
information, refer to
@uref{http://www.physionet.org/physiotools/wfdb/fortran/README,fortran/README}
in the WFDB Software Package.

@unnumberedsubsec SWIG wrappers for Java, Perl, Python, and other languages
@cindex SWIG wrappers
@cindex Java wrappers
@cindex Perl wrappers
@cindex Python wrappers
@cindex C# wrappers

Isaac Henry has contributed WFDB wrappers for Java, Perl, and Python,
as well as for .NET languages such as C#, created using the Simplified
Wrapper Interface Generator (SWIG, @uref{http://www.swig.org/}).  Using
these wrappers, the example program can be written in any of these languages:

@b{Java:}

@example
import wfdb.*;

public class psamples @{
    static @{
        System.loadLibrary("wfdbjava");
    @}

    public static void main(String argv[]) @{
        WFDB_SiginfoArray siarray = new WFDB_SiginfoArray(2);
        if (wfdb.isigopen ("100s", siarray.cast(), 2) < 2)
            System.exit(1);
        WFDB_SampleArray v = new WFDB_SampleArray(2);
        for (int i = 0; i < 10; i++) @{
            if (wfdb.getvec(v.cast()) < 0)
                break;
            System.out.println("\t" + v.getitem(0) +"\t"+ v.getitem(1));
        @}
    @}
@}
@end example

@b{Perl:}

@example
package wfdb;
use wfdb;

$siarray = new wfdb::WFDB_SiginfoArray(2);
if ($nsig = isigopen("100s", $siarray->cast(), 2) < 2) @{
    exit(1);
@}
$v = new wfdb::WFDB_SampleArray(2);
for ($i=0; $i < 10; $i++) @{
    if (getvec($v->cast()) < 0) @{
        exit(2);
    @}
    print "\t", $v->getitem(0), "\t", $v->getitem(1), "\n";
@}
@end example

@b{Python:}

@example
import wfdb, sys

def main(argv):
    siarray = wfdb.WFDB_SiginfoArray(2)
    if wfdb.isigopen("100s", siarray.cast(), 2) < 2: sys.exit(1)
    v = wfdb.WFDB_SampleArray(2)
    for i in range(0,10):
        if wfdb.getvec(v.cast()) < 0: sys.exit(2)
        print "\t%d\t%d" % (v[0], v[1])

if __name__ == "__main__":
    main(sys.argv[1:])

@end example

@b{C#:}

@example
using System;
using Wfdb;

public class psamples @{
    static void Main(string[] argv) @{
        WFDB_SiginfoArray siarray = new WFDB_SiginfoArray(2);
        if (wfdb.isigopen("100s", siarray.cast(), 2) < 2)
            Environment.Exit(1);
        WFDB_SampleArray v = new WFDB_SampleArray(2);
        for (int i = 0; i < 10; i++) @{
            if (wfdb.getvec(v.cast()) < 0)
                break;
	    Console.WriteLine("\t" + v.getitem(0) + "\t" + v.getitem(1));
        @}
    @}
@}
@end example

All SWIG wrappers for the WFDB library are generated using a single
interface file, @file{wfdb.i}.  In principle, this file might be used
to generate wrappers for other programming languages supported by SWIG,
including several versions of LISP, Modula-3, PHP, Ruby, and Tcl.

@unnumberedsubsec Matlab toolbox
@cindex Matlab toolbox
@cindex Toolbox for Matlab
@cindex WFDB toolbox for Matlab

The WFDB Toolbox for Matlab, contributed by Michael Craig and available from
@uref{http://www.@-physionet.@-org/@-physiotools/@-matlab/@-wfdb-swig-matlab/},
is a collection of WFDB applications implemented as functions in Matlab,
built on the SWIG Java wrappers for the WFDB library.  For example, using it
in Matlab, one can read and plot the first five seconds of the same signals
as in the example program above, by:

@example
r = rdsamp('100s', 'maxt', ':5');
plot(r(:,1), r(:,2));
@end example

The WFDB Toolbox for Matlab replaces the @code{wfdb_tools} library of wrapper
functions contributed by Jonas Carlson, since current versions of Matlab are
no longer compatible with the @code{wfdb_tools} library.

The WFDB Software Package includes @code{wfdb2mat}, an application that
converts all or any desired segment of a signal file into a @code{.mat}
file that can be read directly by Matlab. 

Jesus Olivan Palacios has written a tutorial (available at
@uref{http://www.@-neurotraces.@-com/@-scilab/@-sciteam/}) that includes a
detailed section on using the WFDB
Software Package with Scilab (an open-source scientific software
package for numerical computations, with a language similar to that of Matlab,
available from @uref{http://@-www-rocq.inria.fr/@-scilab/}).  The methods
described in this tutorial can also be adapted for use with GNU Octave
(another free language that is mostly compatible with Matlab, available from
@uref{http://@-www.gnu.org/@-software/@-octave/}).

Also available is a reimplementation of a useful subset of the WFDB
library in native m-code (contributed by Jose Garcia Moros and Salvador Olmos)
at @uref{http://www.@-physio@-net.@-org/@-physio@-tools/@-matlab/}.

@node     WFDB path, running example, other languages, Usage
@comment  node-name,  next,  previous,  up
@section The Database Path and Other Environment Variables
@html
<!-- link: database-path.htm -->
@end html
@cindex database path (setting)
@cindex path (database)
@cindex directories for WFDB files
@cindex WFDB files (finding)
@cindex finding WFDB files
@cindex @code{WFDB} (environment variable)
@cindex @code{WFDBCAL} (environment variable)
@cindex @code{WFDBGVMODE} (environment variable)
@cindex @code{WFDBANNSORT} (environment variable)
@cindex WFDB path

WFDB applications make use of several @dfn{environment variables}, which are
named @code{WFDB}, @code{WFDBCAL}, @code{WFDBGVMODE}, and @code{WFDBANNSORT}.
If these variables have not been otherwise defined by the user, their values
are those given by @code{DEFWFDB}, @code{DEFWFDBCAL}, @code{DEFWFDBGVMODE},
and @code{DEFWFDBANNSORT} (defined in @file{wfdblib.h} at the
time the WFDB library was compiled).  Unless you have a non-standard
setup, you may not need to set these variables, but it will be helpful
to read this section to understand how they influence the behavior of
WFDB applications.

When WFDB applications @emph{read} database files, they must be able to
find them in various locations that may vary from system to system.  The
WFDB library refers to a character string that consists of an ordered
list of locations to be searched @emph{for input files}.  This string is
called the @dfn{database path}, or the @dfn{WFDB path}.

On most systems, the environment variable @code{WFDB}, if set, specifies
the value of the WFDB path, and overrides the default value.  If you
need to use a non-default WFDB path, you must set the @code{WFDB}
environment variable appropriately before running any WFDB applications,
so that the WFDB path can be examined by the running program.  The WFDB
software package includes easily customizable shell scripts (batch
files) that illustrate how to do this for popular shells and command
interpreters; see @emph{setwfdb}(1), in the @cite{WFDB Applications
Guide}.  (Under classic Mac OS, for which the concept of environment
variables is foreign, the WFDB path may be set only by using
@code{DEFWFDB}.)  For further information, @pxref{WFDB path syntax}.

@cindex calibration file
@cindex @code{WFDBCAL} (environment variable)
The shell scripts that set @code{WFDB} also set the @code{WFDBCAL}
environment variable, which is important if you make use of records that
contain signals other than ECGs.  @code{WFDBCAL} names a
@dfn{calibration file} located in one of the directories named by
@code{WFDB}.  (The symbol @code{DEFWFDBCAL} is usually defined in
@file{wfdblib.h} to specify the name of a default calibration file, to
be used by the WFDB library if @code{WFDBCAL} has not been set.)  Each
signal type may be represented by an entry in the calibration file.
Entries specify the characteristics of any calibration pulses that may
be present, and customary scales for plotting the signals.

@cindex annotation (canonical order)
@cindex canonical order of annotations
The other environment variables are less frequently used than @code{WFDB}
and @code{WFDBCAL}, and in most cases, the compiled-in defaults will be
appropriate (@pxref{Annotation Order}, and @pxref{Multi-Frequency Records},
for details).

@node     running example, name restrictions, WFDB path, Usage
@section Running the Example Program

If @code{WFDB} is properly set, MIT DB record @file{100s} is on-line and
readable, and the example program was compiled correctly, it can be run
by typing

@example
psamples
@end example

@noindent
(Try @samp{./psamples} if @samp{psamples} doesn't work.)  Its output
will appear as:

@example
995     1011
995     1011
995     1011
995     1011
995     1011
995     1011
995     1011
995     1011
1000    1008
997     1008
@end example

The left column contains samples from signal 0, and the right column
contains those from signal 1.

@node     name restrictions, WFDB path syntax, running example, Usage
@section A Note on Identifiers
@cindex function name restrictions
@cindex external identifiers (restrictions)
@cindex restrictions on function and variable names
@cindex variable name restrictions

External identifiers that begin with the underscore (@samp{_}) character
are reserved under the rules of ANSI C to the compiler and libraries.
In order to make the WFDB library as portable as possible, its own external
identifiers do not begin with underscores (since otherwise they might
conflict with external identifiers used by a standard library).

External identifiers beginning with @samp{wfdb_} are reserved for the use
of the WFDB library.  These names are used for functions and global variables
that are intended for the private use of the WFDB library; your programs
should not need to use them.  You should avoid defining functions or
global variables with such names in your programs.

External identifiers beginning with @samp{WFDB_} are used for constants
and data types defined within @file{<wfdb/wfdb.h>}.  Use these identifiers
as needed in your programs, but avoid redefining them.

@node     WFDB path syntax, exercises 1, name restrictions, Usage
@section More About the WFDB Path

@cindex WFDB path
When a WFDB file must be opened for input, the WFDB library attempts to
locate it by attaching each of the components of the WFDB path (one at a
time) as a prefix to the file name.  If two or more matching files exist
in different locations in the WFDB path, the WFDB library opens only the
file that resides in the first of these locations.  Any other matching
files are effectively invisible to WFDB applications unless the WFDB
path is rearranged.

@cindex database path (default)
@cindex WFDB path

The default WFDB path is specified at the time the WFDB library is
compiled, by defining a value for the symbol @code{DEFWFDB} in
@file{wfdblib.h}.  Current versions of the WFDB library are compiled
with a three-component default WFDB path; the first component is empty
(i.e., it refers to the current directory), the second component names
the @dfn{system-wide database directory} (which contains the sample WFDB
files supplied with the WFDB software package), and the third component
is @uref{http://www.physionet.org/physiobank/data@-base} (referring to the
PhysioBank data archives).  Note that this default may be changed at the
time the WFDB library is compiled.  Normally, however, this means that
any record available from PhysioBank is readable by any WFDB application
provided that PhysioBank is accessible from the user's computer and that
the database name is included in the record name (for example,
@file{slpdb/slp60} or @file{nsrdb/16265}).

Under Unix and VMS, the WFDB path can be given as a colon-separated list
of prefixes, in the format used for the Bourne shell's @code{PATH}
variable.  Under MS-Windows, MS-DOS, and classic Mac OS, the WFDB path can be
given in the format used for the MS-DOS @code{PATH} variable, with
semicolons used to separate prefixes (colons retain their customary
meanings, as drive letter suffixes under MS-DOS, or as directory
separators on the Macintosh).  Alternatively, components of the WFDB
path may be separated by whitespace (under any operating system); this
also implies that embedded spaces are not permitted within path
components.  @strong{For this reason, avoid using directories with names
such as @file{My Documents}, or their subdirectories, to store WFDB files.}

When WFDB applications @emph{write} database files, these files are
generally written to the current directory.  (As an example, an
application that analyzes one or more signals in a record may record its
findings in an annotation file in the current directory.)  If the record
name (as provided by the application to the WFDB library) contains path
information, however, output files are written to the corresponding
subdirectory of the current directory.  (For example, if a WFDB
application writes an annotation file for record @code{edb/e0103}, the
file will be written in the @code{edb} subdirectory of the current
directory.  The @code{edb} subdirectory will be created by the WFDB
library if does not exist already.  This feature was introduced in WFDB
library version 10.2.0.)

Note particularly that the current directory is @emph{not} necessarily
part of the WFDB path.  If you modify your WFDB path, you must
explicitly include an empty (null) component, which corresponds to the
current directory, in order to be sure that your WFDB applications can
read any WFDB files that you have previously written.  In most cases,
this null component should be the first in the WFDB path.  Thus, if you
write into the current directory a modified version of an existing WFDB
file, any later actions that would read this file will read your
modified version rather than the original.

@cindex NETFILES
@cindex curl
@cindex libcurl
@cindex libwww
@cindex W3C libwww

The WFDB path may contain @uref{http://} and @code{ftp://} URL prefixes (other
schema, such as @code{file://} and @code{https://}, may also be supported if
they are supported by your version of @code{libcurl} or @code{libwww}).  If
NETFILES support is not compiled into the WFDB library, any WFDB path
components containing @file{://} are ignored.  (These features were first
introduced in WFDB library version 10.1.0.)

@cindex indirect WFDB path
@cindex file containing WFDB path
@cindex database path file (indirect)

If the WFDB library finds that the value assigned to the WFDB path is of the
form @samp{@@@var{file}}, it replaces that value with the contents of
the specified @var{file}.  (This feature was first introduced in WFDB
library version 8.0.)  Indirect WFDB path files may be nested up to ten
levels (this arbitrary limit is imposed to avoid infinite recursion if
the contents of the indirect file are incorrect).  This method of
indirect assignment is useful under classic Mac OS, where recompilation of
the WFDB library would otherwise be necessary in order to change the WFDB
path.  It may also be useful under MS-DOS to reduce the need for
environment space, or if the length of the command needed to set the
@code{WFDB} environment variable would otherwise approach or exceed the
128-byte limit for MS-DOS commands.

If a WFDB header file (@pxref{Database Files}) specifies that a signal
file is to be found in a directory that is not already in the WFDB path,
that directory is appended to the end of the WFDB path; in this case, if
the WFDB path is not set, it is created with an initial null component
followed by the directory that contains the signal file.  (This feature
was first introduced in WFDB library version 6.2.)

The string @samp{%r} is replaced by the current record name wherever it appears
in the WFDB path;  @samp{%@var{N}r} is replaced by the first @var{N}
digits of the record name, if @var{N} is a non-zero digit.  For example,
if (under Unix) the WFDB path is @samp{:/cdrom/mimicdb/%3r:/cdrom/mitdb},
a request to read a file associated with record 055n will cause the WFDB
library to look first in the current directory (since the WFDB path begins
with an empty component), then in @samp{/cdrom/mimicdb/055}, and then in
@samp{/cdrom/mitdb}.  If @samp can be included in the WFDB
path by `escaping' it as @samp.  (Substitutions of @samp{%}-strings
in the WFDB path were first introduced in WFDB library version 9.7.)

@node     exercises 1, , WFDB path syntax, Usage
@section Exercises

These exercises should require only a few minutes.  If you work through
them, you will have an opportunity to become acquainted with a few of
the most common errors in using the WFDB library.

@enumerate

@item
Compile the example program in this chapter and run it.  If the WFDB
Software Package has not already been installed on your system,
download and install the most recent version from PhysioNet first
(@pxref{Installation, , Installing the WFDB Software Package}).

@item
Find out where database records are kept on your system.  What records
are available locally?

@item
Modify the example program so that you can specify the record to be
opened, either as a command-line argument or by having the program prompt
you to type a record name.  If you are unfamiliar with command-line
argument processing, @pxref{Example 2}.

@item
Use the modified version of the example to read samples from records
@file{mitdb/200}, @file{edb/e0103}, @file{slpdb/slp04}, and
@file{mimicdb/237/237}.  The last two of these records have 4 and 6 signals
respectively, so you will need to make a few additional changes to the program
in order to read these records successfully.

@item
Once again using the modified version of the example, what happens if
you omit the path information from one of the records in the previous
exercise (for example, if you try to open @file{e0103} instead of
@file{edb/e0103}?  Figure out how to set the WFDB path so that the
program will work properly in this case.  (Hint: use the application
@file{wfdbwhich}, included with the WFDB Software Package, to find the
header file for record @file{edb/e0103}; this information will help
you to determine how to set the WFDB path.)

@item
If you use MS-DOS or MS-Windows, explore and explain what happens in
the previous exercise if you type the record name using upper-case
letters, or if you type a @samp{\} (backslash) instead of @samp{/}
(forward slash).  (Hint: record names are @emph{not} filenames!)

@item
What happens when you compile the example program as shown, but with the
@code{#include} statement omitted?  with the @samp{-lwfdb} (@samp{-link wfdb},
etc.) omitted?

@item
What is the type of the argument to @code{getvec}?  Why can't @code{getvec}
simply return the value it reads, as in @samp{v = getvec()}?
@end enumerate

@node     Functions, Data Types, Usage, Top
@chapter WFDB Library Functions
@cindex WFDB library functions
@cindex library functions
@cindex functions in the WFDB library

This chapter describes the functions that are available to programs
compiled with the WFDB library.  The functions are introduced in
several groups, with examples to illustrate their usage.

@menu
* introduction to functions::	General notes on functions.
				This node discusses arguments, return codes,
				and the organization of this section of
				the guide.

The remainder of the nodes in this section describe functions for:
				
* selecting::			Selecting database records (opening files).
* special I/O modes::           Setting the input sampling frequency and more.
* signal and annotation I/O::	Reading and writing signals and annotations.
* non-sequential::		Non-sequential access to WFDB files.
* conversion::			Time and other conversion functions.
* calibration::			Calibrating signals.
* miscellaneous functions::	Attribute-reading and other functions.
* Memory Allocation Macros::    Memory allocation with error-checking.
@end menu

@node     introduction to functions, selecting, Functions, Functions
@unnumberedsec About these functions

@cindex arguments
@cindex function arguments
@cindex pointer arguments
Each function description begins with an ANSI C function prototype, which
specifies the types of any arguments as well as the type of the quantity
returned by the function (see @cite{K&R}, pp. 217--218).  Note that many
of these functions take pointer arguments.  These can be traps for
newcomers to C.  Study the examples carefully!  Often a function will
return information to the caller in a variable or structure to which the
pointer argument points.  @strong{It is necessary in such cases for the
caller to allocate storage for the variables or structures and to
initialize the pointers so that they point to the allocated storage.  If
you fail to do so, the compiler probably will not warn you of your
error; instead your program will fail mysteriously, probably with a core
dump and an ``illegal memory reference'' error message.}

@cindex errors
@cindex function return codes
@cindex return codes
With few exceptions, WFDB library functions return integers that indicate
success or failure.  The tables that follow the function prototypes
list the possible returns and their meanings.  By convention, a
return code of
@ifhtml
-1
@end ifhtml
@ifinfo
-1
@end ifinfo
@iftex
@minus{}1
@end iftex
indicates end-of-file on input files, and no
error message is printed.  Other negative return codes signify other
types of errors, and are usually accompanied by descriptive messages on
the standard error output (but
@ifhtml
@pxref{wfdbquiet and wfdbverbose, , @code{wfdbquiet} and @code{wfdbverbose}}).
@end ifhtml
@ifinfo
@pxref{wfdbquiet and wfdbverbose}).
@end ifinfo
@iftex
@pxref{wfdbquiet and wfdbverbose, , @code{wfdbquiet} and @code{wfdbverbose}}).
@end iftex
Zero may indicate success or failure, depending on context (see the
descriptions of the individual functions below).  Positive codes
(returned by only a few functions) always indicate success.

A comprehensive discussion of database files appears later in this guide
(@pxref{Database Files}).  Most readers should not need to learn about
the gruesome details of how the data are actually stored.  You should
know, however, that there are files that contain digitized signals,
other files that contain annotations, and still others (called
@dfn{header} files) that describe attributes of the signals such as
sampling frequency.  The database path lists directories in which
database files are found; the WFDB library functions can find them given
only the record (and annotator) names, provided that @code{WFDB} has been
properly set (@pxref{WFDB path}).  WFDB library functions responsible for
opening signal files find them by reading the header file (which
contains their names) first.

The first two sections of this chapter describes functions that extract
information from header files in order to gain access to signal
and annotation files, and functions that control how these files
are read and written.  The following two sections describe functions
that read and write signal and annotation files.  Many readers will not need
to go any further; the remaining sections deal with special-purpose
functions that exist to serve unusual applications.

@page
@node selecting,special I/O modes,introduction to functions,Functions
@section Selecting Database Records
@cindex selecting database records
@cindex opening database files
@cindex initialization

@menu
* annopen::		Opening input and output annotation files.
* isigopen::		Opening input signal files.
* osigopen::		Opening output signal files.
* osigfopen::		Opening output signal files by name.
* wfdbinit::		@code{annopen} and @code{isigopen} in one function.
@end menu

@c @group
@node     annopen, isigopen, selecting, selecting
@unnumberedsubsec annopen
@findex annopen
@cindex annotation files (opening)
@cindex creating annotation files
@cindex opening annotation files

@example
int annopen(char *@var{record}, WFDB_Anninfo *@var{aiarray}, unsigned int @var{nann})
@end example
@noindent
@strong{Return:}
@table @asis
@item @t{ 0}
Success
@item @t{-3}
Failure: unable to open input annotation file
@item @t{-4}
Failure: unable to open output annotation file
@item @t{-5}
Failure: illegal @code{stat} (in @var{aiarray}) specified for annotation file
@end table
@c @end group

@noindent
This function opens input and output annotation files for a selected
record.  If @var{record} begins with @samp{+}, previously opened
annotation files are left open, and the record name is taken to be the
remainder of @var{record} after discarding the @samp{+}.  Otherwise,
@code{annopen} closes any previously opened annotation files, and takes
all of @var{record} as the record name.  @var{aiarray} is a pointer to
an array of @code{WFDB_Anninfo} structures
@ifnotinfo
(@pxref{WFDB_Anninfo structures, , Annotator Information Structures}),
@end ifnotinfo
@ifinfo
(@pxref{WFDB_Anninfo structures}),
@end ifinfo
one for each annotator to be opened.  @var{nann} is the number of
@code{WFDB_Anninfo} structures in @var{aiarray}.  The caller must
fill in the @code{WFDB_Anninfo} structures to specify the names of the
annotators, and to indicate which annotators are to be read, and which
are to be written.  Input and output annotators may be listed in any
order in @var{aiarray}.  @dfn{Annotator numbers} (for both input and
output annotators) are assigned in the order in which the annotators
appear in @var{aiarray}.  For example, this code fragment

@example
@dots{}
char *record = "100s";
WFDB_Anninfo a[3];

a[0].name = "a"; a[0].stat = WFDB_READ;
a[1].name = "b"; a[1].stat = WFDB_WRITE;
a[2].name = "c"; a[2].stat = WFDB_READ;
if (annopen(record, a, 3) < 0)
@dots{}
@end example

@noindent
attempts to open three annotation files for record @file{100s}.
Annotator @file{a} becomes input annotator 0, @file{b} becomes output
annotator 0, and @file{c} becomes input annotator 1.  Thus
@code{getann(1, &annot)}
@ifnotinfo
(@pxref{getann, , @code{getann}})
@end ifnotinfo
@ifinfo
(@pxref{getann})
@end ifinfo
will read an annotation from annotator @file{c}, and
@code{putann(0, &annot)} will write an annotation for annotator @file{b}.
Input annotation files will be found if they are located in any of the
directories specified by @code{WFDB} (@pxref{WFDB path}); output annotators
are created in the current directory (but note that, under Unix at
least, it is possible to specify annotator names such as @file{/here} or
@file{zzz/there} or even @file{../somewhere/else}; @pxref{Annotation
Files}, for details of how file names are constructed from annotator and
record names).  Several of the example programs in chapter 6 illustrate
the use of @code{annopen}; for example, @pxref{Example 1}.

As a special case, if @var{nann} is 0, @var{aiarray} can be @code{NULL}.
This can be useful to force open annotation files to be closed without
closing open signal files.

@c @group
@node     isigopen, osigopen, annopen, selecting
@unnumberedsubsec isigopen
@findex isigopen
@cindex signal files (opening)
@cindex opening signal files

@example
int isigopen(char *@var{record}, WFDB_Siginfo *@var{siarray}, int @var{nsig})
@end example
@noindent
@strong{Return:}
@table @asis
@item @t{>0}
Success: the returned value is the number of input signals (i.e., the
number of valid entries in @var{siarray})
@item @t{ 0}
Failure: no input signals available
@item @t{-1}
Failure: unable to read header file (probably incorrect record name)
@item @t{-2}
Failure: incorrect header file format
@end table
@c @end group

@noindent
This function opens input signal files for a selected record.  If
@var{record} begins with @samp{+}, previously opened input signal files
are left open, and the record name is taken to be the remainder of
@var{record} after discarding the @samp{+}.  Otherwise, @code{isigopen}
closes any previously opened input signal files, and takes all of
@var{record} as the record name.  If the record name is @samp{-},
@code{isigopen} reads the standard input rather than a @file{hea}
file.  @var{Siarray} is a pointer to an array of @code{WFDB_Siginfo}
structures
@ifnotinfo
(@pxref{WFDB_Siginfo structures, , Signal Information Structures}),
@end ifnotinfo
@ifinfo
(@pxref{WFDB_Siginfo structures}),
@end ifinfo
one for each signal to be opened.

As a special case, if @var{nsig} is 0, @var{siarray} can be @code{NULL}.
In this case, @code{isigopen} closes any open input signals, then
returns the number of signals in @var{record} without opening them.  Use
this feature to determine the amount of storage needed for
signal-related variables, as in the example below, or to force open
input signal files to be closed without closing open annotation or
output signal files.  This action also sets internal WFDB library
variables that record the base time and date, the length of the record,
and the sampling and counter frequencies, so that time conversion
functions such as @code{strtim} that depend on these quantities will
work properly.

If @code{nsig} is greater than 0, @code{isigopen} normally returns the
number of input signals it actually opened, which may be less than
@var{nsig} but is never greater than @var{nsig}.  The caller must
allocate storage for the @code{WFDB_Siginfo} structures; @code{isigopen}
will fill them in with information about the signals.  @dfn{Signal
numbers} are assigned in the order in which signals are specified in the
@file{hea} file for the record; on return from @code{isigopen},
information for signal @var{i} will be found in @var{siarray[i]}.  For
example, we can read the @code{gain} attributes of each signal in record
@file{100s} like this:

@c @group
@example
#include <stdio.h>
#include <wfdb/wfdb.h>

main()
@{
    int i, nsig;
    WFDB_Siginfo *siarray;

    nsig = isigopen("100s", NULL, 0);
    if (nsig < 1)
        exit(1);
    siarray = (WFDB_Siginfo *)malloc(nsig * sizeof(WFDB_Siginfo));
    nsig = isigopen("100s", siarray, nsig);
    for (i = 0; i < nsig; i++)
        printf("signal %d gain = %g\n", i, siarray[i].gain);
    exit(0);
@}
@end example
@c @end group

@noindent
(See @uref{http://www.physionet.org/physiotools/wfdb/examples/pgain.c}
for a copy of this program.)

@noindent
This program, unlike the example in the previous chapter, does not
assume that the number of signals is known.  The first invocation of
@code{isigopen} determines this number (and the program quits if there
are no signals).  Next, the program allocates the array for the signal
information, and then it opens the signals using the second
invocation of @code{isigopen}, passing in the pointer @code{siarray} and the
number of signals determined from the first call (@code{nsig}).

An error message is produced if @code{isigopen} is unable to open
@emph{any} of the signals listed in the header file.  It is not
considered an error if only some of the signals can be opened, however.
A signal will not be opened if its signal file is unreadable, if an
input buffer cannot be allocated for it, or if opening all of the
signals in its group would exceed the limits defined by @var{nsig}.
(Note, however, that most records have only one signal group; as a
consequence, @code{isigopen} fails if @var{nsig} is less than the
total number of signals in such cases.)  If necessary, the caller can
inspect the file names and signal descriptions in @var{siarray} to
determine which signals were opened; 
@ifnotinfo
@pxref{WFDB_Siginfo structures, , Signal Information Structures}.
@end ifnotinfo
@ifinfo
@pxref{WFDB_Siginfo structures}.
@end ifinfo
Several of the example programs in chapter 6
illustrate the use of @code{isigopen}; for example, @pxref{Example
5}.

If @var{nsig} is less than 0, @code{isigopen} fills in up to
@ifnotinfo
@minus{}@var{nsig}
@end ifnotinfo
@ifinfo
-@var{nsig}
@end ifinfo
members of @var{siarray}, based on information from the header file
for @var{record}, but @emph{no signals are actually opened}.  The value
returned in this case is the number of signals named in the @file{hea} file.
Note, however, that there is no guarantee that all (or indeed any) of the
signals named in the @file{hea} file are available to be opened.
The features described in this paragraph were first introduced in
version 4.4 of the WFDB library.

@c @group
@node     osigopen, osigfopen, isigopen, selecting
@unnumberedsubsec osigopen
@findex osigopen
@cindex signal files (creating)
@cindex creating signal files

@example
int osigopen(char *@var{record}, WFDB_Siginfo *@var{siarray}, unsigned int @var{nsig})
@end example
@noindent
@strong{Return:}
@table @asis
@item @t{>0}
Success: the returned value is the number of output signals; this number
should match @var{nsig}
@item @t{-1}
Failure: unable to read header file
@item @t{-2}
Failure: incorrect header file format
@item @t{-3}
Failure: unable to open output signal(s)
@end table
@c @end group

@noindent
This function opens output signal files.  Use it only if signals are to
be @emph{written} using @code{putvec}.  The signal specifications,
including the file names, are read from the header file for a
specified record.  Unmodified MIT or AHA database @file{hea} files
cannot be used, since @code{osigopen} would attempt to overwrite the
(write-protected) signal files named within.  If @var{record} begins
with @samp{+}, previously opened output signal files are left open, and
the record name is taken to be the remainder of @var{record} after
discarding the @samp{+}.  Otherwise, @code{osigopen} closes any
previously opened output signal files, and takes all of @var{record} as
the record name.  If the record name is @samp{-}, @code{osigopen} reads
the standard input rather than a @file{hea} file.  @var{siarray} is a
pointer to an uninitialized array of @code{WFDB_Siginfo} structures;
@var{siarray} must contain at least @var{nsig} members.  The caller must
allocate storage for the @code{WFDB_Siginfo} structures.  On return,
@code{osigopen} will have filled in the @code{WFDB_Siginfo} structures with
the signal specifications.

No more than @var{nsig} (additional) output signals will be opened by
@code{osigopen}, even if the header file contains specifications for
more than @var{nsig} signals.  For example, this code fragment

@example
@dots{}
WFDB_Siginfo s[2];
int i, nsig;

nsig = osigopen("8l", s, 2);
for (i = 0; i < nsig; i++)
    printf("signal %d will be written into `%s'\n", i, s[i].fname);
@dots{}
@end example

@noindent
creates 2 output signals named @file{data0} and @file{data1}
(@pxref{Piped and Local Records}).  @xref{Example 6}, and @pxref{Example 7},
for illustrations of the use of @code{osigopen}.

As a special case, if @var{nsig} is 0, @var{siarray} can be @code{NULL}.
This can be useful to force open output signal files to be closed
without closing open annotation or input signal files.

@c @group
@node     osigfopen, wfdbinit, osigopen, selecting
@unnumberedsubsec osigfopen
@findex osigfopen
@cindex signal files (creating)
@cindex creating signal files

@example
int osigfopen(WFDB_Siginfo *@var{siarray}, unsigned int @var{nsig})
@end example
@noindent
@strong{Return:}
@table @asis
@item @t{>0}
Success: the returned value is the number of output signals;  this number
should match @var{nsig}
@item @t{-2}
Failure: error in signal specification (@code{fname} or @code{desc} too
long, illegal @code{fmt} or @code{bsize}, or incorrect signal group
assignment)
@item @t{-3}
Failure: unable to open output signal(s)
@end table
@c @end group

@noindent
This function opens output signals as does @code{osigopen}, but the
signal specifications, including the signal file names, are supplied by
the caller to @code{osigfopen}, rather than read from a header
file as in @code{osigopen}.  Any previously open output signals are
closed by @code{osigfopen}.  @var{siarray} is a pointer to an array of
@code{WFDB_Siginfo} structures
@ifnotinfo
(@pxref{WFDB_Siginfo structures, , Signal Information Structures}),
@end ifnotinfo
@ifinfo
(@pxref{WFDB_Siginfo structures}),
@end ifinfo
one for each signal to be opened.  @var{nsig} is the number of
@code{WFDB_Siginfo} structures in @var{siarray}.

Before invoking @code{osigfopen}, the caller must fill in the fields of
the @code{WFDB_Siginfo} structures in @var{siarray} (@pxref{Data Types};
the @code{initval}, @code{nsamp}, and @code{cksum} fields may be left
uninitialized, however).  To make a multiplexed signal file, specify the
same @code{fname} and @code{group} for each signal to be included
(@pxref{Multiplexed Signal Files}).  For ordinary (non-multiplexed)
signal files, specify a unique @code{fname} and @code{group} for each
signal.  @xref{Example 8}, for an illustration of the use of
@code{osigfopen}.

As a special case, if @var{nsig} is 0, @var{siarray} can be @code{NULL}.
This can be useful to force open output signal files to be closed
without closing open annotation or input signal files.

@c @group
@node     wfdbinit, , osigfopen, selecting    
@unnumberedsubsec wfdbinit
@findex wfdbinit

@example
int wfdbinit(char *@var{record}, WFDB_Anninfo *@var{aiarray}, unsigned int @var{nann},
           WFDB_Siginfo *@var{siarray}, unsigned int @var{nsig})
@end example
@noindent
@strong{Return:}
@table @asis
@item @t{>0}
Success: the returned value is the number of input signals (i.e., the
number of valid entries in @var{siarray})
@item @t{ 0}
Annotation files opened successfully, input signals unavailable
(not an error for programs that don't need them; no error
message is printed if @var{nsig} is 0)
@item @t{-1}
Failure: unable to read header file (probably incorrect record
name)
@item @t{-2}
Failure: incorrect header file format
@item @t{-3}
Failure: unable to open input annotation file
@item @t{-4}
Failure: unable to open output annotation file
@item @t{-5}
Failure: illegal @code{stat} (in @var{aiarray}) specified for annotation file
@ifnotinfo
(@pxref{WFDB_Anninfo structures, , Annotator Information Structures})
@end ifnotinfo
@ifinfo
(@pxref{WFDB_Anninfo structures})
@end ifinfo
@end table
@c @end group

@noindent
This function opens database files other than output signal
files for a selected record.  The code

@example
n = wfdbinit(record, a, na, s, ns);
@end example

@noindent
is exactly equivalent to

@example
n = annopen(record, a, na);
if (n == 0)
    n = isigopen(record, s, ns);
@end example

@noindent
Avoid using @code{wfdbinit} and @code{setifreq} in the same program.
@xref{Example 9}, for an illustration of the use of @code{wfdbinit}.
@ifnotinfo
@xref{osigopen, , @code{osigopen}}, and @pxref{osigfopen, , @code{osigfopen}},
@end ifnotinfo
@ifinfo
@xref{osigopen}, and @pxref{osigfopen},
@end ifinfo
for methods of opening output signal files.

@page
@node     special I/O modes, signal and annotation I/O, selecting, Functions
@section Special I/O Modes

@menu
* setifreq::			Setting the input sampling frequency.
* getifreq::                    Determining the input sampling frequency.
* setgvmode::                   Setting the resolution for a multifrequency
                                record and how invalid samples are represented.
* getgvmode::                   Determining the getvec operating mode.
* getspf::                      Determining the number of samples per frame.
* setafreq::			Setting the time resolution for output
                                annotations.
* getafreq::                    Determining the time resolution for output
                                annotations.
@end menu

@c @group
@node     setifreq, getifreq, special I/O modes, special I/O modes
@unnumberedsubsec setifreq
@findex setifreq (10.2.6)
@cindex interpolation
@cindex decimation
@cindex resampling

@example
void setifreq(WFDB_Frequency @var{frequency})
@end example
@noindent
@c @end group

@noindent
This function sets the current input sampling frequency (in samples per
second per signal).  It should be invoked after opening the input
signals (using @code{isigopen} or @code{wfdbinit}), and before using any
of @code{getvec}, @code{getann}, @code{putann}, @code{isigsettime},
@code{isgsettime}, @code{timstr}, @code{mstimstr}, @code{strtim},
or @code{annopen}.
@emph{Note that the operation of @code{getframe} is unaffected by
@code{setifreq}.}

Use @code{setifreq} when your application requires input samples at a
specific frequency.  After invoking @code{setifreq}, @code{getvec}
resamples the digitized signals from the input signals at the desired
frequency (@pxref{getvec}), and all of the WFDB library functions that
accept or return times in sample intervals automatically convert between
the actual sampling intervals and those corresponding to the desired
frequency.  This slightly elaborated version of the example program from
the previous chapter invokes @code{setifreq}, passing it the desired sampling
frequency from the command line, then prints the samples in record 100s,
beginning 1 second (@code{t0}) and ending 2 seconds (@code{t1}) from the
beginning of the record:

@example
#include <stdio.h>
#include <wfdb/wfdb.h>

main(int argc, char **argv)
@{
    WFDB_Frequency f = (WFDB_Frequency)0;
    WFDB_Sample v[2];
    WFDB_Siginfo s[2];
    WFDB_Time t, t0, t1;

    if (argc > 1) sscanf(argv[1], "%lf", &f);
    if (f <= (WFDB_Frequency)0) f = sampfreq("100s");

    if (isigopen("100s", s, 2) < 1)
        exit(1);
    setifreq(f);
    t0 = strtim("1");
    isigsettime(t0);
    t1 = strtim("2");
    for (t = t0; t <= t1; t++) @{
        if (getvec(v) < 0)
            break;
        printf("%d\t%d\n", v[0], v[1]);
    @}
    exit(0);
@}
@end example

@noindent
(See @uref{http://www.physionet.org/physiotools/wfdb/examples/psamplex.c} for a
copy of this program.  Compile it as shown in the previous chapter, then run it
using a command such as @samp{psamplex 100}.)  The QRS detector in chapter 6
also illustrates the use of @code{setifreq} (@pxref{Example 10}).

@noindent
Avoid using @code{wfdbinit} and @code{setifreq} in the same program.

@c @group
@node     getifreq, setgvmode, setifreq, special I/O modes
@unnumberedsubsec getifreq
@findex getifreq (10.2.6)

@example
WFDB_Frequency getifreq(void)
@end example
@noindent
@strong{Return:}
@table @asis
@item @t{(WFDB_Frequency)}
the input sampling frequency
@end table
@c @end group

@noindent
This function returns the current input sampling frequency (in samples
per second per signal), which is either the raw sampling frequency for
the record (as would be returned by @code{sampfreq}, @pxref{sampfreq}),
or the frequency chosen using a previous invocation of @code{setifreq}.

@c @group
@node     setgvmode, getgvmode, getifreq, special I/O modes
@unnumberedsubsec setgvmode
@findex setgvmode (9.0)
@cindex interpolation
@cindex decimation
@cindex resampling
@cindex multi-frequency records (reading)
@cindex padding
@cindex invalid samples
@cindex missing samples
@cindex samples (invalid or missing)

@example
void setgvmode(int *@var{mode})
@end example
@noindent
@c @end group

@noindent
This function sets the mode used by @code{getvec} when reading a
multi-frequency record (@pxref{Multi-Frequency Records}).  If @var{mode} is
@code{WFDB_LOWRES}, @code{getvec} decimates any signals sampled at multiples of
the frame rate, so that one sample is returned per signal per frame (i.e., the
oversampled signals are resampled by simple averaging of the samples for each
signal within each frame).  If @var{mode} is @code{WFDB_HIGHRES}, each sample
of any oversampled signal is returned by successive invocations of
@code{getvec}, and each sample of any signal sampled at a lower frequency is
returned by two or more successive invocations of @code{getvec} (i.e., the less
frequently sampled signals are resampled using zero-order interpolation).
@code{getvec} operates in @code{WFDB_LOWRES} mode by default.
@code{WFDB_LOWRES} and @code{WFDB_HIGHRES} are defined in @file{<wfdb/wfdb.h>}.

In WFDB library version 9.6 and later versions, @code{setgvmode} also affects
how annotations are read and written.  If @code{setgvmode(WFDB_HIGHRES)} is
invoked @emph{before} using @code{annopen}, @code{wfdbinit}, @code{getvec},
@code{sampfreq}, @code{strtim}, or @code{timstr}, then all @code{WFDB_Time}
data (including the @code{time} attributes of annotations read by @code{getann}
or written by @code{putann}) visible to the application are in units of the
high-resolution sampling intervals.  (Otherwise, @code{WFDB_Time} data are in
units of frame intervals.)

Version 10.4 and later versions of the WFDB library support two modes
of handling invalid or missing samples.  By default, @code{getframe},
@code{getvec}, and @code{sample} return the special value
@code{WFDB_INVALID_SAMPLE} in such cases.  If @var{mode} is
@w{@code{WFDB_GVPAD} + @code{WFDB_LOWRES}} or 
@w{@code{WFDB_GVPAD} + @code{WFDB_HIGHRES}},
however, these functions replicate the previous
valid sample whenever they encounter an invalid or missing sample,
which may simplify the design of applications such as digital filters.
The constant @code{WFDB_GVPAD} is defined in @file{<wfdb/wfdb.h>}.

@c @group
@node     setgvmode, getspf, setgvmode, special I/O modes
@unnumberedsubsec getgvmode
@findex getgvmode (10.5.3)
@cindex interpolation
@cindex decimation
@cindex resampling
@cindex multi-frequency records (reading)
@cindex padding
@cindex invalid samples
@cindex missing samples
@cindex samples (invalid or missing)

@example
int setgvmode(void)
@end example
@noindent
@c @end group

@noindent
This function returns the operating mode used by @code{getvec}.  If the
returned value has the @code{WFDB_HIGHRES} bit set, @code{getvec} is operating
in high-resolution mode;  if the returned value has the @code{WFDB_GVPAD} bit
set, @code{getvec} replicates the previous valid sample whenever it encounters
an invalid or missing sample, rather than returning the value
@code{WFDB_INVALID_SAMPLE}.  The constants @code{WFDB_HIGHRES},
@code{WFDB_GVPAD}, and @code{WFDB_INVALID_SAMPLE} are defined in
@file{<wfdb/wfdb.h>}.

@c @group
@node     getspf, setafreq, setgvmode, special I/O modes
@unnumberedsubsec getspf
@findex getspf (9.6)
@cindex interpolation
@cindex decimation
@cindex resampling
@cindex multi-frequency records (reading)

@example
int getspf(void)
@end example
@noindent
@strong{Return:}
@table @asis
@item @t{(int)}
the number of samples per signal per frame
@end table
@c @end group

@noindent
Unless the application is operating in @code{WFDB_HIGHRES} mode
(@pxref{setgvmode}) and has then opened a multi-frequency record, this
function returns 1.  For the case of a multi-frequency record being read
in high resolution mode, however, @code{getspf} returns the number of
samples per signal per frame (hence @code{sampfreq(NULL)/getspf()} is
the number of frames per second).

@node     setafreq, getafreq, getspf, special I/O modes
@unnumberedsubsec setafreq
@findex setafreq (10.4.5)
@cindex interpolation
@cindex decimation
@cindex resampling
@cindex resolution

@example
void setafreq(WFDB_Frequency @var{frequency})
@end example
@noindent
@c @end group

@noindent
This function sets the time resolution, in ticks per second, for any output
annotation files created after it has been invoked.  By default, the time
resolution is equal to the input sampling frequency (and @code{setifreq}
invokes @code{setafreq} to maintain this behavior if the input sampling
frequency is changed).

@noindent
This function has no effect on output annotation files that are already open
when it is invoked, nor on annotations read from input annotation files
(for which the @code{time} fields always are expressed in units equal to
the input sample intervals).

@c @group
@node     getafreq, , setafreq, special I/O modes
@unnumberedsubsec getafreq
@findex getafreq (10.4.5)

@example
WFDB_Frequency getafreq(void)
@end example
@noindent
@strong{Return:}
@table @asis
@item @t{>0}
output annotation time resolution, if previously set by @code{setafreq}
@item @t{0}
otherwise
@end table
@c @end group

@noindent
This function returns the current output annotation time resolution
in ticks per second if it has been set using @code{getafreq}.

@page
@node     signal and annotation I/O, non-sequential, special I/O modes, Functions
@section Reading and Writing Signals and Annotations
@cindex signal I/O

@menu
* getvec::		Reading input signals.
* getframe::            Reading input signals from multifrequency records.
* putvec::		Writing output signals.
* getann::		Reading annotations.
* ungetann::		Pushing an annotation onto an input stream.
* putann::		Writing annotations.
@end menu

@c @group
@node    getvec, getframe, signal and annotation I/O, signal and annotation I/O
@unnumberedsubsec getvec
@findex getvec
@cindex reading signals
@cindex signals (reading)

@example
int getvec(WFDB_Sample *@var{vector})
@end example
@noindent
@strong{Return:}
@table @asis
@item @t{>0}
Success;  the returned value is the number of input signals (the number of
valid entries in @var{vector})
@item @t{-1}
End of data (contents of @var{vector} not valid)
@item @t{-3}
Failure: unexpected physical end of file
@item @t{-4}
Failure: checksum error (detected only at end of file)
@end table
@c @end group

@noindent
This function reads a sample from each input signal.  The caller should
allocate storage for an array of @code{WFDB_Sample}s (integers) and pass a
pointer to this array to @code{getvec}.  (The length of the array must
be no less than the number of input signals, as obtained from
@code{isigopen} or @code{wfdbinit}.)  On return, @var{vector[i]}
contains the next sample from signal @var{i}.  For example, this
modified version of the example from chapter 1 reads and prints the
first ten samples of each available input signal:

@example
#include <stdio.h>
#include <malloc.h>
#include <wfdb/wfdb.h>

main()
@{
    int i, j, nsig;
    WFDB_Sample *v;
    WFDB_Siginfo *s;

    nsig = isigopen("100s", NULL, 0);
    if (nsig < 1)
        exit(1);
    s = (WFDB_Siginfo *)malloc(nsig * sizeof(WFDB_Siginfo));
    if (isigopen("100s", s, nsig) != nsig)
        exit(1);
    v = (WFDB_Sample *)malloc(nsig * sizeof(WFDB_Sample));
    for (i = 0; i < 10; i++) @{
        if (getvec(v) < 0)
            break;
        for (j = 0; j < nsig; j++)
            printf("%8d", v[j]);
        printf("\n");
    @}
    exit(0);
@}
@end example

@noindent
(See @uref{http://www.physionet.org/physiotools/wfdb/examples/exgetvec.c}
for a copy of this program.)

Notice how the value returned by the first invocation of @code{isigopen}
is used to determine how many input signals there are.  Several of the
example programs in chapter 6 illustrate the use of @code{getvec}; for
example, @pxref{Example 6}.

@findex setifreq (10.2.6)
@cindex interpolation
@cindex decimation
@cindex resampling
If @code{setifreq} has been used to modify the input sampling rate,
@code{getvec} resamples the input signals at the desired rate, using
linear interpolation between the pair of samples nearest in time to that
of the sample to be returned.  The results will generally be
satisfactory, provided that the original signals do not contain
frequencies near or above the Nyquist limit (half of the desired
sampling frequency).  If this is a concern, you may wish to low-pass
filter the input signals using, for example, @samp{fir} (see the
@cite{WFDB Applications Guide}) before resampling them.  If you use
@code{setifreq} to @emph{increase} the sampling frequency by a large
factor, you may wish to filter the resampled signals within your
application to remove harmonics of the original sampling frequency
introduced by resampling.

@c @group
@node    getframe, putvec, getvec, signal and annotation I/O
@unnumberedsubsec getframe
@findex getframe (9.0)
@cindex reading signals
@cindex signals (reading)
@cindex frames (reading)
@cindex multifrequency records

@example
int getframe(WFDB_Sample *@var{vector})
@end example
@noindent
@strong{Return:}
@table @asis
@item @t{>0}
Success;  the returned value is the number of input signals
@item @t{-1}
End of data (contents of @var{vector} not valid)
@item @t{-3}
Failure: unexpected physical end of file
@item @t{-4}
Failure: checksum error (detected only at end of file)
@end table
@c @end group

@noindent
This function reads a vector of samples, including at least one sample from
each open input signal.  If all signals are sampled at the same frequency, only
one sample is read from each signal.  Otherwise, signals sampled at multiples
of the frame frequency are represented by two or more consecutive elements of
the returned @var{vector}.  For example, if the frame frequency is 125 Hz,
signal 0 is sampled at 500 Hz, and the remaining 3 signals are sampled at 125
Hz each, then the returned @var{vector} has 7 valid components: the first 4 are
samples of signal 0, and the remaining 3 are samples of signals 1, 2, and 3.
The caller should allocate storage for an array of @code{WFDB_Sample}s
(integers) and pass a pointer to this array to @code{getframe}.  The length of
@var{vector} must be determined by summing the values of the @code{spf}
(samples per frame) fields in the @code{WFDB_Siginfo} structures associated
with the input signals
@ifnotinfo
(@pxref{isigopen, , @code{isigopen}}).
@end ifnotinfo
@ifinfo
(@pxref{isigopen}).
@end ifinfo

@c @group
@node     putvec, getann, getframe, signal and annotation I/O
@unnumberedsubsec putvec
@findex putvec
@cindex signals (writing)
@cindex writing signals

@example
int putvec(WFDB_Sample *@var{vector})
@end example
@noindent
@strong{Return:}
@table @asis
@item @t{>0}
Success: the returned value is the number of output signals (the number
of entries in @var{vector} that were written)
@item @t{ 0}
Slew rate too high for one or more signals (difference format only; the
DC level(s) will be corrected as soon as the slew rate permits)
@item @t{-1}
Failure: write error
@end table
@c @end group

@noindent
This function writes a sample to each input signal.  The caller should fill an
array of @code{WFDB_Sample}s with the samples and pass a pointer to this array
to @code{putvec}.  (The length of the array must be no less than the number of
output signals, as given to @code{osigfopen} or @code{osigopen}.)  On entry,
@var{vector[i]} contains the next sample from signal @var{i}.  For example,
this modified version of the previous example
@ifnotinfo
(@pxref{getvec, , @code{getvec}})
@end ifnotinfo
@ifinfo
(@pxref{getvec})
@end ifinfo
copies the first ten samples of each available input signal:

@example
#include <stdio.h>
#include <wfdb/wfdb.h>

main()
@{
    int i, j, nsig;
    WFDB_Sample *v;
    WFDB_Siginfo *s;

    nsig = isigopen("100s", NULL, 0);
    if (nsig < 1)
        exit(1);
    s = (WFDB_Siginfo *)malloc(nsig * sizeof(WFDB_Siginfo));
    if (isigopen("100s", s, nsig) != nsig || 
        osigopen("8l", s, nsig) != nsig)
        exit(1);
    v = (WFDB_Sample *)malloc(nsig * sizeof(WFDB_Sample));
    for (i = 0; i < 10; i++)
        if (getvec(v) < 0 || putvec(v) < 0)
            break;
    wfdbquit();
    exit(0);
@}
@end example

@noindent
(See @uref{http://www.physionet.org/physiotools/wfdb/examples/exputvec.c}
for a copy of this program.)

All programs that write signals or annotations @emph{must} invoke
@code{wfdbquit} to close the output files properly
@ifnotinfo
(@pxref{wfdbquit, , @code{wfdbquit}}).
@end ifnotinfo
@ifinfo
(@pxref{wfdbquit}).
@end ifinfo
This example uses record @file{8l} (@pxref{Piped and Local Records}) for the
output signal specifications; the output signal files will be named
@file{data0} and @file{data1} in the current directory.  Several of the example
programs in chapter 6 illustrate the use of @code{putvec}; for example,
@pxref{Example 6}.

@cindex annotation I/O
@c @group
@node     getann, ungetann, putvec, signal and annotation I/O
@unnumberedsubsec getann
@findex getann
@cindex annotations (reading)
@cindex reading annotations

@example
int getann(WFDB_Annotator @var{an}, WFDB_Annotation *@var{annot})
@end example
@noindent
@strong{Return:}
@table @asis
@item @t{ 0}
Success
@item @t{-1}
End of file (@var{*annot} is not valid)
@item @t{-2}
Failure: incorrect annotator number specified
@item @t{-3}
Failure: unexpected physical end of file
@end table
@c @end group

@noindent
This function reads the next annotation from the input annotator
specified by @var{an} into the annotation structure
@ifnotinfo
(@pxref{WFDB_Annotation structures, , Annotation Structures})
@end ifnotinfo
@ifinfo
(@pxref{WFDB_Annotation structures})
@end ifinfo
pointed to by @var{annot}.  The caller must allocate storage for the
annotation structure.  Input annotators are numbered 0, 1, 2, etc.  This
short program uses @code{getann} to read the contents of the reference
(@file{atr}) annotation file for record @file{100s}:

@example
#include <stdio.h>
#include <wfdb/wfdb.h>

main()
@{
    WFDB_Anninfo a;
    WFDB_Annotation annot;

    a.name = "atr"; a.stat = WFDB_READ;
    if (annopen("100s", &a, 1) < 0)
        exit(1);
    while (getann(0, &annot) == 0)
        printf("%s %s\n", mstimstr(annot.time), annstr(annot.anntyp));
    exit(0);
@}
@end example

@noindent
(See @uref{http://www.physionet.org/physiotools/wfdb/examples/exgetann.c}
for a copy of this program.)

@ifnotinfo
@xref{WFDB_Anninfo structures, , Annotator Information Structures},
@end ifnotinfo
@ifinfo
@xref{WFDB_Anninfo structures},
@end ifinfo
for information on the contents of the @code{WFDB_Anninfo} structure,
and
@ifnotinfo
@pxref{timstr and strtim, , @code{mstimstr}}, and
@pxref{annstr and strann, , @code{annstr}},
@end ifnotinfo
@ifinfo
@pxref{timstr and strtim}, and @pxref{annstr and strann},
@end ifinfo
for details of the functions used to print portions of the annotations
read by @code{getann} in this example.

@c @group
@node     ungetann, putann, getann, signal and annotation I/O
@unnumberedsubsec ungetann
@findex ungetann (5.3)
@cindex annotations (reading)
@cindex reading annotations
@cindex unreading annotations

@example
int ungetann(WFDB_Annotator @var{an}, WFDB_Annotation *@var{annot})
@end example
@noindent
@strong{Return:}
@table @asis
@item @t{ 0}
Success
@item @t{-1}
Failure: push-back buffer full (@code{*@var{annot}} was not pushed back)
@item @t{-2}
Failure: incorrect annotator number specified
@end table
@c @end group

@noindent
This function arranges for the annotation structure pointed to by
@var{annot} to be the next one read by @code{getann} from input
annotator @var{an}.  The pushed-back annotation need not necessarily be
one originally read by @code{getann}.  No more than one annotation may
be pushed back at a time for each input annotator.  (This function was
first introduced in WFDB library version 5.3.)

@c @group
@node     putann, , ungetann, signal and annotation I/O
@unnumberedsubsec putann
@findex putann
@cindex annotations (writing)
@cindex writing annotations

@example
int putann(WFDB_Annotator @var{an}, WFDB_Annotation *@var{annot})
@end example
@noindent
@strong{Return:}
@table @asis
@item @t{ 0}
Success
@item @t{-1}
Failure: write error
@item @t{-2}
Failure: incorrect annotator number specified
@end table
@c @end group

@noindent
This function writes the next annotation for the output annotator specified by
@var{an} from the annotation structure pointed to by @var{annot}.  Output
annotators are numbered 0, 1, 2, etc.  The caller must fill in all fields of
the annotation structure.  Using version 9.7 and later versions of the WFDB
library, annotations may be written in any order (@pxref{Annotation Order}).
Earlier versions require that annotations be supplied to @code{putann} in
canonical order, and return an error code of @t{-3} if an out-of-order
annotation is supplied.  All programs that write signals or annotations
@emph{must} invoke @code{wfdbquit} to close the output files properly
@ifnotinfo
(@pxref{wfdbquit, , @code{wfdbquit}}).
@end ifnotinfo
@ifinfo
(@pxref{wfdbquit}).
@end ifinfo
Several of the example programs in chapter 6 illustrate the use of
@code{putann}; for example, @pxref{Example 1}.

@page
@node     non-sequential, conversion, signal and annotation I/O, Functions
@section Non-Sequential Access to WFDB Files
@cindex random access
@cindex non-sequential access
@cindex skipping through WFDB files

The next three functions permit random access to signal and annotation
files.  It is not possible, however, to skip backwards on piped
input.

@menu
* isigsettime::			Setting time of next samples read.
* isgsettime::			As above, but for one signal group only.
* tnextvec::                    Finding the next valid sample in a signal.
* iannsettime::			Setting time of next annotations read.
* sample::			A random access interface to input signals.
@end menu

@c @group
@node     isigsettime, isgsettime, non-sequential, non-sequential
@unnumberedsubsec isigsettime
@findex isigsettime
@cindex signals (non-sequential access)

@example
int isigsettime(WFDB_Time @var{t})
@end example
@noindent
@strong{Return:}
@table @asis
@item @t{ 0}
Success
@item @t{-1}
Failure: EOF reached or improper seek
@end table
@c @end group

@noindent
This function resets the input signal file pointers so that the next
samples returned from @code{getvec} will be those with sample number
|@var{t}|.  Only the magnitude of @var{t} is significant, not its sign;
hence values returned by @code{strtim} can always be used safely as
arguments to @code{isigsettime}
@ifnotinfo
(@pxref{timstr and strtim, , @code{timstr} and @code{strtim}}).
@end ifnotinfo
@ifinfo
(@pxref{timstr and strtim}).
@end ifinfo
This function will fail if a pipe is used for input and |@var{t}| is less
than the current sample number.  @xref{Example 7}, and @pxref{Example 9},
for illustrations of the use of @code{isigsettime}.

@c @group
@node     isgsettime, tnextvec, isigsettime, non-sequential
@unnumberedsubsec isgsettime
@findex isgsettime
@cindex signals (non-sequential access)

@example
int isgsettime(WFDB_Group @var{sgroup}, WFDB_Time @var{t})
@end example
@noindent
@strong{Return:}
@table @asis
@item @t{ 0}
Success
@item @t{-1}
Failure: EOF reached or improper seek
@item @t{-2}
Failure: incorrect signal group number specified
@end table
@c @end group

@noindent
This function does the job of @code{isigsettime}, but only for the signal
group specified by @var{sgroup}.  This function may be of use if more than
one record is open simultaneously (@pxref{Multiple Record Access}).

@c @group
@node     tnextvec, iannsettime, isgsettime, non-sequential
@unnumberedsubsec tnextvec
@findex tnextvec
@cindex signals (non-sequential access)
@cindex finding valid samples
@cindex skipping gaps in signals
@cindex invalid samples (skipping)
@cindex valid samples (searching for)

@example
WFDB_Time tnextvec(WFDB_Signal @var{s}, WFDB_Time @var{t})
@end example
@noindent
@strong{Return:}
@table @asis
@item @t{>= 0}
Time of the next valid sample of signal @var{s} at or after @var{t}
@item @t{-1}
Failure: EOF reached or improper seek
@end table
@c @end group

@noindent
This function resets the input signal file pointers so that the next
samples read by @code{getvec} will include the next valid sample of
the specified signal @var{s} occurring at or after @var{t}.  Use
@code{tnextvec} to skip lengthy gaps in a signal of interest
efficiently.

@c @group
@node     iannsettime, sample, tnextvec, non-sequential
@unnumberedsubsec iannsettime
@findex iannsettime
@cindex annotations (non-sequential access)

@example
int iannsettime(WFDB_Time @var{t})
@end example
@noindent
@strong{Return:}
@table @asis
@item @t{ 0}
Success
@item @t{-1}
Failure: EOF reached or improper seek
@item @t{-3}
Failure: unexpected physical end of file
@end table
@c @end group

@noindent
This function resets the input annotation file pointers so that the next
annotation read by @code{getann} from each input annotation file will be
the first occurring on or after sample number |@var{t}| in that file.
Only the magnitude of @var{t} is significant, not its sign; hence values
returned by @code{strtim} can always be used safely as arguments to
@code{iannsettime}
@ifnotinfo
(@pxref{timstr and strtim, , @code{timstr} and @code{strtim}}).
@end ifnotinfo
@ifinfo
(@pxref{timstr and strtim}).
@end ifinfo
This function will fail if a pipe is used for input and |@var{t}| is less
than the time of the most recent annotation read from the pipe.
@xref{Example 9}, for an illustration of the use of @code{iannsettime}.

@c @group
@node     sample, , iannsettime, non-sequential
@unnumberedsubsec sample and sample_valid
@findex sample (10.3.0)
@findex sample_valid (10.3.0)

@example
WFDB_Sample sample(WFDB_Signal @var{s}, WFDB_Time @var{t})
int sample_valid(void)
@end example
@noindent
@strong{Return:}
@table @asis
@item @i{ n}
(from @code{sample}): The value (in raw adus) of sample number @var{t}
in open signal @var{s},if successful, or the value of the previous
successfully read sample.
@item @t{ 1}
(from @code{sample_valid}): The most recent value returned by
@code{sample} was valid
@item @t{ 0}
(from @code{sample_valid}): The most recent @var{t} given to @code{sample}
follows the end of the record
@item @t{-1}
(from @code{sample_valid}): The most recent value returned by
@code{sample} was invalid (because signal @var{s} is not available at time
@var{t})
@end table
@c @end group

@noindent
The @code{sample} function allows the caller to read samples of the
currently open input signals in any order.  The first argument is a
signal number (a non-negative integer between 0 and @var{nsig}-1,
where @var{nsig} is the number of open input signals), and the second
is a time, expressed as a non-negative sample number. If @code{sample}
is invoked with valid input arguments, the companion function
@code{sample_valid} returns 1.

The @code{sample_valid} function can be used to check the results of the most
recent invocation of @code{sample}.  If @code{sample_valid} returns 1, the
latest value returned by @code{sample} is valid.  @code{sample_valid} returns 0
if @code{sample} has most recently returned a padded value following the end of
the record.  This allows code that uses @code{sample} to use the condition
@samp{sample_valid() != 0} (or simply @samp{sample_valid()} to determine if
more samples are available.  If @code{sample_valid} returns -1, the most recent
value returned by @code{sample} was @code{WFDB_INVALID_SAMPLE} (because the
requested signal @var{s} was unavailable at the requested time @var{t}).  Use
the condition @samp{sample_valid() > 0} to check if the most recent value
returned by @code{sample} is a valid value (e.g., suitable for inclusion in a
running average or similar calculation).

For an example of the use of @code{sample} and @code{sample_valid},
@pxref{Example 7}.

Be sure to call @code{wfdbquit} before exiting from any program that
uses @code{sample}, to be certain that dynamically allocated memory
used by @code{sample} is freed.

@page
@node     conversion, calibration, non-sequential, Functions
@section Conversion Functions
@cindex conversion functions

Functions in this section perform various useful conversions:  between
annotation codes and printable strings, between times in sample intervals
and printable strings, between Julian dates and printable strings, and between
ADC units and physical units.

@menu
* annstr and strann::		annotation code <-> string
* timstr and strtim::		time in sample intervals <-> HH:MM:SS string
* datstr and strdat::		Julian date <-> DD/MM/YYYY string
* aduphys and physadu::		ADC units <-> physical units
@end menu

@c @group
@node     annstr and strann, timstr and strtim, conversion, conversion
@unnumberedsubsec annstr, anndesc, and ecgstr
@findex ecgstr
@findex annstr (5.3)
@findex anndesc (5.3)
@cindex annotation code (conversion to and from string)
@cindex string (conversion to and from annotation code)
@cindex mnemonic (annotation)
@cindex conversion between annotation code and string

@example
char *annstr(int @var{code})
char *anndesc(int @var{code})
char *ecgstr(int @var{code})
@end example
@noindent
@strong{Return:}
@table @asis
@item @t{(char *)}
pointer to a printable string that describes the code, or @code{NULL}
@end table
@c @end group

@cindex modification label
@noindent
These functions translate the annotation code specified by their argument
into a string (@pxref{Annotation Codes}).  Illegal or undefined codes
are translated by @code{annstr} and @code{ecgstr} into decimal numerals
surrounded by brackets (e.g., @samp{[55]}); @code{anndesc} returns @code{NULL}
in such cases.  The strings returned by @code{annstr} are mnemonics
(usually only one character), which may be modified either by @code{setannstr}
or by the presence of @dfn{modification labels} in an input annotation file
@ifnotinfo
(@pxref{annstr and strann, , @code{setannstr}}).
@end ifnotinfo
@ifinfo
(@pxref{annstr and strann}).
@end ifinfo
The strings returned by @code{anndesc} are brief descriptive strings,
usually those given in the table of annotation codes
(@pxref{Annotation Codes}).  The strings returned by
@code{ecgstr} are usually the same as those returned by @code{annstr},
but they can be modified only by @code{setecgstr}, and not by the
presence of modification labels as for @code{annstr}.  The intent is
that @code{ecgstr} should be used rather than @code{annstr} only when
it is necessary that a fixed set of mnemonics be used, independent of
any modification labels.

Here is a little program that prints a table of the codes, mnemonic
strings, and descriptions:

@example
#include <stdio.h>
#include <wfdb/wfdb.h>
#include <wfdb/ecgcodes.h>

main()
@{
    int i;

    printf("Code\tMnemonic\tDescription\n");
    for (i = 1; i <= ACMAX; i++) @{
        printf("%3d\t%s", i, annstr(i));
        if (anndesc(i) != NULL)
            printf("\t\t%s", anndesc(i));
        printf("\n");
    @}
@}
@end example

@noindent
(See @uref{http://www.physionet.org/physiotools/wfdb/examples/exannstr.c}
for a copy of this program.)

@code{ACMAX} is defined in @file{<wfdb/ecgcodes.h>}.  The range from 1
through @code{ACMAX} includes all legal annotation codes; if you run
this program, you will find some undefined but legal annotation codes in
this range. @xref{Example 3}, for another illustration of the use of
@code{annstr}.  (@code{annstr} and @code{anndesc} were first introduced
in WFDB library version 5.3.)

@c @group
@unnumberedsubsec strann and strecg
@findex strecg
@findex strann (5.3)
@cindex annotation code (conversion to and from string)
@cindex string (conversion to and from annotation code)
@cindex mnemonic (annotation)
@cindex conversion between annotation code and string

@example
int strann(char *@var{string})
int strecg(char *@var{string})
@end example
@noindent
@strong{Return:}
@table @asis
@item @t{(int)}
annotation code
@end table
@c @end group

@noindent
These functions translate the null-terminated ASCII character strings to
which their arguments point into annotation codes.  Illegal strings are
translated into @code{NOTQRS}.  Input strings for @code{strann} and
@code{strecg} should match those returned by @code{annstr} and
@code{ecgstr} respectively.  @xref{Example 9}, for an illustration of the
use of @code{strann}. (@code{strann} was first introduced in WFDB library
version 5.3.)

@c @group
@unnumberedsubsec setannstr, setanndesc, and setecgstr
@findex setannstr (5.3)
@findex setanndesc (5.3)
@findex setecgstr
@cindex annotation code strings (setting)
@cindex changing annotation code strings
@cindex setting annotation code strings

@example
int setannstr(int @var{code}, char *@var{string})
int setanndesc(int @var{code}, char *@var{string})
int setecgstr(int @var{code}, char *@var{string})
@end example
@noindent
@strong{Return:}
@table @asis
@item @t{ 0}
Success
@item @t{-1}
Failure: illegal @code{code}
@end table
@c @end group

@noindent
These functions modify translation tables used by functions that
convert between annotation codes and strings.  @code{setannstr} modifies
the table shared by @code{annstr} and @code{strann}; @code{setanndesc}
modifies the table used by @code{anndesc}; and @code{setecgstr} modifies
the table shared by @code{ecgstr} and @code{strecg}.  They may be used
to redefine strings for defined annotation codes as well as to define
strings for undefined annotation codes.  For example,
@code{setannstr(NORMAL, "\\267")} redefines the string for normal beats
as a PostScript bullet, @samp{@bullet{}} (@code{NORMAL} is defined in
@file{<wfdb/ecgcodes.h>}).

@cindex modification label
An important difference between @code{setannstr} (or @code{setanndesc})
and @code{setecgstr} is that @code{annopen} and @code{wfdbinit} insert
modification labels in any output annotation files that are created
@emph{after} invoking @code{setannstr} or @code{setanndesc};
@code{setecgstr} does not have this side effect.  By using
@code{setannstr} before @code{annopen}, a WFDB application may create
annotation files with self-contained code tables, which can be read
properly by other WFDB applications without the need to inform them
explicitly about non-standard codes.  For this scheme to work as
intended, all custom code mnemonics and descriptions must be defined
before the output annotation files are opened.

By passing a negative value as @var{code} to @code{setannstr} or
@code{setanndesc}, the translation for
@ifnotinfo
@minus{}@var{code}
@end ifnotinfo
@ifinfo
-@var{code}
@end ifinfo
can be modified without triggering the generation of a modification label.
This feature can be useful for programs that use alternate sets of
mnemonics or descriptions for speakers of different languages.

Note that it is possible, though not desirable, to define identical
strings for two or more codes; the behavior of @code{strann} and
@code{strecg} in such cases is implementation-dependent.
(@code{setannstr} and @code{setanndesc} were first introduced in WFDB
library version 5.3.)

@node     timstr and strtim, datstr and strdat, annstr and strann, conversion

@html
<!-- link: strtim.htm -->
@end html

@ifnotinfo
@sp 2
@end ifnotinfo

The next three functions convert between ``standard time format''
strings and times in units of sample intervals.  Normally they should be
invoked after @code{isigopen}, @code{wfdbinit}, or @code{sampfreq}, any of
which will determine the duration of a sample interval and the base time
from a header file, or after defining these quantities using
@code{setsampfreq} and @code{setbasetime}.  If this is not done, or if
these time-conversion functions are used after @code{wfdbquit}, they will
perform conversions in units of seconds (i.e., the sample interval is
taken to be one second in such cases).

@c @group
@unnumberedsubsec [ms]timstr
@findex timstr
@findex mstimstr
@cindex time (conversion to and from string)
@cindex string (conversion to and from time)
@cindex elapsed time
@cindex conversion between time and string

@example
char *timstr(WFDB_Time @var{t})
char *mstimstr(WFDB_Time @var{t})
@end example
@noindent
@strong{Return:}
@table @asis
@item @t{(char *)}
pointer to a string that represents the time
@end table
@c @end group

@noindent
These functions convert times or time intervals into null-terminated
ASCII strings.  If the argument, @var{t}, is greater than zero, it is
treated as a time interval, and converted directly into @var{HH:MM:SS}
format by @code{timstr}, or to @var{HH:MM:SS.SSS} format by
@code{mstimstr}, with leading zero digits and colons suppressed.  If
@var{t} is zero or negative, it is taken to represent negated elapsed
time from the beginning of the record, and it is converted to a time of
day using the base time for the record as indicated by the @file{hea}
file or the caller
@ifnotinfo
(@pxref{setbasetime, , @code{setbasetime}});
@end ifnotinfo
@ifinfo
(@pxref{setbasetime});
@end ifinfo
in this case, if the base time is defined, the string will contain all
digits even if there are leading zeroes, it will include the date if a
base date is defined, and it will be marked as a time of day by being
bracketed (e.g., @samp{[08:45:00 23/04/1989]}).  The result of the
conversion is truncated to a multiple of a second by @code{timstr}, or
to a multiple of a millisecond by @code{mstimstr}.  Note in each case
that the returned pointer addresses static data (shared by @code{timstr}
and @code{mstimstr}), the contents of which are overwritten by
subsequent calls.  @xref{Example 3}, for an illustration of the use of
@code{mstimstr}; also @pxref{Example 5}, for an example of the use of
@code{timstr}.

@c @group
@unnumberedsubsec strtim
@findex strtim
@cindex time (conversion to and from string)
@cindex string (conversion to and from time)
@cindex elapsed time
@cindex conversion between time and string

@example
WFDB_Time strtim(char *@var{string})
@end example
@noindent
@strong{Return:}
@table @asis
@item @t{(WFDB_Time) >0}
number of sample intervals corresponding to the argument interpreted as
a time interval
@item @t{(WFDB_Time) <0}
(negated) elapsed time in sample intervals from the beginning of the record,
corresponding to the argument interpreted as a time of day
@item @t{(WFDB_Time)  0}
a legal return if the argument matches the base time;  otherwise an error
return indicating an incorrectly formatted argument
@end table
@c @end group

@noindent
@cindex standard time format (examples)
This function converts an ASCII string in @dfn{standard time format} to a time
in units of sample intervals.  Examples of standard time format:
@table @code
@item 2:14.875
2 minutes + 14.875 seconds
@item [13:6:0]
13:06 (1:06 PM)
@item [8:0:0 1]
8 AM on the day following the base date
@item [12:0:0 1/3/1992]
noon on 1 March 1992
@item 143
143 seconds (2 minutes + 23 seconds)
@item 4:02:01
4 hours + 2 minutes + 1 second
@item s12345
12345 sample intervals
@item c350.5
counter value 350.5
@item e
time of the end of the record (if defined)
@item i
time of the next sample in input signal 0
@item o
(the letter `o') time of the next sample in output signal 0
@end table

If the argument is bracketed (as in the second, third, and fourth examples), it
is taken as a time of day, and @code{strtim} uses the base time defined
by the header file or by the caller
@ifnotinfo
(@pxref{setbasetime, , @code{setbasetime}});
@end ifnotinfo
@ifinfo
(@pxref{setbasetime});
@end ifinfo
in this case, the value returned is zero or negative (and can be
converted into elapsed time from the beginning of the record by simply
negating it).  If the argument is not bracketed, it is taken as a time
interval, and converted directly into a positive number of sample
intervals.  These notations match those used by @code{timstr} and
@code{mstimstr}, which are (approximately) inverse functions of
@code{strtim}; in fact, for MIT DB and AHA DB records (and any others
with sampling frequencies below 1 KHz), @code{strtim(mstimstr(@var{t}))}
= @var{t}, for any @var{t}.  The @samp{s}-format (as in the seventh
example above) is provided to allow ``conversion'' of time intervals
already expressed in sample intervals.  The similar @samp{c}-format
converts counter values
@ifnotinfo
(@pxref{counter conversion, , @code{getcfreq}})
@end ifnotinfo
@ifinfo
(@pxref{counter conversion})
@end ifinfo
into sample intervals.  The length of the record in
sample intervals can be obtained using @code{strtim("e")}, which
evaluates to zero if this quantity is undefined.  The sample number of
the next sample to be read or written can be determined using
@code{strtim("i")} or @code{strtim("o")}.  If the argument string is
incorrectly formatted, @code{strtim} returns zero (indistinguishable
from a correct input that evokes a zero output); this may be considered
a feature.  Several of the programs in chapter 6 illustrate the use of
@code{strtim} (for example, @pxref{Example 7}).

@node datstr and strdat, aduphys and physadu, timstr and strtim, conversion

The next two functions convert between Julian dates and ASCII strings.
Julian dates as defined by astronomers begin at noon GMT; these begin at
midnight local time.

@c @group
@unnumberedsubsec datstr
@findex datstr
@cindex date (conversion to and from string)
@cindex Julian date (conversion to and from string)
@cindex string (conversion to and from Julian date)
@cindex conversion between Julian date and string

@example
char *datstr(WFDB_Date @var{date})
@end example
@noindent
@strong{Return:}
@table @asis
@item @t{(char *)}
pointer to a string that represents the date
@end table
@c @end group

@noindent
This function converts the Julian date represented by @var{date} into
an ASCII string in the form @var{DD/MM/YYYY}.

@c @group
@unnumberedsubsec strdat
@findex strdat
@cindex date (conversion to and from string)
@cindex Julian date (conversion to and from string)
@cindex string (conversion to and from Julian date)
@cindex conversion between Julian date and string

@example
WFDB_Date strdat(char *@var{string})
@end example
@noindent
@strong{Return:}
@table @asis
@item @t{(WFDB_Date)}
Julian date corresponding to the argument
@end table
@c @end group

@noindent
This function converts @var{string} into a Julian date.  The argument
should be in the format used by @code{datstr}; if @var{string} is
improperly formatted, @code{strdat} returns zero.  Note that dates such
as @samp{15/3/89} refer to the first century A.D., not the twentieth.
For example, the interval in days between the events commemorated by the
French and American national holidays is @code{strdat("14/7/1789")} --
@code{strdat("4/7/1776")}.

@sp 2

@node aduphys and physadu, , datstr and strdat, conversion

The next four functions convert between analog-to-digital converter
(ADC) units and physical units, using as a conversion factor the gain
for the specified input signal.  The first two (@code{aduphys} and
@code{physadu}) are general-purpose functions that convert absolute
levels (i.e., they account for non-zero @code{baseline} values); the
last two (@code{adumuv} and @code{muvadu}) are for use with
millivolt-dimensioned signals only, and convert potential differences
(i.e., @code{adumuv(@var{s}, 0)} = @code{muvadu(@var{s}, 0)} = 0 for all
@var{s}, irrespective of the @code{baseline} values specified in the
header file).  Normally, these functions should be invoked after
@code{isigopen} or @code{wfdbinit}, either of which will determine the
gain from the @file{hea} file.  If this is not done, or if the
header file indicates that the gain is uncalibrated, or if the
specified input signal is not currently open, a gain of @code{WFDB_DEFGAIN}
(defined in @file{<wfdb/wfdb.h>}) ADC units per millivolt, and a baseline
of zero, are assumed.  If the physical units
@ifnotinfo
(@pxref{WFDB_Siginfo structures, , Signal Information Structures})
@end ifnotinfo
@ifinfo
(@pxref{WFDB_Siginfo structures})
@end ifinfo
are not millivolts, @code{adumuv} and @code{muvadu} convert to and from
thousandths of the defined physical units.  Note that @code{adumuv} and
@code{muvadu} deal exclusively with integers, but @code{aduphys} returns
and @code{physadu} accepts double-precision floating point physical
values.

@c @group
@unnumberedsubsec aduphys
@findex aduphys (6.0)
@cindex adu (conversion to and from physical units)
@cindex physical units (conversion to and from adus)
@cindex conversion between adus and physical units

@example
double aduphys(WFDB_Signal @var{s}, WFDB_Sample @var{a})
@end example
@noindent
@strong{Return:}
@table @asis
@item @t{(double)}
physical value corresponding to a sample value of @var{a} ADC units
@end table
@c @end group

@noindent
This function converts the sample value @var{a} from ADC units to
physical units, based on the @code{gain} and @code{baseline} for input
signal @var{s}.  (@code{aduphys} was first introduced in WFDB library
version 6.0.)

@c @group
@unnumberedsubsec physadu
@findex physadu (6.0)
@cindex adu (conversion to and from physical units)
@cindex physical units (conversion to and from adus)
@cindex conversion between adus and physical units

@example
WFDB_Sample physadu(WFDB_Signal @var{s}, double @var{v})
@end example
@noindent
@strong{Return:}
@table @asis
@item @t{(WFDB_Sample)}
sample value, in ADC units, corresponding to @var{v}, in physical units
@end table
@c @end group

@noindent
This function converts the value @var{v} from physical units to ADC
units, based on the @code{gain} and @code{baseline} for input signal
@var{s}.  (@code{physadu} was first introduced in WFDB library version
6.0.)

@c @group
@unnumberedsubsec adumuv
@findex adumuv
@cindex adu (conversion to and from voltage)
@cindex voltage (conversion to and from adus)
@cindex conversion between adus and voltage

@example
int adumuv(WFDB_Signal @var{s}, WFDB_Sample @var{a})
@end example
@noindent
@strong{Return:}
@table @asis
@item @t{(int)}
number of microvolts corresponding to @var{a} ADC units
@end table
@c @end group

@noindent
This function converts the potential difference @var{a} from ADC units to
microvolts, based on the @code{gain} for input signal @var{s}.

@c @group
@unnumberedsubsec muvadu
@findex muvadu
@cindex adu (conversion to and from voltage)
@cindex voltage (conversion to and from adus)
@cindex conversion between adus and voltage

@example
WFDB_Sample muvadu(WFDB_Signal @var{s}, int @var{v})
@end example
@noindent
@strong{Return:}
@table @asis
@item @t{(int)}
number of ADC units corresponding to @var{v} microvolts
@end table
@c @end group

@noindent
This function converts the potential difference @var{v} from microvolts
to ADC units, based on the @code{gain} for input signal @var{s}.

@page
@node     calibration, miscellaneous functions, conversion, Functions
@section Calibration Functions
@cindex calibration functions

@cindex calibration list
Functions in this section are used to determine specifications for
calibration pulses and customary scales for plotting signals.  All of
them make use of the @dfn{calibration list}, which is maintained in
memory and which contains entries for various types of signals.

@menu
* calopen::			read a calibration file into list
* getcal::			retrieve calibration data from list
* putcal::			append calibration data to list
* newcal::			write calibration list to a file
* flushcal::			discard contents of calibration list
@end menu

@c @group
@node   calopen, getcal, calibration, calibration
@unnumberedsubsec calopen
@findex calopen (6.0)
@cindex calibration file (reading)
@cindex reading calibration files
@cindex @code{WFDBCAL} (environment variable)

@example
int calopen(char *@var{file})
@end example
@noindent
@strong{Return:}
@table @asis
@item @t{ 0}
Success
@item @t{-1}
Failure: insufficient memory for calibration list
@item @t{-2}
Failure: unable to open calibration file
@end table
@c @end group

@noindent
This function reads the specified calibration @var{file} (which must be
located in one of the directories specified by @code{WFDB},
@pxref{WFDB path}) into the calibration list.  If @var{file} is
@code{NULL}, the file named by @code{WFDBCAL} is read.  Normally, the
current contents of the calibration list are discarded before reading
the calibration file; if @var{file} begins with @samp{+}, however, the
@samp{+} is stripped from the file name and the contents of the file are
appended to the current calibration list.  If @var{file} is @samp{-},
@code{calopen} reads the standard input rather than a calibration file.
(This function was first introduced in WFDB library version 6.0.)

@c @group
@node   getcal, putcal, calopen, calibration
@unnumberedsubsec getcal
@findex getcal (6.0)
@cindex calibration (retrieving)
@cindex retrieving calibration data

@example
int getcal(char *@var{desc}, char *@var{units}, WFDB_Calinfo *@var{cal})
@end example
@noindent
@strong{Return:}
@table @asis
@item @t{ 0}
Success;  @code{*@var{cal}} contains the requested data
@item @t{-1}
Failure: no match found
@end table
@c @end group

@noindent
This function attempts to find calibration data for signals of type
@var{desc}, having physical units as given by @var{units}.  If
successful, it fills in the contents of the @code{WFDB_Calinfo} structure
@ifnotinfo
(@pxref{WFDB_Calinfo structures, , Calibration Information Structures})
@end ifnotinfo
@ifinfo
(@pxref{WFDB_Calinfo structures})
@end ifinfo
pointed to by @var{cal}.  The caller must allocate storage for the
@code{WFDB_Calinfo} structure, and must not modify the contents of the
strings addressed by the @code{sigtype} and @code{units} fields of the
@code{WFDB_Calinfo} structure after @code{getcal} returns.  @code{getcal}
returns data from the first entry in the calibration list that contains
a @code{sigtype} field that is either an exact match or a prefix of
@var{desc}, and a @code{units} field that is an exact match of
@var{units}; if either @var{desc} or @var{units} is @code{NULL},
however, it is ignored for the purpose of finding a match.
@code{getcal} cannot succeed unless the calibration list has been
initialized by a previous invocation of @code{calopen} or @code{putcal}.
(This function was first introduced in WFDB library version 6.0.)

@c @group
@node    putcal, newcal, getcal, calibration
@unnumberedsubsec putcal
@findex putcal (6.0)
@cindex calibration (storing)
@cindex storing calibration data

@example
int putcal(WFDB_Calinfo *@var{cal})
@end example
@noindent
@strong{Return:}
@table @asis
@item @t{ 0}
Success
@item @t{-1}
Failure: insufficient memory
@end table
@c @end group

@noindent
This function adds the @code{WFDB_Calinfo} structure pointed to by @var{cal}
to the end of the calibration list.  (This function was first introduced
in WFDB library version 6.0.)

@c @group
@node    newcal, flushcal, putcal, calibration
@unnumberedsubsec newcal
@findex newcal (6.0)
@cindex calibration list (writing)
@cindex writing calibration list

@example
int newcal(char *@var{file})
@end example
@noindent
@strong{Return:}
@table @asis
@item @t{ 0}
Success
@item @t{-1}
Failure: unable to open @var{file}
@end table
@c @end group

@noindent
This function creates a new calibration @var{file} (in the current
directory) containing the contents of the calibration list (which is not
modified).  @var{file} must satisfy the standard conditions for a WFDB
file name, i.e., it may contain letters, digits, or underscores.  (This
function was first introduced in WFDB library version 6.0.)

@c @group
@node    flushcal, , newcal, calibration
@unnumberedsubsec flushcal
@findex flushcal (6.0)
@cindex calibration list (discarding)
@cindex discarding calibration list
@cindex emptying calibration list
@cindex flushing calibration list

@example
void flushcal()
@end example
@c @end group

@noindent
This function discards the current calibration list and returns the
memory that it occupied to the heap.  Note that @code{wfdbquit} does
@emph{not} perform the function of @code{flushcal}.  (This function was
first introduced in WFDB library version 6.0.)

@page
@node miscellaneous functions, Memory Allocation Macros, calibration, Functions
@section Miscellaneous WFDB Functions

@menu
* newheader::			Creating a @file{hea} file for a new WFDB
				record.
* setheader::			Creating or changing a @file{hea} file.
* setmsheader::                 Creating a @file{hea} for a multi-segment
                                record.
* getseginfo::                  Get information about segments of a
                                multi-segment record.
* wfdbquit::			Closing WFDB files.
* iannclose and oannclose::     Closing annotation files.
* wfdbquiet and wfdbverbose::	Suppressing error messages from the WFDB library.
* wfdberror::			Retrieving error messages from the WFDB library.
* wfdbmemerr::                  Setting WFDB library behavior on memory errors.
* sampfreq::			Reading the sampling frequency of a WFDB record.
* setsampfreq::			Setting the sampling frequency.
* setbasetime::			Setting the base time.
* findsig::                     Finding a signal by name.
* counter conversion::		Functions for reading and setting counter
				conversion parameters.
* setwfdb::			Dynamically changing the database path.
* getwfdb::			Reading the database path.
* wfdbfile::			Obtaining the pathname of a WFDB file.
* wfdbflush::			Flushing buffered output annotations and
				samples.
* getinfo::			Reading info strings from a @file{hea} file.
* putinfo::			Writing info strings into a @file{hea} file.
* setibsize::			Setting the default input buffer size.
* setobsize::			Setting the default output buffer size.
* wfdbgetskew::                 Reading intersignal skew.
* wfdbsetskew::                 Recording intersignal skew.
* wfdbgetstart::                Reading the prolog size in a signal file.
* wfdbsetstart::                Recording the prolog size in a signal file.
* wfdbputprolog::               Writing a prolog to a signal file.

@end menu

@c @group
@node     newheader, setheader, miscellaneous functions, miscellaneous functions
@unnumberedsubsec newheader
@findex newheader
@cindex header files (creating)
@cindex creating header files

@example
int newheader(char *@var{record})
@end example
@noindent
@strong{Return:}
@table @asis
@item @t{ 0}
Success
@item @t{-1}
Failure: unable to create header file
@end table
@c @end group

@cindex record names (restrictions)
@noindent
This function creates a @file{hea} file (in the current directory, unless
@var{record} includes path information).
Use @code{newheader} just after you have finished writing the signal
files, but before calling @code{wfdbquit}.  If @var{record} begins with
@samp{+}, the @samp{+} is discarded and the remainder of @var{record} is
taken as the record name.  Otherwise, all of @var{record} (excluding any
path information) is taken to be
the record name.  If the record name is @samp{-}, the header file
is written to the standard output.  Record names may include letters in
lower or upper case, digits, and underscores (@samp{_}); they may not
include any other characters.  If @var{record} does not conform to these
requirements, @code{newheader} will return
@ifnotinfo
@minus{}1;
@end ifnotinfo
@ifinfo
-1;
@end ifinfo
@pxref{Example 8},
for an illustration of the use of @code{newheader} to check the validity
of a record name.  For compatibility with the widest range of operating
systems, keep record names short (6 characters or less) and avoid those
that are distinguished by case alone.  To avoid confusion with MIT DB
and AHA DB records, do not use three- or four-digit record names.

@c @group
@node     setheader, setmsheader, newheader, miscellaneous functions
@unnumberedsubsec setheader
@findex setheader (5.0)
@cindex header files (creating)
@cindex header files (modifying)
@cindex creating header files
@cindex modifying header files

@example
int setheader(char *@var{record}, WFDB_Siginfo *@var{siarray}, unsigned int @var{nsig})
@end example
@noindent
@strong{Return:}
@table @asis
@item @t{ 0}
Success
@item @t{-1}
Failure: unable to create header file
@end table
@c @end group

@noindent
This function creates or recreates a header file (in the current
directory) for the specified @var{record}, based on the contents of the
first @var{nsig} members of @var{siarray}.  The preferred way to create
a header file for a new record is using @code{newheader}, which records
signal checksum and length variables maintained by @code{putvec}.  The
intended use of @code{setheader} is for editing header files,
e.g., to change recorded signal gains from a calibration program, or to add
signal descriptions or ``info'' strings.  In the following code fragment, the
header file for record @file{old} is used to create a header file for
record @file{new}:

@example
@dots{}
int nsig, status;
WFDB_Siginfo *s;

nsig = isigopen("old", NULL, 0);
s = (WFDB_Siginfo *)malloc(nsig * sizeof(WFDB_Siginfo));
nsig = isigopen("old", s, -nsig);
if (nsig > 0) @{
    s[0].gain = 100.0;
    status = setheader("new", s, (unsigned int)nsig);
@}
@dots{}
@end example

@noindent
The header file for record @file{new} will contain the same signal
information as that for record @file{old}, except that the @code{gain}
for signal 0 will have been changed as shown.  Any ``info'' strings in
the @file{hea} file for record @file{old} must be copied explicitly;
@ifnotinfo
@pxref{getinfo, , @code{getinfo}}, and @pxref{putinfo, , @code{putinfo}}.
@end ifnotinfo
@ifinfo
@pxref{getinfo}, and @pxref{putinfo}.
@end ifinfo
(This function was first
introduced in WFDB library version 5.0.)

@c @group
@node     setmsheader, getseginfo, setheader, miscellaneous functions
@unnumberedsubsec setmsheader
@findex setmsheader (9.1)
@cindex header files (creating)
@cindex multi-segment records (creating)
@cindex creating header files

@example
int setmsheader(char *@var{record}, char *@var{snarray}[], unsigned int @var{nsegments})
@end example
@noindent
@strong{Return:}
@table @asis
@item @t{ 0}
Success
@item @t{-1}
Failure: illegal record name, or no segments specified, or header not writable
@item @t{-2}
Failure: segment name too long, or insufficient memory
@item @t{-3}
Failure: attempt to nest multi-segment records, or unreadable segment
header
@item @t{-4}
Failure: segment length unspecified, or numbers of signals or sampling
frequencies don't match between segments
@end table
@c @end group

@noindent
This function creates a header file (in the current directory) for a
multi-segment @var{record} (@pxref{Multi-Segment Records}.  @var{snarray}
contains the names of the segments, each of which must be an existing
(single-segment) record; @var{nsegments} specifies the number of segments in
@var{snarray}.  Once a header has been created by @code{setmsheader}, any WFDB
application can read the concatenated signal files of the constituent segment
simply by opening the multi-segment record (using @code{isigopen} or
@code{wfdbinit}).  Note that the signal files themselves are not modified in
any way, nor are they copied; rather, the other WFDB library functions that
read signals (@code{getvec}, @code{getframe}, @code{isigsettime}, and
@code{isgsettime}) automatically switch among the signal files of the segments
as required.  For an example of the use of @code{setmsheader}, see
@file{app/wfdbcollate.c} in the WFDB Software Package.  (This function was
first introduced in WFDB library version 9.1.)

@c @group
@node     getseginfo, wfdbquit, setmsheader, miscellaneous functions
@unnumberedsubsec getseginfo
@findex getseginfo (10.5.4)
@cindex multi-segment header (reading)
@cindex segments (in multi-segment records)

@example
int getseginfo(WFDB_Segment **@var{psegarray})
@end example
@noindent
@strong{Return:}
@table @asis
@item @t{(int)}
number of segments belonging to the current input record
@end table
@c @end group

@noindent
Invoking @code{getseginfo(@var{psegarray})}, where @var{psegarray} is
declared to be of type @code{**WFDB_Seginfo}, sets @var{*psegarray} so
that it points to an array of @code{WFDB_Seginfo} structures that
describe the segments of the currently open (multi-segment) record.
The return value indicates the number of segments (i.e., the number
of valid @code{WFDB_Seginfo} structures in @var{*psegarray}.  If
there is no current input record, or if the current input record is
not a multi-segment record, this function returns 0 and does not modify
@var{*psegarray}.

contain the segments' names, lengths, and starting sample numbers.

@c @group
@node     wfdbquit, iannclose and oannclose, getseginfo, miscellaneous functions
@unnumberedsubsec wfdbquit
@findex wfdbquit
@cindex closing WFDB files
@cindex flushing WFDB I/O
@cindex I/O (completing)
@cindex annotation (canonical order)
@cindex canonical order of annotations

@example
void wfdbquit(void)
@end example
@c @end group

@noindent
This function closes all open WFDB files and frees any memory allocated by
other WFDB library functions.  It also resets the following:

@itemize @bullet
@item
the factors used for converting between samples, seconds, and counter
values (reset to 1), the base time (reset to 0, i.e., midnight), and the
base counter value (reset to 0);
@ifnotinfo
@pxref{timstr and strtim, , @code{timstr} and @code{strtim}}
@end ifnotinfo
@ifinfo
(@pxref{timstr and strtim})
@end ifinfo
@item
the parameters used for converting between adus and physical units (reset to
@code{WFDB_DEFGAIN} adu/mV, a quantity defined in @file{<wfdb/wfdb.h>});
@ifnotinfo
@pxref{aduphys and physadu, , @code{aduphys} and @code{physadu}}
@end ifnotinfo
@ifinfo
(@pxref{aduphys and physadu})
@end ifinfo
@item
internal variables used to determine output signal specifications;
@ifnotinfo
@pxref{newheader, , @code{newheader}}.
@end ifnotinfo
@ifinfo
@pxref{newheader}.
@end ifinfo
@end itemize

If any annotations have been written out-of-order (@pxref{Annotation Order}),
this function attempts to run @file{sortann} (see the @cite{WFDB
Applications Guide}) as a subprocess to restore the annotations to canonical
order.  If this cannot be done, it prints a warning message indicating that the
annotations are not in order, and providing instructions for putting them in
order.

Programs that do not write annotations or signals need not use
@code{wfdbquit}.  Note, however, that several WFDB library functions allocate
memory that is maintained for later use by the library.  This is not generally
a problem, since these functions also free such memory if it is no longer
needed on a subsequent call;  thus these `memory leaks' do not grow over time.
Virtually all operating systems reclaim memory allocated by user-level
applications on exit, so that a small and self-limiting leak is not
a problem.  Nevertheless, there are embedded systems and other
environments in which memory is not reclaimed when a user application
exits, and in these cases it is best to invoke @code{wfdbquit()} on exit from
any WFDB application, even those that do not write output using the
library.  In an ANSI/ISO C environment, this can be ensured by
including the line

@example
	atexit(wfdbquit);
@end example

@noindent
early in the code, before the first exit.


@c @group
@node     iannclose and oannclose, wfdbquiet and wfdbverbose, wfdbquit, miscellaneous functions
@unnumberedsubsec iannclose
@findex iannclose (9.1)
@cindex closing annotation files
@cindex annotation (canonical order)
@cindex canonical order of annotations

@example
void iannclose(WFDB_Annotator @var{an})
@end example
@c @end group

@noindent
This function closes the annotation file associated with input annotator
@var{an}.  It was first introduced in WFDB library version 9.1.

@c @group
@unnumberedsubsec oannclose
@findex oannclose (9.1)

@example
void oannclose(WFDB_Annotator @var{an})
@end example
@c @end group

@noindent
This function closes the annotation file associated with output
annotator @var{an}.  It was first introduced in WFDB library version 9.1.

If any annotations have been written out-of-order (@pxref{Annotation Order}),
this function attempts to run @file{sortann} (see the @cite{WFDB
Applications Guide}) as a subprocess to restore the annotations to canonical
order.  If this cannot be done, it prints a warning message indicating that the
annotations are not in order, and providing instructions for putting them in
order.

@c @group
@node     wfdbquiet and wfdbverbose, wfdberror, iannclose and oannclose, miscellaneous functions
@unnumberedsubsec wfdbquiet
@findex wfdbquiet
@cindex error suppression
@cindex suppressing errors

@example
@code{void wfdbquiet(void)}
@end example
@c @end group

@noindent
This function suppresses error reporting on the standard error output
from the WFDB library functions.

@c @group
@unnumberedsubsec wfdbverbose
@findex wfdbverbose (4.0)

@example
@code{void wfdbverbose(void)}
@end example
@c @end group

@noindent
This function can be used to restore normal error reporting after using
@code{wfdbquiet}.  (This function was first introduced in WFDB library
version 4.0.)

@c @group
@node     wfdberror, wfdbmemerr, wfdbquiet and wfdbverbose, miscellaneous functions
@unnumberedsubsec wfdberror
@findex wfdberror (4.5)

@example
@code{char *wfdberror(void)}
@end example
@noindent
@strong{Return:}
@table @asis
@item @t{(char *)}
pointer to error string
@end table
@c @end group

@noindent
This function returns a pointer to a string containing the text of the
most recent WFDB library error message (or to a string containing the WFDB
library version number, if there have been no errors).  Function
@code{wfdberror} is primarily intended for use in applications for which the
standard error output is unavailable or inadequate, such as in X Window
System applications.  (Note that this function may be unnecessary for
MS-Windows applications, since the MS-Windows version of the WFDB library
generates a message box for error messages, unless @code{wfdbquiet} has
been used to silence them.)  This function was first introduced in WFDB
library version 4.5.  Versions earlier than 9.4 return an empty string
rather than the library version number if there have been no errors.

@c @group
@node     wfdbmemerr, sampfreq, wfdberror, miscellaneous functions
@unnumberedsubsec wfdbmemerr
@findex wfdbmemerr (10.4.6)

@example
@code{void wfdbmemerr(int exit_on_error)}
@end example
@noindent
@c @end group

@noindent
This function sets how the WFDB library behaves in the event of a memory
allocation error.  If @var{exit_on_error} is true (any non-zero value), then
such an event causes the WFDB library to emit an appropriate error message
and then terminate the running program.  (This behavior is the default.)

If @var{exit_on_error} is false (zero), a subsequent memory allocation error
will cause the WFDB library function in which it occurs to continue running
if possible (after emitting an error message as above).

This function was first introduced in WFDB library version 10.4.6.

@c @group
@node     sampfreq, setsampfreq, wfdbmemerr, miscellaneous functions
@unnumberedsubsec sampfreq
@findex sampfreq

@example
WFDB_Frequency sampfreq(char *@var{record})
@end example
@noindent
@strong{Return:}
@table @asis
@item @t{(WFDB_Frequency)>0.}
Success: the returned value is the sampling frequency in Hz
@item @t{(WFDB_Frequency)-1.}
Failure: unable to read header file
@item @t{(WFDB_Frequency)-2.}
Failure: incorrect header file format
@end table
@c @end group

@noindent
This function determines the sampling frequency (in Hz) for the record
specified by its argument.  If its argument is NULL, @code{sampfreq}
returns the currently defined sampling frequency, if any.  It also sets
the internal variables used by the time-conversion functions
@ifnotinfo
(@pxref{timstr and strtim, , @code{timstr} and @code{strtim}})
@end ifnotinfo
@ifinfo
(@pxref{timstr and strtim})
@end ifinfo
for converting between sample intervals and seconds.  @xref{Example 3},
for an illustration of the use of @code{sampfreq}.  Note that the value
returned by @code{sampfreq} for a multifrequency record depends on the
current @code{getvec} mode
@ifnotinfo
(@pxref{setgvmode, , @code{setgvmode}}).
@end ifnotinfo
@ifinfo
(@pxref{setgvmode}).
@end ifinfo

@c @group
@node     setsampfreq, setbasetime, sampfreq, miscellaneous functions
@unnumberedsubsec setsampfreq
@findex setsampfreq
@cindex sampling frequency (changing)
@cindex changing sampling frequency

@example
int setsampfreq(WFDB_Frequency @var{freq})
@end example
@noindent
@strong{Return:}
@table @asis
@item @t{ 0}
Success
@item @t{-1}
Failure:  illegal sampling frequency specified (@var{freq} must not be
negative)
@end table
@c @end group

@noindent
This function sets the sampling frequency used by the time-conversion
functions
@ifnotinfo
(@pxref{timstr and strtim, , @code{timstr} and @code{strtim}}).
@end ifnotinfo
@ifinfo
(@pxref{timstr and strtim}).
@end ifinfo
Use @code{setsampfreq} before creating a new @file{hea} file
@ifnotinfo
(@pxref{newheader, , @code{newheader}}).
@end ifnotinfo
@ifinfo
(@pxref{newheader}).
@end ifinfo
@xref{Example 8}, for an illustration of the use of @code{setsampfreq}.

@c @group
@node     setbasetime, findsig, setsampfreq, miscellaneous functions
@unnumberedsubsec setbasetime
@findex setbasetime
@cindex current time
@cindex time of day (setting)
@cindex base time (setting)

@example
int setbasetime(char *@var{string})
@end example
@noindent
@strong{Return:}
@table @asis
@item @t{ 0}
Success
@item @t{-1}
Failure: incorrect string format
@end table
@c @end group

@noindent
This function sets the base time used by the time-conversion functions
@code{timstr} and @code{strtim}.  Its argument is a null-terminated
ASCII string in @var{HH:MM:SS} format.  An optional base date in
@var{dd/mm/yyyy} format can follow the time in @var{string}; if present,
the date should be separated from the time by a space or tab character.  If
@var{string} is empty or @code{NULL}, the current date and time are read
from the system clock.  Use @code{setbasetime} after defining the sampling
frequency and before creating a header file.
@ifnotinfo
(@pxref{newheader, , @code{newheader}}).
@end ifnotinfo
@ifinfo
(@pxref{newheader}).
@end ifinfo
@xref{Example 8}, for an illustration of the use of @code{setbasetime}.

There is no @code{getbasetime} function;  use @code{mstimstr(0)} at any
time after opening a record to convert the base date and time into a string.

@sp 2

@c @group
@node     findsig, counter conversion, setbasetime, miscellaneous functions
@findex findsig (10.4.12)
@cindex signal name
@cindex signal number
@cindex finding signal by name
@cindex signal lookup

@example
int findsig(char *@var{string})
@end example
@noindent
@strong{Return:}
@table @asis
@item @t{ WFDB_Signal}
Success
@item @t{-1}
Failure: signal not found
@end table
@c @end group

@noindent
This function converts its argument to an input signal number.
If @var{string} is numeric and can be interpreted as a valid
input signal number, it is taken as such;  otherwise, it is
assumed to be a signal name, and if it is an exact match to the
@code{desc} field of a currently open input signal's @code{siginfo}
structure, @code{findsig} returns the corresponding signal number.
If two or more signals have identical matching names, @code{findsig}
returns the lowest matching signal number.


@sp 2

@c @group
@node     counter conversion, setwfdb, findsig, miscellaneous functions
@cindex base counter value
@cindex counter (base)
@cindex counter frequency
@cindex counter value
@cindex frequency (counter)
@cindex tape counter

Database records are sometimes obtained from analog tapes for which a
tape counter is available.  Since many analog tape recorders lack
elapsed time indicators, it is often useful to identify events in the
analog tape using counter values.  A similar situation may arise if a
chart recording or other hard copy with numbered pages is to be compared
with a database record.  To simplify cross-referencing between the
analog tape or chart and the digital database record, the WFDB library
supports conversion of counter values (or page numbers) to time.  For this
to be possible, the counter must be linear (i.e., it must change at the
same rate throughout the tape; this is not true of those that count the
number of revolutions of the supply or take-up reel), and the base
counter value (the counter value or page number corresponding to sample
0) and the counter frequency (the difference between counter values
separated by a one-second interval, or the reciprocal of the number of
seconds per page) must be defined.  The following four functions,
first introduced in WFDB library version 5.2, are used to obtain or set
the values of these parameters.

@unnumberedsubsec getcfreq
@findex getcfreq (5.2)
@example
WFDB_Frequency getcfreq(void)
@end example
@noindent
@strong{Return:}
@table @asis
@item @t{(WFDB_Frequency)}
the counter frequency in Hz
@end table
@c @end group

@noindent
This function returns the currently-defined counter frequency.  The
counter frequency is set by the functions that read header
files, or by @code{setcfreq}.  If the counter frequency has not been
defined explicitly, @code{getcfreq} returns the sampling frequency.

@c @group
@unnumberedsubsec setcfreq
@findex setcfreq (5.2)
@example
void setcfreq(WFDB_Frequency @var{freq})
@end example
@c @end group

@noindent
This function sets the counter frequency.  Use @code{setcfreq}
before creating a @file{hea} file
@ifnotinfo
(@pxref{newheader, , @code{newheader}}).
@end ifnotinfo
@ifinfo
(@pxref{newheader}).
@end ifinfo
The effect of @code{setcfreq} is nullified by later invoking any of the
functions that read header files.  If @var{freq} is zero or
negative, the counter frequency is treated as equivalent to the sampling
frequency.

@c @group
@unnumberedsubsec getbasecount
@findex getbasecount (5.2)
@example
double getbasecount(void)
@end example
@noindent
@strong{Return:}
@table @asis
@item @t{(double)}
base counter value
@end table
@c @end group

@noindent
This function returns the base counter value, which is set by the
functions that read header files, or by @code{setbasecount}.
If the base counter value has not been set explicitly,
@code{getbasecount} returns zero.

@c @group
@unnumberedsubsec setbasecount
@findex setbasecount (5.2)
@example
void setbasecount(double @var{count})
@end example
@c @end group

@noindent
This function sets the base counter value.  Use @code{setbasecount}
before creating a header file
@ifnotinfo
(@pxref{newheader, , @code{newheader}}).
@end ifnotinfo
@ifinfo
(@pxref{newheader}).
@end ifinfo
The effect of @code{setbasecount} is nullified by later invoking any of the
functions that read @file{hea} files.

@c @group
@node     setwfdb, getwfdb, counter conversion, miscellaneous functions
@unnumberedsubsec setwfdb
@findex setwfdb
@cindex database path (changing)
@cindex changing the WFDB path
@cindex WFDB path

@example
void setwfdb(char *@var{string})
@end example
@c @end group

@cindex NETFILES
@noindent
This function may be used to set or change the database path
(@pxref{WFDB path}) within a running program.  The argument points to a
null-terminated string that specifies the desired database path (but see
the next paragraph for an exception).  The string contains a list of
locations where input files may be found.  These locations may be
absolute directory names (such as @samp{/usr/local/database} under Unix,
or @samp{d:/database} under MS-DOS), relative directory names (e.g.,
@code{../mydata}), or URL prefixes (e.g.,
@samp{http://www.physio@-net.org/physio@-bank/database}).  If NETFILES
support is unavailable, any URL prefixes in the string are ignored.  The
special form @samp{.} refers to the current directory.  Entries in the
list may be separated by whitespace or by semicolons; under Unix, colons
may also be used as separators.  An empty component, indicated by an
initial or terminal separator, or by two consecutive separators, will be
understood to specify the current directory (which may also be indicated
by a component consisting of a single @samp{.}).  If the string is empty or
@code{NULL}, the database path is limited to the current directory.

@cindex WFDB path
@cindex indirect WFDB path
@cindex file containing WFDB path
@cindex database path file (indirect)

If @var{string} begins with @samp{@@}, the remaining characters of
@var{string} are taken as the name of a file from which the WFDB path is
to be read.  This file may contain either the WFDB path, as described in
the previous paragraph, or another indirect WFDB path specification.
Indirect WFDB path specifications may be nested no more than ten levels
deep (an arbitrary limit imposed to avoid infinite recursion).
Evaluation of indirect WFDB paths is deferred until @code{getwfdb} is
invoked, either explicitly or by the WFDB library while attempting to open
an input file (e.g., using @code{annopen} or @code{isigopen}).  (The
features described in this paragraph were first introduced in WFDB library
version 8.0.)
@ifinfo
@xref{getwfdb}
@end ifinfo
@ifnotinfo
@xref{getwfdb, , @code{getwfdb}}
@end ifnotinfo
for an example of the use of @code{setwfdb}.

@c @group
@node     getwfdb, wfdbfile, setwfdb, miscellaneous functions
@unnumberedsubsec getwfdb
@findex getwfdb
@cindex database path (reading)
@cindex reading the WFDB path
@cindex WFDB path

@example
char *getwfdb(void)
@end example
@noindent
@strong{Return:}
@table @asis
@item @t{(char *)}
pointer to the database path string
@end table
@c @end group

@noindent
This function returns the current database path.  For example, this code
fragment

@example
@dots{}
char *oldp, *newp;

oldp = getwfdb();
if (newp = malloc(strlen("/usr/mydb;") + strlen(oldp) + 1)) @{
    sprintf(newp, "/usr/mydb;%s", oldp);
    setwfdb(newp);
@}
@dots{}
@end example

@noindent
adds the directory @samp{/usr/mydb} to the beginning of the database path.
(The standard @samp{/} directory separator can be used, even under MS-DOS; if
you elect to use the alternate @samp{\}, remember to quote it within a C string
as @samp{\\}.)

@c @group
@node     wfdbfile, wfdbflush, getwfdb, miscellaneous functions
@unnumberedsubsec wfdbfile
@findex wfdbfile (4.3)
@cindex filenames of WFDB files (obtaining)
@cindex pathnames of WFDB files (obtaining)
@cindex NETFILES
@cindex WFDB path

@example
char *wfdbfile(char *@var{type}, char *@var{record})
@end example
@noindent
@strong{Return:}
@table @asis
@item @t{(char *)}
pointer to a filename, or @code{NULL}
@end table
@c @end group

@noindent
This function attempts to locate an existing WFDB file by searching the
database path (@pxref{WFDB path}).  Normally, the file is specified by its
@var{type} (e.g., @file{hea}, or an annotator name such as @file{atr})
and by the @var{record} to which it belongs.  A file that does not include a
record name as part of its name can be found by @code{wfdbfile} if the name is
passed in the @var{type} variable and @var{record} is @code{NULL}.  The string
returned by @code{wfdbfile} includes the appropriate component of the database
path; since the database path may include empty or non-absolute components, the
string is not necessarily an absolute pathname.  If the WFDB library has been
compiled with NETFILES support, and the WFDB path includes one or more
URL prefixes, the string returned may be a URL rather than a pathname.  If the
file cannot be found, @code{wfdbfile} returns @code{NULL}.  (This function was
first introduced in WFDB library version 4.3.)

@c @group
@node     wfdbflush, getinfo, wfdbfile, miscellaneous functions
@unnumberedsubsec wfdbflush
@findex wfdbflush
@cindex flushing output annotations and samples

@example
void wfdbflush(void)
@end example
@c @end group

@noindent
This function brings database output files up-to-date by forcing any
output annotations or samples that are buffered to be written to the
output files.

@c @group
@node     getinfo, putinfo, wfdbflush, miscellaneous functions
@unnumberedsubsec getinfo
@findex getinfo (4.0)
@cindex header info (reading) 
@cindex info (in header files)

@example
char *getinfo(char *@var{record})
@end example
@noindent
@strong{Return:}
@table @asis
@item @t{(char *)}
pointer to an ``info'' string, or @code{NULL}
@end table
@c @end group

@noindent
This function reads an ``info'' string from the @file{hea} file for
the specified @var{record}.  If @var{record} is @code{NULL}, then
@code{getinfo} reads the next available info string for the currently
open record.  Info strings are null-terminated and do not
contain newline characters.  The @file{hea} files of some records may
contain no info strings;  others may contain more than one info string.
For example, the following code fragment may be used to read and print
all of the info for record @file{100s}:

@example
@dots{}
char *info;

if (info = getinfo("100s"))
    do @{
        puts(info);
    @} while (info = getinfo(NULL));
@dots{}
@end example

@noindent
If the @file{hea} file was opened by another WFDB function, such as
@code{isigopen}, @code{annopen}, @code{sampfreq}, or @code{wfdbinit},
the @var{record} argument can be @code{NULL} even for the first @code{getinfo},
like this:

@example
@dots{}
char *info;
WFDB_Frequency sps;

if (sps = sampfreq("100s")
    while (info = getinfo(NULL))
        puts(info);
@dots{}
@end example

(This function was first introduced in WFDB library version 4.0.)

@c @group
@node     putinfo, setibsize, getinfo, miscellaneous functions
@unnumberedsubsec putinfo
@findex putinfo (4.0)
@cindex header info (writing)
@cindex info (in header files)

@example
int putinfo(char *@var{s})
@end example
@noindent
@strong{Return:}
@table @asis
@item @t{ 0}
Success
@item @t{-1}
Failure: header not initialized
@end table
@c @end group

@noindent
This function writes @var{s} as an ``info'' string into the
@file{hea} file that was created by the most recent invocation of
@code{newheader}.  The string argument, @var{s}, must be null-terminated
and should not contain newline characters.  No more than 254 characters
may be written in a single invocation of @code{putinfo}.  Two or more
info strings may be written to the same header by successive
invocations of @code{putinfo}.  Note that @code{newheader} or
@code{setheader} must be used before @code{putinfo}.  (This function was
first introduced in WFDB library version 4.0.)

@c @group
@node     setibsize, setobsize, putinfo, miscellaneous functions
@unnumberedsubsec setibsize
@findex setibsize (5.0)
@cindex buffer size (setting)
@cindex input buffer size
@cindex @code{getvec} buffer size

@example
int setibsize(int @var{size})
@end example
@noindent
@strong{Return:}
@table @asis
@item @t{>0}
Success: the returned value is the new input buffer size in bytes
@item @t{-1}
Failure: buffer size could not be changed
@item @t{-2}
Failure: illegal value for @var{size}
@end table
@c @end group

@noindent
This function can be used to change the default size of the input
buffers allocated by @code{getvec}.  It cannot be used while input
signals are open (i.e., after invoking @code{isigopen} or @code{wfdbinit}
and before invoking @code{wfdbquit}).  If @var{size} is positive, the
default input buffers will be @var{size} bytes;  if @var{size} is zero,
the system default buffer size (@code{BUFSIZ}) is used.  Note that
the default buffer size has no effect on reading signals for which
an explicit buffer size is given in the header file, i.e.,
those for which the @code{bsize} field of the @code{WFDB_Siginfo} structure
@ifnotinfo
(@pxref{WFDB_Siginfo structures, , Signal Information Structures})
@end ifnotinfo
@ifinfo
(@pxref{WFDB_Siginfo structures})
@end ifinfo
is non-zero.
(This function was first introduced in WFDB library version 5.0.)

@c @group
@node     setobsize, wfdbgetskew, setibsize, miscellaneous functions
@unnumberedsubsec setobsize
@findex setobsize (5.0)
@cindex buffer size (setting)
@cindex output buffer size
@cindex @code{putvec} buffer size

@example
int setobsize(int @var{size})
@end example
@noindent
@strong{Return:}
@table @asis
@item @t{>0}
Success: the returned value is the new output buffer size in bytes
@item @t{-1}
Failure: buffer size could not be changed
@item @t{-2}
Failure: illegal value for @var{size}
@end table
@c @end group

@noindent
This function can be used to change the default size of the output
buffers allocated by @code{putvec}.  It cannot be used while output
signals are open (i.e., after invoking @code{osigopen} or
@code{osigfopen} and before invoking @code{wfdbquit}).  If @var{size} is
positive, the default output buffers will be @var{size} bytes; if
@var{size} is zero, the system default buffer size (@code{BUFSIZ}) is
used.  Note that the default buffer size has no effect on writing
signals for which an explicit buffer size is given in the @file{hea}
file read by @code{osigopen}, or in the @code{bsize} field of the
@code{WFDB_Siginfo} structure
@ifnotinfo
(@pxref{WFDB_Siginfo structures, , Signal Information Structures})
@end ifnotinfo
@ifinfo
(@pxref{WFDB_Siginfo structures})
@end ifinfo
passed to @code{osigfopen}.
(This function was first introduced in WFDB library version 5.0.)

@c @group
@node     wfdbgetskew, wfdbsetskew, setobsize, miscellaneous functions
@unnumberedsubsec wfdbgetskew
@findex wfdbgetskew (9.4)
@cindex intersignal skew
@cindex skew

@example
int wfdbgetskew(WFDB_Signal @var{s})
@end example
@noindent
@strong{Return:}
@table @asis
@item @t{(int)}
the skew (in frames) for input signal @var{s}
@end table
@c @end group

@noindent
This function returns the @dfn{skew} (as recorded in the @file{hea}
file, but in frame intervals rather than in sample intervals) of the
specified input signal, or 0 if @var{s} is not a valid input signal
number.  Since sample vectors returned by @code{getvec} or
@code{getframe} are already corrected for skew, @code{wfdbgetskew} is
useful primarily for programs that need to rewrite existing
@file{hea} files, where it is necessary to preserve the previously
recorded skews.  The following code fragment demonstrates how this can
be done:

@example
char *record;
int nsig;
WFDB_Signal s;
static WFDB_Siginfo *si;

@dots{}

if ((nsig = isigopen(record, NULL, 0)) < 1)
    exit(1);
si = (WFDB_Siginfo *)malloc(nsig * sizeof(WFDB_Siginfo));
if (si == NULL || isigopen(record, siarray, nsig)!= nsig)
    exit(1);
for (s = 0; s < nsig; s++) @{
    wfdbsetskew(s, wfdbgetskew(s));
    wfdbsetstart(s, wfdbgetstart(s));
@}
setheader(record, siarray, (unsigned)nsig);
@end example

@noindent
Note that this function does not @emph{determine} the skew between
signals;  the problem of doing so is not possible to solve in the
general case.  @code{wfdbgetskew} merely reports what has previously been
determined by other means and recorded in the header file for the input
record.  (This function was first introduced in WFDB library version 9.4.)

@c @group
@node     wfdbsetskew, wfdbgetstart, wfdbgetskew, miscellaneous functions
@unnumberedsubsec wfdbsetskew
@findex wfdbsetskew (9.4)
@cindex intersignal skew
@cindex skew

@example
void wfdbsetskew(WFDB_Signal @var{s}, int @var{skew})
@end example
@c @end group

@noindent
This function sets the specified @var{skew} (in frames) to be recorded
by @code{newheader} or @code{setheader} for signal @var{s}.  For an
example of the use of @code{wfdbsetskew}, @pxref{wfdbgetskew}.  Note that
@code{wfdbsetskew} has no effect on the skew correction performed by
@code{getframe} (or @code{getvec}), which is determined solely by the
skews that were recorded in the header file at the time the input
signals were opened.  (This function was first introduced in WFDB library
version 9.4.)

@c @group
@node     wfdbgetstart, wfdbsetstart, wfdbsetskew, miscellaneous functions
@unnumberedsubsec wfdbgetstart
@findex wfdbgetstart (9.4)
@cindex byte offset
@cindex cruft (in signal files)
@cindex prolog (in signal files)
@cindex start of sample data

@example
long wfdbgetstart(WFDB_Signal @var{s})
@end example
@noindent
@strong{Return:}
@table @asis
@item @t{(long)}
the length of the prolog of the file that contains input signal @var{s}
@end table
@c @end group

@noindent
This function returns the number of bytes in the @dfn{prolog} of the
signal file that contains the specified input signal, as recorded in the
header file.  Note that @code{wfdbgetstart} does not
@emph{determine} the length of the prolog by inspection of the signal
file; it merely reports what has been determined by other means and
recorded in the @file{hea} file.  Since the prolog is not readable
using the WFDB library, and since functions such as @code{isigopen} and
@code{isigsettime} take the prolog into account when calculating byte
offsets for @code{getframe} and @code{getvec}, @code{wfdbgetstart} is
useful primarily for programs that need to rewrite existing
@file{hea} files, where it is necessary to preserve the previously
recorded byte offsets.  For an example of how this can be done,
@pxref{wfdbgetskew}.  (This function was first introduced in WFDB library
version 9.4.)

@c @group
@node     wfdbsetstart, wfdbputprolog, wfdbgetstart, miscellaneous functions
@unnumberedsubsec wfdbsetstart
@findex wfdbsetstart (9.4)
@cindex byte offset
@cindex cruft (in signal files)
@cindex prolog (in signal files)
@cindex start of sample data

@example
void wfdbsetstart(WFDB_Signal @var{s}, long @var{bytes})
@end example
@c @end group

@noindent
This function sets the specified prolog length (@var{bytes}) to be
recorded by @code{newheader} or @code{setheader} for signal @var{s}.
For an example of the use of @code{wfdbsetstart}, @pxref{wfdbgetskew}.  Note
that @code{wfdbsetstart} has no effect on the calculations of byte offsets
within signal files as performed by @code{isigsettime}, which are
determined solely by the contents of the @file{hea} file at the time
the signals were opened.  (This function was first introduced in WFDB library
version 9.4.)

@c @group
@node     wfdbputprolog, , wfdbgetstart, miscellaneous functions
@unnumberedsubsec wfdbputprolog
@findex wfdbputprolog (10.4.15)
@cindex byte offset
@cindex cruft (in signal files)
@cindex prolog (in signal files)
@cindex start of sample data

@example
void wfdbputprolog(char *@var{prolog}, long @var{bytes}, WFDB_Signal @var{s})
@end example
@c @end group

@noindent
This function writes the specified @var{prolog} of length @var{bytes} to the
signal file for the specified output signal @var{s}, and invokes
@code{wfdbsetstart(}@var{s, bytes}@code{)}.  (This function was first
introduced in WFDB library version 10.4.15.)

@page
@node Memory Allocation Macros, , miscellaneous functions, Functions
@section Memory Allocation Macros

@menu
* MEMERR::                      Handling a memory allocation error.
* SFREE::                       Safely release allocated memory.
* SUALLOC::                     Safely allocate memory using a new pointer.
* SALLOC::                      Safely allocate memory, reusing a pointer.
* SREALLOC::                    Safely expand a chunk of allocated memory.
* SSTRCPY::                     Safely copy a string into a new buffer.

@end menu

These macros use the standard ANSI/ISO C functions @code{calloc()},
@code{realloc()}, @code{free()}, @code{strlen()}, and @code{strcpy()} to handle
dynamic memory allocation tasks for the WFDB library.  They can also be used by
applications that include @file{wfdb/wfdb.h}, where they are defined.

These macros provide safe handling of insufficient memory and double free
errors (either condition results in a descriptive error message, which by
default is followed by an @code{exit(1)} to end the process with a signal
to the parent shell or other process).


@c @group
@node     MEMERR, SFREE, memory allocation, memory allocation
@unnumberedsubsec MEMERR
@findex MEMERR (10.5.0)
@cindex insufficient memory
@cindex memory errors
@cindex allocation errors

@example
MEMERR(@var{object_name}, size_t @var{n_elements}, size_t @var{element_size})
@end example
@c @end group

@noindent
This macro uses @code{wfdb_error} to send a short error message of the form
@code{WFDB: can't allocate (}@var{n_elements}*@var{element_size} @code{) bytes
for }@var{object_name}@code{)}.  Unless @code{wfdbmemerr(1)} has been invoked
previously, the process that invoked @code{MEMERR} exits immediately.

@c @group
@node     SFREE, SUALLOC, MEMERR, memory allocation
@unnumberedsubsec SFREE
@findex SFREE (10.5.0)
@cindex release allocated memory
@cindex freeing allocated memory
@cindex memory deallocation

@example
SFREE(@var{object} *@var{pointer})
@end example
@c @end group

@noindent
This macro releases memory previously allocated to the @var{object} addressed
by the specified @var{pointer}, somewhat more safely than by invoking the
standard @code{free()} function.  On completion, @var{pointer} is set to NULL.

@code{SFREE} does nothing if @var{pointer} is initially NULL (unlike
@code{free()}, which may cause the process to crash).  If @code{SFREE} receives
a non-NULL pointer, it passes that pointer to @code{free()}, which may cause a
crash if the pointer does not point to a block of memory that was previously
allocated using one of the macros below, or directly using @code{malloc()},
@code{calloc()}, or @code{realloc()}.

@c @group
@node     SUALLOC, SALLOC, SFREE, memory allocation
@unnumberedsubsec SUALLOC
@findex SUALLOC (10.5.0)
@cindex allocating memory
@cindex memory allocation

@example
SUALLOC(@var{object_name}, size_t @var{n_elements}, size_t @var{element_size})
@end example
@c @end group

@noindent
This macro allocates memory sufficient for @var{n_elements} items of
@var{element_size} bytes each, and sets the pointer given by @var{object_name}
to point to the allocated memory.  If there is not enough available memory,
@code{SUALLOC} invokes @code{MEMERR} (above).

The newly allocated memory block is filled with zeroes.

@code{SUALLOC} does not check to see if @var{object_name} already points to
allocated memory, which will lead to memory leaks if so.

@c @group
@node     SALLOC, SREALLOC, SUALLOC, memory allocation
@unnumberedsubsec SALLOC
@findex SALLOC (10.5.0)
@cindex allocating memory
@cindex memory allocation

@example
SALLOC(@var{object_name}, size_t @var{n_elements}, size_t @var{element_size})
@end example
@c @end group

@noindent
This macro allocates memory sufficient for @var{n_elements} items of
@var{element_size} bytes each, and sets the pointer given by @var{object_name}
to point to the allocated memory.  If there is not enough available memory,
@code{SALLOC} invokes @code{MEMERR} (above).

The newly allocated memory block is filled with zeroes.

Unless @var{object_name} is initially NULL, @code{SALLOC} frees it using
@code{SFREE} before allocating the requested memory.

@c @group
@node     SREALLOC, SSTRCPY, SALLOC, memory allocation
@unnumberedsubsec SREALLOC
@findex SREALLOC (10.5.0)
@cindex allocating memory
@cindex expanding allocated memory
@cindex memory allocation
@cindex reallocating memory

@example
SREALLOC(@var{object_name}, size_t @var{n_elements}, size_t @var{element_size})
@end example
@c @end group

@noindent
This macro allocates memory sufficient for @var{n_elements} items of
@var{element_size} bytes each, and sets the pointer given by @var{object_name}
to point to the allocated memory.  If there is not enough available memory,
@code{SREALLOC} invokes @code{MEMERR} (above).

Use @code{SREALLOC} to expand a previously allocated block of memory,
preserving its contents.  @code{REALLOC} usually allocates a new block of the
desired size, moving the contents of the previously allocated block into
the beginning of the new block and then freeing the original block.  Pointers
to locations in the original block will no longer be valid in this case.

The portion of the newly allocated block that extends beyond the previous
contents is uninitialized.

If @var{object_name} is initially NULL, @code{SUALLOC}, @code{SALLOC}, and
@code{REALLOC} are functionally equivalent, except that @code{REALLOC} does
not fill the allocated block with zeroes.

@c @group
@node     SSTRCPY, , SREALLOC, memory allocation
@unnumberedsubsec SSTRCPY
@findex SSTRCPY (10.5.0)
@cindex allocating strings
@cindex string allocation

@example
SSTRCPY(char *destination, char *source)
@end example
@c @end group

@noindent
This macro copies the @var{source} string (including a trailing null character)
into newly-allocated memory, and it sets @var{destination} to point to the copy.
If @var{destination} is not NULL on entry, @code{SSTRCPY} uses @code{SFREE}
to release the previously allocated memory.

@node     Data Types, Annotation Codes, Functions, Top
@chapter Data Types

@noindent
@emph{Simple data types} used by the WFDB library are defined in
@file{<wfdb/wfdb.h>}.  These include:

@table @code
@item WFDB_Sample
@cindex @code{WFDB_Sample} (defined)
a signed integer type (at least 16 bits) used to represent sample
values, in units of adus.

@item WFDB_Time
@cindex @code{WFDB_Time} (defined)
a signed integer type (at least 32 bits) used to represent times and
time intervals, in units of sample intervals.  Only the magnitude is
significant;  the sign of a @code{WFDB_Time} variable indicates how it
is to be printed by @code{timstr} or @code{mstimstr}.

@item WFDB_Date
@cindex @code{WFDB_Date} (defined)
a signed integer type (at least 32 bits) used to represent Julian dates,
in units of days.

@item WFDB_Frequency
@cindex @code{WFDB_Frequency} (defined)
a floating point type used to represent sampling and counter
frequencies, in units of Hz.

@item WFDB_Gain
@cindex @code{WFDB_Gain} (defined)
a floating point type used to represent signal gains, in units of adus
per physical unit.

@item WFDB_Group
@cindex @code{WFDB_Group} (defined)
an unsigned integer type used to represent signal group numbers.

@item WFDB_Signal
@cindex @code{WFDB_Signal} (defined)
an unsigned integer type used to represent signal numbers.

@item WFDB_Annotator
@cindex @code{WFDB_Annotator} (defined)
an unsigned integer type used to represent annotator numbers.

@end table

@noindent
@emph{Composite data types} used by the WFDB library are also defined in
@file{<wfdb/wfdb.h>}.  These types, described in detail in the following
sections, include:

@table @code
@item WFDB_Siginfo
an object containing the name and global attributes of a given signal.

@item WFDB_Calinfo
an object containing calibration specifications for signals of a given
type.

@item WFDB_Anninfo
an object containing the name and attributes of a given annotator.

@item WFDB_Annotation
an object describing one or more attributes of one or more signals at a
given time.

@end table

@menu
* WFDB_Siginfo structures::		Signal names and attributes.
* WFDB_Calinfo structures::		Signal calibration specifications.
* WFDB_Anninfo structures::		Annotator names and file types.
* WFDB_Annotation structures::		Annotation contents.
* WFDB_Seginfo structures::              Segment information.
@end menu

@node WFDB_Siginfo structures, WFDB_Calinfo structures, Data Types, Data Types
@section Signal Information Structures
@cindex @code{WFDB_Siginfo} structure (defined)
@cindex signal information structure
@cindex information structure (signal)
@cindex structure (signal information)
@cindex attributes of signals (global)

The @var{siarray} argument for @code{isigopen}, @code{osigopen},
@code{wfdbinit}, and @code{osigfopen} is a pointer to an array of
objects of type @code{WFDB_Siginfo}.  The first three of these functions
fill in the @code{WFDB_Siginfo} objects to which @var{siarray} points, but
the caller must supply initialized @code{WFDB_Siginfo} objects to
@code{osigfopen}.  Each object specifies the attributes of a
signal:

@table @code
@item char *fname
@cindex signal file name
a pointer to a null-terminated string that names the file in which
samples of the associated signal are stored.  Input signal files are
found by prefixing @code{fname} with each of the components of the
database path in turn (@pxref{WFDB path}).  @code{fname} may include
relative or absolute path specifications if necessary; the use of an
absolute pathname, combined with an initial null component in @code{WFDB},
reduces the time needed to find the signal file to a minimum.  If
@code{fname} is @samp{-}, it refers to the standard input or
output.

@item char *desc
@cindex signal file description
a pointer to a null-terminated string without embedded newlines (e.g.,
@samp{ECG lead V1} or @samp{trans-thoracic impedance}).  The length of
the @code{desc} string is restricted to a maximum of @code{WFDB_MAXDSL}
(defined in @file{<wfdb/wfdb.h>}) characters, not including the null.

@item char *units
@cindex physical units
@cindex units (physical)
@cindex scale (amplitude)
a pointer to a null-terminated string without embedded whitespace.  The
string specifies the physical units of the signal; if NULL, the units
are assumed to be millivolts.  The length of the @code{units} string is
restricted to a maximum of @code{WFDB_MAXUSL} (defined in
@file{<wfdb/wfdb.h>}) characters (not including the null).

@item WFDB_Gain gain
@cindex gain
@cindex adu
@cindex units (ADC)
@cindex scale (amplitude)
the number of analog-to-digital converter units (adus) per physical unit
(see previous item) relative to the original analog signal; for an ECG,
this is roughly equal to the amplitude of a normal QRS complex.  If
@code{gain} is zero, no amplitude calibration is available; in this
case, a @code{gain} of @code{WFDB_DEFGAIN} (defined in
@file{<wfdb/wfdb.h>}) may be assumed.

@item WFDB_Sample initval
@cindex initial value of signal
the initial value of the associated signal (i.e., the value of sample
number 0).

@item WFDB_Group group
@cindex signal group
@cindex multiplexed signal file
the signal group number.  All signals in a given group are stored in the
same file.  If there are two or more signals in a group, the file is
called a @dfn{multiplexed signal file}.  Group numbers begin at 0;
arrays of @code{WFDB_Siginfo} structures are always kept ordered with respect
to the group number, so that signals belonging to the same group are
described by consecutive entries in @var{siarray}.

@item int fmt
@cindex signal file format
@cindex format (signal file)
@cindex difference format
@cindex first difference
the signal storage format.  The most commonly-used formats are format 8
(8-bit first differences), format 16 (16-bit amplitudes), and format 212
(pairs of 12-bit amplitudes bit-packed into byte triplets).  See
@file{<wfdb/wfdb.h>} for a complete list of supported formats.  All signals
belonging to the same group must be stored in the same format.

@item int spf
@cindex multi-frequency records
@cindex frame rate
@cindex oversampled signals
@cindex signals (oversampled)
@cindex samples per frame
the number of samples per frame.  This is 1, for all except oversampled
signals in multi-frequency records, for which @code{spf} may be any
positive integer.  Note that non-integer values are not permitted (thus
the frame rate must be chosen such that all sampling frequencies used in
the record are integer multiples of the frame rate).

@item int bsize
@cindex block size
the block size, in bytes.  For signal files that reside on Unix character
device special files (or their equivalents), the @code{bsize} field
indicates how many bytes must be read or written at a time
(@pxref{Special Files}).  For ordinary disk files, @code{bsize} is zero.
All signals belonging to a given group have the same
@code{bsize}.

@item int adcres
@cindex resolution
@cindex ADC resolution
the ADC resolution in bits.  Typical ADCs have resolutions between 8 and 16
bits inclusive.

@item int adczero
@cindex ADC zero
the ADC output given an input that falls exactly at the center of the
ADC range (normally 0 VDC).  Bipolar ADCs produce two's complement
output; for these, @code{adczero} is usually zero.  For the MIT DB,
however, an offset binary ADC was used, and @code{adczero} was 1024.

@item int baseline
@cindex baseline
@cindex isoelectric level
@cindex physical zero level
the value of ADC output that would map to 0 physical units input.
The value of @code{adczero} is not synonymous with that of
@code{baseline} (the isoelectric or physical zero level of the signal);  the
@code{baseline} is a characteristic of the @emph{signal}, while
@code{adczero} is a characteristic of the @emph{digitizer}.  The value
of @code{baseline} need not necessarily lie within the output range of
the ADC;  for example, if the @code{units} are @samp{degrees_Kelvin},
and the ADC range is 200--300 degrees Kelvin, @code{baseline} corresponds to 
absolute zero, and lies well outside the range of values actually produced
by the ADC.

@item long nsamp
@cindex length of signal file
@cindex duration of signal file
@cindex signal file length
the number of samples in the signal.  (Exception: in multi-frequency records,
@code{nsamp} is the number of samples divided by @code{spf}, see above, i.e.,
the number of frames.)  All signals in a given record must have the same
@code{nsamp}.  If @code{nsamp} is zero, the number of samples is unspecified,
and the @code{cksum} (see the next item) is not used; this is useful for
specifying signals that are obtained from pipes, for which the length may not
be known.

@item int cksum
@cindex checksum of signal file
@cindex signal file checksum
a 16-bit checksum of all samples.  This field is not usually accessed by
application programs; @code{newheader} records checksums calculated by
@code{putvec} when it creates a new @file{hea} file, and
@code{getvec} compares checksums that it calculates against @code{cksum}
at the end of the record, provided that the entire record was read
through without skipping samples.
@end table

@cindex signal number
The number of @code{WFDB_Siginfo} structures in @var{siarray} is given by
the @var{nsig} argument of the functions that open signal files.  Input
and output signal numbers are assigned beginning with 0 in the order in
which the signals are given in @var{siarray}.  Note that input signal 0
and output signal 0 are distinct.  Input signal numbers are supplied to
@code{aduphys}, @code{physadu}, @code{adumuv}, and @code{muvadu} in
their first arguments.  @xref{Example 5}, for an illustration of how to
read signal specifications from @code{WFDB_Siginfo} structures.

@node WFDB_Calinfo structures, WFDB_Anninfo structures, WFDB_Siginfo structures, Data Types
@section Calibration Information Structures
@cindex @code{WFDB_Calinfo} structure (defined)
@cindex calibration information structure

The @var{cal} argument for @code{getcal} and @code{putcal} is a pointer
to an object of type @code{WFDB_Calinfo}.  A @code{WFDB_Calinfo} object
contains information about signals of a specified type:

@table @code
@item char *sigtype
@cindex signal type
a pointer to a null-terminated string without embedded tabs or newlines.
This field describes the type(s) of signals to which the calibration
specifications apply.  Usually, @code{sigtype} is an exact match to
(or a prefix of) the @code{desc} field of the @code{WFDB_Siginfo} object
that describes a matching signal.

@item char *units
@cindex physical units
@cindex units (physical)
a pointer to a null-terminated string without embedded whitespace.
This field specifies the physical units of signals to which the
calibration specifications apply.  Usually, the @code{units} field of
a @code{WFDB_Calinfo} structure must exactly match the @code{units} field of
the @code{WFDB_Siginfo} structure that describes a matching signal.

@item double scale
@cindex scale (amplitude)
@cindex plotting scale
@cindex display scale
the customary plotting scale, in physical units per centimeter.  WFDB
applications that produce graphical output may use @code{scale} as a
default.  Except in unusual circumstances, signals of different types
should be plotted at equal multiples of their respective @code{scale}s.

@item double low
@itemx double high
@cindex calibration pulse limits
@cindex pulse limits (calibration)
values (in physical units) corresponding to the low and high levels of
a calibration pulse.  If the signal is AC-coupled (see below),
@code{low} is zero, and @code{high} is the pulse amplitude.

@item int caltype
@cindex AC-coupled signals
@cindex DC-coupled signals
@cindex calibration pulse shape
@cindex pulse shape (calibration)
a small integer that specifies the shape of the calibration pulse (see
@file{<wfdb/@-wfdb.h>} for definitions).  @code{caltype} is even if signals of
the corresponding @code{sigtype} are AC-coupled, and odd if they are
DC-coupled.
@end table

@cindex calibration list
The calibration list is a memory-resident linked list of @code{WFDB_Calinfo}
structures.  It is accessible only via @code{calopen}, @code{getcal},
@code{putcal}, @code{newcal}, and @code{flushcal}.

@node WFDB_Anninfo structures, WFDB_Annotation structures, WFDB_Calinfo structures, Data Types
@section Annotator Information Structures
@cindex @code{WFDB_Anninfo} structure (defined)
@cindex annotator information structure
@cindex annotator name
@cindex attributes of annotators
@cindex AHA-format annotation file
@cindex WFDB-format annotation file
@cindex standard annotation file
@cindex format (annotation file)
@cindex information structure (annotator)
@cindex structure (annotator information)

The @var{aiarray} argument for @code{annopen} and @code{wfdbinit} is a
pointer to an array of objects of type @code{WFDB_Anninfo}.  Each member
of the array contains information provided to @code{annopen} and
@code{wfdbinit} about an annotation file associated with the
record:

@table @code
@item char *name
@cindex annotator name
the annotator name.  The name @file{atr} is reserved for a @dfn{reference
annotation file} supplied by the creator of the database record to document its
contents as accurately and thoroughly as possible.  You may use other annotator
names to identify annotation files that you create; unless there are compelling
reasons not to do so, follow the convention that the annotator name is the name
of the file's creator (a program or a person).  To avoid confusion, do not use
@samp{dat}, @samp{data@var{n}}, @samp{d@var{n}}, or @samp{hea}
(all of which are commonly used as parts of WFDB file names) as annotator
names.  The special name @samp{-} refers to the standard input or output.
Other annotator names may contain upper- or lower-case letters, digits, and
underscores.  Annotation files are normally created in the current directory
and found in any of the directories in the database path (@pxref{WFDB path}).

@item int stat
@cindex @code{WFDB_AHA_READ}
@cindex @code{WFDB_AHA_WRITE}
@cindex @code{WFDB_READ}
@cindex @code{WFDB_WRITE}
the file type/access code.  Usually, @code{stat} is either
@code{WFDB_READ} or @code{WFDB_WRITE}, to specify standard (``WFDB
format'') annotation files to be read by @code{getann} or to be
written by @code{putann}.  Both MIT DB and AHA DB annotation files can
be (and generally are) stored in WFDB format.  The symbols
@code{WFDB_READ} and @code{WFDB_WRITE} are defined in
@file{<wfdb/wfdb.h>}. An AHA-format annotation file can be read by
@code{getann} or written by @code{putann} if the @code{stat} field is
set to @code{WFDB_AHA_READ} or @code{WFDB_AHA_WRITE} before calling
@code{annopen} or @code{wfdbinit} (@pxref{Example 2}).  Other formats
may be supported via a similar mechanism; consult @file{<wfdb/wfdb.h>}
for more information.
@end table

@cindex annotator number
The number of @code{WFDB_Anninfo} objects in @var{aiarray} is given by
the @var{nann} argument of @code{annopen} and @code{wfdbinit}.  The
annotation-reading function, @code{getann}, knows the annotators by
number only; @code{annopen} and @code{wfdbinit} assign input annotator
numbers beginning with 0 in the order in which they are given in the
array of @code{WFDB_Anninfo} objects.  Output annotator numbers used by
@code{putann} also start at 0; note that input annotator 0 and output
annotator 0 are distinct.  Annotator numbers are supplied to
@code{getann} and @code{putann} in their first arguments.
@ifnotinfo
@xref{annopen, , @code{annopen}},
@end ifnotinfo
@ifinfo
@xref{annopen},
@end ifinfo
for an example of how to set the contents of an array of @code{WFDB_Anninfo}
objects.

@node WFDB_Annotation structures, WFDB_Seginfo structures, WFDB_Anninfo structures, Data Types
@section Annotation Structures
@cindex @code{WFDB_Annotation} structure (defined)
@cindex annotation structure
@cindex structure (annotation)
@cindex attributes of signals (local)

The @var{annot} argument of @code{getann} and @code{putann} is an
object of type @code{WFDB_Annotation} containing these fields:

@table @code
@item long time
@cindex annotation time
@cindex time of annotation
@cindex reference point (on QRS)
time of the annotation, in samples from the beginning of the record.
The times of beat annotations in the @file{atr} files for the MIT DB
generally coincide with the R-wave peak in signal 0; for the AHA DB,
they generally coincide with the PQ-junction.

@item char anntyp
@cindex annotation code field
@cindex annotation type
@cindex type (annotation)
annotation code; an integer between 1 and @code{ACMAX}.
@xref{Annotation Codes}, for a list of legal annotation codes.
@code{ACMAX} is defined in @file{<wfdb/@-ecgcodes.h>}.

@item signed char subtyp
@itemx unsigned char chan
@itemx signed char num
@cindex annotation subtype
@cindex subtype (annotation)
@cindex user-defined fields in annotation
@cindex signal (associating annotation with)
numbers between
@ifnotinfo
@minus{}128
@end ifnotinfo
@ifinfo
-128
@end ifinfo
and 127.  In MIT DB @file{atr} files, the
@code{subtyp} field is used with noise and artifact annotations to
indicate which signals are affected (@pxref{Annotation Codes}).
The @code{chan} field is intended to indicate the signal to which the
annotation is attached.  More than one annotation may be written with
the same @code{time} if the @code{num} or @code{chan} fields are distinct and in
ascending order.  The semantics of the @code{chan} field are
unspecified, however; users may assign any desired meaning, which need
not have anything to do with signal numbers.  In user-created annotation
files, these fields can be used to store arbitrary small integers.  The
@code{subtyp} field requires no space in a standard annotation file
unless it is non-zero; the @code{chan} and @code{num} fields require no
space unless they have changed since the previous annotation.

@item char *aux
@cindex annotation @code{aux} string
@cindex @code{aux} string (annotation)
a free text string.  The first byte is interpreted as an @code{unsigned
char} that specifies the number of bytes that follow (up to 255).  In
MIT DB @file{atr} files, the @code{aux} field is used with rhythm
change annotations to specify the new rhythm, and with comment
annotations to store the text of the comment
@ifnotinfo
(@pxref{Annotation Codes}).
@end ifnotinfo
The string can contain arbitrary binary
data, including embedded nulls.  It is unwise to store anything but ASCII
strings, however, if the annotation file may be transported to a system with
a different architecture (e.g., on which multiple-byte quantities may have
different sizes or byte layouts).  The @code{aux} field requires no
space in a standard annotation file if it is @code{NULL}.  Note that
conversion of annotation files to other formats may entail truncation or
loss of the @code{aux} string.  Note also that the @code{aux} pointer
returned by @code{getann} points to a small static buffer (separately
allocated for each input annotator beginning with WFDB library version
9.4) that may be overwritten by subsequent calls.
@end table

@xref{Example 3}, for a short program that examines the contents of a
@code{WFDB_Annotation}.

@node WFDB_Seginfo structures, , WFDB_Annotation structures, Data Types
@section Segment Information Structures
@cindex @code{WFDB_Seginfo} structure (defined)
@cindex segment information structure
@cindex structure (segment)

Objects of type @code{WFDB_Seginfo} contain these fields:

@table @code
@item char recname[WFDB_MAXRNL+1]
@cindex segment name

Segment name (the name of the simple record corresponding
to a segment of a multi-segment record), unless @code{recname}
has the special value '@code{~}'.  In the latter case, the
segment is a gap (i.e., it corresponds to an interval during
which no signals are available).

@item WFDB_Time nsamp
@cindex segment length

Segment length in samples.

@item WFDB_Time samp0
@cindex segment start

Number of samples that precede the segment in the multi-segment record
to which it belongs.  If the segment is opened as an individual
record, its @code{n}th sample has sample number @code{n-1}, just as
for any record.  If the record to which the segment belongs is opened,
the @code{n}th sample in the segment has sample number @code{n-1+samp0}.
@end table

@node     Annotation Codes, Database Files, Data Types, Top
@chapter Annotation Codes

@menu
* Mapping macros::		Macros for mapping annotation codes.
@end menu

@cindex ECG annotation code
@cindex annotation code
@cindex code (annotation)

Application programs that deal with annotations should include the
line

@example
#include <wfdb/ecgcodes.h>
@end example

@noindent
which provides the symbolic definitions of annotation codes given in the
first column of the table below.  (The second column of the table shows
the strings returned by @code{annstr} and @code{ecgstr}.)

@display
@emph{Beat annotation codes:}
@t{NORMAL   N  } Normal beat
@t{LBBB     L  } Left bundle branch block beat
@t{RBBB     R  } Right bundle branch block beat
@t{BBB      B  } Bundle branch block beat (unspecified)
@t{APC      A  } Atrial premature beat
@t{ABERR    a  } Aberrated atrial premature beat
@t{NPC      J  } Nodal (junctional) premature beat
@t{SVPB     S  } Supraventricular premature or ectopic beat (atrial or nodal)
@t{PVC      V  } Premature ventricular contraction
@t{RONT     r  } R-on-T premature ventricular contraction
@t{FUSION   F  } Fusion of ventricular and normal beat
@t{AESC     e  } Atrial escape beat
@t{NESC     j  } Nodal (junctional) escape beat
@t{SVESC    n  } Supraventricular escape beat (atrial or nodal) [1]
@t{VESC     E  } Ventricular escape beat
@t{PACE     /  } Paced beat
@t{PFUS     f  } Fusion of paced and normal beat
@t{UNKNOWN  Q  } Unclassifiable beat
@t{LEARN    ?  } Beat not classified during learning

@emph{Non-beat annotation codes:}
@t{VFON     [  } Start of ventricular flutter/fibrillation
@t{FLWAV    !  } Ventricular flutter wave
@t{VFOFF    ]  } End of ventricular flutter/fibrillation
@t{NAPC     x  } Non-conducted P-wave (blocked APC) [4]
@t{WFON     (  } Waveform onset [4]
@t{WFOFF    )  } Waveform end [4]
@t{PWAVE    p  } Peak of P-wave [4]
@t{TWAVE    t  } Peak of T-wave [4]
@t{UWAVE    u  } Peak of U-wave [4]
@t{PQ       `  } PQ junction
@t{JPT      '  } J-point
@t{PACESP   ^  } (Non-captured) pacemaker artifact
@t{ARFCT    |  } Isolated QRS-like artifact [2]
@t{NOISE    ~  } Change in signal quality [2]
@t{RHYTHM   +  } Rhythm change [3]
@t{STCH     s  } ST segment change [1,3]
@t{TCH      T  } T-wave change [1,3,4]
@t{SYSTOLE  *  } Systole [1]
@t{DIASTOLE D  } Diastole [1]
@t{MEASURE  =  } Measurement annotation [1,3]
@t{NOTE     "  } Comment annotation [3]
@t{LINK     @@  } Link to external data [5]
@end display

@c @group
@noindent
@strong{Notes:}
@enumerate
@item
Codes @code{SVESC}, @code{STCH}, and @code{TCH} were first introduced
in WFDB library version 4.0.  Codes @code{SYSTOLE}, @code{DIASTOLE}, and
@code{MEASURE} were first introduced in WFDB library version 7.0.

@cindex @code{subtyp} (in @code{NOISE} annotation)
@cindex noisy signals (annotating)
@item
In MIT and ESC DB @file{atr} files, each non-zero bit in the @code{subtyp}
field indicates that the corresponding signal contains noise (the least
significant bit corresponds to signal 0).

@cindex annotation @code{aux} string
@cindex @code{aux} string (annotation)
@item
The @code{aux} field contains an ASCII string (with prefixed byte count)
describing the rhythm, ST segment, T-wave change, measurement, or the
nature of the comment.  By convention, the character that follows the
byte count in the @code{aux} field of a @code{RHYTHM} annotation is @samp{(}.
See the @cite{MIT-BIH Arrhythmia Database Directory} for a list of rhythm
annotation strings.

@item
Codes @code{WFON}, @code{WFOFF}, @code{PWAVE}, @code{TWAVE}, and
@code{UWAVE} were first introduced in DB library version 8.3.  The
@samp{p} mnemonic now assigned to @code{PWAVE} was formerly assigned to
@code{NAPC}, and the @samp{t} mnemonic now assigned to @code{TWAVE} was
formerly assigned to @code{TCH}.  The obsolete codes @code{PQ}
(designating the PQ junction) and @code{JPT} (designating the J-point)
are still defined in @file{<wfdb/ecgcodes.h>}, but are identical to
@code{WFON} and @code{WFOFF} respectively.

@item
The @code{LINK} code was first introduced in WFDB library version 9.6.  The
@code{aux} field of a @code{LINK} annotation contains a URL (a uniform
resource locator, in the form @file{http://machine.name/some/data},
suitable for passing to a Web browser such as Netscape or Mosaic).  @code{LINK}
annotations may be used to associate extended text, images, or other data with
an annotation file.  If the @code{aux} field contains any whitespace, text
following the first whitespace is taken as descriptive text to be displayed by
a WFDB browser such as @code{WAVE}.

@end enumerate
@c @end group

@cindex @code{NOTQRS} (annotation code)
The annotation codes in the table above are the predefined values of the
@code{anntyp} field in a @code{WFDB_Annotation}.  Other values in the range
of 1 to @code{ACMAX} (defined in @file{<wfdb/ecgcodes.h>}) are legal but
do not have preassigned meanings.  The constant @code{NOTQRS}, also defined in
@file{<wfdb/ecgcodes.h>}, is not a legal value for @code{anntyp}, but is
a possible output of the macros discussed below.

@node    Mapping macros, , Annotation Codes, Annotation Codes
@section Macros for Mapping Annotation Codes
@cindex macros
@cindex annotation code mapping
@cindex mapping annotation codes

Application programs that use the macros described in this section
should include the line
@example
#include <wfdb/ecgmap.h>
@end example

@noindent
which will make their definitions, and those in @file{<wfdb/ecgcodes.h>},
available.

@table @code
@findex isann
@cindex annotation code (legal)
@cindex legal annotation code
@item isann(@var{c})
true (1) if @var{c} is a legal annotation code, false (0)
otherwise

@cindex QRS annotation code
@findex isqrs
@item isqrs(@var{c})
true (1) if @var{c} denotes a QRS complex, false (0) otherwise

@findex map1
@item map1(@var{c})
maps @var{c} into one of the set @{@code{NOTQRS}, @code{NORMAL},
@code{PVC}, @code{FUSION}, @code{LEARN}@}

@findex map2
@item map2(@var{c})
maps @var{c} into one of the set @{@code{NOTQRS}, @code{NORMAL},
@code{SVPB}, @code{PVC}, @code{FUSION}, @code{LEARN}@}

@findex annpos (6.0)
@item annpos(@var{c})
maps @var{c} into one of the set @{@code{APUNDEF}, @code{APSTD},
@code{APHIGH}, @code{APLOW}, @code{APATT}, @code{APAHIGH},
@code{APALOW}@} (see @file{<wfdb/ecgmap.h>} for definitions of these
symbols;  this macro was first introduced in WFDB library version 6.0)
@end table

@findex setisqrs (6.0)
@findex setmap1 (6.0)
@findex setmap2 (6.0)
@findex setannpos (6.0)
If you define your own annotation codes, you may wish to modify the
tables used by the macros above.  The file @file{<wfdb/ecgmap.h>} also defines
@code{setisqrs(@var{c}, @var{x})}, @code{setmap1(@var{c}, @var{x})},
@code{setmap2(@var{c}, @var{x})}, and @code{setannpos(@var{c}, @var{x})}
for this purpose.  In each case, @var{x} is the value to be returned
when the corresponding mapping macro is invoked with an argument of
@var{c}.  (These macros were first introduced in WFDB library version
6.0.)

The macros below convert between AHA and MIT annotation codes;  they are
also defined in @file{<wfdb/ecgmap.h>}.

@findex ammap
@findex mamap
@cindex AHA annotation code
@table @code
@item ammap(@var{a})
maps @var{a} (an AHA annotation code) into an MIT annotation code
(one of the set @{@code{NORMAL}, @code{PVC}, @code{FUSION}, @code{RONT},
@code{VESC}, @code{PACE}, @code{UNKNOWN}, @code{VFON}, @code{VFOFF},
@code{NOISE}, @code{NOTE}@}), or @code{NOTQRS}

@item mamap(@var{c}, @var{s})
maps @var{c} (an MIT annotation code) into an AHA annotation code
(one of the set @{@samp{N}, @samp{V}, @samp{F}, @samp{R}, @samp{E},
@samp{P}, @samp{Q}, @samp{[}, @samp{]}, @samp{U}, @samp{O}@}); @var{s}
is the MIT annotation @code{subtyp} (significant only if @var{c} is
@code{NOISE})
@end table

@node     Database Files, Examples, Annotation Codes, Top
@chapter Database Files

@cindex WFDB path
The WFDB library has been constructed to provide a standard interface
between the database files and application programs.  Alternate means of
access to database files is strongly discouraged, since file formats may
change.  Database files are located in the directories specified by
@code{WFDB} (@pxref{WFDB path}).

Recall that a WFDB record is not a file; rather, it is an extensible
@emph{collection} of database files (@pxref{Concepts 1, , Records}).  Thus, for
example, record 100 of the MIT-BIH Arrhythmia Database consists of the files
named @file{100.hea}, @file{100.dat}, and @file{100.atr} in the @file{mitdb}
directory of the MIT-BIH Arrhythmia Database CDROM (or in PhysioBank, within
@uref{http://@-www.@-physio@-net.@-org/@-physio@-bank/@-data@-base/@-mitdb/}),
together with any additional files in other directories that you may have
associated with record 100 (such as your own annotation file).  All files
associated with a given record include the record name as the first part of the
file name.  No explicit action (other than choosing the file name, and locating
the file in the WFDB path) is needed in order to associate a new file with an
existing WFDB record.

To find the location of a database file easily, you can use @file{wfdbwhich},
an application included with the WFDB Software Package.  Type
@code{wfdbwhich} for brief instructions on its use, or see the @cite{WFDB
Applications Guide}.

@section File Types
@cindex file types

There are four types of files supported by the WFDB library:

@menu
* Header Files::		contain signal file names and attributes.
* Signal Files::		contain signals.
* Annotation Files::		contain annotations.
* Calibration Files::		contain signal calibration specifications.

The other topics in this section deal with special types of database access:

* EDF Files::                   a popular format readable by WFDB 10.4.5 and
                                 later.
* AHA Format Files::		not used for on-line WFDB records.
* Standard I/O::		pipes and I/O redirection for WFDB files.
* Multiplexed Signal Files::	signal groups.
* Multi-Frequency Records::     signals sampled at different frequencies within
                                 within a given record.
* Multi-Segment Records::       concatenated records.
* Multiple Record Access::	how to have more than one record open.
* Special Files::		signal I/O using block and character devices.
* Piped and Local Records::	records for use with user-created signals.
* NETFILES::			input directly from web and FTP servers.
* Annotation Order::            the canonical order for annotations:
                                 how and why to break the rules, and how
                                 to deal with the consequences of doing so.
@end menu

@node     Header Files, Signal Files, Database Files, Database Files
@unnumberedsubsec Header Files
@cindex @code{hea} file

@dfn{Header files} have names of the form @file{@var{record}.hea}, where
@var{record} is the record name.  (MIT DB records are named 100--124 and
200--234 with some numbers missing.  AHA DB records are named
1001--1010, 2001--2010, 3001--3010, 4001--4010, 5001--5010, 6001--6010,
7001--7010, and 8001--8010.  ESC DB records are named e0103--e1304, with
many numbers missing.)  Header files are text files, with lines
terminated by ASCII carriage-return/line-feed pairs, created by
@code{newheader}, @code{setheader}, or @code{setmsheader}, from which
@code{isigopen}, @code{osigopen}, and @code{wfdbinit} read the names of
the signal files and their attributes as given in the array of
@code{WFDB_Siginfo} objects; @code{sampfreq} also reads a header
file to determine the sampling frequency used for a record.

@node     Signal Files, Annotation Files, Header Files, Database Files
@unnumberedsubsec Signal Files
@cindex signal file

@dfn{Signal files} usually have names of the form
@file{@var{record}.dat}.  (The @code{.dat} suffix is conventional, but
not required; any file name acceptable to the operating system is
permissible.)  Signal files are binary, and usually contain either
16-bit amplitudes (format 16), pairs of 12-bit amplitudes bit-packed
into byte triplets (format 212), or 8-bit first differences (format 8).
(See @file{<wfdb/wfdb.h>} for information about other formats that are
supported.)  The functions that read and write signal files perform
appropriate transformations so that the samples visible to the
application program are always amplitudes of type @code{int} (at least
16 bits), regardless of the signal file format.

@node     Annotation Files, Calibration Files, Signal Files, Database Files
@unnumberedsubsec Annotation Files
@cindex annotation file

@dfn{Annotation files} have names of the form
@file{@var{record}.@var{annotator}}.  Those named
@file{@var{record}.atr} are reference annotation files (assumed to be
correct).  Annotation files are binary, and contain records of variable
length that average slightly over 16 bits per annotation.

@node     Calibration Files, EDF Files, Annotation Files, Database Files
@unnumberedsubsec Calibration Files
@cindex calibration file

Unlike header, signal, and annotation files, @dfn{calibration files} are
not associated with individual records.  A calibration file is needed
only if you have records containing signals other than ECGs;  in this
case, it is likely that a single calibration file will be adequate for use
with all of your records.  Calibration files are text files, with lines
terminated by ASCII carriage-return/line-feed pairs, created by
@code{newcal}, from which @code{calopen} reads the calibration list
@ifnotinfo
(@pxref{WFDB_Calinfo structures, , Calibration Information Structures}).
@end ifnotinfo
@ifinfo
(@pxref{WFDB_Calinfo structures}).
@end ifinfo
The WFDB Software Package includes a standard calibration file,
@file{wfdbcal}, in the @file{data} directory.

@node     EDF Files, AHA Format Files, Calibration Files, Database Files
@unnumberedsubsec EDF Files
@cindex EDF
@cindex European Data Format

European Data Format (EDF) was defined in 1990, and it has become a very
widely supported open format for exchange of recorded physiologic signals,
especially polysomnograms. EDF files encapsulate functional equivalents of
header and signal files, and EDF+ files can also include annotation streams
(stored as signals).

EDF files begin with an embedded (text) header containing specifications of the
signals and a limited amount of demographic information, followed by the binary
samples of the signals.  Within each block of samples, typically one second to
one minute in length, all samples of the first signal are stored consecutively,
followed by all samples of the second signal, etc.

EDF files can be read directly using WFDB library version 10.4.5 and later.
The name of the file, which must include a '.' and cannot end in '.hea', can
be passed as a record name to any WFDB library function that accepts record
names.  Although the WFDB library does not support EDF as an output format, the
@file{mit2edf} application included in the WFDB Software Package can convert
anything readable by the WFDB library into EDF.

EDF+, defined in 2003, is backwards-compatible with EDF (any EDF reader,
including the WFDB library, can read EDF+), but the additional features of
EDF+, including methods for reading annotations and recognizing signal
discontinuities (which are marked by annotations), are available only from
EDF+-specific readers.  The WFDB library does not currently include built-in
support for the additional features of EDF+, but annotation streams are
available as signals, so it is possible for an application using the WFDB
library to provide its own means of decoding annotation streams as they are
read.

Further information about EDF and EDF+ is available at
@uref{http://www.edfplus.info/}.

@node     AHA Format Files, Standard I/O, EDF Files, Database Files
@unnumberedsubsec AHA Format Files
@cindex AHA format

The ``AHA Format'' was defined in 1980 for storage of database records
on 9-track digital tape.  Signal files in AHA format are in format 16,
with two signals multiplexed into one file (@pxref{Multiplexed Signal Files}),
and may be read and written using @code{getvec} and @code{putvec}.
AHA-format annotation files are binary, and contain fixed-length
(16-byte) annotation records.  An annotation file in AHA format may be
read or written using @code{getann} or @code{putann}, if the @code{stat}
field of the @code{WFDB_Anninfo} object is set to @code{WFDB_AHA_READ} or
@code{WFDB_AHA_WRITE} before opening the file.  @code{annopen} recognizes the
format of input annotation files automatically and prints a warning if
the format does not match what was expected on the basis of @code{stat}.
AHA format annotation files may be converted to standard format without
loss of information, and doing so reduces the storage requirement by a
factor of eight.

Yet another format has been used more recently for distribution of AHA DB files
on floppy diskettes and CDROMs.  This format is compatible with neither the
original AHA format nor with any of the formats supported directly by the WFDB
library.  Programs @file{a2m} and @file{ad2m}, supplied with the WFDB Software
Package, can convert files in this format (as well as those in the original AHA
format) to the standard formats.

@node Standard I/O, Multiplexed Signal Files, AHA Format Files, Database Files
@section Using Standard I/O for Database Files
@cindex standard I/O (as WFDB files)
@cindex pipes (as WFDB files)

If @samp{-} is supplied as a record name to any of the functions that
read or write header files, the @file{hea} file is taken to be
the standard input or output, as appropriate.  If the name of a signal
file is specified in the @file{hea} file (or in the array of
@code{WFDB_Siginfo} objects passed to @code{osigfopen}) as @samp{-}, the
standard input (output) is used by @code{getvec} (@code{putvec}).  If
the name of an annotator is given in the array of @code{WFDB_Anninfo} objects
as @samp{-}, the standard input (output) is used by @code{getann}
(@code{putann}).  If the name of a calibration file is given as
@samp{-}, the standard input (output) is used by @code{calopen}
(@code{newcal}).

Under MS-DOS, these features may not always be usable, since the
standard input and output are usually opened in ``text'' mode (which
is unsuitable for binary database files).

Although the WFDB library does not forbid the use of the standard
input or output for more than one function (e.g., as both a signal file
and an annotation file), such use is in general a gross error that is
likely to lead to unintended results.

@node     Multiplexed Signal Files, Multi-Frequency Records, Standard I/O, Database Files
@section Multiplexed Signal Files
@cindex multiplexed signal file
@cindex signal group

Multiplexed signal files may be identified by examining the @code{group}
fields of the array of @code{WFDB_Siginfo} objects returned by
@code{isigopen} or @code{wfdbinit}.  Signals belonging to the same group
are multiplexed together in the same file.  If all signals in a given signal
file have been sampled at the same frequency, and there are @var{n} signals
in the file, then each group of @var{n} successive samples in that file
contains a sample from each signal, always in the same order
(but @pxref{Multi-Frequency Records}).

Multiplexed signal files can be useful if the storage device is
sequential-access only (e.g., 9-track tape), if the storage device has lengthy
seek times (e.g., optical disk), if many signals must be recorded and Unix's
per-process limit on open files would otherwise be exceeded, or if very high
speed is required while the file is being created (because of sampling
constraints).  CDROM signal files, and those available from PhysioNet, are
multiplexed unless the record contains only one signal.

@node     Multi-Frequency Records, Multi-Segment Records, Multiplexed Signal Files, Database Files
@section Multi-Frequency Records
@cindex multi-frequency records
@cindex sampling frequency
@cindex frame (of samples)
@cindex sample frame
@cindex frequency multiplier

When signals of different types are recorded simultaneously and for
lengthy periods, it may be appropriate to choose different sampling
frequencies in order to reduce the storage requirements for signals of
limited bandwidth.  The support for multi-frequency records provided in
WFDB library version 9.0 (and later versions) allows application programs
to read and write records containing signals digitized at multiple
sampling frequencies.  In a multi-frequency record, a @dfn{frame} of
samples contains one @emph{or more} samples from each signal.  The
@dfn{frame rate} (base sampling frequency) of the record, as recorded in the
header file and as normally returned by @code{sampfreq}, is defined as
the number of frames per second.  Signals sampled at multiples of the
frame rate are referred to as @dfn{oversampled signals}.  For each signal, a
frequency multiplier specifies how many samples are included in each
frame.  The frequency multiplier (1 by default) is an integer, encoded
within the format field in the header file, and specified in the
@code{spf} field of the @code{WFDB_Siginfo} structure for the signal.

A frame can be read as it was written
@ifnotinfo
(@pxref{getframe, , @code{getframe}})
@end ifnotinfo
@ifinfo
(@pxref{getframe})
@end ifinfo
by an application that has been written to make use of multi-frequency
records.  Applications that are not ``multi-frequency aware'' can still
read signals using the standard @code{getvec} interface, which returns
(as always) one sample per signal on each invocation.  By default,
@code{getvec} reads multi-frequency records in @dfn{low-resolution} mode.
In this mode, each oversampled signal is resampled at the frame rate by
averaging all of its samples in each frame.

The function @code{setgvmode} can be used to select @dfn{high-resolution} mode,
in which @code{getvec} replicates samples of signals digitized at less than the
maximum sampling frequency (i.e., using zero-order interpolation) so that each
sample of an oversampled signal appears in at least one sample vector returned
by @code{getvec}.  In this mode, @code{sampfreq} returns the number of samples
per signal returned by @code{getvec} per second of the record.  Furthermore
(when using WFDB library version 9.6 and later versions), all time quantities
passed to and from the WFDB library functions are understood to be in units of
these shorter sampling intervals; thus, for example, @code{getann} converts
times in frame numbers (as recorded in annotation files) into times in sample
numbers before filling in the caller's annotation structure, and @code{putann}
converts times in sample numbers into times in frame numbers before writing
annotations into annotation files.  This permits applications that are not
``multi-frequency aware'' to read multi-frequency records with the highest
possible resolution.

@cindex @code{WFDBGVMODE} (environment variable)
The operating mode used by @code{getvec}, if not specified by an explicit call
to @code{setgvmode}, is determined by the value of the environment variable
@code{WFDBGVMODE} if it is set, and otherwise by the value of
@code{DEFWFDBGVMODE} in @file{wfdblib.h} at the time the library was compiled..
In either case, a value of 0 selects low-resolution mode, and any other value
selects high-resolution mode.

@node     Multi-Segment Records, Multiple Record Access, Multi-Frequency Records, Database Files
@section Multi-Segment Records
@cindex multi-segment records
@cindex concatenating records
@cindex nested records
@cindex layout segment
@cindex fixed layout
@cindex variable layout

A multi-segment record consists of two or more concatenated segments.
Each segment is an ordinary WFDB record, with its own header file and
signal file(s).  There are two types of multi-segment records:
@dfn{fixed-layout} records, in which all signals must appear in the
same order within each segment (signals may not be omitted, added, or
swapped), and the gain and baseline of any given signal may not change
from segment to segment; and @dfn{variable-layout} records, which are
not bound by these constraints.  If the first segment of a
multi-segment record has a length of zero samples, that segment is a
@dfn{layout segment}, and the record is a variable-layout record.
Version 9.1 of the WFDB library is the first to support reading and
writing fixed-layout multi-segment records, and version 10.3.17 is the
first to support reading variable-layout multi-segment records.

In both types of multi-segment records, the sampling frequency of any
given signal must be the same in each segment.  Segments of
multi-segment records must be ordinary records (it is not permitted to
nest one multi-segment record within another, for example), and the
length of each segment must be specified (the WFDB library does not
impose this requirement on ordinary records that are not part of a
multi-segment record).  There are no other restrictions on segments;
specifically, it is permitted to mix segments with different storage
formats, and for any segment to appear more than once.  A special
header file (created either manually or by using @code{setmsheader})
specifies the record name for each segment in a multi-segment record.
Once this special header exists, the multi-segment record can be read
by any WFDB application.  Note that only the signal files of the
segments are ``linked'' by the multi-segment record's header;
annotation files associated with the individual segments are
@emph{not} readable as part of the multi-segment record (although an
annotation file associated directly with the multi-segment record can
be created and read just as for an ordinary record).  From the point
of view of a WFDB application, reading a multi-segment record is
exactly like reading an ordinary record; specifically,
@code{isigsettime} works as expected, permitting jumps forward and
backward between as well as within segments.

Unlike ordinary segments, no signal file is associated with a layout segment;
only the header file is needed.  In the header file of a layout segment, all of
the signals present in any of the other segments of the record are listed, in
the desired order and with the desired gains and baselines.  When the WFDB
library reads the record, @code{getframe} assembles the frame, scaling and
shifting each component as needed.  If any signals are missing during a
segment, the resulting gaps are filled with the sample value
@code{WFDB_INVALID_SAMPLE}.  In this way, WFDB applications do not need to be
aware of signal changes; rather, they may read variable-layout records as if
they were ordinary (fixed-layout) records.

WFDB applications generally assume fixed-layout records, i.e., that
the number and types of available signals (and their sampling
frequencies, gains, and baselines) are constant throughout the record.
These conditions do not always apply in clinical settings, in which
signals may be added, removed, or recalibrated to meet clinical needs,
resulting in variable-layout recordings.  A variable-layout recording
may be divided into segments such that each segment is a fixed-layout
WFDB record.  The segments can then be reassembled into a
multi-segment WFDB record.  Version 10.3.17, and later versions, of
the WFDB library contain code in @code{signal.c} to create a virtual
fixed-layout record on the fly when reading a variable-layout record.
This feature can also be used to transform ordinary records on the fly, if
it is desirable to rearrange, delete, duplicate, scale, or shift
signals.

@node     Multiple Record Access, Special Files, Multi-Segment Records, Database Files
@section Simultaneous Access to Multiple Records
@cindex simultaneous records
@cindex access to multiple records
@cindex multiple record access

Selection functions that accept @var{record} arguments (@code{annopen},
@code{isigopen}, @code{osigopen}, and @code{wfdbinit}) normally close any
active database files of the types with which each deals before
proceeding.  The argument @code{+}@var{record} is synonymous with
@var{record}, but has the effect of causing these functions to leave any
active files open.  (For convenience, the other functions that accept
@var{record} arguments --- @code{sampfreq}, @code{newheader}, and
@code{setheader} --- also treat @var{record} and @code{+}@var{record} as
synonymous, but without any noticeable effect.)  The restrictions on the
total numbers of signals and annotation files still apply.

If the sampling frequencies or lengths of the records do not match, a warning
message will be produced (unless @code{wfdbquiet} was invoked).  The
time-conversion functions
@ifnotinfo
(@pxref{timstr and strtim, , @code{timstr} and @code{strtim}})
@end ifnotinfo
@ifinfo
(@pxref{timstr and strtim})
@end ifinfo
will continue to use the sampling frequency and base
time defined for the first record that was opened, unless these
attributes are reset by @code{sampfreq}, @code{setsampfreq}, or
@code{setbasetime}.

Function @code{calopen} uses the @samp{+} convention for calibration
file names.  Although it normally creates the calibration list from
scratch each time it is called, it retains the current calibration list
if the calibration file name is prefixed by @samp{+}.

@node     Special Files, Piped and Local Records, Multiple Record Access, Database Files
@section Signals That Are Not Stored in Disk Files
@cindex character devices (as signal files)
@cindex nine-track tape
@cindex reading 9-track tape
@cindex signal file (on tape)
@cindex special files (as signal files)
@cindex Unix character devices (as signal files)
@cindex writing 9-track tape

The @code{fname} component of a @code{WFDB_Siginfo} object can be any
string acceptable as a file name to your operating system.  Under Unix,
for example, signals can be read from (or written to) `special' files
such as @file{/dev/rmt0} (the raw tape drive).  If I/O must be performed
in fixed-size blocks (such as for Unix character devices), the
@code{bsize} component of the @code{WFDB_Siginfo} object must contain the
appropriate block size in bytes.  In such cases, the WFDB library must
obtain (using @code{malloc} (see @cite{K&R}, page 167) an amount of
memory equal to the size of one block when the signal file is first
opened.  For large programs running on 16-bit machines, this can cause
problems if signal files with large block sizes are read.  (In such
cases, @code{isigopen} or @code{osigopen} will not open the signal file
if there is not enough memory to allocate a buffer.)  Under Unix, if
this problem occurs, use the ``piped records'' (@pxref{Piped and Local
Records}) instead.  The usual method is to read or write the signal file
using a utility such as Unix's @code{dd} and to pipe the data to or from
the application program.  Although this approach is flexible, there are
a few drawbacks:

@enumerate
@item
While reading piped input, the standard input cannot be used for other
purposes by the application program.  Interactive programs can avoid
problems by opening @file{/dev/tty} for I/O, however.

@item
Programs that use @code{isigsettime} or @code{isgsettime} cannot perform
backward skips on piped input, and forward skips can be quite slow.

@item
Additional system resources (computation time, process slots, and
memory) are needed when using pipes, in comparison with the usual
method of operation.
@end enumerate

Several special-purpose header files allow application programs to read data
directly from 9-track tape.  When the WFDB Software Package is installed, these
files are copied into the @file{tape} subdirectory of the system-wide database
directory.  The record names associated with most of these header files
(@file{tape/512}, @file{tape/1024}, @file{tape/4096}, @file{tape/10240})
specify the block size in bytes.  These use 16-bit format, 250 Hz samples, 12
bit ADC with zero ADC offset, two signals multiplexed into one, and data to be
read from @file{/dev/rmt0}.  Record @file{tape/6144d} uses 8-bit difference
format, 6144 bytes/block, and is otherwise similar to the others.  Records
@file{tape/ahatape} and @file{tape/mittape} can be used to read or write an
AHA-format signal file on a 9-track tape that has been positioned to the
beginning of the correct file; the signal file for these is @file{/dev/nrmt0}
(the non-rewinding raw tape drive).  If the tape density is encoded into the
tape drive name on your system, additional header files may be needed.

@node     Piped and Local Records, NETFILES, Special Files, Database Files
@section Piped and Local Records
@cindex piped record
@cindex record (piped)
@cindex signal file (piped)
@cindex standard I/O (as WFDB files)

@dfn{Piped record} header files allow application programs to read signals from
the standard input, or write them to the standard output.  Record @file{8}
specifies 8-bit format, a 10-bit ADC, zero ADC offset, and two signals sampled
at 250 Hz, both of which are to be acquired from the standard input, or written
to the standard output.  Record @file{16} specifies 16-bit format and a 12-bit
ADC, and is otherwise identical to record @file{8}.  ADCs from several
manufacturers can produce output in the format specified by record @file{16};
thus such output can be piped directly into an application program using record
@file{16}.  Signal files in AHA format also match these specifications.  Piped
records for reading or writing other numbers of signals are provided in the
@file{pipe} subdirectory of the system-wide database directory; they are named
@file{pipe/8x@var{n}} and @file{pipe/16x@var{n}}, where @var{n} is the number
of signals (@var{n} = 1, 2, @dots{}, 16; piped record header files can be
created with larger numbers of signals (use the existing files as a model).

@cindex local record
@cindex signal file (local)
Application programs may also read or write signal files in the current
directory using @dfn{local record} header files.  Record
@file{16l} (``one-six-ell'') specifies up to sixteen format 16
files, and record @file{8l} (``eight-ell'') specifies up to
sixteen format 8 files, named @file{data0}, @file{data1},
@file{data2}, @dots{}, @file{data@var{n}} in the current directory.
When opened using @code{isigopen} or @code{wfdbinit}, these signal files
will be readable by @code{getvec} as signals 0, 1, 2, @dots{} 16
respectively.  These files should be created by the user, with the use
of @code{putvec}.  It is necessary to create only as many signal files
as will be used; if, for example, only one signal is needed, only
@file{data0} need be created.


@node     NETFILES, Annotation Order, Piped and Local Records, Database Files
@section NETFILES
@cindex NETFILES
@cindex URL
@cindex http
@cindex ftp
@cindex WFDB path
@cindex curl
@cindex libcurl
@cindex libwww
@cindex W3C libwww

If the symbol @code{WFDB_NETFILES} is defined at the time the WFDB library is
compiled, then input files located on remote web (HTTP) and FTP servers can be
read directly.  This capability was originally implemented using the World Wide
Web Consortium's @code{libwww} library (which is available on many of the
platforms supported by the WFDB library).  In WFDB library version 10.3.16 and
later, this capability can alternatively be provided by the @code{libcurl}
library, which is smaller, faster, and more robust than @code{libwww}, and is
actively supported and its creators on all of the popular platforms and many
others as well.  NETFILES support, if available, is transparent to WFDB
applications.  To make use of this feature, simply link to the NETFILES-enabled
WFDB library (the necessary @code{libcurl} or @code{libwww} functions will be
loaded automatically), and incorporate one or more URL prefixes in the WFDB
path.

@cindex WFDB path
@cindex PhysioBank
In current versions of the WFDB library, the default WFDB path (defined in the
WFDB library source file @code{wfdblib.h}, and used as the WFDB path if the
WFDB environment variable is undefined) is @samp{. /usr/database
http://www.physio@-net.org/physio@-bank/data@-base}.  (The second component,
after the @samp{.} that specifies the current directory, may vary, depending on
your platform and the choices made during installation.)  The third component
is a URL prefix pointing to PhysioBank, an on-line archive for a wide variety
of standard databases of physiologic signals.
For example, the MIT-BIH Polysomnographic Database is kept in
@uref{http://www.physio@-net.org/physio@-bank/data@-base/slpdb}, so it is
possible to read record @code{slp37} of that database directly from PhysioBank
by passing @code{slpdb/slp37} as the @var{record} argument to @code{wfdbinit}
(or @code{isigopen}, @code{annopen}, etc.).

Current implementations of @code{libcurl} and @code{libwww} permit input from
@code{http://} URLs in much the same way that local files are read, provided
that the remote web server supports HTTP 1.1 range requests (most, including
PhysioNet's, do).  This means that it is not necessary to download an entire
file in order to examine part of it, and you may notice little or no speed
difference between local file and network file input for many applications.  If
the remote server does not support range requests, however, or if input is from
an @code{ftp://} URL, the current implementations download the entire file to
memory, so you may notice a significant startup delay if the file is long and
your network connection is slow, or if the file does not fit into physical
memory.

Currently, NETFILES support is limited to input files; as always, any output
files created by the WFDB library are written into the current directory,
unless the record name contains local path information.

NETFILES support was introduced in WFDB library version 10.0.1.

@node     Annotation Order, , NETFILES, Database Files
@section Annotation Order
@cindex annotation order
@cindex canonical order of annotations
@cindex order of annotations

WFDB applications may generally assume (and most of them do assume) that all
annotations in any given annotation file are in @dfn{canonical order}.
Successful use of @code{iannsettime} requires that this assumption be correct.
The earliest versions of the WFDB library (before version 6.1) defined
canonical order as time order.  Versions 6.1 through 10.4.11 define canonical
order as @code{time} and @code{chan} order, and versions 10.4.12 and later also
use @code{num} for ordering (thus annotations are arranged first in @code{time}
order, and any simultaneous annotations are arranged according to the value of
their @code{num} fields, from smallest to largest, and those with identical
@code{num} fields are similarly arranged in @code{chan} order).

@cindex location (of annotation)
@cindex annotation location
@cindex virtual array of annotations
@cindex annotation (canonical order)
@cindex canonical order of annotations
The combination of the @code{time}, @code{num}, and @code{chan} fields of an
annotation defines a unique @dfn{location} in a virtual array of
annotations which an annotation file represents.  No two annotations may
occupy the same location in this virtual array.  This restriction was
enforced by versions of the WFDB library earlier than version 9.7.  In
these versions of the WFDB library, @code{putann} required that
annotations be written in canonical order, and refused to write any
out-of-order annotations supplied to it.

@cindex @code{sortann}
Current versions of the WFDB library do not impose this requirement.  In version
9.7 and later versions, @code{putann} accepts and records out-of-order
annotations and multiple annotations that occupy the same location.  If any
such annotations have been written, the completed annotation file is rewritten
in canonical order by @code{wfdbquit} or @code{oannclose}.  This is accomplished
by running @file{sortann} (see the @cite{WFDB Applications
Guide}) as a separate process using the ANSI C @code{system} function.  If this
function is not available, or if @file{sortann} cannot be run, @code{wfdbquit}
(or @code{oannclose}) emits a warning message describing how to post-process
the annotations to put them into canonical order.

@cindex changing an annotation
@cindex deleting an annotation
@cindex annotation (changing or deleting)
Although it is possible using current versions of the WFDB library to write two
or more annotations to the same location, @emph{only the last annotation
written to any given location is retained} in the canonically-ordered
annotation file.  Thus that an application that generates an annotation file
can change the @code{anntyp}, @code{subtyp}, or @code{aux} fields of a
previously-written annotation simply by writing another annotation to the same
location (i.e, with the same @code{time}, @code{num}, and @code{chan} fields).
As a special case, an application may @emph{delete} a previously-written
annotation by writing a @code{NOTQRS} annotation to the same location.  To move
an annotation to a different location (i.e., to change its @code{time},
@code{num}, or @code{chan} fields), it is necessary to delete it from the
original location, and then to insert it at the desired location, using two
separate invocations of @code{putann}.

@cindex unsorted annotation files
@cindex @code{WFDBANNSORT} (environment variable)
In unusual circumstances, an unsorted annotation file may be useful (for
example, as an aid for debugging the application that produced it; @file{rdann}
can be used to list all of the annotations in such a file, in the order in
which they were written).  In some environments, the use of the ANSI C
@code{system} function may be a security problem, and you may wish to avoid
automatic sorting of annotations for this reason.  Set the environment variable
@code{WFDBANNSORT} to 0 at run time, or define the symbol @code{DEFWFDBANNSORT}
as 0 when compiling the WFDB library, if you wish to suppress automatic
annotation sorting by @code{wfdbquit} and @code{oannclose}.

@node Examples, Exercises, Database Files, Top
@chapter Programming Examples
@cindex examples
@cindex programming examples
The programs in this chapter are useful as models for a variety of
applications that use the WFDB library.  The line numbers are for
reference only; they are not part of the programs.  Any of these
examples can be compiled (under Unix) using a command of the form
@example
cc @var{file.c} -lwfdb
@end example
@noindent
or, if the WFDB library or its @code{*.h} files are not in the standard
locations:
@example
cc `wfdb-config --cflags` @var{file.c} `wfdb-config --libs`
@end example
@noindent
where @var{file.c} is the name of the file containing the source;
@pxref{Usage, , Using the WFDB Library}, for further information.
The sources for these examples are included in the WFDB Software Package,
within the @file{examples} directory.

@menu
* Example 1::			An annotation filter.
* Example 2::			A WFDB-to-AHA format annotation translator.
* Example 3::			An annotation printer.
* Example 4::			Generating an R-R interval histogram.
* Example 5::			A program that prints signal specifications.
* Example 6::			A differentiator.
* Example 7::			A general-purpose FIR filter.
* Example 8::			Creating a new WFDB record.
* Example 9::			A signal averager.
* Example 10::			A QRS detector.
@end menu

@node     Example 1, Example 2, Examples, Examples
@unnumberedsec Example 1:  An Annotation Filter

The following program copies an annotation file, changing all QRS annotations
to @code{NORMAL} and deleting all non-QRS annotations.

@example
 @i{1}  #include <stdio.h>
 @i{2}  #include <wfdb/wfdb.h>
 @i{3}  #include <wfdb/ecgmap.h>
 @i{4}
 @i{5}  main()
 @i{6}  @{
 @i{7}      WFDB_Anninfo an[2];
 @i{8}      char record[8], iann[10], oann[10];
 @i{9}      WFDB_Annotation annot;
@i{10}
@i{11}      printf("Type record name: ");
@i{12}      fgets(record, 8, stdin); record[strlen(record)-1] = '\0';
@i{13}      printf("Type input annotator name: ");
@i{14}      fgets(iann, 10, stdin); iann[strlen(iann)-1] = '\0';
@i{15}      printf("Type output annotator name: ");
@i{16}      fgets(oann, 10, stdin); oann[strlen(oann)-1] = '\0';
@i{17}      an[0].name = iann; an[0].stat = WFDB_READ;
@i{18}      an[1].name = oann; an[1].stat = WFDB_WRITE;
@i{19}      if (annopen(record, an, 2) < 0) exit(1);
@i{20}      while (getann(0, &annot) == 0)
@i{21}          if (isqrs(annot.anntyp)) @{
@i{22}              annot.anntyp = NORMAL;
@i{23}              if (putann(0, &annot) < 0) break;
@i{24}          @}
@i{25}      wfdbquit();
@i{26}  @}
@end example

@noindent
(See @uref{http://www.physionet.org/physiotools/wfdb/examples/example1.c}
for a copy of this program.)

@c @group
@noindent
@strong{Notes:}

@table @emph
@item Line 2:
All programs that use the WFDB library must include
@file{<wfdb/wfdb.h>}.

@item Line 3:
The @code{#include} statement makes available not only the mapping
macros, one of which will be used in line 21, but also the
annotation code symbols in @file{<wfdb/ecgcodes.h>}, one of which
will be needed in line 22.

@item Line 7:
Since there will be two annotators (one each for input and output),
the array of @code{WFDB_Anninfo} objects has two members.

@item Line 9:
This structure will be filled in by @code{getann}, modified, and passed
to @code{putann} for output.

@item Lines 11--16:
The record name and the annotator names are filled into the character
arrays.  The code in lines 12, 14, and 16 illustrates a C idiom for
reading a string of limited length; the second statement in each of
these lines replaces the trailing newline character (which @code{fgets}
copies into the string) with a null.  String arguments to WFDB library
functions should not include newline characters.

@item Lines 17--18:
Pointers to the character arrays (strings) containing the annotator
names are filled into the @code{name} fields of the array of
@code{WFDB_Anninfo} objects. Note that the @code{name} fields are only
pointers and do not contain storage for the strings themselves.  If this
is not clear to you, review the discussion of pointers and arrays in
@cite{K&R}, pp.  97--100.  The input annotator is to be read, the output
annotator is to be written.  @code{WFDB_READ} and @code{WFDB_WRITE} are
defined in @file{<wfdb/wfdb.h>}.

@item Line 19:
Note that the first and second arguments of @code{annopen} are
the names of the respective arrays; thus @code{annopen} receives pointers
rather than values in its argument list.

@item Line 20:
An annotation is read from annotator 0 into @code{annot}.
The @samp{&} is necessary since @code{getann} requires a pointer to
the structure in order to be able to modify its contents.  When
@code{getann} returns a negative value, no more annotations remain to be
read and the loop ends.

@item Line 21:
The macro @code{isqrs} is defined in @file{<wfdb/ecgmap.h>};
@code{isqrs(@var{x})} is true if @var{x} is an annotation code that
denotes a QRS complex, false if @var{x} is not a QRS annotation
code.

@item Line 22:
@code{NORMAL} is defined in @file{<wfdb/ecgcodes.h>}.

@item Line 23:
The call to @code{putann} now writes the modified annotation in the
output annotator 0 file.  As for @code{getann}, a pointer to
@code{annot} must be passed using the @samp{&} operator.

@item Line 25:
All files are closed prior to exiting.  This is mandatory since the
program creates an output file with @code{putann}.
@end table
@c @end group

@node     Example 2, Example 3, Example 1, Examples
@unnumberedsec Example 2: An Annotation Translator

This program translates the @file{atr} annotations for the record
named in its argument into an AHA-format annotation file with the
annotator name @file{aha}.

@example
 @i{1}  #include <stdio.h>
 @i{2}  #include <wfdb/wfdb.h>
 @i{3}
 @i{4}  main(argc, argv)
 @i{5}  int argc;
 @i{6}  char *argv[];
 @i{7}  @{
 @i{8}      WFDB_Anninfo an[2];
 @i{9}      WFDB_Annotation annot;
@i{10}
@i{11}      if (argc < 2) @{
@i{12}          fprintf(stderr, "usage: %s record\n", argv[0]);
@i{13}          exit(1);
@i{14}      @}
@i{15}      an[0].name = "atr"; an[0].stat = WFDB_READ;
@i{16}      an[1].name = "aha"; an[1].stat = WFDB_AHA_WRITE;
@i{17}      if (annopen(argv[1], an, 2) < 0) exit(2);
@i{18}      while (getann(0, &annot) == 0 && putann(0, &annot) == 0)
@i{19}          ;
@i{20}      wfdbquit();
@i{21}      exit(0);
@i{22}  @}
@end example

@noindent
(See @uref{http://www.physionet.org/physiotools/wfdb/examples/example2.c}
for a copy of this program.)

@noindent
@strong{Notes:}

@table @emph
@item Lines 4--6:
If this doesn't look familiar, see @cite{K&R}, pp. 114--115.

@item Lines 11--14:
This is the standard idiom for producing those cryptic error messages
for which Unix programs are notorious; @code{argv[0]} is the name by
which the program was invoked.

@item Lines 15--16:
These lines set up the annotator information.  Input annotator 0 is the
@file{atr} annotation file, and output annotator 0 will be written
in AHA format.

@item Line 17:
If we can't read the input or write the output, quit with an error
message from @code{annopen}.

@item Line 18:
Here's where the work is done.  The format translation is handled
entirely by @code{getann} and @code{putann}.  The loop ends normally
when @code{getann} reaches the end of the input file, or prematurely if
there is a read or write error.

@item Line 21:
Since we have carefully defined non-zero exit codes for the various
errors that this program might encounter, we also define this
successful exit here.  If this program is run as part of a Unix shell
script, the exit codes are accessible to the shell, which can determine
what to do next as a result.  If this line were omitted (as in example
1), the exit code would be undefined.
@end table

@node     Example 3, Example 4, Example 2, Examples
@unnumberedsec Example 3: An Annotation Printer

This program prints annotations in readable form.  Its first argument is
an annotator name, and its second argument is a record name.

@example
 @i{1}  #include <stdio.h>
 @i{2}  #include <wfdb/wfdb.h>
 @i{3}
 @i{4}  main(argc, argv)
 @i{5}  int argc;
 @i{6}  char *argv[];
 @i{7}  @{
 @i{8}      WFDB_Anninfo a;
 @i{9}      WFDB_Annotation annot;
@i{10}
@i{11}      if (argc < 3) @{
@i{12}          fprintf(stderr, "usage: %s annotator record\n", argv[0]);
@i{13}          exit(1);
@i{14}      @}
@i{15}      a.name = argv[1]; a.stat = WFDB_READ;
@i{16}      (void)sampfreq(argv[2]);
@i{17}      if (annopen(argv[2], &a, 1) < 0) exit(2);
@i{18}      while (getann(0, &annot) == 0)
@i{19}          printf("%s (%ld) %s %d %d %d %s\n",
@i{20}                 timstr(-(annot.time)),
@i{21}                 annot.time,
@i{22}                 annstr(annot.anntyp),
@i{23}                 annot.subtyp, annot.chan, annot.num,
@i{24}                 (annot.aux != NULL && *annot.aux > 0) ?
@i{25}                  annot.aux+1 : "");
@i{26}      exit(0);
@i{27}  @}
@end example

@noindent
(See @uref{http://www.physionet.org/physiotools/wfdb/examples/example3.c}
for a copy of this program.)

@noindent
@strong{Notes:}
@table @emph
@item Line 16:
The invocation of @code{sampfreq} here sets the internal variables needed
by @code{timstr} below.

@item Line 20:
This line gives the annotation time as a time of day.  If the base time
is omitted in the header file, or if we used
@code{timstr(annot.time)} instead, we would obtain the elapsed time from
the beginning of the record.

@item Lines 24--25:
This expression evaluates to an empty string unless the @code{aux}
string is non-empty.  It makes the assumption that @code{aux} is a
printable ASCII string; the printable part follows the length
byte.
@end table

@node     Example 4, Example 5, Example 3, Examples
@unnumberedsec Example 4: Generating an R-R Interval Histogram

This program reads an annotation file, determines the intervals between
beat annotations (assumed to be the R-R intervals), and accumulates a
histogram of them.

@example
 @i{1}  #include <stdio.h>
 @i{2}  #include <wfdb/wfdb.h>
 @i{3}  #include <wfdb/ecgmap.h>
 @i{4}
 @i{5}  main(argc, argv)
 @i{6}  int argc;
 @i{7}  char *argv[];
 @i{8}  @{
 @i{9}      int rr, *rrhist, rrmax;
@i{10}      long t;
@i{11}      WFDB_Anninfo a;
@i{12}      WFDB_Annotation annot;
@i{13}      void *calloc();
@i{14}
@i{15}      if (argc < 3) @{
@i{16}          fprintf(stderr, "usage: %s annotator record\n", argv[0]);
@i{17}          exit(1);
@i{18}      @}
@i{19}      a.name = argv[1]; a.stat = WFDB_READ;
@i{20}      if (annopen(argv[2], &a, 1) < 0) exit(2);
@i{21}      if ((rrmax = (int)(3*sampfreq(argv[2]))) <= 0) exit(3);
@i{22}      if ((rrhist = (int *)calloc(rrmax+1, sizeof(int))) == NULL) @{
@i{23}          fprintf(stderr, "%s: insufficient memory\n", argv[0]);
@i{24}          exit(4);
@i{25}      @}
@i{26}      while (getann(0, &annot) == 0 && !isqrs(annot.anntyp))
@i{27}          ;
@i{28}      t = annot.time;
@i{29}      while (getann(0, &annot) == 0)
@i{30}          if (isqrs(annot.anntyp)) @{
@i{31}              if ((rr = annot.time - t) > rrmax) rr = rrmax;
@i{32}              rrhist[rr]++;
@i{33}              t = annot.time;
@i{34}          @}
@i{35}      for (rr = 1; rr < rrmax; rr++)
@i{36}          printf("%4d %s\n", rrhist[rr], mstimstr((long)rr));
@i{37}      printf("%4d %s (or longer)\n", rrhist[rr], mstimstr((long)rr));
@i{38}      exit(0);
@i{39}  @}
@end example

@noindent
(See @uref{http://www.physionet.org/physiotools/wfdb/examples/example4.c}
for a copy of this program.)

@c @group
@noindent
@strong{Notes:}
@table @emph
@item Lines 21--25:
Here we allocate storage for the histogram.  The value returned by
@code{sampfreq}, if positive, specifies the number of sample intervals
per second; we will allocate 3 seconds' worth of bins, initialized to
zero.  See @cite{K&R}, page 167, for a description of
@code{calloc}.

@item Lines 26--28:
This code sets @code{t} to the time of the first annotated beat in the
record.

@item Lines 29--34:
Here we read the remainder of the annotations, skipping any non-beat
annotations.  The difference between the values of @code{annot.time} for
consecutive beat annotations defines an R-R interval (@code{rr}).  Each
possible value of @code{rr} up to @code{rrmax} is assigned a bin in
@code{rrhist}.  Intervals longer than 3 seconds (@code{rrmax}) are
counted in the bin corresponding to @code{rr} = @code{rrmax}.

@item Lines 35--37:
The histogram is printed as a two-column table, with the number of
intervals in the first column and the length of the interval (with
millisecond resolution) in the second column. (What happens if
@code{rr} starts at 0 rather than 1 in line 35?)
@end table
@c @end group

@node     Example 5, Example 6, Example 4, Examples
@unnumberedsec Example 5: Reading Signal Specifications

This program reads the signal specifications of the record named as its
argument:

@example
 @i{1}  #include <stdio.h>
 @i{2}  #include <wfdb/wfdb.h>
 @i{3}
 @i{4}  main(argc, argv)
 @i{5}  int argc;
 @i{6}  char *argv[];
 @i{7}  @{
 @i{8}      WFDB_Siginfo *s;
 @i{9}      int i, nsig;
@i{10}
@i{11}      if (argc < 2) @{
@i{12}          fprintf(stderr, "usage: %s record\n", argv[0]);
@i{13}          exit(1);
@i{14}      @}
@i{15}      nsig = isigopen(argv[1], NULL, 0);
@i{16}      if (nsig < 1) exit(2);
@i{17}      s = (WFDB_Siginfo *)malloc(nsig * sizeof(WFDB_Siginfo));
@i{18}      if (s == NULL) @{
@i{19}          fprintf(stderr, "insufficient memory\n");
@i{20}	        exit(3);
@i{21}      @}
@i{22}      if (isigopen(argv[1], s, nsig) != nsig) exit(2);
@i{23}      printf("Record %s\n", argv[1]);
@i{24}      printf("Starting time: %s\n", timstr(0L));
@i{25}      printf("Sampling frequency: %g Hz\n", sampfreq(argv[1]));
@i{26}      printf("%d signals\n", nsig);
@i{27}      for (i = 0; i < nsig; i++) @{
@i{28}          printf("Group %d, Signal %d:\n", s[i].group, i);
@i{29}          printf(" File: %s\n", s[i].fname);
@i{30}          printf(" Description: %s\n", s[i].desc);
@i{31}          printf(" Gain: ");
@i{32}          if (s[i].gain == 0.)
@i{33}              printf("uncalibrated; assume %g", WFDB_DEFGAIN);
@i{34}          else printf("%g", s[i].gain);
@i{35}          printf(" adu/%s\n", s[i].units ? s[i].units : "mV");
@i{36}          printf(" Initial value: %d\n", s[i].initval);
@i{37}          printf(" Storage format: %d\n", s[i].fmt);
@i{38}          printf(" I/O: ");
@i{39}          if (s[i].bsize == 0) printf("can be unbuffered\n");
@i{40}          else printf("%d-byte blocks\n", s[i].bsize);
@i{41}          printf(" ADC resolution: %d bits\n", s[i].adcres);
@i{42}          printf(" ADC zero: %d\n", s[i].adczero);
@i{43}          if (s[i].nsamp > 0L) @{
@i{44}              printf(" Length: %s (%ld sample intervals)\n",
@i{45}                     timstr(s[i].nsamp), s[i].nsamp);
@i{46}              printf(" Checksum: %d\n", s[i].cksum);
@i{47}          @}
@i{48}          else printf(" Length undefined\n");
@i{49}      @}
@i{50}      exit(0);
@i{51}  @}
@end example

@noindent
(See @uref{http://www.physionet.org/physiotools/wfdb/examples/example5.c}
for a copy of this program.)

@noindent
@strong{Notes:}

@table @emph
@item Line 15:
The command-line argument, @code{argv[1]}, is the record name.  The
number of signals listed in the header file for the record is returned by
@code{isigopen} as @code{nsig}.  If @code{nsig} < 1, @code{isigopen} will
print an error message; in this case the program can't do anything useful,
so it exits.

@item Line 17:
We allocate @code{nsig} signal information (@code{WFDB_Siginfo}) objects.

@item Line 22:
On the second invocation of @code{isigopen}, we pass the pointer to the
signal information objects and the number of signals we expect to open.
@code{isigopen} returns the number of signals it is able to open;  if
any of those named in the header file are unreadable, the return value
will not match @code{nsig}, and the program exits.

@item Line 24:
Invoking @code{timstr} with an argument of zero (here written @samp{0L}
to emphasize to the compiler that the argument is a @code{long} integer)
will obtain the starting time of the record.  If no starting time is
defined, @code{timstr} will return ``@code{0:00:00}''.

@item Lines 31--34:
Notice how a zero value for @code{gain} is interpreted.

@item Line 35:
If the @code{units} field is NULL, the physical units are assumed to be
millivolts (``mV'').

@item Lines 38--40:
If @code{bsize} is zero, I/O can be performed in blocks of any reasonable
size;  otherwise it must be performed in blocks of exactly the specified
@code{bsize}.

@item Lines 43--48:
If the length of the record is defined, it is printed in both hours,
minutes, and seconds, and in sample intervals.  Since the argument of
@code{timstr} in line 39 is positive, it is interpreted as a time
interval.  The checksum is defined only if the record length is
defined.
@end table

@node     Example 6, Example 7, Example 5, Examples
@unnumberedsec Example 6: A Differentiator

@cindex digital filter
@cindex filter (digital)
The program below inverts and differentiates the signals read by
@code{getvec} and writes the results with @code{putvec}.  The output is
readable as record @file{dif}.  A wide variety of simple digital filters
can be modelled on this example; @pxref{Example 7}, for a more general
approach.

@example
 @i{1}  #include <stdio.h>
 @i{2}  #include <wfdb/wfdb.h>
 @i{3}  
 @i{4}  main(argc, argv)
 @i{5}  int argc;
 @i{6}  char *argv[];
 @i{7}  @{
 @i{8}      WFDB_Siginfo *s;
 @i{9}      int i, nsig, nsamp=1000;
@i{10}      WFDB_Sample *vin, *vout;
@i{11}  
@i{12}      if (argc < 2) @{
@i{13}          fprintf(stderr, "usage: %s record\n", argv[0]); exit(1);
@i{14}      @}
@i{15}      if ((nsig = isigopen(argv[1], NULL, 0)) <= 0) exit(2);
@i{16}      s = (WFDB_Siginfo *)malloc(nsig * sizeof(WFDB_Siginfo));
@i{17}      vin = (WFDB_Sample *)malloc(nsig * sizeof(WFDB_Sample));
@i{18}      vout = (WFDB_Sample *)malloc(nsig * sizeof(WFDB_Sample));
@i{19}      if (s == NULL || vin == NULL || vout == NULL) @{
@i{20}          fprintf(stderr, "insufficient memory\n");
@i{21}          exit(3);
@i{22}      @}
@i{23}      if (isigopen(argv[1], s, nsig) != nsig) exit(2);
@i{24}      if (osigopen("8l", s, nsig) <= 0) exit(3);
@i{25}      while (nsamp-- > 0 && getvec(vin) > 0) @{
@i{26}          for (i = 0; i < nsig; i++)
@i{27}              vout[i] -= vin[i];
@i{28}          if (putvec(vout) < 0) break;
@i{29}          for (i = 0; i < nsig; i++)
@i{30}              vout[i] = vin[i];
@i{31}      @}
@i{32}      (void)newheader("dif");
@i{33}      wfdbquit();
@i{34}      exit(0);
@i{35}  @}
@end example

@noindent
(See @uref{http://www.physionet.org/physiotools/wfdb/examples/example6.c}
for a copy of this program.)

@noindent
@strong{Notes:}

@table @emph
@item Line 24:
Here we attempt to open as many output signals as there are input
signals; if we cannot do so, the program exits after @code{osigopen}
prints an error message.

@item Line 25:
The main loop of the program begins here.  If 1000 samples can be read
from each signal, the loop will end normally; if @code{getvec} fails
before 1000 samples have been read, the loop ends prematurely.

@item Lines 26--27:
For each signal, we compute the negated first difference by subtracting
the new sample from the previous sample.

@item Line 28:
One sample of each output signal is written here.

@item Lines 29--30:
The new input samples are copied into the output sample vector in preparation
for the next iteration.

@item Line 32:
This step is optional.  It creates a header file for a new record
to be called @file{dif}, which we can then open with another program if
we want to read the signals that this program has written.  Since the
@var{record} argument for @code{osigopen} was @file{8l}, we can also
read these files using record @file{8l}; one reason for making a new
@file{hea} file here is that the @file{hea} file for @file{8l} may
not necessarily indicate the proper sampling frequency for these
signals.

@item Line 33:
Since the program writes output signals, it must invoke @code{wfdbquit} to
close the files properly.
@end table

@node     Example 7, Example 8, Example 6, Examples
@unnumberedsec Example 7: A General-Purpose FIR Filter

@cindex digital filter
@cindex filter (digital)
This program illustrates the use of @code{sample} to obtain random
access to signals, a technique that is particularly
useful for implementing digital filters.  The first argument is the
record name, the second and third arguments are the start time and the
duration of the segment to be filtered, and the rest of the arguments
are finite-impulse-response (FIR) filter coefficients.  For example, if
this program were compiled into an executable program called
@samp{filter}, it might be used by
@example
filter 100 5:0 20 .2 .2 .2 .2 .2
@end example
@noindent
which would apply a five-point moving average (rectangular window) filter
to 20 seconds of record @file{100}, beginning 5 minutes into the record.
The output of the program is readable as record @file{out}, for which a
header file is created in the current directory.

@example
 @i{1}  #include <stdio.h>
 @i{2}  #include <wfdb/wfdb.h>
 @i{3}  
 @i{4}  main(argc, argv)
 @i{5}  int argc;
 @i{6}  char *argv[];
 @i{7}  @{
 @i{8}      double *c, one = 1.0, vv, atof();
 @i{9}      int i, j, nc = argc - 4, nsig;
@i{10}      WFDB_Time nsamp, t, t0, t1;
@i{11}      static WFDB_Sample *v;
@i{12}      static WFDB_Siginfo *s;
@i{13}  
@i{14}      if (argc < 4) @{
@i{15}          fprintf(stderr,
@i{16}            "usage: %s record start duration [ coefficients ... ]\n",
@i{17}                  argv[0]);
@i{18}          exit(1);
@i{19}      @}
@i{20}      if (nc < 1) @{
@i{21}          nc = 1; c = &one;
@i{22}      @}
@i{23}      else if ((c = (double *)calloc(nc, sizeof(double))) == NULL) @{
@i{24}          fprintf(stderr, "%s: too many coefficients\n", argv[0]);
@i{25}          exit(2);
@i{26}      @}
@i{27}      for (i = 0; i < nc; i++)
@i{28}          c[i] = atof(argv[i+4]);
@i{29}      if ((nsig = isigopen(argv[1], NULL, 0)) < 1)
@i{30}          exit(3);
@i{31}      s = (WFDB_Siginfo *)malloc(nsig * sizeof(WFDB_Siginfo));
@i{32}      v = (WFDB_Sample *)malloc(nsig * sizeof(WFDB_Sample));
@i{33}      if (s == NULL || v == NULL) @{
@i{34}          fprintf(stderr, "insufficient memory\n");
@i{35}          exit(4);
@i{36}      @}
@i{37}      if (isigopen(argv[1], s, nsig) != nsig)
@i{38}          exit(5);
@i{39}      t0 = strtim(argv[2]);
@i{40}      if (t0 < (WFDB_Time)0) t0 = -t0;
@i{41}      (void)sample(0, t0);
@i{42}      if (!sample_valid()) @{
@i{43}          fprintf(stderr, "%s: inappropriate value for start time\n",
@i{44}                 argv[0]);
@i{45}          exit(6);
@i{46}      @}
@i{47}      if ((nsamp = strtim(argv[3])) < 1) @{
@i{48}          fprintf(stderr, "%s: inappropriate value for duration\n",
@i{49}                  argv[0]);
@i{50}          exit(7);
@i{51}      @}
@i{52}      t1 = t0 + nsamp;
@i{53}      if (osigopen("16l", s, nsig) != nsig)
@i{54}          exit(8);
@i{55}  
@i{56}      for (t = t0; t < t1 && sample_valid(); t++) @{
@i{57}          for (j = 0; j < nsig; j++) @{
@i{58}              for (i = 0, vv = 0.; i < nc; i++)
@i{59}                  if (c[i] != 0.) vv += c[i]*sample(j, t+i);
@i{60}              v[j] = (WFDB_Sample)vv;
@i{61}          @}
@i{62}          if (putvec(v) < 0) break;
@i{63}      @}
@i{64}  
@i{65}      (void)newheader("out");
@i{66}      wfdbquit();
@i{67}      exit(0);
@i{68}  @}
@end example

@noindent
(See @uref{http://www.physionet.org/physiotools/wfdb/examples/example7.c}
for a copy of this program.)

@noindent
@strong{Notes:}

@table @emph
@item Lines 20--22:
If no coefficients are provided on the command line, the program will
simply copy the selected segment of the input signals.

@item Lines 23--28:
If there are more coefficients than there are samples in the circular
buffer, or if memory cannot be allocated for the coefficient vector, the
program cannot work properly, so it exits with an error message.  In
lines 27 and 28, the ASCII strings that represent the coefficients are
converted to @code{double} format and stored in the coefficient
vector.

@item Lines 29--40:
The record name is @code{argv[1]}, and the start time is @code{argv[2]};
if the record can't be opened, the program exits.  See the previous example
for details on how @code{isigopen} is used. If the user provides an absolute
start time
@ifnotinfo
(@pxref{timstr and strtim, , @code{timstr} and @code{strtim}}),
@end ifnotinfo
@ifinfo
(@pxref{timstr and strtim}),
@end ifinfo
the negative value returned by @code{strtim} is converted to a sample number
in line 40.

@item Lines 41--46:
Here, @code{sample} is invoked only for its side effect;  if any samples can
be read from the specified record beginning at sample number @code{t0}, then
@code{sample(0, 0L)} returns a valid sample, so that the value returned by
@code{sample_valid} is true (1).  If not, the program exits.

@item Lines 47--52:
The @var{duration} argument should be a time interval in @var{HH:MM:SS}
format; @code{strtim} converts it to the appropriate number of
samples, and @code{t1} is set to the calculated end time in line 52.

@item Lines 53--54:
The output signals will be written to files in the current directory
according to the specifications for record @file{16l}
(@pxref{Piped and Local Records}).  If we can't write as many output
signals as there are input signals, the program exits.

@item Lines 56--63:
Here's where the work is done.  The outer loop is executed once per
sample vector, the middle loop once per signal, and the inner loop
once per coefficient.  In line 59, we retrieve an input sample,
multiply it by a filter coefficient, and add it to a running sum.
The sum (@code{vv}) is initialized to zero in line 58 before we begin, and is
converted to a @code{WFDB_Sample} in line 60 when we are finished.  Once
an entire output sample vector ia ready, it is written in line 62.
The entire process is repeated until we reach input sample number @code{t1},
or we run out of input samples.

@item Line 65:
The program creates a header file for record @file{out}, using
the signal specifications from record @file{16l} and the sampling
frequency from the input record.
@end table
@node     Example 8, Example 9, Example 7, Examples
@unnumberedsec Example 8: Creating a New Database Record
@cindex creating a record
@cindex records (creating)

This program creates a new record from scratch.  It asks the user for
information about the signals to be sampled, then records them, and
finally creates a @file{hea} file for the new record.  Details of
data acquisition are hardware-dependent and are not shown here. 

@example
  @i{1}  #include <stdio.h>
  @i{2}  #include <wfdb/wfdb.h>
  @i{3}  
  @i{4}  main()
  @i{5}  @{
  @i{6}      char answer[32], record[8], directory[32];
  @i{7}      int i, nsig = 0;
  @i{8}      long nsamp, t;
  @i{9}      double freq = 0.;
 @i{10}      char **filename, **description, **units;
 @i{11}      WFDB_Sample *v;
 @i{12}      WFDB_Siginfo *s;
 @i{13}  
 @i{14}      do @{
 @i{15}          printf("Choose a record name [up to 6 characters]: ");
 @i{16}          fgets(record, 8, stdin); record[strlen(record)-1] = '\0';
 @i{17}      @} while (newheader(record) < 0);
 @i{18}      do @{
 @i{19}          printf("Number of signals to be recorded [>0]: ");
 @i{20}          fgets(answer, 32, stdin); sscanf(answer, "%d", &nsig);
 @i{21}      @} while (nsig < 1);
 @i{22}      s = (WFDB_Siginfo *)malloc(nsig * sizeof(WFDB_Siginfo));
 @i{23}      v = (WFDB_Sample *)malloc(nsig * sizeof(WFDB_Sample));
 @i{24}      filename = (char **)malloc(nsig * sizeof(char *));
 @i{25}      description = (char **)malloc(nsig * sizeof(char *));
 @i{26}      units = (char **)malloc(nsig * sizeof(char *));
 @i{27}      if (s == NULL || v == NULL || filename == NULL ||
 @i{28}          description == NULL || units == NULL) @{
 @i{29}          fprintf(stderr, "insufficient memory\n");
 @i{30}          exit(1);
 @i{31}      @}
 @i{32}      for (i = 0; i < nsig; i++) @{
 @i{33}          if ((filename[i] = (char *)malloc(32)) == NULL ||
 @i{34}              (description[i] = (char *)malloc(32)) == NULL ||
 @i{35}              (units[i] = (char *)malloc(32)) == NULL) @{
 @i{36}              fprintf(stderr, "insufficient memory\n");
 @i{37}              exit(1);
 @i{38}          @}
 @i{39}      @}
 @i{40}      do @{
 @i{41}          printf("Sampling frequency [Hz per signal, > 0]: ");
 @i{42}          fgets(answer, 32, stdin); sscanf(answer, "%lf", &freq);
 @i{43}      @} while (setsampfreq(freq) < 0);
 @i{44}      do @{
 @i{45}          printf("Length of record (H:M:S): ");
 @i{46}          fgets(answer, 32, stdin);
 @i{47}      @} while ((nsamp = strtim(answer)) < 1L);
 @i{48}      printf("Directory for signal files [up to 30 characters]: ");
 @i{49}      fgets(directory, 32, stdin);
 @i{50}      directory[strlen(directory)-1] = '\0';
 @i{51}      printf("Save signals in difference format? [y/n]: ");
 @i{52}      fgets(answer, 32, stdin);
 @i{53}      s[0].fmt = (answer[0] == 'y') ? 8 : 16;
 @i{54}      printf("Save all signals in one file? [y/n]: ");
 @i{55}      fgets(answer, 32, stdin);
 @i{56}      if (answer[0] == 'y') @{
 @i{57}          sprintf(filename[0], "%s/d.%s", directory, record);
 @i{58}          for (i = 0; i < nsig; i++) @{
 @i{59}               s[i].fname = filename[0];
 @i{60}               s[i].group = 0;
 @i{61}          @}
 @i{62}      @}
 @i{63}      else @{
 @i{64}          for (i = 0; i < nsig; i++) @{
 @i{65}               sprintf(filename[i], "%s/d%d.%s", directory,i,record);
 @i{66}               s[i].fname = filename[i];
 @i{67}               s[i].group = i;
 @i{68}          @}
 @i{69}      @}
 @i{70}      for (i = 0; i < nsig; i++) @{
 @i{71}          s[i].fmt = s[0].fmt; s[i].bsize = 0;
 @i{72}          printf("Signal %d description [up to 30 characters]: ", i);
 @i{73}          fgets(description[i], 32, stdin);
 @i{74}          description[i][strlen(description[i])-1] = '\0';
 @i{75}          s[i].desc = description[i];
 @i{76}          printf("Signal %d units [up to 20 characters]: ", i);
 @i{77}          fgets(units[i], 22, stdin);
 @i{78}          units[i][strlen(units[i])-1] = '\0';
 @i{79}          s[i].units = (*units[i]) ? units[i] : "mV";
 @i{80}          do @{
 @i{81}              printf(" Signal %d gain [adu/%s]: ", i, s[i].units);
 @i{82}              fgets(answer, 32, stdin);
 @i{83}              sscanf(answer, "%lf", &s[i].gain);
 @i{84}          @} while (s[i].gain < 0.);
 @i{85}          do @{
 @i{86}              printf(" Signal %d ADC resolution in bits [8-16]: ",i);
 @i{87}              fgets(answer, 32, stdin);
 @i{88}              sscanf(answer, "%d", &s[i].adcres);
 @i{89}          @} while (s[i].adcres < 8 || s[i].adcres > 16);
 @i{90}          printf(" Signal %d ADC zero level [adu]: ", i);
 @i{91}          fgets(answer, 32, stdin);
 @i{92}          sscanf(answer, "%d", &s[i].adczero);
 @i{93}      @}
 @i{94}      if (osigfopen(s, nsig) < nsig) exit(1);
 @i{95}      printf("To begin sampling, press RETURN;  to specify a\n");
 @i{96}      printf(" start time other than the current time, enter\n");
 @i{97}      printf(" it in H:M:S format before pressing RETURN: ");
 @i{98}      fgets(answer, 32, stdin); answer[strlen(answer)-1] = '\0';
 @i{99}      setbasetime(answer);
@i{100}  
@i{101}      adinit();
@i{102}  
@i{103}      for (t = 0; t < nsamp; t++) @{
@i{104}          for (i = 0; i < nsig; i++)
@i{105}              v[i] = adget(i);
@i{106}          if (putvec(v) < 0) break;
@i{107}      @}
@i{108}  
@i{109}      adquit();
@i{110}  
@i{111}      (void)newheader(record);
@i{112}      wfdbquit();
@i{113}      exit(0);
@i{114}  @}
@end example

@noindent
(See @uref{http://www.physionet.org/physiotools/wfdb/examples/example8.c}
for a copy of this program.)

@noindent
@strong{Notes:}
@table @emph
@item Lines 14--17:
This code uses @code{newheader} to determine if a legal record name was
entered (since we don't want to digitize the signals and then find out
that we can't create the header file).  The header file
created in line 17 will be overwritten in line 111.

@item Lines 57--62:
This code generates a file name and initializes the @code{fname} and
@code{group} fields of the array of @code{WFDB_Siginfo} objects so that all
signals will be saved in one file.

@item Lines 63--69:
This code generates unique file names and groups for each signal.

@item Lines 70--93:
Here, information specific to individual signals is gathered.

@item Line 94:
If the signal files can't be created, this program can do nothing else
useful, so it quits with an error message from @code{osigfopen}.

@item Lines 95--99:
Just before sampling begins, we set the base time.  Note that an empty
string argument for @code{setbasetime} gives us the current time read from
the system clock.

@item Line 101:
What goes here will be hardware dependent.  Typically it is necessary to
set up a timer for the ADC, allocate DMA buffers, specify interrupt vectors,
and initiate the first conversion(s).  This program might also be used to
create a database record from prerecorded data in a non-supported format;
in this case, we might simply open the file containing the prerecorded data
here.

@item Lines 103--107:
Here is where the samples are acquired (using hardware-dependent code
not shown here) and recorded (using @code{putvec}).  At high sampling
frequencies, it is critical to make this code as fast as possible.  It
could be made faster by judicious use of @code{register} and pointer
variables if necessary.  In an extreme case the entire loop, possibly
including @code{putvec} itself, can be written in assembly language;
since it is only a small fraction of the entire program, doing so is
within reason.

@item Line 109:
This final piece of hardware-dependent code typically clears the ADC
control register, stops the timer, and frees any system resources such as
DMA channels or interrupts.

@item Line 111:
All of the information needed to generate the header file has been
stored in WFDB library internal data structures by @code{osigfopen} and 
@code{putvec};  we call @code{newheader} here (before @code{wfdbquit}) to
create the new @file{hea} file.

@item Line 112:
It is still necessary to use @code{wfdbquit} to close the signal file(s),
even after calling @code{newheader}.  (In fact, it would be possible,
though not likely to be useful, to record more samples and to generate
another header file before calling @code{wfdbquit}.)
@end table

@node     Example 9, Example 10, Example 8, Examples
@unnumberedsec Example 9: A Signal Averager

The following program is considerably more complex than the previous examples
in this chapter.  It reads an annotation file (for which the annotator name
is specified in its first argument, and the record name in the second
argument) and selects beats of a specified type to be averaged.  The
program selects segments of the signals that are within 50 milliseconds of the
time of the specified beat annotations, subtracts a baseline estimate
from each sample, and calculates an average waveform (by default, the
average normal QRS complex).

@example
 @i{1}  #include <stdio.h>
 @i{2}  #include <wfdb/wfdb.h>
 @i{3}  #include <wfdb/ecgmap.h>
 @i{4}  
 @i{5}  main(argc, argv)
 @i{6}  int argc;
 @i{7}  char *argv[];
 @i{8}  @{
 @i{9}      int btype, i, j, nbeats = 0, nsig, hwindow, window;
@i{10}      long stoptime = 0L, **sum;
@i{11}      WFDB_Anninfo a;
@i{12}      WFDB_Annotation annot;
@i{13}      WFDB_Sample *v, *vb;
@i{14}      WFDB_Siginfo *s;
@i{15}      void *calloc();
@i{16}  
@i{17}      if (argc < 3) @{
@i{18}          fprintf(stderr,
@i{19}                  "usage: %s annotator record [beat-type from to]\n",
@i{20}                  argv[0]);
@i{21}          exit(1);
@i{22}      @}
@i{23}      a.name = argv[1]; a.stat = WFDB_READ;
@i{24}      if ((nsig = isigopen(argv[2], NULL, 0)) < 1) exit(2);
@i{25}      s = (WFDB_Siginfo *)malloc(nsig * sizeof(WFDB_Siginfo));
@i{26}      v = (WFDB_Sample *)malloc(nsig * sizeof(WFDB_Sample));
@i{27}      vb = (WFDB_Sample *)malloc(nsig * sizeof(WFDB_Sample));
@i{28}      sum = (long **)malloc(nsig * sizeof(long *));
@i{29}      if (s == NULL || v == NULL || vb == NULL || sum == NULL) @{
@i{30}          fprintf(stderr, "%s: insufficient memory\n", argv[0]);
@i{31}          exit(2);
@i{32}      @}
@i{33}      if (wfdbinit(argv[2], &a, 1, s, nsig) != nsig) exit(3);
@i{34}      hwindow = strtim(".05"); window = 2*hwindow + 1;
@i{35}      for (i = 0; i < nsig; i++)
@i{36}          if ((sum[i]=(long *)calloc(window,sizeof(long))) == NULL) @{
@i{37}              fprintf(stderr, "%s: insufficient memory\n", argv[0]);
@i{38}              exit(2);
@i{39}          @}
@i{40}      btype = (argc > 3) ? strann(argv[3]) : NORMAL;
@i{41}      if (argc > 4) iannsettime(strtim(argv[4]));
@i{42}      if (argc > 5) @{
@i{43}          if ((stoptime = strtim(argv[5])) < 0L)
@i{44}              stoptime = -stoptime;
@i{45}          if (s[0].nsamp > 0L && stoptime > s[0].nsamp)
@i{46}              stoptime = s[0].nsamp;
@i{47}      @}
@i{48}      else stoptime = s[0].nsamp;
@i{49}      if (stoptime > 0L) stoptime -= hwindow;
@i{50}      while (getann(0, &annot) == 0 && annot.time < hwindow)
@i{51}          ;
@i{52}      do @{
@i{53}          if (annot.anntyp != btype) continue;
@i{54}          isigsettime(annot.time - hwindow - 1);
@i{55}          getvec(vb);
@i{56}          for (j = 0; j < window && getvec(v) > 0; j++)
@i{57}              for (i = 0; i < nsig; i++)
@i{58}                  sum[i][j] += v[i] - vb[i];
@i{59}          nbeats++;
@i{60}      @} while (getann(0, &annot) == 0 &&
@i{61}               (stoptime == 0L || annot.time < stoptime));
@i{62}      if (nbeats < 1) @{
@i{63}          fprintf(stderr, "%s: no `%s' beats found\n",
@i{64}                  argv[0], annstr(btype));
@i{65}          exit(4);
@i{66}      @}
@i{67}      printf("Average of %d `%s' beats:\n", nbeats, annstr(btype));
@i{68}      for (j = 0; j < window; j++)
@i{69}          for (i = 0; i < nsig; i++)
@i{70}              printf("%g%c", (double)sum[i][j]/nbeats,
@i{71}                     (i == nsig-1) ? '\n' : '\t');
@i{72}      exit(0);
@i{73}  @}
@end example

@noindent
(See @uref{http://www.physionet.org/physiotools/wfdb/examples/example9.c}
for a copy of this program.)

@noindent
@strong{Notes:}

@table @emph
@item Line 34:
The ``half-window'' is 50 milliseconds wide, and the ``window'' (the
duration of a segment to be entered into the average) is one sample more
than twice that amount (i.e., 50 milliseconds to either side of the
fiducial point defined by the annotation).

@item Lines 35--39:
Here we allocate memory for the @code{sum} vectors that will be used to
store the running totals.  See @cite{K&R}, page 167, for a description
of @code{calloc}.

@item Line 40:
If a third argument is present on the command line, it is taken as an
annotation code mnemonic for the desired beat type;  otherwise, the
program will average @code{NORMAL} QRS complexes.

@item Line 41:
If a fourth argument is present on the command line, it is taken as the
start time; we arrange for the first annotation to be read by
@code{getann} to be the first annotation that occurs after the chosen
start time.

@item Lines 42--49:
This code similarly determines when the averaging should stop.  Unless
no stop time was specified on the command line and the signal length is
not defined in the @file{hea} file for the record, @code{stoptime}
will have a positive value in line 49, which makes a tiny adjustment so
that if a beat annotation occurs within 50 milliseconds of the end of the
averaging period, the beat will not be included in the average.

@item Lines 50-51:
This code addresses the (admittedly unlikely) prospect that the first
annotation(s) might occur within the first 50 milliseconds of the
record; any such annotations will be excluded from the average.

@item Lines 52--61:
Here we read annotations (the first is already in @code{annot} when we
enter the loop, and subsequent annotations are read in line 60); select
the desired ones (line 53); skip to the correct spot in the signals (line
54; the sample selected there is the one just before the beginning of
the window); read a sample from each signal (line 55) into the @code{vb}
vector, which will be used as a crude baseline estimate; read
@code{window} samples from each signal (line 56), subtracting the
baseline from each and adding the result into the running totals; update
a beat counter (line 59); and check for loop termination conditions
(line 61).

@item Lines 62--71:
This is the output section.  If no beats of type @code{btype} were
found, obviously no average can be printed; note that the message goes
to the standard error output, so the user will notice it even if the
standard output has been redirected to a file.  In the usual case, the
averages are printed out as a table, with a column allocated to each
signal.  Note the cast in line 70 (necessary to preserve precision), and
the trick used in line 71 to print a tab after each column but the last
in each line.
@end table

@node     Example 10, , Example 9, Examples
@unnumberedsec Example 10: A QRS Detector

@cindex QRS detector
@cindex detector (QRS)
@cindex digital filter
@cindex filter (digital)
This program reads a single ECG signal, attempts to detect QRS complexes,
and records their locations in an annotation file.  The detector algorithm
is based on a Pascal program written by W.A.H. Engelse and C. Zeelenberg,
``A single scan algorithm for QRS-detection and feature extraction'',
@cite{Computers in Cardiology} @b{6}:37-42 (1979).

@example
 @i{1}  #include <stdio.h>
 @i{2}  #include <wfdb/wfdb.h>
 @i{3}  #include <wfdb/ecgcodes.h>
 @i{4}  
 @i{5}  #define abs(A)  ((A) >= 0 ? (A) : -(A))
 @i{6}  
 @i{7}  main(argc, argv)
 @i{8}  int argc;
 @i{9}  char *argv[];
@i{10}  @{
@i{11}      int filter, time=0, slopecrit, sign, maxslope=0, nsig, nslope=0,
@i{12}          qtime, maxtime, t0, t1, t2, t3, t4, t5, t6, t7, t8, t9,
@i{13}          ms160, ms200, s2, scmax, scmin = 0;
@i{14}      WFDB_Anninfo a;
@i{15}      WFDB_Annotation annot;
@i{16}      WFDB_Sample *v;
@i{17}      WFDB_Siginfo *s;
@i{18}  
@i{19}      if (argc < 2) @{
@i{20}          fprintf(stderr, "usage: %s record [threshold]\n", argv[0]);
@i{21}          exit(1);
@i{22}      @}
@i{23}      a.name = "qrs"; a.stat = WFDB_WRITE;
@i{24}      if ((nsig = isigopen(argv[1], NULL, 0)) < 1) exit(2);
@i{25}      s = (WFDB_Siginfo *)malloc(nsig * sizeof(WFDB_Siginfo));
@i{26}      v = (WFDB_Sample *)malloc(nsig * sizeof(WFDB_Sample));
@i{27}      if (s == NULL || v == NULL) @{
@i{28}          fprintf(stderr, "%s: insufficient memory\n", argv[0]);
@i{29}          exit(2);
@i{30}      @}
@i{31}      if (isigopen(argv[1], s, nsig) != nsig) exit(2);
@i{32}      if (sampfreq((char *)NULL) < 240. ||
@i{33}          sampfreq((char *)NULL) > 260.)
@i{34}          setifreq(250.);
@i{35}      if (annopen(argv[1], &a, 1) < 0) exit(2);
@i{36}      if (argc > 2) scmin = muvadu(0, atoi(argv[2]));
@i{37}      if (scmin < 1) scmin = muvadu(0, 1000);
@i{38}      slopecrit = scmax = 10 * scmin;
@i{39}      ms160 = strtim("0.16"); ms200 = strtim("0.2"); s2 = strtim("2");
@i{40}      annot.subtyp = annot.chan = annot.num = 0; annot.aux = NULL;
@i{41}      (void)getvec(v);
@i{42}      t9 = t8 = t7 = t6 = t5 = t4 = t3 = t2 = t1 = v[0];
@i{43}  
@i{44}      do @{
@i{45}          filter = (t0 = v[0]) + 4*t1 + 6*t2 + 4*t3 + t4
@i{46}                  - t5         - 4*t6 - 6*t7 - 4*t8 - t9;
@i{47}          if (time % s2 == 0) @{
@i{48}              if (nslope == 0) @{
@i{49}                  slopecrit -= slopecrit >> 4;
@i{50}                  if (slopecrit < scmin) slopecrit = scmin;
@i{51}              @}
@i{52}              else if (nslope >= 5) @{
@i{53}                  slopecrit += slopecrit >> 4;
@i{54}                  if (slopecrit > scmax) slopecrit = scmax;
@i{55}              @}
@i{56}          @}
@i{57}          if (nslope == 0 && abs(filter) > slopecrit) @{
@i{58}              nslope = 1; maxtime = ms160;
@i{59}              sign = (filter > 0) ? 1 : -1;
@i{60}              qtime = time;
@i{61}          @}
@i{62}          if (nslope != 0) @{
@i{63}              if (filter * sign < -slopecrit) @{
@i{64}                  sign = -sign;
@i{65}                  maxtime = (++nslope > 4) ? ms200 : ms160;
@i{66}              @}
@i{67}              else if (filter * sign > slopecrit &&
@i{68}                       abs(filter) > maxslope)
@i{69}                  maxslope = abs(filter);
@i{70}              if (maxtime-- < 0) @{
@i{71}                  if (2 <= nslope && nslope <= 4) @{
@i{72}                      slopecrit += ((maxslope>>2) - slopecrit) >> 3;
@i{73}                      if (slopecrit < scmin) slopecrit = scmin;
@i{74}                      else if (slopecrit > scmax) slopecrit = scmax;
@i{75}                      annot.time = strtim("i") - (time - qtime) - 4;
@i{76}                      annot.anntyp = NORMAL; (void)putann(0, &annot);
@i{77}                      time = 0;
@i{78}                  @}
@i{79}                  else if (nslope >= 5) @{
@i{80}                      annot.time = strtim("i") - (time - qtime) - 4;
@i{81}                      annot.anntyp = ARFCT; (void)putann(0, &annot);
@i{82}                  @}
@i{83}                  nslope = 0;
@i{84}              @}
@i{85}          @}
@i{86}          t9 = t8; t8 = t7; t7 = t6; t6 = t5; t5 = t4;
@i{87}          t4 = t3; t3 = t2; t2 = t1; t1 = t0; time++;
@i{88}      @} while (getvec(v) > 0);
@i{89}  
@i{90}      wfdbquit();
@i{91}      exit(0);
@i{92}  @}
@end example

@noindent
(See @uref{http://www.physionet.org/physiotools/wfdb/examples/example10.c}
for a copy of this program.)

@noindent
@strong{Notes:}

@table @emph
@item Line 5:
A macro that evaluates to the absolute value of its argument.

@item Lines 11--12:
The names of these variables match those in the original Pascal program.

@item Lines 32--35:
Most of this program is independent of sampling frequency, but the filter
(lines 45--46) and the threshold are as specified by the authors
of the original program for human ECGs sampled at 250 Hz (e.g., the AHA DB).
If the sampling frequency of the input record is significantly
different, we use @code{setifreq} to specify that we want @code{getvec}
to give us data resampled at 250 Hz.  The output annotation file is
created in line 35 only after invoking @code{setifreq} if necessary.

@item Lines 36--38:
The threshold is actually a slope criterion (with units of amplitude/time);
these lines normalize the threshold with respect to the signal gain.
The default value is used unless the user supplies an acceptable
alternative.  The variables @code{scmin} and @code{scmax} are lower and
upper bounds for the adaptive threshold @code{slopecrit}.

@item Lines 41--42:
Here we read the first sample and copy it into the variables that will
be used to store the ten most recent samples.

@cindex digital filter
@cindex filter (digital)
@item Lines 45--46:
This FIR filter differentiates and low-pass filters the input signal.

@item Lines 47--56:
Here we adjust the threshold if more than two seconds have elapsed since
a QRS was detected.  In line 49, @code{slopecrit} is set to 15/16 of its
previous value if no slopes have been found;  in line 53, it is set to
17/16 of its previous value if 5 or more slopes were found (suggesting
the presence of noise).

@item Lines 57--61:
If the condition in line 48 is satisfied, we may have found the beginning
of a QRS complex.  We record that a slope has been found, set the timer
@code{maxtime} to 160 msec, and save the sign of the slope and the current
time relative to the previous beat.

@item Lines 62--85:
This code is executed once we have found a slope.  Each time the filter
output crosses the threshold, we record another slope and begin looking
for a threshold crossing of the opposite sign (lines 63--66), which must
occur within a specified time.  We record the maximum absolute value of
the filter in @code{maxslope} (lines 67--69) for eventual use in updating
the threshold (lines 72--74).  Once a sufficient interval has elapsed
following the last threshold crossing (line 70), if there were between
2 and 4 slopes, we have (apparently) found a QRS complex, and the program
records a @code{NORMAL} annotation (lines 75--76).  If there were 5 or more
slopes, the program records an artifact annotation (lines 80--81).  If
only 1 slope was found, it is assumed to represent a baseline shift and
no output is produced.

@item Lines 86--88:
At the end of the loop, the samples are shifted through the @code{t@var{n}}
variables and another sample is read.
@end table

@node     Exercises, Glossary, Examples, Top
@unnumbered Exercises

These exercises are based on the material in the previous chapters.
Answers to some of them are at the back of the book, but try to work
through them first.

@enumerate
@item
Type in the first program from the previous chapter, compile it, and run
it.  If you know that you will need to read WFDB files from non-standard
locations, remember to set and export the environment variable @code{WFDB}
(@pxref{WFDB path}).  It is a good idea to include this step in your
@file{.profile}, @file{.cshrc}, or @file{autoexec.bat}.  As input, try
record @file{100s}, input annotator @file{atr}, and output annotator
@file{normal}.  The program should finish in five seconds or less.
The annotations will have been written into a file called
@file{100s.nor} in the current directory.  Now type
``@code{rdann -r 100s -a atr}'' and observe the output
for a few seconds, then try ``@code{rdann -r 100s -a nor}'' and notice the
difference.

@item
Modify the program from the previous exercise so that the non-QRS
annotations are put into a second output annotation file.  Remember that
you will need three annotation files in all (one input and two
output).

@item
The next five short exercises are to be worked out on paper, although you
may wish to check your work on the computer.  All of them assume that
we are given a signal sampled at 100 Hz with the following
specifications:
@example
fname = "signal.dat"
desc = "BP"
units = "mmHg"
gain = 10
initval = 80
group = 0
fmt = 212
spf = 1
bsize = 0
adcres = 12
adczero = 0
baseline = -300
nsamp = 1000000
cksum = 3109
@end example
For starters, convert a sample value of 280 into physical units.

@item
Convert 120 mmHg into adus.

@item
What are the maximum and minimum possible sample values in adu?  in
mmHg?

@item
How large is @file{signal.dat}, in bytes?  How much space could we save
if we converted it to format 8 (eight-bit first-differences)?  What is
the maximum slew rate (in mmHg/second) that we can represent in that
format?

@item
Oops!  We have just discovered that the maximum slew rate in our signal
is 1500 mmHg/sec.  Is there any way to store it at full precision in one
of the supported formats, that saves space compared to its present
format?

@item
Figure out how to plot or display the first 1000 points from signal 0 of
a record in amplitude vs. time format.  You may wish to begin with the
example program from the first chapter.  Arrange for the record name to
be read from the command line (see @cite{K&R}, pp. 114--115, if you
don't know how to do this).

@item
Try plotting VCGs by modifying the program from the previous exercise to
plot pairs of samples from each of two signals rather than sample
number/value pairs.

@item
Modify the program from the previous exercise, or Example 2 from the
previous chapter, so that you can specify a segment of the record to be
processed with start and end times.  For example, the command
@example
@var{your-program record} 10:0 10:10
@end example
should skip the first ten minutes, then process the next ten seconds of
signals from @var{record}.

@item
Using @code{isigsettime} on a format 8 signal introduces a random offset
into the signal, since the contents of a format 8 signal file are first
differences rather than amplitudes.  For an AC-coupled signal such as an
ECG, this is usually inconsequential, but a DC-coupled signal such as a
blood pressure signal is usually useful only if absolute levels are
known.  If we store such a signal in format 8, we must read it
sequentially from the beginning in order to get correct sample values.
If we intend to do a lot of non-sequential processing of such a signal,
it may be worthwhile to build a table containing the correct sample
values at periodic intervals; then we can use @code{isigsettime} to
skip to a sample in the table, and read forward sequentially from that
point.  Write a program to build such a table, and wrappers for
@code{isigsettime} and @code{getvec} to give random access to format 8
signal files without introducing offset errors.  On your system, how
many sample intervals should be allowed between table entries in order
to obtain an @code{isigsettime} equivalent that executes in an average
of 100 msec or less?

@item
This exercise and the next assume that you have access to the web, so
that you can obtain the freely available input records needed from PhysioNet.
Since the 360 Hz sampling frequency used in the MIT-BIH Arrhythmia
Database is an integer multiple of the 60 Hz mains frequency, it is
quite easy to design a 60 Hz notch filter that can be applied to the
database to suppress power-line interference (for example, by
averaging pairs of samples that are 180 degrees out of phase).  Write
a program that filters two input signals and writes out the filtered
data using @code{putvec} (@pxref{Example 7}, for a model program).
Try it out on MIT DB record @file{122} (if you have a NETFILES-enabled
WFDB library, use the default WFDB path, and open record
@file{mitdb/122}; otherwise, download the record from
@uref{http://www.physio@-net.org/physio@-bank/data@-base/mitdb/}.)
Use your programs from the previous exercises to display your output
and compare it with the original signals.

@item
(Non-trivial) Write a QRS detector that is independent of sampling
frequency without using @code{setifreq}.  Some useful constants (for
adult human ECGs): average normal QRS duration = 80 milliseconds,
average QRS amplitude = 1 millivolt, average R-R interval = 1 second;
assume that upper and lower limits for these quantities are within a
factor of 3 of the average values.  Run your detector on MIT-BIH
Arrhythmia Database record @file{200}.  (If you have a
NETFILES-enabled WFDB library, use the default WFDB path, and open
record @file{mitdb/200}; otherwise, download the record from
@uref{http://www.physio@-net.org/physio@-bank/data@-base/mitdb/}.)
Read the documentation on the annotation comparator, @file{bxb}, and
figure out how to use it to compare the annotation file produced by
your program against the reference annotator @file{atr}.  How does
your detector compare to Example 10?
@end enumerate

@node     Glossary, Installation, Exercises, Top
@appendix Glossary

@table @emph
@item AC-coupled signal
@cindex AC-coupled signal (defined)
A signal, such as an ECG, for which only variations in level, rather
than absolute levels, are significant.  Such signals are usually passed
through high-pass filters before they are digitized, in order to remove
any DC component (baseline offset), so that the gain can be chosen
optimally for the range of variation in the signal.

@item ADC
@cindex ADC (defined)
Analog-to-digital converter.

@item ADC resolution
@cindex ADC resolution (defined)
@cindex resolution
The number of significant bits per sample.  Typical ADCs yield between 8
and 16 bits of resolution.

@item ADC zero
@cindex ADC zero (defined)
The value produced by the ADC given a 0 volt input.  For bipolar ADCs,
this value is usually 0, but for the unipolar (offset binary) converter
used for the MIT DB, the ADC zero was 1024.

@item adu
@cindex adu (defined)
The unit of amplitude for samples.

@item AHA DB
@cindex AHA DB (defined)
The American Heart Association Database for the Evaluation of Ventricular
Arrhythmia Detectors, consisting of 80 records identified by four-digit
record names.

@item AHA format
@cindex AHA format (defined)
The format formerly used for interchange of AHA DB and MIT DB records
on 9-track tape between institutions, not used for on-line files
because it is relatively wasteful of storage space compared to other
WFDB-compatible formats.  (q.v.).

@item Annotation
@cindex annotation (defined)
A label, associated with a particular sample, which describes a feature
of the signal at that time.  Most annotations are QRS annotations and
indicate the QRS type (normal, PVC, SVPB, etc.).  Annotations are
written by @code{putann} and read by @code{getann}.

@item Annotation code
@cindex annotation code (defined)
An integer in the range of 1 to @code{ACMAX} (defined in
@file{<wfdb/ecgcodes.h>}) inclusive, which denotes an event type.

@item Annotation file
@cindex annotation file (defined)
A set of annotations in time order.

@item Annotator name
@cindex annotator name (defined)
A name associated with an annotation file.  The annotation file name is
constructed from the record name by appending a @samp{.} and the annotator
name.  On CDROMs and MS-DOS file systems, the annotator name is restricted
to three characters.

@item Annotator [number]
@cindex annotator number (defined)
An integer by which an annotation file, once opened, is known.  Input
annotators and output annotators each have their own series of annotator
numbers assigned in serial order beginning with 0.

@item Application program
In this guide, a program that uses the WFDB library to do
something.

@item @file{atr}
@cindex @code{atr} (defined)
The annotator name for the reference annotation files (originally,
@file{atruth}, i.e., the "truth" annotations).

@item Base counter value
@cindex base counter value (defined)
@cindex counter (base)
The counter value (q.v.) that corresponds to sample 0.  The base
counter value is read by @code{getbasecount}, and set by @code{setbasecount}
(or by any of the functions that read header files).  If not
defined explicitly, the base counter value is taken to be 0.

@item Base time
@cindex base time (defined)
The time of day that corresponds to sample 0 in a given record.  For
MIT, AHA, and ESC DB records, the base time was not recorded and is taken to
be 0:0:0 (midnight).

@item Baseline [amplitude]
@cindex baseline amplitude (defined)
The sample value that corresponds to the baseline (isoelectric level or
physical zero level) in the signal.  This quantity may drift during the
record for a variety of reasons, in which case the @code{baseline} field
of the @code{WFDB_Siginfo} object that describes the signal is only an
approximation.  The baseline is @emph{not} the same as the ADC zero
(q.v.), which is a fixed characteristic of the digitizer.

@item Calibration file
@cindex calibration file (defined)
A file containing data used to build a calibration list (q.v.).

@item Calibration list
@cindex calibration list (defined)
A memory-resident linked list of @code{WFDB_Calinfo} objects
@ifnotinfo
(@pxref{WFDB_Calinfo structures, , Calibration Information Structures}).
@end ifnotinfo
@ifinfo
(@pxref{WFDB_Calinfo structures}).
@end ifinfo
Each such structure specifies the size and type of the calibration pulse,
and the customary plotting scale, for a particular type of signal.

@item CDROM
@cindex CDROM (defined)
A read-only medium used for distribution of a number of databases
readable using the WFDB library, including the ESC and Long-Term ST
databases, among others.  CDROMs are physically identical in
appearance to audio compact disks.

@item Closing [a record]
The process of completing I/O associated with a record.

@item Counter frequency
@cindex counter frequency (defined)
@cindex frequency (counter)
The difference between counter values (q.v.) that are separated
by an interval of one second.  The counter frequency is constant
throughout any given record.  It may be undefined, in which case it is
treated as equivalent to the sampling frequency (q.v.) by the WFDB
library.  The counter frequency is read by @code{getcfreq},
and set by @code{setcfreq} (or by any of the functions that read
header files).

@item Counter value
@cindex counter value (defined)
@cindex tape counter
A number that serves as a time reference, in a record for which a
counter frequency is defined.  A counter value may be converted to the
time in seconds from the beginning of the record by subtracting the base
counter value (q.v.) and dividing the remainder by the counter
frequency.  The units of @samp{c}-prefixed @code{strtim} arguments are
counter values.

@item Database files
Those files (annotation files, header files, signal files, and
calibration files) that are accessed via the WFDB library.

@item Database path
@cindex WFDB path
@cindex database path (defined)
The names of the directories in which header, annotation, and calibration files
are kept.  (Signal files may be located in these directories or elsewhere;
header files specify their locations.)  To modify the database path, the
environment variable @code{WFDB} must be set by the user and exported
accordingly.

@item DC-coupled signal
@cindex DC-coupled signal (defined)
A signal, such as a blood pressure signal, for which absolute levels are
significant.  Such signals must be digitized without being passed
through high-pass filters, in order to preserve absolute levels.

@item ESC DB
@cindex ESC DB (defined)
The European ST-T Database, consisting of 90 records identified by
@samp{e}-prefixed four-digit record names.

@item Frame
@cindex frame (defined)
A set of samples, containing all samples that occur within a given
frame interval.  For an ordinary record, a frame contains exactly one
sample of each signal;  for a multi-frequency record, a frame contains
@emph{at least} one sample of each signal, and more than one sample of
each oversampled signal (q.v.).

@item Frame interval
@cindex frame interval (defined)
A time interval during which at least one sample exists for each signal.
For an ordinary record, the frame interval and the sampling interval are
identical.  For a multi-frequency record, the frame interval is chosen
to be an integer multiple of each sampling frequency used.

@item Frame rate
@cindex frame rate (defined)
The basic sampling frequency defined for a multi-frequency record;  the
reciprocal of the frame interval.  The frame rate is usually the lowest
sampling frequency used for any signal included in the record.

@item Gain
@cindex gain (defined)
In this context, the number of adus (q.v.) per physical unit, referred to the
original analog signal.  Gain in this sense is directly proportional to the
degree of amplification (the usual meaning of the word) of the analog
signal prior to digitization.  Gain may vary between signals in a
record.

@item @file{hea}
@cindex @code{hea} (defined)
The suffix (extension) that designates a header file (originally
@file{header}).

@item header file
@cindex header file (defined)
A file accessible via the WFDB library that describes the signal files
associated with a given database record.  A header file has a name of
the form `@var{record}@code{.hea}', where @var{record} is the record name
(q.v.).

@item High-resolution mode
@cindex high-resolution mode (defined)
An alternative mode for reading a multi-frequency record using
@code{getvec}, that can be selected using @code{setgvmode}.  In
high-resolution mode, @code{getvec} replicates samples of signals
digitized at less than the maximum sampling frequency, so that each
sample of any oversampled signals appear in at least one sample vector.

@item Info string
@cindex info string (defined)
Free text within a header file.  Info strings can be read using
@code{getinfo} and written using @code{putinfo}.

@item Local record
@cindex local record (defined)
A record for which the signal files reside in the current directory,
typically used for user-created signals.  Records @file{8l} and
@file{16l} are local records.

@item Location [of an annotation]
@cindex location (of annotations)
@cindex annotation location (defined)
Every annotation has @code{time}, @code{num}, and @code{chan} attributes that
define its location within a virtual array of annotations.  See
@dfn{Canonical order of annotations}.

@item Canonical order of annotations
@cindex annotation (canonical order)
@cindex canonical order of annotations
Normally, annotations are arranged in time order within an annotation file.
Annotations that have identical @code{time} attributes are arranged in
@code{num} and @code{chan} order.  Annotations that have identical
@dfn{locations} (i.e., identical @code{time}, @code{num}, and @code{chan}
attributes) should not normally occur in a single annotation file; if this does
happen, the last annotation at any given location is treated as a replacement
of any previous annotations at that location.

@item Low-resolution mode
@cindex low-resolution mode (defined)
The default mode for reading a multi-frequency record using
@code{getvec}.  In low-resolution mode, @code{getvec} returns one
sample per signal per frame, by decimating any oversampled signals
to the frame rate.

@item MIT DB
@cindex MIT DB (defined)
The Massachusetts Institute of Technology--Beth Israel Hospital
Arrhythmia Database, consisting of 48 records identified by three-digit
record names.

@item MIT format
@cindex MIT format (defined)
One of a set of WFDB-compatible formats for header, signal, and
annotation files that were first used for the MIT-BIH Arrhythmia
Database.  The term WFDB-compatible format should be used unless referring
specifically to the formats used for the MIT DB (single-segment header format,
signal format 212, and bit-packed annotation format).

@item Modification label
@cindex modification label (defined)
An ``invisible'' annotation at the beginning of an annotation file.  A
modification label defines an annotation mnemonic and a corresponding
description.  When @code{annopen} (or @code{wfdbinit}) opens an annotation
file that contains modification labels, it automatically calls
@code{setannstr} and @code{setanndesc} to add the mnemonics and
descriptions to the translation tables used by @code{annstr},
@code{strann}, and @code{anndesc}.  When @code{annopen} (or
@code{wfdbinit}) creates an annotation file, it automatically generates
modification labels, for each annotation code that has been (re)defined
using @code{setannstr} or @code{setanndesc}.  For this reason, you
should normally make all of your calls to @code{setannstr} and
@code{setanndesc} @emph{before} calling @code{annopen} or @code{wfdbinit}.
(An exception is if you are simply @emph{translating} mnemonics and
descriptions into another language, rather than @emph{redefining} them.)
Version 5.3 and later versions of the WFDB library support reading and
writing modification labels;  earlier versions read modification labels
as @code{NOTE} annotations.

@item Multi-frequency record
@cindex multi-frequency record (defined)
A record containing signals sampled at two or more sampling frequencies.
Version 9.0 and later versions of the WFDB library support reading and
writing multi-frequency records.

@item Multi-segment record
@cindex multi-segment record (defined)
A composite record that is the concatenation of two or more ordinary
(single-segment) records.  Multi-segment records do not have their own
signal files (the signal files of their constituent segments are read
when it is necessary to read signals of multi-segment records), but they
have their own header files (created using @code{setmsheader}),
and may have their own annotation files as well (annotation files for
the constituent segments of a multi-segment record are @emph{not}
concatenated automatically when the record is read).  The WFDB Software
Package includes @code{wfdbcollate}, an
application that can create multi-segment records from sets of
single-segment records.  Version 9.1 and later versions of the WFDB
library support reading and writing multi-segment records.

@item Multiplexed signal file
@cindex multiplexed signal file (defined)
A set of vectors in time order, each consisting of two or more integer
samples, thus representing an equal number of signals.

@item NETFILES
@cindex NETFILES (defined)
@cindex curl
@cindex libcurl
@cindex libwww
@cindex W3C libwww
WFDB files made available by an FTP or HTTP (web) server;  readable
by applications linked with a NETFILES-enabled WFDB library.  A
NETFILES-enabled WFDB library can be created by compiling the WFDB
library sources with the symbol @code{WFDB_NETFILES} defined (to anything;
its value is not important, only that it is defined) and then linking
them with the @code{libwww} library available from the World Wide Web
Consortium (@uref{http://www.w3.org/Library/}) or the @code{libcurl}
library available from @uref{http://curl.haxx.se/}.

@item 9-track tape
@cindex nine-track tape (defined)
A medium used for archival storage of WFDB records, which was once nearly
universally available on minicomputers and larger systems.  The
important parameters are tape density (typically 800 or 1600 bpi) and
block size (typically some multiple of 512 bytes).  Higher tape density
and larger block size permit more data to be stored on a tape.

@item Opening [a database record or a file]
The process of making a database record or a file accessible, if
necessary by creating it.

@item Oversampled signal
@cindex oversampled signal (defined)
In a multi-frequency record, any signal recorded at a sampling frequency
greater than the frame rate (q.v.).

@item Physical unit
@cindex physical unit (defined)
The natural unit of measurement of the original analog signal (e.g.,
millivolts, liters per second, degrees).  To convert samples into
physical units, subtract the ADC zero and divide the remainder by the
gain.

@item Physical zero
@cindex physical zero (defined)
The level (in physical units) that corresponds to the baseline (in adu),
normally zero physical units.  For example, physical zero for a pressure
signal with units of mmHg is 0 mmHg.

@item PhysioNet
@cindex PhysioNet
The home of the WFDB library, and a source for recorded physiologic signals and
software for use with them.  All materials on PhysioNet are freely available.
The main PhysioNet server is @uref{http://www.@-physio@-net.@-org/}, located at
MIT in Cambridge, Massachusetts; PhysioNet mirror sites are located around the
world (see @uref{http://www.@-physio@-net.@-org/@-mirrors/} for a list).

@item Piped record
@cindex piped record (defined)
A database record for which a signal file is designated as @file{-},
signifying that it is to be read from the standard input or written to
the standard output.  Records @file{8} and @file{16} are piped
records, as are those defined within the @file{pipe} subdirectory of
the system-wide database directory (q.v.)

@item Prolog
@cindex prolog (defined)
@cindex byte offset
@cindex cruft (in signal files)
@cindex prolog (in signal files)
@cindex start of sample data
Extraneous bytes at the beginning of a signal file that are not to be
read as samples.  Signal files created using the WFDB library do not
contain prologs, but signal files created using other means may contain
prologs.  To read such a signal file using the WFDB library, provided that
the sample data are in a supported format, it is sufficient to record the
length of the prolog (in bytes) in the appropriate locations in a
header file that names the signal file.  If you need to create
such a header file, refer to the description of the byte offset
field in @cite{header(5)} (the specification of the header file
format in the @cite{WFDB Applications Guide}) or
@pxref{wfdbsetstart}.

@item Record
@cindex record (defined)
An extensible set of files that may include signal files, annotation
files, and a header file, all of which are associated with the
same original signals.  Only the header file is mandatory.
Although records are sometimes called tapes for historical reasons,
records are now more commonly maintained on CDROMs or magnetic disks
than on tape.

@item Record name
@cindex record name (defined)
A character string that identifies a database record.  Record names of
MIT DB records are 3-digit numerals, those of AHA DB records are 4-digit
numerals, and those of ESC DB records are 4-digit numerals with a
prefixed @samp{e}.  Record names may contain up to @code{WFDB_MAXRNL}
(defined in @file{<wfdb/wfdb.h>}) characters, including any combination of
letters, digits, and underscores.  Case (the difference between @samp{e}
and @samp{E}, for example) is significant in record names, even under
operating systems such as MS-DOS that do not treat case as significant
in file names.

@item Reference annotation file
@cindex reference annotation file (defined)
An annotation file supplied by the creator of a record to document its
contents as accurately and thoroughly as possible.  The annotator name
@file{atr} is reserved for reference annotation files.

@item Sample
@cindex sample (defined)
An integer (of at least 16 bits) that corresponds to a voltage measured
at a given instant by an analog-to-digital converter.  Samples are
written by @code{putvec} and read by @code{getvec}.

@item Sample interval
@cindex sample interval (defined)
The unit of time;  the interval between consecutive samples of a given
signal.

@item Sample number
@cindex sample number (defined)
An attribute of a sample defined as the number of samples of the same
signal that precede it; thus the first sample of any signal has sample
number 0.  Sample numbers are long integers (32 bits).  Samples that
have the same sample number in different signals of a given record may
be treated as having been observed simultaneously.

@item Sampling frequency
@cindex sampling frequency (defined)
The number of samples of a given signal that represent one second of
the original analog signal.  The sampling frequency is constant
throughout a signal file, and is the same for all signals in a given
record.

@item Signal
@cindex signal (defined)
A continuously varying function of time that is approximated by
discrete samples.

@item Signal file
@cindex signal file (defined)
A set of samples in time order, which represent a signal or signal
group.  Signal files usually have names of the form @var{record}@code{.dat},
but this is only a convention and is not required.

@item Signal group
@cindex signal group (defined)
A set of signals that are multiplexed together and stored in the same
file.  It is possible to reset input pointers for all signals in a
given signal group
@ifnotinfo
(@pxref{isgsettime, , @code{isgsettime}}),
@end ifnotinfo
@ifinfo
(@pxref{isgsettime}),
@end ifinfo
but not independently for individual signals within a signal group.

@item Signal group number
@cindex signal group number (defined)
A number by which a signal file, once opened, is known.

@item Signal number
@cindex signal number (defined)
An integer by which a signal, once opened, is known.  Input and output
signals each have their own series of signal numbers assigned in serial
order beginning with 0.

@item Skew
@cindex skew (defined)
@cindex intersignal skew
The time difference between samples having the same sample number but
belonging to different signals.  Ideally the skew is zero (or less than
one sample interval), but in some cases this is not so. For example, if
the signals were originally recorded on multitrack analog tape, very
small differences in the azimuth of the recording and playback heads may
result in measurable skew among signals.  If the skew can be measured
(for example, by reference to features of two signals with a known time
difference), it can be recorded in the header file for a record;
once this has been done, @code{getvec} and @code{getframe} correct for
skew automatically.  If you need to correct for skew, see
@cite{skewedit(1)} and @cite{header(5)} (in the @cite{WFDB
Applications Guide}), or @pxref{wfdbsetskew}.  Prospectively, if you
anticipate that skew may be a problem, it is a good idea to apply an
easily identifiable synchronization pulse to all your inputs
simultaneously while recording;  you can then locate this pulse in
each digitized signal and use these measurements to correct for skew.

@item Standard time format
@cindex standard time format (defined)
Any string format legal as an argument for @code{strtim}
@ifnotinfo
(@pxref{timstr and strtim, , @code{timstr} and @code{strtim}}).
@end ifnotinfo
@ifinfo
(@pxref{timstr and strtim}).
@end ifinfo

@item System-wide database directory
@cindex system-wide database directory (defined)
@cindex WFDB path
@cindex database path
The directory that contains local copies of the default WFDB calibration file,
WFDB sample record @code{100s}, and local, piped, and tape header files.
This directory is created when the WFDB Software Package is installed, and
by default it is included in the WFDB path (as the second component, following
the user's current directory).  It is called the ``system-wide'' database
directory because it is shared by all users of the system on which it resides.
Under Unix, the system-wide database directory is usually @file{/usr/database}
or @file{/usr/local/database};  under MS-DOS or MS-Windows, it is usually
@file{c:\database}.

@item Tape
@cindex tape (defined)
A database record.

@item Time
@cindex time (defined)
In this guide, synonymous with sample number (q.v.).  Thus the ``time of
an annotation'' is the sample number of the sample to which the
annotation ``points''.

@item WFDB-compatible format
@cindex WFDB-compatible format (defined)
Any of the standard formats readable and writable by the WFDB library,
for storage of WFDB records in PhysioBank and on CDROMs.

@item WFDB library
A set of functions (subroutines), able to read and write database files,
callable by C and C++ programs, and described in this guide.

@item WFDB path
The database path (q.v.).
@end table

@node     Installation, WFDB Applications, Glossary, Top
@appendix Installing the WFDB Software Package

This appendix briefly describes how to install the WFDB Software Package
on a new system.  The package includes C-language sources for the WFDB
library and for a variety of applications (@pxref{WFDB Applications}) including
@code{WAVE}, sources for this manual, the @cite{WFDB Applications Guide}, and
the @cite{WAVE User's Guide}, and a one-minute sample record (@samp{100s}).

The latest version of the WFDB Software Package can always be downloaded in
source form from @uref{http://www.physio@-net.org/physio@-tools/@-wfdb.shtml},
the WFDB home page on PhysioNet.  Binaries for popular operating systems and
development snapshots are also usually available there.

The process for installing the package is the same on all platforms, and
is documented in detail in the quick-start guides for the popular platforms
that can be found on the WFDB home page.  In brief:

@cindex curl
@cindex libcurl
@cindex libwww
@cindex W3C libwww
@enumerate
@item
@emph{Install any prerequisites needed for your platform.}  These include
@code{gcc} (the GNU Compiler Collection), related software development tools
such as @code{make}, a supported HTTP client library (either @code{libcurl} or
@code{libwww}; this can be omitted if NETFILES support is not desired), the
XView libraries (needed for WAVE only), and X11 (needed by XView). All of these
components are free (open-source) software available for all popular platforms,
including GNU/Linux, Mac OS X, MS Windows, and Unix.  The quick start guides
list recommended packages and where to find them.

@item
@emph{Download and unpack the WFDB Software Package.}  Versions for all
platforms are built from a single package of portable sources;  the most
recent package is always available at
@uref{http://www.physio@-net.org/physio@-tools/wfdb.tar.gz}.

@item
@emph{Configure the package for your system.}  The @code{configure} script
creates a customized building procedure for your system and allows you
a few choices about where to install the package.

@item
@emph{Make and verify a test build.}  The package includes a set of test
scripts that are run to verify basic operations of the WFDB library and
many of the applications, permitting them to be tested before installation.

@item
@emph{Make, install, and test a final build.}
@end enumerate

See the quick start guide for your platform for detailed step-by-step
instructions.

@emph{Important:} Although you may be able to compile the WFDB Software Package
using a proprietary compiler, this is @emph{not supported}.

@unnumberedsec The WFDB library and languages other than C

If you wish to use or develop WFDB applications in C, C++ or Fortran,
everything necessary is included in the WFDB Software Package.

The separate @code{wfdb-swig} package provides wrappers for the WFDB library
so that it can be used by applications written in a variety of other languages
supported by SWIG, the Simplified Wrapper Interface Generator.  At the time
of writing, the languages that are known to work with @code{wfdb-swig} wrappers
are Perl, Python,, Java, and C#.  Other languages supported by SWIG include
Guile, mzScheme, PHP, Ruby, and Tcl. You will need to install @code{wfdb-swig}
wrappers for your platform and language after installing the WFDB Software
Package.  Download the @code{wfdb-swig} package from
@uref{http://physio@-net.org/physio@-tools/wfdb-swig.shtml}, the
@code{wfdb-swig} home page on PhysioNet, and build and install the wrappers
you need following the instructions on that page.

The WFDB Toolbox for Matlab provides a set of WFDB applications for Matlab,
based on the SWIG Java wrappers;  install it with one click from the link
you will find on its home page, 
@uref{http://@-physionet.org/@-physiotools/@-matlab/@-wfdb-swig-matlab/}.

@node     WFDB Applications, Extensions, Installation, Top
@appendix WFDB Application Programs

This appendix briefly describes the application programs that are
included with the WFDB Software Package.  Except where noted otherwise,
these applications are usable on all systems for which the WFDB library is
available.  For details on using these programs, refer to the @cite{WFDB
Applications Guide}.  (On Unix systems, the contents of the
@cite{Applications Guide} may also be available as on-line @code{man}
pages.)

@menu
* Using::		Notes on using these programs.
* Annotation I/O::	Programs that read, write, summarize, and
			otherwise process annotation files.
* Evaluation::		Programs for evaluating the performance of ECG
			analysis programs.
* Signal processing::	Programs that read, excerpt, reformat, resample,
			filter, combine, analyze, acquire, and replay signals.
* Graphics::		Programs for viewing or plotting signals and
			annotations.
@end menu

@node     Using, Annotation I/O, WFDB Applications, WFDB Applications
@unnumberedsec How to use these programs

These programs are kept in directories that vary from system to system;
they may not be in the default search path.  If you cannot find them,
consult an expert (such as the person who installed the WFDB library on
your system).  If you use these programs often, you may wish to include
the directory in which they are kept in your search path.

@cindex WFDB path
To use any of these programs, you will need to set the database path first
(@pxref{WFDB path}), unless the default database path
(@samp{. /usr/database http://www.physio@-net.org/physio@-bank/data@-base})
is suitable. Programs that accept @emph{time} arguments or commands (usually
shown as @var{from} and @var{to} below) use @code{strtim} to convert these
strings into sample intervals; hence they accept any of the varieties of
standard time format described earlier
@ifnotinfo
(@pxref{timstr and strtim, , @code{timstr} and @code{strtim}}).
@end ifnotinfo
@ifinfo
(@pxref{timstr and strtim}).
@end ifinfo
Programs that accept annotation mnemonics as arguments or commands
(usually shown as @var{code} below) use @code{strann} to interpret them;
for a list of legal mnemonics, @pxref{Annotation Codes}.  Where record
or annotator names are required as command arguments, they are indicated
below as @var{record} or @var{annotator}.

In the remainder of this appendix, you will find usage examples and
capsule descriptions of the standard WFDB application programs.  The
square brackets (`@var{[ ]}') in some of the usage examples surround
arguments that may be omitted;  the brackets themselves are not
to be included in the command line.  Where an ellipsis (`@var{...}')
appears, it indicates that the previous argument may be repeated.
If invoked without any arguments, or with a @samp{-h} (help) option, most of
these programs print a brief synopsis of how they are used.

@node     Annotation I/O, Evaluation, Using, WFDB Applications
@unnumberedsec Annotation File Processing

@example
ann2rr -a @var{annotator} -r @var{record [ options ... ]}
rr2ann -a @var{annotator} -r @var{record [ options ... ]}
rdann -a @var{annotator} -r @var{record [} -f @var{from} -t @var{to} -p @var{type ... ]}
wrann -a @var{annotator} -r @var{record}
sumann -a @var{annotator} -r @var{record}
tach -a @var{annotator} -r @var{record [ options ... ]}
@end example

Programs @file{ann2rr} and @file{rr2ann} respectively list RR (inter-beat)
intervals in text format from an annotation file, and create an annotation
file from a text-format list of RR intervals.

The program @file{rdann} is an annotation printer similar to the one
shown in chapter 6 (@pxref{Example 3}).  The optional @var{from} and
@var{to} arguments (in standard time format) specify a portion of the
annotation file to be printed, and one or more @var{type} arguments
(annotation mnemonics) can be given to restrict the output to
annotations that are of the specified type(s).

The output of @file{rdann} can be converted back into an annotation file
by providing it as the standard input of @file{wrann}.  This can be
useful for editing annotation files in some cases;  they can be
converted to ASCII format by @file{rdann}, edited using any text editor,
and converted back into annotation files by @file{wrann}.

A summary of the contents of an annotation file can be obtained using
@file{sumann}.  The summary includes the number of annotations of each
type, and the duration and number of episodes of each rhythm and signal
quality.

@file{tach} generates a uniformly sampled, smoothed, instantaneous heart
rate sequence from an annotation file.

@node     Evaluation, Signal processing, Annotation I/O, WFDB Applications
@unnumberedsec Evaluation of ECG Analyzers

@cindex annotation comparator
@cindex comparator (annotation)
@example
bxb -r @var{record} -a @var{reference-annotator test-annotator [ options ... ]}
rxr -r @var{record} -a @var{reference-annotator test-annotator [ options ... ]}
mxm -r @var{record} -a @var{reference-annotator test-annotator [ options ... ]}
epic -r @var{record} -a @var{reference-annotator test-annotator [ options ... ]}
sumstats @var{file}
plotstm @var{file}
ecgeval
nst @var{[ options ... ]}
@end example

The motivation for developing the MIT and AHA databases was to provide
material for evaluating the accuracy of arrhythmia detectors,
particularly with respect to ventricular arrhythmias.  Between 1984
and 1987, the Association for the Advancement of Medical
Instrumentation (AAMI) sponsored the development of a recommended
practice (designated ECAR) for using the databases for this purpose.
The aim of ECAR was to specify the evaluation methodology in
sufficient detail to permit reproducible testing, and to encourage
informed comparisons of the performance of ventricular arrhythmia
detectors in the analysis of these standard test recordings.  More
recently, the AAMI has developed, and ANSI has adopted as American
National Standards, a standard (ANSI/AAMI EC38:1998) for ambulatory
electrocardiographs, and a companion standard (ANSI/AAMI EC57:1998)
for testing and reporting performance results of cardiac rhythm and ST
segment measurement algorithms.  EC38 and EC57 specify standard
protocols for evaluating the automated analysis algorithms that are
included in many such devices.  These protocols include those
developed for the earlier recommended practice, and extend them to
evaluation of supraventricular arrhythmia and ischemia detection.
EC38 and EC57 specify the use of @file{bxb}, @file{rxr}, @file{mxm},
and @file{epic} to perform evaluations, and further specifies the use
of the MIT DB (as well as two other databases formerly available on
the MIT-BIH Arrhythmia Database CDROM and now freely available on
PhysioNet), the AHA DB, and (for devices that perform analysis of the
ST segment) the ESC DB.  If you are interested in this subject, obtain
copies of the American National Standards for @cite{Ambulatory
Electrocardiographs} (ANSI/AAMI EC38:1998) and for @cite{Testing and
Reporting Performance Results of Cardiac Rhythm and ST Segment
Measurements Algorithms} (ANSI/AAMI EC57:1998; @pxref{Sources}).

@cindex QRS detector
@cindex detector (QRS)
To evaluate an arrhythmia detector using this software, obtain for each
WFDB record to be used in the test an annotation file containing the
detector's analysis of each beat.  These are referred to as the `test'
annotation files (or the `algorithm' annotation files, in EC38 and EC57).
The placement of the beat annotations must match those in the reference
annotations within 150 msec; thus it is not necessary to place
annotations precisely at the PQ junction (as in the AHA DB reference
annotations) or on the major local extremum (as in the MIT DB reference
annotations).  If the detector is capable of shut-down (i.e., if it
inhibits its QRS detection function during periods that it judges are
unreadable), the test annotation files should include a @code{NOISE}
annotation with @code{subtyp = -1} at the beginning of each period of
shut-down, and a @code{NOISE} annotation with any other @code{subtyp}
at the end of each such period.  (If the record ends while the detector
is shut down, the annotation file should include a final `end of shut-down'
annotation as above to permit correct shut-down accounting.)  If the
detector is capable of ventricular fibrillation detection, the test
annotation files should also include @code{VFON} and @code{VFOFF}
annotations; it is not necessary to mark flutter waves (use @code{FLWAV}
annotations to do so if desired).  See the @code{man} page for @file{epic},
in the @cite{WFDB Applications Guide}, for information on marking
atrial fibrillation, ischemic ST episodes, and ST deviation measurements
in test annotation files.  Any annotations that appear in the
first five minutes of an annotation file are treated as belonging to the
detector's learning period, and are not used in the evaluation.  The
evaluation software examines such annotations only to determine the
detector's state (normal, shut down, or in VF) at the beginning of the
test period.

@cindex annotation comparator
@cindex comparator (annotation)
Program @file{bxb} implements the beat-by-beat comparison algorithm described
in EC38 (section 4.2.14.2.2) and EC57 (section 4.3.2).  By default, the output
is in a self-explanatory matrix format.  The @samp{-L} option, which must be
followed by two file names, specifies that the output of @file{bxb} should be
written in line format, for further processing by @file{sumstats}.  The
line-format output includes column headings only if the output file must be
created from scratch.  In this way, @file{bxb} can be used repeatedly to build
up a line-format tables for multiple records.  Among the other options is
@samp{-o}, which causes @file{bxb} to generate an output annotation file (with
annotator name @file{bxb}) indicating agreements and discrepancies between the
input annotators.

@file{rxr} can be used to performed the run-by-run comparison described in EC38
(section 5.2.14) and in EC57 (sections 4.4.3 and 4.4.4).  @file{mxm} compares
heart rate, HRV, or other measurements, as described in EC38 (section
4.2.14.2.3).  @file{epic} evaluates VF and AF detection, and ST analysis, as
described in EC38 (sections 5.2.14), and EC57 (sections 4.5 and 4.6).  These
programs also accept a @samp{-L} option to produce line-format output as for
@file{bxb}.

@file{sumstats} derives the record-by-record, episode-by-episode, and
aggregate performance statistics described in EC38 (section 4.2.14.3) and in
EC57 (sections 3.5.2 and 3.5.3) from line-format output files produced
by @file{bxb}, @file{rxr}, @file{mxm}, and @file{epic}.  The input file
must include the column headings so that @file{sumstats} can recognize
the file type.  The output includes a copy of the input, with aggregate
statistics appended at the end.  @file{plotstm} generates a PostScript
scatter plot of ST measurement comparisons gathered by @file{epic}, as
described in EC57 (section 4.6.2).

The easiest way to use these programs is to run @file{ecgeval}, which
generates a script (batch) file to run @file{bxb}, @file{rxr}, etc., for
each record in a database.  See @cite{Evaluating ECG Analyzers} (in the
@cite{WFDB Applications Guide}) for details.

@cindex noise stress test
By adding noise to annotated ECG records, the noise tolerance of an
arrhythmia detector can be measured.  This idea was described by the
author, along with W.K. Muldrow and R.G. Mark, in ``A noise stress test
for arrhythmia detectors'', @cite{Computers in Cardiology}
@strong{11}:381-384 (1984).  Program @file{nst} adds calibrated amounts
of noise to ECGs (or other signals), generating an output record in WFDB
format.  @file{nst} was used to generate the graded series of noisy ECG
records in the MIT-BIH Noise Stress Test Database (see
@uref{http://physio@-net.org/physio@-bank/data@-base/nstdb/}).
These records are among those specified as standard test
material by EC38 (section 4.2.14.2) and EC57 (section 3.2).

@node     Signal processing, Graphics, Evaluation, WFDB Applications
@unnumberedsec Signal Processing Applications

@example
rdsamp -r @var{record [ options ... ]}
wrsamp -r @var{record [ options ... ]}
snip -i @var{input-record} -n @var{new-record [ options ... ]}
xform -i @var{input-record [ options ... ]}
fir @var{[ options ... ]} -c @var{coefficient ...}
sigamp -r @var{record [ options ...]}
sqrs -r @var{record [ options ... ]}
sample @var{[ options ... ]}
calsig -r @var{record [ options ... ]}
@end example

@file{rdsamp} prints samples from the specified record;  @samp{-f} and
@samp{-t} options may be used to specify a range of sample numbers, and
a subset of signal numbers may be selected using the @samp{-s}
option.  The output of @file{rdsamp}, or any similar text, can be
converted into a WFDB record using @file{wrsamp}.

To copy an excerpt of a longer record, use @file{snip}, which creates
new
header and signal files for @var{new-record} in the current directory.
The beginning and end of the excerpt are specified using @samp{-f} and
@samp{-t} options as for @file{rdsamp}.  Annotator names may follow a @samp{-a}
option; in this case excerpts from the specified annotation files are copied as
well (the annotations are appropriately time-shifted).

@file{xform} is a more general version of @file{snip}; its main uses are for
reformatting, rescaling, and sampling rate conversion.  You may create a
@file{hea} file specifying the desired format, sampling frequency, ADC zero
levels, signal gains, etc., and supply it to @file{xform} using the @samp{-o}
option; if you do not do so, @file{xform} obtains the required information
interactively.  @file{xform} accepts all of the options used by @file{snip},
as well as several others.

@cindex digital filter
@cindex filter (digital)
Program @file{fir} is a general-purpose FIR filter for WFDB records,
similar to the one discussed in chapter 6 (@pxref{Example 7}).

@file{sigamp} measures signal amplitudes (either baseline-corrected RMS
amplitudes or peak-to-peak amplitudes);  it may be useful for calibrating
signals (together with @file{calsig}) or for determining signal gains
for @file{nst}.

@cindex QRS detector
@cindex detector (QRS)
@file{sqrs} is a slightly modified version of the QRS detector
discussed in chapter 6 (@pxref{Example 10}).  Options allow
specification of the signal and interval to be analyzed and the
detection threshold.

Program @file{sample} is an MS-DOS application that uses a Microstar
Laboratories DAP 1200- or 2400-series ISA (AT bus) analog interface
board (@pxref{Sources}) to generate database records from analog
signals, or to generate analog signals from database records.  If you
wish to use other hardware for these purposes, refer to chapter 6
(@pxref{Example 8}) and to the source for @file{sample} as
models.

If you create your own database records using @file{sample} or other
means, program @file{calsig} may be useful for determining signal
gains and offsets if your signals include standard calibration pulses or
identifiable signal levels. @file{calsig} incorporates two
independent algorithms for measuring calibration pulses; it rewrites
header files based on its measurements.

@node     Graphics, , Signal processing, WFDB Applications
@unnumberedsec Graphical Applications

@example
wave -r @var{record [ }-a @var{annotator ]}
pschart @var{[ [ options ... ] script ... ]}
psfd @var{[ [ options ... ] script ... ]}
@end example

@cindex ECG waveform editor
@cindex waveform editor
@file{wave} is an X Window System client application for viewing
and editing WFDB records.  (@file{wave} is included in the WFDB software
package.)  @file{wave} can be
run on all popular platforms, including FreeBSD, GNU/Linux, Mac OS X,
MS Windows, Solaris, and other systems for which X11 servers are available.
Run @file{wave} without any arguments to obtain instructions for printing its
on-line manual.

@file{pschart} and @file{psfd} produce annotated ``chart recordings''
and ``full-disclosure'' plots that can be printed on PostScript devices.
These programs were used to prepare the @cite{MIT-BIH Arrhythmia
Database Directory} and the @cite{European ST-T Database
Directory}.  The popular Chart-O-Matic web service
(@uref{http://physionet.org/cgi-bin/chart}) creates browser-viewable plots
using @file{pschart}.

@node     Extensions, Sources, WFDB Applications, Top
@appendix Extensions

This section may be helpful if you wish to extend the capabilities of
the WFDB library, or if you wish to port it to another environment.  In
order to make use of the information in this section, you should have
the WFDB library sources (@pxref{Sources}).  The sources are distributed
among four `include' (@file{.h}) files and five @file{.c} files:

@display
@t{wfdb.h        } Constant and structure definitions, and function prototypes
@t{ecgcodes.h    } Annotation codes
@t{ecgmap.h      } Annotation code mapping macros
@t{wfdblib.h     } External definitions for private WFDB library functions

@t{wfdbinit.c    } Functions @code{wfdbinit}, @code{wfdbquit}, and @code{wfdbflush}
@t{signal.c      } Functions for signals
@t{calib.c       } Functions for signal calibration
@t{annot.c       } Functions for annotations
@t{wfdbio.c      } Low-level I/O and operating system-dependent functions
@end display

The first three of these files are the standard `include' files that
are usually obtained by @samp{#include <wfdb/@var{file}.h>} statements.
When modifying the WFDB library, however, make any necessary changes in
the copies of these files that are kept in the library source
directory.  Install the modified versions of the @file{.h} files in the
system's @file{include} directory after installing the modified WFDB
library.

The cleanest mechanism for adding additional fields to @file{hea} files is
to include them in `info' strings
@ifnotinfo
(@pxref{getinfo, , @code{getinfo}}),
@end ifnotinfo
@ifinfo
(@pxref{getinfo}),
@end ifinfo
rather than by modifying the code that reads and writes @file{hea}
files (in @file{signal.c}).

A common problem is the need to import signal files generated by other
software.  Often this problem can be solved by writing a format
conversion program that uses input functions provided with the other
software to read the signal files, and @code{putvec} to write them
in one of the formats supported by the WFDB library.  This solution is
unlikely to be satisfactory if you have many large signal files to
import, however, and you may wish to arrange for @code{getvec} to read
the imported files directly.  This may be done by defining a new signal
file format, as outlined below.

To define a new format for signal files, choose a numeric code to
represent your format.  (Values between 900 and 999 are reserved for
user-defined signal file format codes.)  In @file{wfdb.h}, add your format
code to @code{FMT_LIST} and increment @code{NFMTS}.  In @file{signal.c},
define functions (macros if possible for efficiency) for reading and
writing single samples; these should be named @code{r@var{nnn}} and
@code{w@var{nnn}}, where @var{nnn} is your format code.  Follow the
examples in @file{signal.c}; it will almost certainly be easier to make
use of the existing macros @code{r8} and @code{w8} than to begin from
scratch.  Add additional @code{case} statements in @code{getvec} and
@code{putvec}, again following the existing models.  You will also need
to add a @code{case} in @code{isgsettime}, including a formula to
determine the number of bytes needed per sample, given the number of
signals multiplexed.  (All currently-defined formats use fixed-length
encoding.  If you wish to implement variable-length encoding, it may be
easiest to implement an indexed-search method for @code{isgsettime} in
such cases.)  If the ADC resolution exceeds the number of bits in a C
@code{int} on your system, change the @code{typedef} for @samp{WFDB_Sample}
in @file{<wfdb/wfdb.h>} as necessary;  be aware that this change is likely
to require additional changes to application programs (use @file{lint}
or an ANSI C compiler to check your code).

Although the WFDB library generally assumes that signal files are ``pure'', it
is possible to read imported signal files that contain prologs (data that
precede the first sample).  To do so, you must construct a header file in which
the @code{format} fields encode the length of the prolog in bytes (you can do
this manually, or use @code{wfdbsetstart}, @pxref{wfdbsetstart}, for this
purpose).  For example, a signal file with a 512-byte prolog followed by format
16 samples would be specified using @samp{16+512} in the @code{format} field or
fields (if the file contains more than one signal, the @code{format} fields for
all signals in the file must be identical).  Note that this facility is
provided only for signal file import; the WFDB library is not equipped to
create signal files with embedded prologs.

In a similar fashion, though with substantially more effort in most
cases, you may define a new format for annotation files.  Add additional
@code{stat} values for reading and writing to the list in @file{wfdb.h}.
In @file{annot.c}, add additional @code{case} statements and code to
@code{annopen}, @code{getann}, @code{putann}, and @code{wfdb_anclose}.  If
you are designing a new format, you may wish to specify a `magic number'
with which your files will begin, to allow @code{annopen} to recognize
the format automatically; a good choice of such a number is one in which
the first byte is non-zero (to distinguish it from AHA format files) and
the high six bits of the second byte are zero (to distinguish it from WFDB
format files).

Some users may wish to define additional annotation codes.  An easy and
portable way to accomplish this is to use @code{setannstr} and
@code{setanndesc} within programs that create your annotation files,
before opening them using @code{annopen} (or @code{wfdbinit}).  Annotation
files created in this way contain modification labels at the beginning
that document the non-standard code definitions, and that permit them to
be read properly by standard WFDB applications.  Another solution is to
modify the WFDB library.  This method has the disadvantage that all of
your applications that read annotation files must be recompiled, and
they may no longer read standard annotation files properly.  If despite
this disadvantage you prefer to modify the WFDB library, begin by defining
symbolic names and numeric values for your new codes in
@file{ecgcodes.h}.  (Values between 42 and 49 are reserved for
user-defined annotation codes.  Unused values less than 42 may be
assigned in future versions of the WFDB library, and values greater than
49 are reserved to indicate the presence of optional fields such as
@code{subtyp}.)  Next, decide how the new codes are to be mapped by
@code{isqrs}, @code{map1}, @code{map2}, @code{mamap}, and @code{annpos},
and set the appropriate entries in each of the code map arrays in
@file{ecgmap.h}.  Finally, add mnemonic and descriptive strings for the
new codes in the @code{cstring}, @code{astring}, and @code{tstring}
arrays in @file{annot.c}.

The modular design of the library makes it fairly easy to remove
unneeded functionality in order to conserve memory for special
applications.  The @file{calib.c} package is not referenced by any other
WFDB library modules.  For signal processing applications that do not involve
annotations, the entire @file{annot.c} package may be removed (with trivial
modifications to the functions in @file{wfdbinit.c}).  If you wish to add
functions to the library, you will find that it will be easier to
maintain your modified version and to merge updates if you preserve the
existing arrangement of functions, which requires no global variables.
Rather than defining global variables, consider implementing query
functions (global-scope functions that read or write local
variables).  If you wish to define new types of binary files, consider
using the low-level I/O routines in @file{wfdbio.c} for reading and writing
them in a machine-independent format.

@cindex WFDB path
Porting the WFDB library to another environment is a straightforward
operation if an ANSI C compiler is available in the target environment.
Since all direct access to database files is performed using the
(private) function @code{wfdb_open}, it is possible to include file name
translation in that function if needed, to accommodate file naming
schemes that may be imposed by the operating system or other
requirements.  If the notion of environment variables is foreign to the
target environment, @code{getwfdb} can be modified to read the WFDB path
from a file.  You may wish to modify the private function
@code{wfdb_error} (which is responsible for all error reporting from WFDB
library functions) if the `standard error output' is unavailable or
inadequate for use in the target environment.  All of these functions are
contained within @file{wfdbio.c}; it is unlikely that any other code will
require changes for a port.

If you encounter errors while compiling @file{signal.c}, you may wish to try
using the functions provided in that file as alternatives to the standard
macros @code{r16} and @code{w16}; the fully-expanded versions of these macros
are quite complex and are known to cause difficulty for at least one (now
obsolete) C compiler.  (Define the symbol @samp{BROKEN_CC} while compiling
@file{signal.c} in order to obtain the function versions of @code{r16} and
@code{w16}.)  While compiling @file{signal.c}, it may be necessary to disable
code optimization for some C compilers;  no current compilers are known to
have such limitations, however.

@node     Sources, Answers, Extensions, Top
@appendix Sources

This section is a compendium of sources for databases and related
materials that may be useful to readers of this guide.  Please send any
corrections to the author (@email{george@@mit.edu}).

@table @emph
@cindex MIT DB
@cindex waveform editor
@cindex libwww
@cindex W3C libwww
@cindex XView toolkit
@item WFDB Programmer's Guide (this guide)
@itemx WFDB Applications Guide
@itemx WAVE User's Guide
@itemx MIT-BIH Arrhythmia Database
@itemx MIT-BIH Arrhythmia Database Directory
@itemx MIT-BIH Polysomnographic Database
@itemx MIMIC Database
@itemx MGH/Marquette Foundation Waveform Database CROMs
@item Long-Term ST Database
@itemx Other reference databases of physiologic signals
@itemx WFDB Software Package
@itemx W3C libwww sources
@itemx XView toolkit (needed for WAVE)
@sp 1
@display
WWW: @uref{http://www.physionet.org/}
@end display

PhysioNet offers free access via the web to large collections of
recorded physiologic signals and related open-source software.  The
PhysioNet web site is a public service of the PhysioNet Resource
funded by the National Institutes of Health's NIBIB and NIGMS.  The
master PhysioNet web server is located at MIT in Cambridge,
Massachusetts; about ten public mirrors are located elsewhere in the
US and around the world (see
@uref{http://@-www.@-physio@-net.@-org/@-mirrors/} for a list).

@cindex CDROM
@cindex ESC DB
@item European ST-T Database CDROM
@itemx European ST-T Database Directory
@itemx VALE Database Directory
@sp 1
@display
National Research Council (CNR) Institute of Clinical Physiology
Dept. of Bioengineering and Medical Informatics
via Trieste, 41
56126 PISA, Italy

email: @email{taddei@@ifc.pi.cnr.it}
telephone: +39 050 501145
telefax: +39 050 503596
@end display

Over half of this database has been contributed to PhysioNet (see above), from
which it may be downloaded freely.

@cindex AHA DB
@item AHA Database for Evaluation of Ventricular Arrhythmia Detectors
@sp 1
@display
ECRI
5200 Butler Pike
Plymouth Meeting, PA 19462 USA

email: @email{bduffin@@ecri.org}
WWW: @uref{http://www.ecri.org/}
telephone: +1 610 825 6000
@end display

@item American National Standard ANSI/AAMI EC38:1998, Ambulatory Electrocardiographs
@itemx American National Standard ANSI/AAMI EC57:1998 Testing and Reporting Performance
@itemx @ @ @ Results of Cardiac Rhythm and ST Segment Measurement Algorithms
@sp 1
@display
Association for the Advancement of Medical Instrumentation
1110 N Glebe Road, Suite 220
Arlington, VA 22201 USA

WWW: @uref{http://www.aami.org/}
telephone: +1 703 525 4890
telefax: +1 703 276 0793
@end display

@item Computers in Cardiology
@sp 1
@display
WWW: @uref{http://www.cinc.org/}
@end display

CinC is the major scientific meeting at which current research in ECG
signal processing and modelling is discussed; the proceedings of the
conference are probably the single best source of information in print
about these topics.  CinC conferences have taken place annually since
1974, usually in September.  They are usually convened in North
America and in Europe in alternate years.  The deadline for submission
of abstracts is on or about 1 May each year.  Proceedings of CinC
conferences since 2006 are available on-line at
@uref{http://cinc.mit.edu/archives/}, and usually appear about a month
after the date of the conference.  CinC will be in in Bologna (Italy)
in 2008, and in Park City, Utah (USA) in 2009.

Since 2000, Computers in Cardiology and PhysioNet have jointly sponsored
an annual series of open challenges that invite participants to address
topics of interest to researchers and clinicians, often involving the
study of WFDB-compatible data sets made available via PhysioNet, and
culminating with research presentations and awards at CinC.  See
@uref{http://physionet.org/challenge/} for further information.

@item Proceedings of Computers in Cardiology (ISSN 0276-6574)
@sp 1

CinC proceedings from 2006 to present are available on-line at
@uref{http://cinc.mit.edu/archives/}.

IEEE members can find CinC proceedings from 1988 to the
present using IEEEXplore (@uref{http://ieeexplore.ieee.org/}). Many
universities provide access to these services for their students, faculty, and
staff.  Printed volumes of CinC proceedings are available from:

@display
IEEE Customer Service
445 Hoes Lane
P.O. Box 1331
Piscataway, NJ 08855-1331 USA

email: @email{customer.service@@ieee.org}
WWW: @uref{http://www.ieee.org/ieeestore/}
telephone: 1 800 678 IEEE (USA and Canada) or +1 732 981 0060
telefax: +1 732 981 9667
@end display

@cindex GNU emacs
@item GNU emacs
@itemx gcc (the GNU portable C/C++ compiler)
@itemx ghostscript
@itemx GNU tar
@itemx GNU gzip (free and improved replacement for `compress')
@itemx Larry Wall's `patch' program, with GNU revisions
@itemx GNU groff, gtbl, and related text formatting utilities
@itemx GNU info and makeinfo (standalone hypertext browser and formatter)
@sp 1

@display
Free Software Foundation
59 Temple Place - Suite 330
Boston, MA 02111-1307 USA

email: @email{gnu@@gnu.org}
WWW: @uref{http://www.gnu.org/}
telephone: +1 617 542 5942
@end display
GNU software is included in all Linux distributions (in fact, since Linux
is the name of the kernel only, and the largest component of these
distributions is actually GNU software, it is proper to refer to them as
GNU/Linux distributions).  GNU software for all popular (and many other)
operating systems is available on CDROM or tape from the address above, and
is also freely available by anonymous FTP from @code{ftp.gnu.org} and and many
other archive sites.  Please support the FSF with a donation if you use GNU
software.

@item @TeX{} for Unix systems
@sp 1

This software is available by anonymous FTP from CTAN (Comprehensive TeX
Archive Network) mirrors, including @code{ftp.tex.ac.uk}, @code{ftp.dante.de},
and @code{ctan.tug.org}.  Many of the sources of GNU software (above) also make
@TeX{}, etc. available.  CTAN is indexed on the World Wide Web (one such index
is @uref{http://www.ctan.org/}).

The Unix @TeX{} distribution is also distributed on CDROM and in other tape
formats by the Free Software Foundation (address above) and others.  It is
also included with most GNU/Linux distributions (see below).

Several commercial implementations of @TeX{} for MS-DOS, MS-Windows, and Mac OS
are widely available;  visit the web site of the @TeX{} Users Group (below) for
pointers.

@item General information on @TeX{}
@sp 1
@display
TeX Users Group
PO Box 2311
Portland, OR 97208-2311 USA

WWW: @uref{http://www.tug.org/}
email: @email{office@@tug.org}
telephone: +1 503 223 3960
telefax: +1 503 223 9994
@end display

@cindex curl
@cindex libcurl
@item libcurl
@sp 1
@display

WWW: @uref{http://curl.haxx.se/}
@end display

The @code{libcurl} library is a modern and enhanced replacement for the
@code{libwww} libraries (see below).  Only one of @code{libcurl} and
@code{libwww} is needed for NETFILES support.

@cindex libwww
@cindex W3C libwww
@item W3C libwww libraries
@sp 1
@display

WWW: @uref{http://www.w3.org/Library/}
@end display

The @code{libwww} libraries, created and maintained by the World Wide
Web Consortium, provide the low-level functions needed to support
the WFDB library's (optional) NETFILES capability.  The @code{libwww}
libraries are also available from PhysioNet.

@cindex X11
@cindex X Window System
@item X11 (the X Window System, Version 11)
@sp 1
@display

email: @code{xorg_info@@x.org}
WWW: @uref{http://www.x.org/}
telephone: +1 781 376 8200
telefax: +1 781 376 9358
@end display

Sources for XView are available from PhysioNet.

@cindex GNU/Linux
@cindex Linux
@item GNU/Linux

GNU/Linux is a POSIX-compliant reimplementation of the Unix operating system,
written by Linus Torvalds and a cast of thousands.  It runs on Intel 386,
486, and Pentium PCs, among others.  For information about GNU/Linux, visit
the web site of the Linux Documentation Project:
@display
WWW: @uref{http://www.linuxdoc.org/}
@end display

GNU/Linux is freely available by anonymous FTP in source and binary form from
many sites, including:
@display
www.kernel.org
metalab.unc.edu
ftp.funet.fi
@end display

Many low-cost (typically US$10 to US$30) distributions of GNU/Linux on CDROMs
are widely available.  Among the more popular are:

@display
Debian (non-commercial)

WWW: @uref{http://www.debian.org/}


Fedora (non-commercial)

WWW: @uref{http://fedoraproject.org/}


Gentoo (non-commercial)

WWW: @uref{http://www.gentoo.org/}


Knoppix (non-commercial, live CD)

WWW: @uref{http://www.knoppix.org/}


Mandriva

WWW: @uref{http://www.mandriva.com/}


Red Hat

WWW: @uref{http://www.redhat.com/}


Slackware (non-commercial)

WWW: @uref{http://www.slackware.com/}


SuSE (Novell)

WWW: @uref{http://www.novell.com/linux/}


Ubuntu

WWW: @uref{http://www.ubuntu.com/}
@end display

@item Compilers and software development systems

Any ANSI/ISO C compiler can be used to compile the WFDB library and
applications that use it.  Under Unix and GNU/Linux, high-quality free
compilers and development tools are universally available and taken for
granted.  Even if you must work in the MS-Windows or MS-DOS environment,
however, there is no reason to purchase expensive, bloated, and inferior
proprietary compilers and software development systems, since there are several
excellent, highly recommended, and completely free alternatives based on the
GNU C/C++ compiler (gcc).

Using any of these packages does not limit you to creating free (open
source) software, although you are certainly encouraged to do so.  If you
wish to develop and sell proprietary software using @code{gcc}, this is
certainly possible, with fewer and less severe restrictions than you
will encounter if using a commercial compiler.

@cindex Cygwin
@item Cygwin

WWW: @uref{http://www.cygwin.com/}

This is a freely available software development platform for MS-Windows
9x/NT/2000/ME/XP, based on GNU @code{gcc} and a large set of Unix
utilities developed by the GNU project and ported to MS-Windows by
Cygnus Software (now part of Red Hat, see above).  Cygwin itself is
open-source software and is highly recommended in preference to
commercial C/C++ compilers if you must work in the MS-Windows
environment.  The WFDB Software Package binaries available via PhysioNet
are created using Cygwin.

@cindex MinGW
@item MinGW

WWW: @uref{http://www.mingw.org/}

This is another freely available software development platform for MS-Windows,
also based on @code{gcc} and many of the same utilities as Cygwin.  The
difference is that applications built by MinGW gcc use the native MS-Windows C
library, while those built by Cygwin gcc generally use Cygwin's Unix-compatible
standard C library.  

@cindex djgpp
@item djgpp

WWW: @uref{http://www.delorie.com/djgpp/}

A freely available software development platform for MS-DOS, including
@code{gcc}, a free 32-bit DOS extender, and many of the same utilties
as Cygwin and MinGW.

@item Microstar DAP analog interface boards for PCs
@sp 1
@display
Microstar Laboratories
2265 116th Avenue N.E.
Bellevue, WA 98004 USA

email: @email{info@@mstarlabs.com/}
WWW: @uref{http://www.mstarlabs.com/}
telephone: +1 425 453 2345
telefax: +1 425 453 3199
@end display

@cindex Web browser
@item Web browsers
@sp 1

The most popular Web browsers may be downloaded by anonymous FTP.

@display
Firefox
FTP: @code{ftp.mozilla.org}
WWW: @uref{http://www.mozilla.org/firefox/}

Netscape
FTP: @code{ftp.netscape.com}
WWW: @uref{http://www.netscape.com/}

Mozilla
FTP: @code{ftp.mozilla.org}
WWW: @uref{http://www.mozilla.org/}

MS Internet Explorer
FTP: @code{ftp.microsoft.com}
WWW: @uref{http://www.microsoft.com/}

Opera
FTP: @code{ftp.opera.com}
WWW: @uref{http://www.opera.com/}
@end display
@end table

@node     Answers, Recent changes, Sources, Top
@comment This node isn't in the top-level menu;  why make it too easy?
@unnumbered Answers to Selected Exercises

@table @asis
@item 3.
@iftex
280 adu = (280 adu @minus{}(@minus{}300 adu)) / 10 adu/mmHg = 58 mmHg.
@end iftex
@ifinfo
280 adu = (280 adu - (-300 adu)) / 10 adu/mmHg = 58 mmHg.
@end ifinfo
@ifhtml
280 adu = (280 adu - (-300 adu)) / 10 adu/mmHg = 58 mmHg.
@end ifhtml

@item 4.
@iftex
120 mmHg = 120 mmHg * 10 adu/mmHg + ( @minus{}300 adu) = 900 adu.
@end iftex
@ifinfo
120 mmHg = 120 mmHg * 10 adu/mmHg + (- 300 adu) = 900 adu.
@end ifinfo
@ifhtml
120 mmHg = 120 mmHg * 10 adu/mmHg + (- 300 adu) = 900 adu.
@end ifhtml

@item 5.
The range of sample values is
@iftex
@minus{}2047 to +2047 adu, or @minus{}174.7 to +234.7 mmHg.  The
special value @minus{}2048
@end iftex
@ifinfo
-2047 to +2047 adu, or -174.7 to +234.7 mmHg.  The special value -2048
@end ifinfo
@ifhtml
-2047 to +2047 adu, or -174.7 to +234.7 mmHg.  The special value -2048
@end ifhtml
adu, if found in the input, is replaced with @code{WFDB_INVALID_SAMPLE}
(-32768) to indicate missing or out-of-range samples.

@item 6.
We don't know how big @file{signal.dat} is, because we don't know how
many other signals are multiplexed with the @code{BP} signal.  If there
are no others, @file{signal.dat} is 1,500,000 bytes (@code{nsamp} * 1.5
bytes/sample).  One-third of the space occupied by @file{signal.dat} could be
saved if it were converted to format 8.  The maximum slew rate
representable in format 8 is 127 adu/sample interval * 100 sample
intervals/sec / 10 adu/mmHg = 1270 mmHg/sec.

@item 7.
One way to save a little space is to resample the signal at 120 Hz, and
then change to format 8 (maximum slew rate = 1524 mmHg/sec).  This can be
done using @file{xform};  it reduces the storage requirement by
one-fifth.

@item 8.
If you have installed PhysioToolkit's @code{plt} package, a simple
solution is to write the sample numbers and values on the standard
output in two-column ASCII format.  The plotting is then performed by
the pipeline:
@example
@var{your-program} | plt 0 1
@end example

@end table

@node     Recent changes, Concept Index, Answers, Top
@comment  node-name,  next,  previous,  up
@unnumbered Recent changes

This section contains a brief summary of changes to the WFDB library and
to this guide since the first printing of the tenth edition of this
guide in June, 1999.  See @file{NEWS}, in the top-level directory of the
WFDB Software Package distribution, for information on any more recent
changes that may not be described here.

@unnumberedsec WFDB 10.5
@unnumberedsubsec Changes in version 10.5.4 (13 July 2010)

Function @code{getseginfo()} has been introduced in WFDB library version
10.5.4, to allow applications to obtain information about the
segments belonging to the current (multi-segment) input record.

In previous versions, integer arithmetic overflow was possible
when converting format 32 samples using @code{aduphys()}, if the difference
between the baseline value and the sample to be converted exceeded
the range of signed 32-bit integers.  Although @code{rdsamp} does not
use @code{aduphys()} similar code in @code{rdsamp} also exhibited this problem,
which has now been corrected;  thanks to Ikaro Silva for
reporting it and providing a test case.

@unnumberedsubsec Changes in version 10.5.3 (22 June 2010)

Function @code{getgvmode()} has been introduced in WFDB library version
10.5.3, to allow querying the current operating mode of @code{getvec()}.

@unnumberedsubsec Changes in version 10.5.2 (18 April 2010)

When reading annotations of multifrequency records opened in
high-resolution mode, a time shift was introduced by @code{getann()} in
WFDB library versions 10.4.5 to 10.5.1.  This problem has been
corrected in 10.5.2.

Certain malformed segment .hea files were able to cause null pointer
errors in @code{isigopen()}.  This problem has been
corrected.  Thanks to Mauro Villarroel for reporting this problem
together with a test case.

@unnumberedsubsec Changes in version 10.5.1 (19 March 2010)

In version 10.5.0, signals in formats 80 and 160 with amplitudes of 0 were
incorrectly treated as invalid by @code{getframe()} and @code{getvec()}.
Thanks to Isaac Henry for reporting this problem, which has been corrected in
@file{lib/signal.c}.

@unnumberedsubsec Changes in version 10.5.0 (16 March 2010)

The WFDB library now supports signals with 24 and 32 bits of precision, using
new formats 24 and 32, as well as BDF and BDF+ files (24-bit EDF and EDF+
variants), so that WFDB applications can now read these formats.  Note that
these formats, unlike all previously defined formats, require more than 16 bits
per sample value.  If the WFDB software is compiled on a 16-bit platform
(unusual except for embedded processors), the excess high bits of signals in
these formats are not read on input, and they are replaced by zeroes on output,
unless the @code{WFDB_Sample} data type has been redefined as @code{long} (in
@file{wfdb.h}).  This is not done by default since it would increase memory and
computational requirements unnecessarily in embedded applications that do not
require 24- or 32-bit precision.

Since support for extended precision samples cannot be introduced
without this limitation in backward compatibility for 16-bit
platforms, the minor version number of the library has been
incremented to 5.  Most users will not be affected by this change,
however, apart from the new functionality it provides.

Memory allocation macros that have been defined previously in
@file{wfdblib.h} have been moved to @file{wfdb.h} so they are
accessible to WFDB applications, including user-written applications.
For information about using these macros (MEMERR, SFREE, SUALLOC,
SALLOC, SREALLOC, and SSTRCPY), @pxref{Memory Allocation Macros}.

A buffer overflow in the WFDB library's internal function @code{edfparse()}
(in @file{signal.c}) has been corrected, thanks to a bug report and patch
from Joonas Paalasmaa.

@unnumberedsec WFDB 10.4
@unnumberedsubsec Changes in version 10.4.25 (21 January 2010)

In WFDB library versions 10.4.5 through 10.4.24, @code{strtim("e")} did not
work properly when the open input record was an EDF file and high
resolution input mode had not been selected.  This problem has been
corrected.

(WFDB library version 10.4.24 was identical to version 10.4.23.)

@unnumberedsubsec Changes in version 10.4.23 (7 August 2009)

Several changes in @file{lib/signal.c} eliminate unintended interactions that
occurred in previous versions between the input modes that can be
selected using @code{setgvmode()}.  The effects of these interactions were
first observed in newly written code; existing WFDB applications were
apparently unaffected.

On Cygwin/Windows, the WFDB library is now installed in both @file{/usr/bin}
and @file{/usr/lib}, to simplify building user applications that use the WFDB
library and to simplify running compiled WFDB applications outside of
the Cygwin environment.
	
@unnumberedsubsec Changes in version 10.4.22 (28 July 2009)

When reading multifrequency records in WFDB_LOWRES mode, invalid
sample values occurring in signals that need to be decimated were not
always handled properly in WFDB library function @code{getvec()}, resulting
in spurious artifacts during intervals that contained a mixture of
valid and invalid samples.  This bug has been corrected; thanks to
Omar Abdala for a report that brought this problem to light.

A declaration for WFDB library function @code{findsig()} has been added
to @file{wfdb.h}.  Thanks to Thomas Heldt for reporting a warning that
resulted from its omission in previous versions.

@unnumberedsubsec Changes in version 10.4.21 (14 May 2009)

Changes in @file{lib/Makefile.tpl} were needed in order to pass configuration
constants to the functions added to @file{lib/wfdbio.c} in 10.4.20, but were
omitted from that release; they have been included in this version.

@unnumberedsubsec Changes in version 10.4.20 (4 May 2009)

Several new functions have been added to @file{lib/wfdbio.c}, to make
configuration constants accessible at run time; these are needed by
the WFDB Toolkit for Matlab.

(WFDB library version 10.4.19 was identical to version 10.4.18.)

@unnumberedsubsec Changes in version 10.4.18 (15 March 2009)

The WFDB library no longer reports spurious checksum errors when reading
EDF files, which don't have checksums.

@unnumberedsubsec Changes in version 10.4.17 (5 March 2009)

Internal WFDB library function @code{wfdb_checkname} now allows hyphens
(@code{-}) within record and annotator names.

@unnumberedsubsec Changes in version 10.4.16 (3 March 2009)

WFDB library function @code{strtim} now rounds rather than truncating when
the sampling frequency is not an integer.

@unnumberedsubsec Changes in version 10.4.15 (26 February 2009)

WFDB library function @code{mstimstr} now outputs time to the nearest
millisecond, rather than truncating its calculation to the next lowest number
of milliseconds.

A new WFDB library function, @code{wfdbputprolog}, can write a prolog at the
beginning of a signal file.  WFDB applications ignore embedded prologs.

@unnumberedsubsec Changes in version 10.4.14 (23 February 2009)

WFDB library function @code{setwfdb()} now exports the WFDB library
environment variables (@code{WFDB}, @code{WFDBCAL},
@code{WFDBANNSORT,} and @code{WFDBGVMODE}) on all modern platforms, so
that a process started by a WFDB application inherits the WFDB
environment of its parent.  (This is not possible on platforms that do
not provide a working @code{putenv()} function, such as MacOS 9 and
earlier, and 16-bit versions of MS-Windows.)  Some previous versions
also included this capability, but the older implementation caused a
memory leak, and it was disabled in version 10.4.6.  Thanks to Omar
Abdala and Dan Scott for reporting the problem and help in identifying
its cause.

@unnumberedsubsec Changes in version 10.4.13 (16 February 2009)

A new WFDB library function, @code{tnextvec()}, finds the next valid sample
from a chosen signal, occurring at or after a specified time.  This
function is particularly useful when reading variable-layout
multi-segment records that may have lengthy gaps in signals of interest.

For portability, the WFDB library has always stored parameters such as
sampling frequency and gain as strings rather than as floating-point
numbers.  Although the resultant loss of precision has been almost
always negligible, it has been observable in the specific case of
converting very large time intervals from sample intervals to seconds,
when the sampling frequency in Hz cannot be represented exactly in
binary (e.g., if the sampling frequency is once per minute, or 1/60
Hz).  This situation cannot be wholly avoided, but it can be
minimized.  Changes in this release, in @code{setheader()} and in functions
that record time units in annotation files, increase the precision
with which non-integer parameters are recorded, so that loss of
precision as a result of converting them to and from strings should
almost never be observable.

@unnumberedsubsec Changes in version 10.4.12 (20 January 2009)

The rule for sorting annotations within a file has been changed to allow a much
larger number of simultaneous annotations (i.e., annotations in a given
annotation file with identical @code{time} fields) than was previously
possible.  Since version 6.1, the WFDB library sorts simultaneous annotations
according to the value of their @code{chan} fields, allowing for 256
simultaneous annotations at any given time.  Beginning in version 10.4.12, the
library sorts simultaneous annotations according to their @code{num} fields,
then sorts those with identical @code{num} fields according to their
@code{chan} fields, allowing up to 65,536 simultaneous annotations at any given
time.

A new WFDB library function, @code{findsig}, returns the signal number of the
input signal matching its string argument, or -1 if no such input signal
exists.  If the string argument could be interpreted as an input signal number,
it is taken as such; otherwise, the string argument must be an exact match to a
signal name (@code{desc} field in the @code{siginfo} structure).

Previous versions of WFDB library function @code{setifreq} entered an infinite
loop if invoked (contrary to specifications) before opening an input record.
@code{setifreq} now detects the error, emits an appropriate warning, and
returns.

If a WFDB application that uses WFDB library version 10.4.5 through 10.4.11
attempted to read an annotation file before reading the sampling frequency of
the associated record (for example, by invoking @code{isigopen} or
@code{sampfreq}), the annotation times might all appear to be zero.  This may
occur when reading annotations created using WFDB 10.4.5 or later.  The times
supplied when creating the file are correctly written but may be incorrectly
read in these cases.  This problem was corrected in this release; thanks to
Thomas Heldt for reporting it and providing a reproducible example of it.

(WFDB library version 10.4.11 was identical to version 10.4.10.)

@unnumberedsubsec Changes in version 10.4.10 (31 October 2008)

EDF digital maximum and minimum values are now read properly in 64-bit
builds;  previous versions had a bug in @code{edfparse} (an internal WFDB
library function defined in @file{lib/signal.c}) that did not appear in
32-bit builds.	Thanks to Joe Mietus for reporting the problem.

@unnumberedsubsec Changes in version 10.4.9 (10 October 2008)

The WFDB library once again correctly interprets a hyphen (@code{-})
(used in place of a record name, annotator name, signal file name,
or calibration file name) as a reference to the standard input or
output, for platforms that support POSIX standard I/O (@pxref{Standard I/O}).
This feature was broken in 10.4.5 as a side effect of changes
in @code{wfdb_open} (an internal WFDB library function defined in
@file{lib/wfdbio.c}).

(WFDB library version 10.4.8 was identical to version 10.4.7.)

@unnumberedsubsec Changes in version 10.4.7 (15 July 2008)

Yinqi Zhang reported and contributed a fix for a memory leak in
@code{make_vsd()} (an internal WFDB library function defined in
@file{signal.c}).

@unnumberedsubsec Changes in version 10.4.6 (9 April 2008)

The WFDB functions @code{setafreq()} and @code{getafreq()} (for
setting and getting the time resolution of newly-created output
annotation files in ticks per second) were new in version 10.4.5, but
were undocumented.  They are now described in this Guide, and wrappers
for these functions are now included in @file{fortran/wfdbf.c}.

An important change in the WFDB library: memory allocation errors are
now treated as fatal by default (in previous versions, the functions
that encountered them returned error values that permitted the
application to handle them).  These errors occur when there is
insufficient memory available to the WFDB library. To obtain the old
behavior, in which the calling function will continue execution if
possible after a memory allocation error, invoke @code{wfdbmemerr(0)}.
By default, however, such an error will cause the process to
terminate.  In either case, the WFDB library emits an appropriate
error message to aid in troubleshooting.

New macros for handling dynamically allocated memory are defined in
@file{lib/wfdblib.h} and used throughout the WFDB library, eliminating most
known memory leaks.  Three known leaks remain (in @code{setecgstr()}, 
@code{setannstr()}, and @code{setanndesc()});  these are documented and
harmless in current applications.  Thanks to Yinqi Zhang for reporting a leak
in @code{copysi()} (an internal WFDB library function defined in
@file{signal.c}), which prompted the cleanup.
	
WFDB functions @code{strecg()}, @code{setecgstr()}, @code{strann()},
@code{setannstr()}, and @code{setanndesc()} now handle NULL string inputs
properly. (Previous versions passed NULL strings to @code{strcmp()}, with
undesirable results.)

@unnumberedsubsec Changes in version 10.4.5 (6 February 2008)

Bob Farrell and Tony Ricke chased down and provided fixes for memory leaks
in several WFDB library functions, and also provided revisions to permit
additional type checking and to avoid type mismatch warnings.

Changes in the build system make it easier to build WFDB using Cygwin
gcc (with or without the Cygwin POSIX library).

When creating annotation files, if the input sampling frequency differs
from the frame rate of the input record (either because of using
@code{WFDB_HIGHRES} mode while reading a multifrequency record, or because
of having used @code{setifreq()} to modify the sampling frequency), a comment
is written to the beginning of the annotation file indicating the
resolution of the annotation times in ticks per second (thus allowing
the application to store its annotations with whatever time resolution
is desired).  When reading an annotation file, if such a resolution
comment is found, getann adjusts the times of annotations to match the
currently defined sampling frequency.  The resolutions are kept
independently for each annotation file, so (for example) @code{bxb} can
compare two annotation files written with different resolutions.

The ability to set the time resolution of annotation files has required
a minor change in the semantics of @code{setifreq()}.  It is now necessary
to invoke @code{setifreq()} before creating an annotation file that will
have a resolution matching the (modified) input sampling frequency.
Since @code{setifreq()} must be invoked after opening the input signals,
this implies that @code{wfdbinit()} cannot be used to open both input signals
and output annotation files if @code{setifreq()} is to be used;  rather, the
sequence should be @code{isigopen()}, @code{setifreq()}, and finally
@code{annopen()}.

If a string that includes a @samp{.} is supplied to a WFDB library function
where a record name is expected, the WFDB library assumes that it is
the name of a file located in the WFDB path. If the name ends in @samp{.hea},
the file is assumed to be a WFDB-format header file, and its record name
is assumed to be the first part of the string, exclusive of the @samp{.hea'}.

This version also includes support for reading EDF files natively.
If a string supplied as a record name contains a @samp{.} but does not end
in @samp{.hea}, it is assumed to be the record name of an EDF file of the
same name.

(WFDB library versions 10.4.3 and 10.4.4 were identical to version 10.4.2.)

@unnumberedsubsec Changes in version 10.4.2 (4 May 2006)

Mathias Gruber reported a line in wfdbio.c that used void pointer arithmetic
(permitted as an extension by gcc but not allowed by ANSI/ISO C or most other C
compilers).  This operation has been replaced by ANSI/ISO C-conformant code.

@unnumberedsubsec Changes in version 10.4.1 (6 April 2006)

A bug caused incorrect output from WFDB library function strtim() when called
with the argument "i", following use of setifreq() to change the effective
sampling frequency, resulting in incorrect output from example 10 in the WFDB
Programmer's Guide.  This has now been corrected.

@unnumberedsubsec Changes in version 10.4.0 (2 March 2006)

Version 10.4.0 and later versions of the WFDB library are intended to
be compiled using ANSI/ISO C (and C++) compilers only;  previous
versions also supported the use of traditional (K&R) C compilers.  The
most obvious change resulting from this decision is in the use of
prototypes in function declarations, an innovation of ANSI C that
permits better error-checking by compilers.  The ANSI/ISO C standard is
now more than 15 years old, and it has been over 10 years since a C
compiler that does not support function prototypes was used for
development of the WFDB library.  Code in @file{wfdbio.c} that provides
limited support for compilers that do not provide an ANSI/ISO C library
has been retained for now, and @file{wfdb.h} still includes a set of K&R C
function declarations; both of these features are deprecated, however,
and may be removed in future versions of the WFDB library.  Users who
still need to use a K&R C compiler to compile the library itself may
find 'unprotoize' (included in the GNU gcc distribution) to be helpful.

The mapping of lowest expressible sample values to @code{WFDB_INVALID_SAMPLE}
performed by @code{getframe()} (in @file{lib/signal.c}) did not work properly
for signal formats 80 and 160 (in which samples are recorded as unsigned
integers); this has now been corrected.

The symbol @code{WFDB_GVPAD} is newly defined in @file{<wfdb/wfdb.h>}.
It may be added to @code{WFDB_HIGHRES} or @code{WFDB_LOWRES} and given
as input to @code{setgvmode()}.
The effect of doing so is that missing samples, and samples recognized
as invalid, are replaced by @code{getframe()}, @code{getvec()}, and
@code{sample()} with the most recently read valid values rather than by
the special value @code{WFDB_INVALID_SAMPLE}.
This behavior allows applications such as digital filters to remain
ignorant of missing data without significant performance penalties.

@code{sample()} now checks that its signal number input
is valid, and returns @code{WFDB_INVALID_SAMPLE} if not.  In previous
versions, @code{sample} returned a sample from signal 0 if the requested
signal was unavailable.

@code{sample_valid()} now returns -1 in the case of a signal that becomes
unavailable before the end of the record (previous versions returned 1 in this case).

The FIR filter example (@pxref{Example 7}) now works properly.  The
previous version always began processing the input at sample 0,
regardless of start time specified in its argument list.

@unnumberedsec WFDB 10.3
@unnumberedsubsec Changes in version 10.3.17 (20 August 2005)

This version is the first to support reading variable-layout records
(multi-segment records in which the number, arrangement, gains, and baselines
of the signals may vary from one segment to the next; @pxref{Multi-Segment
Records}).

Rounding errors in the WFDB library's @code{mstimstr} function have been
reduced.  Previous versions did not always round appropriately when the
sampling frequency was much less than 1 Hz.

The maximum length for a record name (@code{WFDB_MAXRNL}, defined in
@file{wfdb.h}) has been increased from 11 to 20.

A new constant, @code{WFDB_INVALID_SAMPLE}, is now defined in @file{wfdb.h}.
It is used to identify padding inserted to fill in for missing data.  When
writing in any format that uses fewer than 16 bits per sample, @code{putvec}
maps @code{WFDB_INVALID_SAMPLE} to the lowest (most negative) value expressible
in that format; when reading a signal file in such a format, @code{getframe}
performs the inverse mapping, so that missing data can be identified regardless
of the data format.

A side effect of this change is that (for example) any samples that had the
most negative value (for example, -2048 in a format 212 signal file) are now
flagged as invalid.  To treat such samples as invalid is reasonable, however,
since these occur only when the input level falls below the working range of
the analog-to-digital converter.

@unnumberedsubsec Changes in version 10.3.16 (13 June 2005)

Benjamin Moody has added an interface between the WFDB library and
@code{libcurl} as an alternative to the existing @code{libwww} interface, and
has updated @file{configure} and @file{conf/*.def} to search for and use
@code{libcurl} if it is available.  The primary advantages of @code{libcurl}
over @code{libwww} are that @code{libcurl} is smaller and faster, it supports
access to password-protected files, and it is actively maintained.  Both
libraries are freely available on all popular platforms.

Isaac Henry has updated @code{configure} to support building a native
MS-Windows version of the WFDB library using either Cygwin gcc or MinGW
gcc.

A number of minor changes, mostly involving conditional use of
@file{malloc.h}, @file{stdlib.h}, and @file{string.h}, were made to
eliminate warnings from gcc 4.x.

@unnumberedsubsec Changes in version 10.3.15 (31 January 2005)

Rules for generating the binary tarball for MS-Windows have been
fixed so that the Cygwin DLLs are now included with correct
permissions.

Installation of shared libraries under GNU/Linux requires an extra
step if SELinux is enabled (as under Fedora Core 2 and later);  this
has been incorporated into @file{conf/linux-slib.def}.

@unnumberedsubsec Changes in version 10.3.14 (29 December 2004)

Guido Muesch reported that @code{getspf()} did not always
return correct results if the frame frequency does not have an exact
representation as a double precision floating point number.  This problem has
now been corrected.

@unnumberedsubsec Changes in version 10.3.13 (5 May 2004)

Using an indirect WFDB path (i.e., setting the WFDB environment
variable to a value such as @samp{@@FILE}, where @samp{FILE} contains the
desired path) was broken in WFDB library versions 10.3.9 through
10.3.12;  it now works again, thanks to a patch contributed by
Fred Geheb.

@unnumberedsubsec Changes in version 10.3.12 (9 March 2004)

Okko Willeboordse pointed out an incompatibility between the native
MS-Windows API and the ANSI/ISO C library function @code{mkdir}, which is
used by the WFDB library.  This does not present a problem when
compiling the WFDB library using the supported Cygwin/gcc compiler
under MS-Windows, nor does any related problem occur on any other
platform.  It should now be a little easier to compile the WFDB library
using unsupported compilers, thanks to a new @code{MKDIR} macro that hides
the incompatibility (see @file{lib/wfdblib.h0}).

Piotr Wlodarek initiated a discussion about memory leaks in the WFDB
library, citing as an example the 'trivial example program in C' from
this Guide, which does not free memory it allocates in @code{isigopen()}
when reading the signal specifications.  This problem can be avoided by
invoking @code{wfdbquit()} in the example program, just before exiting.
Further discussion of this point has been added to this Guide following
the presentation of the 'trivial example', and in the description of
@code{wfdbquit}.

@unnumberedsubsec Changes in version 10.3.11 (17 October 2003)

In @file{signal.c}, several bugs have been identified and fixed.  Thanks
to Piotr Wlodarek, who found a buffer overrun in @code{isigopen}.  Also,
@code{isgsettime} sometimes performed incorrect seeks on multifrequency
records that had been opened in high-resolution mode;  this has been
fixed, together with a related bug that caused the value returned by
@code{strtim("e")} to be calculated incorrectly in some such cases.

@unnumberedsubsec Changes in version 10.3.10 (3 August 2003)

In version 10.3.9, the functions @code{setannstr}, @code{setanndesc},
and @code{setecgstr} did not contain necessary checks to avoid invoking
@code{strcmp} with a @code{NULL} argument.  These checks have been added in
version 10.3.10.  Thanks to Thomas Heldt for reporting this problem.

@unnumberedsubsec Changes in version 10.3.9 (16 July 2003)

The WFDB library functions @code{setwfdb}, @code{setannstr},
@code{setanndesc}, and @code{setecgstr} now copy their input string
arguments, so that it is no longer necessary for WFDB applications to
keep these strings valid.  If you have created applications that rely on
being able to modify these strings, it will be necessary to invoke the
corresponding functions again before such changes will take effect
within the WFDB library.

Previous versions of the WFDB library function @code{putinfo} did not
flush their output until either a new header file was created (via
@code{setheader} or @code{newheader}) or the process exited.  This has
now been corrected, and putinfo output is now flushed before
@code{putinfo} returns.  Thanks to Jonas Carlson for reporting this
problem.

@unnumberedsubsec Changes in version 10.3.8 (12 July 2003)

The WFDB library function @code{setbasetime()} now properly
accepts arguments specifying midnight (e.g., @samp{0:0:0}), which previous
versions rejected, and the function @code{setheader()}
records such times correctly in the @file{.hea} files it creates.

(WFDB library version 10.3.7 was identical to 10.3.6.)

@unnumberedsubsec Changes in version 10.3.6 (7 April 2003)

The fix applied in @code{isigclose()} in 10.3.5 was incomplete but is now
(really!) fixed.  Applications that use @code{sample()} should call 
@code{wfdbquit()} to be certain that @code{sample}'s buffer is freed before
exiting.

Some long-standing problems in the code (in @file{lib/wfdbio.c}) that handles
http range requests for NETFILES-enabled versions of the library have
been partially addressed.  The underlying issue is that http servers
do not always return the range of bytes requested;  when this happens,
it is not difficult to determine that there is a problem, but it is
tricky to figure out what to do about it.  Based on experiments with
several different http servers, the strategy for handling such problems
within the WFDB NETFILES code has been improved substantially, but
there may be further room for improvement.

@unnumberedsubsec Changes in version 10.3.5 (31 March 2003)

Fixed a bug in WFDB library function @code{isigclose()} (in
@file{lib/signal.c}) that had caused @code{sample()}'s buffer to be freed
inappropriately when switching segments in a multi-segment record.
Thanks to Dave Schaffer for the bug report and for a test case that
illustrated the bug.

(WFDB library versions 10.3.3 and 10.3.4 were identical to 10.3.2.)

@unnumberedsubsec Changes in version 10.3.2 (25 February 2003)

Fixed a WFDB library bug that caused annotation sorting to fail
if a new header file had been written.  Thanks to Winton Baker for
reporting this problem and for providing an example that illustrated
the bug.

(WFDB library version 10.3.1 was identical to 10.3.0.)

@unnumberedsubsec Changes in version 10.3.0 (26 November 2002)

Fixed bugs in @file{lib/signal.c} that caused improper accounting of signal
group numbers when reading from two or more records at the same time (as in
@file{nst}), a bug that caused a segfault in @file{nst}, and a bug that
referenced uninitialized memory in @code{newheader} if @code{nsig} = 0.

The WFDB Software Package has been ported to Mac OS X (Darwin),
version 10.2 (the port should also work under 10.1 but this
has not been tested and will not be supported).	

It is now possible to generate a shared WFDB library (DLL)
under MS-Windows using Cygwin/gcc.

Added functions @code{sample} and @code{sample_valid} to the WFDB library
(in @file{lib/signal.c}).  @code{sample(s, t)} returns the sample at
time (sample number) @code{t} from signal @code{s}, handling all
necessary buffering internally and allowing the caller to treat the
signal file as a virtual array of randomly accessible samples.
@code{sample_valid} can be invoked to check if the most recent value
returned by @code{sample} was valid (e.g., to see if the end of the
input was reached).  For an example of the use of these functions, see
@file{app/wqrs.c}.

@unnumberedsec WFDB 10.2
@unnumberedsubsec Changes in version 10.2.9 (27 October 2002)

Fixed a bug in example 9 in this guide (introduced in version 10.2.0).

Updated @file{lib/wfdbdll.def} and the @file{Makefile.dos} files in several
directories.  These have not been tested in recent years and may need further
revisions; feedback is welcome.

Corrected persistent problems with generating PDF versions of the manuals for
the desired page size, and added hyperlinks to the PDF version of this guide.

(WFDB library version 10.2.8 was identical to 10.2.7.)

@unnumberedsubsec Changes in version 10.2.7 (14 October 2002)

Added a workaround to @code{wfdb_fclose} (in @file{lib/wfdbio.c}) so that
closing @code{stdin} after using @code{freopen} doesn't trigger a core dump.

If out-of-order annotations were written and automatic annotation sorting was
suppressed, the warning produced by @code{oannclose} (in @file{lib/annot.c})
once again includes the correct @code{sortann} command needed to put the
annotations into order.  (This feature was broken by a previous revision.)

@unnumberedsubsec Changes in version 10.2.6 (24 June 2002)
@findex setifreq (10.2.6)
@findex getifreq (10.2.6)
@cindex interpolation
@cindex resampling

The new functions @code{setifreq} and @code{getifreq} allow an
application to choose any convenient sampling frequency for reading
input signals.  Samples read from signal files using @code{getvec} are
buffered, resampled, and delivered to the calling application as if the
original signals had been sampled at the desired frequency.  Times
expressed in sample intervals passed to or from other WFDB library
functions (@code{getann}, @code{putann}, @code{mstimstr}, @code{timstr},
and @code{strtim}) are rescaled as needed to match intervals
corresponding to the chosen frequency.  Thanks to Pat Hamilton for the
inspiration!

The WFDB library now records the base time with millisecond precision
(previous versions did so with one-second precision), and @file{xform}
provides starting times to the library function @code{setbasetime} with
millisecond precision.  Thanks to Allavatam Venugopal for providing
examples that illustrated the need for these features.

Fixed deskewing buffer initialization in @code{getframe}, broken by the
10.2.0 update, which introduced an infinite loop when reading a record
that requires skew correction starting at sample 0.  Thanks to Andrew
Walsh for finding an example that triggered this bug.

Fixed rounding errors in @code{adumuv}, @code{muvadu}, and @code{physadu}.
Previous versions rounded negative values toward zero;  to obtain consistent
conversions, however, it is necessary to round all values down (e.g., from
-1.5 to -2 rather than up to -1).

Fixed a memory leak in @code{wfdb_fclose} (in @file{lib/wfdbio.h}).  Thanks to
Ion Gazta@~naga.

@unnumberedsubsec Changes in version 10.2.5 (10 March 2002)

Additions and fixes in @file{wfdbf.c} (the Fortran wrappers for the WFDB
library).

@unnumberedsubsec Changes in version 10.2.4 (20 December 2001)

Code in @file{wfdbio.c} that required the use of the string
@code{header} to identify a header file has been revised so that the
standard @code{hea} is now usable for this purpose in all cases.

@unnumberedsubsec Changes in version 10.2.3 (14 December 2001)

Portability fixes in @file{wfdblib.h}.  (WFDB library version 10.2.2 was
identical to 10.2.1.)

@unnumberedsubsec Changes in version 10.2.1 (16 November 2001)

@cindex WFDB path
Most users will no longer need to set the WFDB path explicitly, as a result
of several minor changes in the default path and in the installer for the
WFDB Software Package.

The environment variable @code{WFDBNOSORT} was replaced by @code{WFDBANNSORT},
and the environment variable @code{WFDBGVMODE} was introduced
(@pxref{Annotation Order}, and @pxref{Multi-Frequency Records}, for details).

@unnumberedsubsec Changes in version 10.2.0 (15 October 2001)

There are no longer any fixed limits on the numbers of signals or annotation
files that can be opened simultaneously, or on the number of samples per signal
per frame.  In previous versions of the WFDB library, the symbols
@code{WFDB_MAXSIG}, @code{WFDB_MAXANN}, and @code{WFDB_MAXSPF} (all defined in
@file{<wfdb/wfdb.h>}) specified limits on these parameters that could be
modified only by recompiling the WFDB library.  These symbols are still defined
for compatibility with older applications that use them (typically to determine
the size of static arrays).

@cindex WFDB path
Since version 10.1.1, record names may include path information (see the notes
for version 10.1.1 below), but if such names are used to generate names of WFDB
output files, the user has been required to ensure that the target directory
exists.  This requirement is eliminated in version 10.2.0.  If an output file
is specified to be located in a non-existent directory, the WFDB library will
attempt to create the directory (including, if necessary, any non-existent
parent directories).  This feature simplifies the use of record names that
include directory information, as is common when reading data from a CDROM or
a web server such as PhysioNet.  For example, using the WFDB path
(@file{. http://@-www.@-physio@-net.@-org/@-physio@-bank/@-data@-base}), if the
current directory, @file{.}, does not contain a subdirectory named
@file{mitdb}, the command:

@example
sqrs -r mitdb/100
@end example

@noindent
will read its input from
@uref{http://@-www.@-physio@-net.@-org/@-physio@-bank/@-data@-base/@-mitdb/},
will create a directory named @file{mitdb} within the current directory, and
will write its output annotation file (@file{100.qrs}) into this newly-created
directory.  If we then use the command:

@example
rdann -r mitdb/100 -a qrs
@end example

@noindent
the header file is still read from the remote directory, but the annotation
file is read from @file{./mitdb}.  (The programs @file{sqrs} and @file{rdann}
are standard applications that use the WFDB library;  see the @cite{WFDB
Applications Guide} for details.)

Also new is the WFDB test suite (located in the @file{checkpkg} directory of
the WFDB source tree, at the same level as the @file{lib} directory containing
the WFDB library sources).  This set of programs can be used to help verify
that a newly-installed version of the WFDB library behaves properly.

@unnumberedsec WFDB 10.1
@unnumberedsubsec Changes in version 10.1.6 (1 August 2001)

The WFDB library requires that the record name specified in the first line
of a header file must match the name of the record with which the header file
is associated (this is done in order to detect corrupted or erroneously
renamed header files).  Version 10.1.6 requires that only the final portion of
the record name (stripped of any path information) must match.

@unnumberedsubsec Changes in version 10.1.5 (11 June 2000)

More changes in the @file{make} description files, for Cygwin compatibility.

@unnumberedsubsec Changes in version 10.1.4 (6 June 2000)

The symbol @code{WFDB_NETFILES} replaces the old @code{NETFILES}.

@unnumberedsubsec Changes in version 10.1.3 (26 April 2000)

More changes in the @file{make} description files, to support a configuration
script.

@unnumberedsubsec Changes in version 10.1.2 (11 March 2000)

Changes in the @file{make} description files.

@unnumberedsubsec Changes in version 10.1.1 (30 January 2000)

@cindex WFDB path
Record names may contain (absolute or relative) path information as a prefix,
and if (as a result) an input file is found in a location that does not appear
explicitly in the WFDB path, that location is appended to the end of the WFDB
path.  For example, if the WFDB path is
@file{. http://@-www.@-physio@-net.@-org/@-physio@-bank/@-data@-base}, and the
record name @file{mitdb/100} is supplied to @code{wfdbinit}, the WFDB library
will find the header file at
@uref{http://@-www.@-physio@-net.@-org/@-physio@-bank/@-data@-base/@-mitdb/@-100.hea},
and will then add @uref{http://@-www.@-physio@-net.@-org/@-physio@-bank/data@-base/@-mitdb/} to
the end of the WFDB path so that the signal file (specified as @file{100.dat}
in the header file) can be found.

@unnumberedsubsec Changes in version 10.1.0 (15 January 2000)
@cindex NETFILES

Version 10.1.0 supports a new signal file format (311), and contains
numerous minor changes in the NETFILES support code introduced in 10.0.1.

@unnumberedsec WFDB 10.0
@unnumberedsubsec Changes in version 10.0.1 (19 November 1999)
@cindex NETFILES
@cindex libwww
@cindex W3C libwww

Beginning with version 10.0.1, the WFDB library supports reading not only local
files, but also remote files made available by web (HTTP) or FTP servers.  To
make use of this feature, link your application with both the WFDB library and
the @code{libwww} library (freely available for all versions of Unix, and for
most recent versions of MS Windows, from @uref{http://www.w3.org/Library}, or
from @uref{http://@-www.physio@-net.org/physio@-tools/libwww/}).  (In some
cases, notably under GNU/Linux, @code{libwww} is linked together with the
dynamically-loaded version of the WFDB library, so that you do not need to link
@code{libwww} explicitly.)  All access to remote files is read-only.  If you do
not wish to allow access to remote files, or if @code{libwww} is not available
for your OS, simply do not define the symbol NETFILES when compiling the WFDB
library.  For further details, see @file{wfdbio.c} in the WFDB library sources.

@cindex WFDB path
The WFDB environment variable may now contain whitespace (space, tab, or
newline characters) as path component separators under any OS.  Multiple
consecutive whitespace characters are treated as a single path component
separator.  Use a @samp{.} to specify the current directory as a path component
when using whitespace as a path component separator.  A semicolon (@samp{;}) is
also acceptable as a path component separator under any OS.  A colon (@samp{:})
is still acceptable as a path component separator under Unix (Linux, etc.),
provided only that the colon is not immediately followed by @samp{//}.

If the WFDB path includes components of the forms
@file{http://somewhere.net/mydata} or @file{ftp://somewhere.else/yourdata}, the
sequence @samp{://} is explicitly recognized as part of a URL prefix (under any
OS), and the @samp{:} and @samp{/} characters within the @samp{://} are not
interpreted further.  Note that the MS-DOS @samp{\} is @emph{not} acceptable as
an alternative to @samp{/} in a URL prefix.  To make WFDB paths containing URL
prefixes more easily (human) readable, use whitespace for path component
separators.

Previous versions of the WFDB library that were compiled for environments other
than MS-DOS used file names in the format @var{type.record}.  This file
name format is no longer supported.

@unnumberedsubsec Changes in version 10.0.0 (25 June 1999)

Beginning with version 10.0.0, the name of the library is WFDB.  All
earlier versions were named DB.  All library symbols have been similarly
renamed, with @code{WFDB} and @code{wfdb} replacing @code{DB} and @code{db}
everywhere, in names of library functions, constants, type and structure
definitions, library source file names, and names of environment variables
(e.g., the DB environment variable is now the WFDB environment variable).

Version 10.0.0 of the WFDB library is functionally identical with the
final release (version 9.7.4) of the DB library, except for the name
changes.  It should be possible to recompile existing applications
written for DB library version 9.x without modification, and to link
them with WFDB library version 10.0.0.  This is possible because two
sets of @code{#include} files are provided with the WFDB library.  The
first set, accessible via @code{#include <wfdb/...>}, works with
applications written as described in this guide.  The alternate set,
accessible via @code{#include <ecg/...>}, is compatible with DB 9.x
applications as described in previous editions of this guide.


@node     Concept Index, Function and Macro Index, Recent Changes, Top
@unnumbered Concept Index

@printindex cp

@node     Function and Macro Index, Copying, Concept Index, Top
@unnumbered Function and Macro Index

For a number of entries below, the function name is followed by the version
number of the WFDB library in which the function first appeared.
Functions for which no such number appears have been present in all
numbered versions of the WFDB library.


@printindex fn

@node     Copying, , Function and Macro Index, Top
@ifinfo
WFDB Programmer's Guide

Tenth Edition (revised and with additions for WFDB library version VERSION)
LONGDATE

George B. Moody

Copyright (C) 1989 -- 2006 George B. Moody

Permission is granted to make and distribute verbatim copies of this
guide provided that the copyright notice and this permission notice are
preserved on all copies.

@ignore
Permission is granted to process this file through TeX and to print the
results, provided that the printed document carries a copying permission
notice identical to this one except for the removal of this paragraph
(this paragraph not being relevant to the printed document).

@end ignore
Permission is granted to copy and distribute modified versions of this
guide under the conditions for verbatim copying and under the conditions
that follow in this paragraph.  Each copy of the resulting derived work
must contain a notice that it is a modified version of this guide.  The
notice must state which edition of this guide was the source for the derived
work, and it must credit the authors of this guide and of the
modifications.  The entire resulting derived work must be distributed
under the terms of a permission notice identical to this one.

Permission is granted to copy and distribute translations of this guide
into another language, under the above conditions for modified versions.

The author would appreciate receiving copies of any modified or
translated versions of this guide for reference purposes.
@end ifinfo

@bye
