\documentstyle[twocolumn]{article}

\title{Evaluating ECG Analyzers}
\author{George B. Moody\\
Harvard-MIT Division of Health Sciences and Technology, Cambridge, MA, USA}
\date{}

\begin{document}
\topmargin -0.5625in
\oddsidemargin -0.25in
\evensidemargin -0.25in
\def\columnsep{0.3125 in}
\def\textwidth{6.875 in}
\def\textheight{224 mm}
\flushbottom
% \pagestyle{empty}	% uncomment to suppress page numbering
\setcounter{page}{100}

\maketitle

\section*{Summary}
This paper describes how to evaluate an automated ECG analyzer using
available annotated ECG databases and software, in compliance with
standard evaluation protocols.  These protocols have been adopted as
parts of the {\em American National Standard for Ambulatory
Electrocardiographs} (ANSI/AAMI EC-38, 1994);  they include earlier
evaluation protocols developed for an AAMI Recommended Practice,
{\em Testing and Reporting Performance Results of Ventricular Arrhythmia
Detection Algorithms} (AAMI ECAR, 1987).  It will be most useful
to readers who plan to use the suite of evaluation software included
in the WFDB Software Package available from MIT;  this suite of software
includes the reference implementations of the evaluation protocols
specified in EC-38.

\section{Introduction}
Continuous monitoring of the electrocardiogram in both inpatients and
ambulatory subjects has become a very common procedure during the past
thirty years, with diverse applications ranging from screening for cardiac
arrhythmias or transient ischemia, to evaluation of the efficacy of
antiarrhythmic drug therapy, to surgical and critical care monitoring.
Since the first intensive care units were established in the 1960s,
the need for automated data reduction and analysis of the ECG has been
apparent, motivated by the very large amount of data that must be
analyzed (on the order of $10^{5}$ cardiac cycles per patient per
day).  As clinical experience has led to the identification of more
and more prognostic indicators in the ECG, clinicians have demanded
and received increasingly sophisticated automated ECG analyzers.  The
early heart rate monitors rapidly evolved into devices that were
designed first to detect ventricular fibrillation, then other
``premonitory'' ventricular arrhythmias.  Many newer devices attempt
to detect supraventricular arrhythmias and transient ischemic ST
changes.

Visual analysis of the ECG is far from simple.  Accurate diagnosis of
ECG abnormalities requires attention to subtle features of the
signals, features that may appear only rarely, and which are often
obscured by or mimicked by noise.  Diagnostic criteria are complicated
by inter- and intra-patient variability of both normal and abnormal
ECG features.  Given these considerations, it is not surprising that
developers are faced with a difficult task in the design of algorithms
for automated ECG analysis, and that the results of their efforts are
imperfect.  Certain parts of the problem --- QRS detection in the
absence of noise, for example --- are well-solved by most current
algorithms;  others --- detection of supraventricular arrhythmias, for
example --- remain exceedingly difficult.  Just as we may find it easiest
to analyze ``textbook'' examples, automated ECG analyzers may perform
better while analyzing the recordings used during their development
than when applied to ``real-world'' signals.

Since automated ECG analyzers vary in performance, and since their
performance is dependent on the characteristics of their input,
quantitative evaluations of these devices are essential in order to
assess the usefulness of their outputs.  At one extreme, a device's
outputs in the context of a particular type of signal may be so
unreliable as to be worthless;  unfortunately, the other extreme ---
an output so reliable it can be accepted uncritically --- is not a
characteristic of any existing monitor, nor can it be expected in
the future.

\subsection{ECG Databases}
Several databases of ECG recordings are generally available
for evaluating ECG analyzers.  They serve several important needs:
\begin{itemize}
   \item  They contain {\em representative} signals.  Wide variations in
ECG characteristics among subjects severely limit the value of
synthesized waveforms for testing purposes.  Realistic tests of ECG
analyzers require large sets of ``real-world'' signals.

   \item They contain {\em rarely observed but clinically significant}
signals.  Although it is not particularly difficult to obtain
recordings of common ECG abnormalities, often those that are most
significant are rarely recorded.  Both developers and evaluators of
ECG analyzers need examples of such recordings.

   \item  They contain {\em standard} signals.  System comparisons
are meaningless unless performance is measured using the same test
data in each case, since performance is so strongly data-dependent.

   \item  They contain {\em annotated} signals.  Typically, each QRS
complex has been manually annotated by two or more cardiologists
working independently.  The {\em reference} annotations produced
as a result serve as a ``gold standard'' against which a device's
analysis can be compared quantitatively.

   \item They contain {\em digitized, computer-readable} signals.  It is
therefore possible to perform a fully automated, strictly reproducible
test in the digital domain if desired, allowing one to establish with
certainty the effects of algorithm modifications on performance.
\end{itemize}

At present, the following ECG databases are available:\footnote{Sources:
ECRI, 5200 Butler Pike, Plymouth Meeting, PA 19462 USA (AHA DB); MIT-BIH
Database Distribution, MIT Room 20A-113, 77 Massachusetts Avenue, Cambridge,
MA 02139 USA (MIT, ESC, NST, and CU databases).  Outside of North America, the
ESC DB is available from: CNR Institute of Clinical Physiology, Computer
Laboratory, via Trieste, 41, 56100 Pisa, Italy.  Except for the AHA DB,
all are available in CD-ROM format.}
\begin{itemize}
   \item {\bf AHA DB}: The American Heart Association Database for
Evaluation of Ventricular Arrhythmia Detectors (80 records, 35 minutes
each)

   \item {\bf MIT DB}: The Massachusetts Institute of Technology--Beth
Israel Hospital Arrhythmia Database (48 records, 30 minutes each)

   \item {\bf ESC DB}: The European Society of Cardiology ST-T
Database (90 records, two hours each)

   \item {\bf NST DB}: The Noise Stress Test Database (12 records, 30
minutes each; supplied with the MIT DB)

   \item {\bf CU DB}: The Creighton University Sustained Ventricular
Arrhythmia Database (35 records, 8 minutes each; supplied on the
second edition of the MIT DB CD-ROM)

\end{itemize}
Each of these databases represents a very substantial effort by many
workers; in particular, the AHA, MIT, and ESC databases each required
more than five years of sustained effort by large teams of researchers
and clinicians from many institutions.  Nevertheless, it should be
recognized that even these databases do not fully represent the
variety of ``real-world'' ECGs observed in clinical practice.
Although these databases permit standardized, quantitative, automated,
and fully reproducible evaluations of analyzer performance, it is
risky to extrapolate from the results of such evaluations to
expectations of real-world performance.  Such extrapolations can be
particularly error-prone if the evaluation data were also used for
development of the analysis algorithm, since the algorithm may have
been (perhaps unintentionally) ``tuned'' to its training set.  It
should also be noted that the first four of the databases listed above were
obtained from Holter ECG recordings; although the frequency response
of the Holter recording technique is not usually a limiting factor in
the performance of an ECG analyzer, it may tend to favor devices that
are designed to analyze Holter recordings over devices that have been
designed to analyze higher-fidelity input signals.

\subsection{Evaluation Protocols}
Between 1984 and 1987, the Association for the Advancement of Medical
Instrumentation (AAMI) sponsored the development of a protocol for the
use of the first two of these databases, which was published as an
AAMI Recommended Practice.\footnote{{\it Testing and Reporting Performance
Results of Ventricular Arrhythmia Detection
Algorithms}. Publication AAMI ECAR (1987); available from AAMI, 3330
Washington Boulevard, Suite 400, Arlington, VA 22201 USA.}  Between
1990 and 1994, the ambulatory ECG subcommittee of the AAMI ECG committee
developed a standard for ambulatory ECG monitors, significant
portions of which address the issue of the accuracy of automated analysis
performed by some of these devices.\footnote{{\it American National Standard
for Ambulatory Electrocardiographs}.  Publication ANSI/AAMI EC-38 (1994);
available from AAMI (address above).}  The ambulatory ECG standard (EC-38)
builds on the evaluation protocol adopted for the earlier Recommended Practice
(ECAR), incorporating provisions for the use of all five of the databases
listed above, with extensions for assessing detection of
supraventricular arrhythmias and transient ischemic ST changes.  The
standard breaks new ground in establishing specific reporting requirements
for the performance of automated ECG analyzers on standard tests using
the databases listed above.

A significant constraint imposed on evaluators by the EC-38 standard
is that they must obtain annotation files containing the analysis
results of the device under test.  Although the device itself need not
produce these files, EC-38 specifically requires that they be produced
by an automated procedure, which must be fully disclosed.  The intent
of this requirement is to permit reproducible independent evaluations
in which neither the proprietary data of the developers (the analysis
algorithms) nor that of the evaluators (the test signals and reference
annotations) need necessarily to be disclosed.  By defining the
interface between the developer and the evaluator to be the annotation
file, the responsibilities of each party are clearly defined: the
developer must make certain that the device's outputs are recorded in
the annotation file in the manner intended by the developer, but in
the language of the standard; the evaluator must make certain
that the algorithms used to compare the device's annotation files with
the reference annotation files conform to the specification of the
standard.  The format and content of these annotation files is
specified in detail below.  For many existing devices, it may be
difficult or impossible to obtain such annotation files without the
cooperation of the developers.  Newly-designed devices should
incorporate the necessary ``hooks'' for producing annotation files.

\subsection{Software to Support Evaluations}
This paper describes a suite of programs that support evaluations
of automated ECG analyzers in accordance with the methods described in
the EC-38 standard (as well as those in the earlier ECAR Recommended
Practice).  These methods are sufficiently complex that the
development of such a suite of programs is not an afternoon's work.
By making generally available reference implementations of the
evaluation algorithms, much needless duplication of effort may be
avoided.  By circulating them in source form to other users, we may
hope to find and correct any bugs, with the eventual result that
evaluators of devices should not have to bear the burden of evaluating
the evaluation technique itself.  By using them for evaluations, any
ambiguities in the English specification of the evaluation algorithms
are resolved in a consistent manner for each device tested.  These
programs are written in C and run under MS-DOS or UNIX.  They have
been made available as part of the WFDB Software Package
from MIT\footnote{Source: MIT-BIH Database Distribution
(address above).} and are included in the current edition of the
MIT-BIH Arrhythmia Database CD-ROM.  In this paper, the names of these
programs are printed {\tt like this}.

\section{Evaluating an ECG Analyzer}
The major task facing an evaluator is that of presenting the reference signals
to the device under test, and collecting annotation files from the device.  The
details of this task will vary for each device, but a few general hints are
given below.  A second task, that of obtaining reference heart rate
measurements, should be a much simpler job.  Once all of this information has
been gathered, the remaining work required --- that of comparing the device's
analysis against the ``gold standard'' --- can be performed automatically.

\subsection{Presenting Signals to the Analyzer}
Two distinctly different types of tests are possible.  If the device can accept
digital inputs, the reference signals can be supplied in that form (perhaps
after resampling with {\tt xform} to convert the digitized samples to the
expected sampling frequency and numerical range, and possibly with additional
digital signal processing to simulate the signal conditioning normally
performed by the device's front-end data acquisition hardware).  The primary
advantage of testing in the digital domain is that the test is (or should be)
strictly reproducible, since no noise or additional quantization error can be
introduced in this way.  This method usually avoids the issue of
synchronization of the test annotations with the reference signals discussed
below.

Testing in the analog domain requires that analog signals be recreated from
the digital signals.  (It should be noted that even the analog versions of the
MIT and AHA databases that have been available in the past were recreated
from the digitized signals by the database developers.)  The advantage of this
approach is that it exercises the entire system, including the front-end data
acquisition hardware.  It is often difficult, however, to establish
synchronization between the signal source and the analyzer, needed in order to
permit comparisons of annotations.  One way of dealing with this problem is to
arrange for the analyzer's sampling clock to trigger the digital-to-analog
converter used to recreate the analog signals, or to arrange for an external
clock to trigger both D/A conversion in the playback system and A/D conversion
in the analyzer.  Another method is to begin and end the signal generation
process by delivering signals from the analyzer to the playback device, and
recording the analyzer's clock time at the times of the signals; assuming that
both the analyzer and the playback device have stable clocks, event times in
the analyzer's frame of reference can be converted to database sample numbers
by linear interpolation.  The WFDB software package includes a program ({\tt
sample}) that uses a Microstar DAP 2400-series analog interface
board\footnote{
Source: Microstar Laboratories, 2863 152nd Avenue NE, Redmond,
WA 98052 USA.  External analog anti-aliasing filters (to reduce
``staircasing'') and attenuators (to obtain patient-level signals) may also be
required, depending on the system to be evaluated.  DAP boards can also be
used with {\tt sample} to create new database records.}
and an MS-DOS PC to recreate analog signals from digital database records on
CD-ROMs or magnetic disk files.

\subsection{Obtaining Test Annotation Files}
For any ambulatory ECG monitor that incorporates automated analysis functions,
the EC-38 standard requires the manufacturer to implement and disclose a
method for producing test annotation files.  Independent evaluators should seek
assistance from the manufacturer in any case, since the manufacturer's
interpretation of the device's outputs in the language of EC-38 is definitive
(in effect, the annotation file generation technique becomes part of
the system under test).  Note that generation of annotation files need not be
synchronous with data acquisition; a device might conceivably store all of the
necessary data until the end of the test, and only then write the file.
Neither does the standard require that an annotation be determined within any
fixed amount of time, as would be expected of devices designed to trigger
pacing, for example.  Furthermore, EC-38 specifically allows for the
possibility that the device under test might not produce the annotation file
directly.  If any external hardware or software is required to do so, however,
it must be made generally available or specified in sufficient detail by the
manufacturer to permit an independent evaluator to obtain test annotation
files.

Annotation files contain a label (an annotation) for each beat and for certain
other features of the signals, such as rhythm and ST changes.  Annotations are
stored in time order in annotation files.  The ``time'' of an annotation is
that of the sample in the signal file with which the annotation is
associated.\footnote{Times in annotation and signal files are usually expressed
as {\em sample numbers} (the number of samples in the signal file that precede
the sample in question).} The WFDB library (included in the WFDB software package)
includes C-callable functions ({\tt getann} and {\tt putann}) for reading and
writing annotations.  In a C program, annotations appear as data structures
containing a 32-bit {\tt time} field together with a pair of 8-bit fields that
encode the annotation type and sub-type ({\tt anntyp} and {\tt subtyp} [sic],
respectively), and a variable-length {\tt aux} field usually used to store
text.  In annotation files, these annotation structures are usually stored in
a variable-length bit-packed format averaging slightly more than 16 bits per
annotation.\footnote{Test annotations that include heart rate or ST
measurements require substantially more storage.  {\tt getann} and
{\tt putann} can also use the original AHA DB format (containing fixed-length
annotations, 16 bytes each), but this format should not be used for
evaluations of devices that incorporate ST analysis functions, since the
space available for the {\tt aux} data is too small to store ST measurements.}

Test annotation files may include the following:
\begin{itemize}
   \item  {\em Beat annotations}.  These need not coincide precisely with the
reference beat annotations, since the evaluation protocol allows a time
difference of up to 150 ms between each pair of matching beat annotations.
All beat annotations are mapped during the evaluation process into the set
\{ N, V, F, S, Q \} (corresponding to normal, ventricular ectopic, ventricular
fusion, supraventricular ectopic, and unclassifiable or paced beats
respectively); devices need not be capable of producing all of these
annotations, but any beat annotations that they do produce will be translated
into one of these types.  The standard specifies the mapping used for the
{\tt anntyp} values defined in {\tt <ecg/ecgcodes.h>}. (This file is
supplied with all of the existing CD-ROM databases.)  Any beat annotations that
appear in the first five minutes of a record (the ``learning period'') are
ignored in the evaluation process.  The remainder of the record (the ``test
period'') must be fully annotated.  Note in particular that the last beat of
some records may be very close to the last sample;  since the analyzer may
reach the end of the input signals before producing an annotation for the last
beat, it may be necessary to ``pad'' the input data for a few seconds at the
end of the record to permit the analyzer to emit its final beat annotation.

   \item {\em Shutdown annotations}.  If the device suspends its analysis
because of poor signal quality or for any other reason, it should mark the
periods during which analysis is suspended.  The evaluation software tallies
beats missed during such periods separately from beats missed at other times.
The beginning of each period of shutdown may be marked using a {\tt NOISE}
annotation with ${\tt subtyp} = -1$, and the end of each period of shutdown
may be marked using a {\tt NOISE} annotation with ${\tt subtyp} = 0$ (see
the source for {\tt bxb} for notes on other acceptable methods of marking
shutdown).

   \item {\em Ventricular fibrillation annotations}.  The beginning and end
of each detected episode of ventricular fibrillation should be marked using
{\tt VFON} and {\tt VFOFF} annotations.

   \item {\em Other rhythm annotations}.  These should include
{\tt RHYTHM} annotations marking the beginning and end of each detected episode
of atrial fibrillation.  The beginning of each episode should be marked with an
``{\tt (AFIB}'' rhythm annotation, i.e., an annotation with {\tt anntyp} = {\tt
RHYTHM} and {\tt aux} = \verb|"\05(AFIB"|, where ``\verb|\05|'' is C notation
for a byte with the value 5 (ASCII control-E).  Non-empty {\tt aux} fields
always begin with a byte that specifies the number of data bytes that follow;
in this case, the five characters ({\tt ( A F I B}) of the string.  The end of
each episode should be marked with any other rhythm annotation (for example,
``{\tt (N}'').

   \item {\em Heart rate measurements}.  Each type of heart rate measurement
(including any heart rate or RR interval variability measurements) made by the
device under test should be assigned a measurement number, $m$, between 0 and
127.  A {\tt MEASURE} annotation should be recorded for each heart rate
measurement, with ${\tt subtyp} = m$ and with the measurement in the {\tt aux}
field, as an ASCII-coded decimal number.

   \item {\em ST deviation measurements}.  If available, these should be
provided in the {\tt aux} fields of beat annotations, as ASCII-coded decimal
numbers indicating the deviations in microvolts from reference levels
established for each signal from the first 30 seconds of each record.  For
example, ``25 -104'' indicates a 25 $\mu$V elevation in signal 0 and a 104
$\mu$V depression in signal 1.  If ST measurements are omitted from any beat
annotation, the evaluation software assumes they are unchanged from their
previous values.

   \item {\em Ischemic ST change annotations}.  These {\tt STCH} annotations
should mark the beginning and end of each detected episode of ischemic ST
change.  ST change annotations have additional information in the {\tt aux}
field as for rhythm annotations: the beginning of each episode is marked by an
``{\tt (ST}{\it ns}'' annotation, and the end of each episode by a
``{\tt ST{\it ns})}'' annotation, where {\it n} indicates the signal affected
 (``{\tt 0}'' or ``{\tt 1}''), and {\it s} indicates ST elevation (``{\tt +}'')
or depression (``{\tt -}'').  {\it n} may be omitted if the episode detection
criteria depend on features of both signals.  The extremum of each episode may
optionally be marked with an ``{\tt AST}{\it nsm}'' annotation, where {\it n}
and {\it s} are defined as above, and {\it m} is the ST deviation in
microvolts, relative to a reference level established as above.
\end{itemize}
Note that only beat annotations are absolutely required in test annotation
files.  ST deviation measurements within beat annotations, and the other
types of annotations listed above, only need to be recorded for devices
that are claimed by their manufacturers to provide optional features for
detection of ventricular or atrial fibrillation, measurement of ST deviations,
or detection of ischemic ST changes.

If the time units in the test annotation files are not the same as those in
the reference annotation files (for example, because {\tt xform} was used to
change the sampling frequency of the signal files in a digital-domain test),
the time units must be rescaled before proceeding with the comparison.  This
may be done by using {\tt xform} to rewrite the test annotation files with
the original sampling frequency.\footnote{
The obvious alternative, using {\tt xform} to rewrite the reference annotation
files at the time the signal files are resampled, should not be used in a
formal evaluation.  Because of the possibility that resampling the reference
annotation files might result in moving reference annotations into or out of
the test period, or changing the lengths of episodes, doing so might produce
results that could not be directly compared with those obtained in a standard
evaluation.}

Details of the ST deviation measurement and episode detection criteria used in
producing the reference annotation files for the ESC database may be found in
several sources.\footnote{ See, for example, the {\it European ST-T Database
Directory}, pp. vi-vii, supplied with the ESC DB; or Taddei, A., et al., ``The
European ST-T database: development, distribution, and use'', {\it Computers
in Cardiology} {\bf 17}:177-180 (1990).} Note, however, that many techniques
for measuring ST deviation and for detecting transient ischemic ST changes are
possible, and that to date the best evaluation results have been obtained for
analyzers using criteria that do not attempt to mimic those used by the human
experts who annotated the database.

\subsection{Obtaining Reference Heart Rate Data}
The final step of preparation for the evaluation is to process the
reference annotation files to obtain reference heart rate annotation
files.  These files must contain heart rate measurement annotations
with the same measurement numbers assigned as for the test heart rate
annotations; they need not necessarily contain beat or other
annotations from the reference annotation files.  Quoting from EC38,
\begin{quote}
To evaluate the accuracy of heart rate measurement, the evaluator
shall implement and disclose a method for obtaining heart rate
measurements using the reference annotation files (the `reference
heart rate').  This method need not be identical to the method used by
the device under test, but in general it will be advantageous if it
matches that method as closely as possible.
\end{quote}
It will generally be in the manufacturer's interest to provide a
program for generating reference heart rate annotation files, to avoid
the need for an independent evaluator to do so, with a likely result
of less than optimal agreement with the test heart rate measurements.
The WFDB software package includes a sample implementation of such a
program ({\tt examples/refhr.c}); note that it will need to be
customized for each device to be tested.

Note that measurement errors are normalized by the mean value of the reference
measurements in each record.  Be certain that this mean value cannot be
zero!\footnote{
For certain types of HRV or RRV measurements (though not for heart rate
measurements), this is a potential problem.  One solution is to add a small
positive offset to any measurement with an expected zero mean.  It is within
the letter, though not the spirit, of the standard protocol, to add a
very large number in such a case, so as to make the error percentage
arbitrarily small.  The mean value of the reference measurements must be
reported; this should serve as a disincentive to this sort of creative abuse
of the standard.  An honest approach might be to add an offset on the order of
the expected standard deviation of the individual measurements.}

\section{Comparing Annotation Files}

Once the test annotation files and the reference heart rate annotation files
have been obtained, the remainder of the evaluation procedure is
straightforward.  All of the information needed to characterize the analysis
performed by the device under test is encoded in the test annotation files;
similarly, all of the information needed to characterize the actual contents
of the test signals is encoded in the reference annotation and reference
heart rate annotation files.  The evaluation procedure thus entails comparison
of the test and reference annotation files for each record.

Four programs are provided in the WFDB Software Package for this purpose:
\begin{itemize}
   \item {\tt bxb} compares annotation files beat by beat; its output
includes QRS, VEB, and (optionally) SVEB sensitivity and positive
predictivity, as well as RR interval error and shutdown statistics.

   \item {\tt rxr} compares annotation files run by run; its output
includes ventricular (and, optionally, supraventricular) ectopic couplet, short
run (3--5 beats), and long run (6 or more beats) sensitivity and positive
predictivity.

   \item {\tt epic} compares annotation files episode by episode; its
output includes ventricular fibrillation, atrial fibrillation, and ischemic ST
detection statistics as well as comparisons of ST deviation measurements.

   \item {\tt mxm} compares measurements from a test annotation
file and a reference heart rate annotation file;  its output indicates
measurement error.\footnote{
{\tt mxm} is not restricted to comparison of heart rate measurements;  if
other types of measurements are available, they may be compared in the same
manner as heart rates by {\tt mxm}.}
\end{itemize}

The WFDB Software Package also includes three related programs:
\begin{itemize}
   \item {\tt sumstats} reads certain output files generated by {\tt bxb},
{\tt rxr}, {\tt epic}, and {\tt mxm}, and calculates aggregate statistics for
a set of records.
   \item {\tt plotstm} generates scatter plots of ST deviation measurements
collected by {\tt epic}.
   \item {\tt ecgeval} automates the entire comparison procedure by running
{\tt bxb}, {\tt rxr}, {\tt epic}, and {\tt mxm} for each record, collecting
their output, then running {\tt sumstats} (and optionally {\tt plotstm}), and
finally printing the results.
\end{itemize}

To obtain a concise summary of how to use any of these programs, including
a list of any command-line options, simply run the program without any
command-line arguments.  Refer to the {\it WFDB Applications Guide},
which accompanies the WFDB Software Package, for details.

In most cases, it will be easiest to collect all of the annotation files
before beginning the comparison, and then to perform the comparison by typing:
\begin{verbatim}
ecgeval
\end{verbatim}
The program asks for the test annotator name, the names of the databases
used for testing, and what optional detector outputs should be evaluated.

Only the statistics required by ECAR and EC38 are reported by {\tt
ecgeval}.  If more detailed evaluation data are needed, it will be
necessary to run {\tt bxb}, {\tt rxr}, etc., separately.  If file
space is extremely limited, it may be necessary to delete each test
annotation file after it has been compared against the reference file,
before the next test annotation file can be created; in such cases, it
may also be necessary to prompt the user to change media containing
signal or reference annotation files, or to reset the device under
test before beginning each record.  Optionally, {\tt ecgeval} can
generate a script (batch) file of commands, which can be edited to
accommodate special requirements such as these.

For example, suppose we have obtained a set of test annotation files with the
annotator name ``{\tt yow}'', which we wish to compare against the reference
annotation files (annotator name ``{\tt atr}'')\footnote{
Annotation files for any given record are distinguished by annotator names,
which, under MS-DOS, correspond to the ``extension'' of the file name.  The
reference annotation files supplied with the databases have the annotator
name ``{\tt atr}'' (from ``{\tt atruth}'', as used in environments with less
restrictive file-naming schemes, originally because ``{\tt a}'' was intended to
indicate the file type, and ``{\tt truth}'' because \ldots well, because the
annotations are supposed to be The Truth).}
and reference heart rate annotation files (annotator name ``{\tt htr}'').
The portion of the evaluation script generated by {\tt ecgeval} for MIT DB
record 100 is:
\begin{verbatim}
bxb -r 100 -a atr yow -L bxb.out sd.out
rxr -r 100 -a atr yow -L vruns.out sruns.out
mxm -r 100 -a htr yow -L hr0.out -m 0
epic -r 100 -a atr yow -L -A af.out
    -V vf.out -S st.out stm.out
\end{verbatim}
(The last two lines shown above form a single command.  The {\tt mxm} command
gathers statistics on measurement number 0; if other heart rate measurements
are defined, {\tt mxm} should be run once for each such measurement,
substituting the appropriate measurement numbers for {\tt 0} in the output
file name, {\tt hr0.out}, and the final argument.)  Statistics for the
remainder of the MIT DB are obtained by repeating these commands, substituting
in each the appropriate record names for {\tt 100}.  Once these commands have
been run for all of the records, the record-by-record statistics will be found
in nine files ({\tt bxb.out}, {\tt sd.out}, {\tt vruns.out}, {\tt sruns.out},
{\tt hr0.out}, {\tt af.out}, {\tt vf.out}, {\tt st.out}, and {\tt stm.out}).
The first eight of these files contain one line for each record.\footnote{
{\tt stm.out} contains one line for each ST deviation measurement that was
compared; in this example, {\tt stm.out} would be empty since the reference
annotation files of the MIT DB do not contain ST deviation measurements.} {\tt
sumstats} can read any of these files, and calculates aggregate performance
statistics; to use it, type ``{\tt sumstats} {\it file}'', where {\it file} is
the name of one of these files.  The output of {\tt sumstats} contains a copy
of its input, with aggregate statistics appended to the end.  Typically this
output might be saved in a file to be printed later, e.g.,
\begin{verbatim}
sumstats bxb.out >>report.out
\end{verbatim}

A scatter plot of the ST measurement comparisons performed by {\tt epic} can be
produced using {\tt plotstm}, the output of which can be printed directly on
any PostScript printer.  For example, to make a plot file for {\tt stm.out},
type:
\begin{verbatim}
plotstm stm.out >stm.ps
\end{verbatim}

\section{Studying Discrepancies}

Having conducted an evaluation as described above, a common question
is ``what were the errors?''  {\tt bxb} and {\tt rxr} can help
answer such questions.

{\tt bxb} can generate an output annotation file (with annotator name
``{\tt bxb}'') in which all matching beat annotations are copied from
the test annotation file, and each mismatch is indicated by a {\tt
NOTE} annotation, with the {\tt aux} field indicating the element of
the confusion matrix in which the mismatch is tallied (e.g., ``{\tt Vn}''
represents a beat called a VEB by the reference annotator and a normal
beat by the test annotator).  Programs such as {\tt view},
{\tt wave}, and {\tt wview}\footnote{
{\tt view} (for MS-DOS),  {\tt wave} (for Linux, Solaris, and SunOS) and
{\tt wview} (for MS Windows) are included in the WFDB Software Package.}
can be used to search for and display the waveforms associated with the
mismatches.  To generate an output annotation file, add the {\tt -o} option to
the {\tt bxb} command line, as in:
\begin{verbatim}
bxb -r 100 -a atr yow -L bxb.out sd.out -o
\end{verbatim}
A particularly useful way to document an evaluation is to print a
full disclosure report with {\tt bxb} output annotations, using the
program {\tt psfd} (also included in the WFDB Software Package).  This may
be accomplished by preparing a file containing a list of the names of
the records to be printed (call it {\tt list}), and then using the
command:
\begin{verbatim}
psfd -a bxb list >output.ps
\end{verbatim}
The file {\tt output.ps} can be printed on any PostScript printer.
Run {\tt psfd} without any arguments for a summary of its (numerous)
options;  try a short test before making a large set of printouts,
which can take a long time.

Both {\tt bxb} and {\tt rxr} accept a {\tt -v} option to run in
``verbose'' mode, in which each discrepancy is reported in the
standard error output.  When running {\tt rxr}, this feature is useful
for finding missed and falsely detected ectopic couplets and runs.

\section{Acknowledgements}

Having been involved in the production of most of the databases as
well as the design of the evaluation protocols, it has been my
privilege to receive the benefits of the sustained contributions of
many colleagues who have supported these projects with their dedicated
efforts.  I would like especially to thank
Paul Albrecht,
Jim Bailey,
Ted Baker,
Rich Bowser,
Don Brodnick,
Jerry Cox,
Phil Devlin,
Charlie Feldman,
Scott Greenwald,
Russ Hermes,
David Israel,
Franc Jager,
Carlo Marchesi,
Roger Mark,
Joe Mietus,
Warren Muldrow,
Diane Perry,
Scott Peterson,
Ken Ripley,
Paul Schluter,
Alessandro Taddei,
Roy Wallen,
and Cees Zeelenberg.

\appendix
\section{Using the AHA Database}

Since the AHA DB is not available in CD-ROM format, the WFDB Software
Package includes a pair of programs that convert files read from AHA
DB distribution tapes or floppy disks into files formatted in the same
way as those in the CD-ROM databases.  {\tt a2m} converts AHA
annotation files, and {\tt ad2m} converts AHA signal files and also
generates header ({\tt *.hea}) files.  (Run these programs without
command-line arguments to obtain instructions on their use.)  Using
{\tt a2m} and {\tt ad2m}, all 80 AHA DB records can be stored in
roughly 130 Mb of disk space (assuming use of the standard 35-minute
records).  These programs can also reformat old (pre-1989) MIT DB
tapes written in the AHA DB distribution format.

It is also possible to read and write AHA tape-format files directly using
the WFDB library;  refer to the {\it WFDB Programmer's Guide}
for details.

\section{Noise stress testing}

With respect to many tasks performed by an ECG analyzer, dealing with
noise is the major problem faced by system designers.  Although
measurements such as ST deviation may be obtained reliably in clean
signals, the presence of noise may render them inaccurate.  In some
instances, it is sufficient to recognize the presence of noise and either
to mark measurements as unreliable or to avoid making measurements
altogether.  In other cases, excluding noisy data is inappropriate
(for example, given the multiple correlations among physical
activity, noise, and transient ischemia, excluding noisy signals is
likely to introduce sampling bias in an ischemia detector).

It is difficult to measure the effects of noise on an ECG analyzer
using ordinary recordings.  Even if existing databases include an
adequate variety of both ECG signals and noise, the sample size is
certainly too small to include all combinations of noise and ECG
signals that may be encountered in clinical use.  In ordinary
recordings, it is difficult or impossible to separate the effects of
noise from the intrinsic problems of analyzing clean signals of the
same type.

The noise stress test circumvents these problems.  By adding noise
in calibrated amounts to clean signals, any combination of noise and
signal types is possible.  Since both the noise-corrupted signal and
the clean signal can be analyzed (in separate experiments) by the
same analyzer, the effects of noise on the analysis are readily
separable from any other problems that may arise while analyzing the
clean signals.  Finally, since the test can be repeated using
different amounts of noise, it is possible to characterize analyzer
performance as a function of signal-to-noise ratio.

The major criticisms of the noise stress test are that not all noise
is additive, and that the characteristics of the added noise may not
perfectly match those of noise observed in clinical practice.  These
points, though formally irrefutable, do not negate the value of the
test.  In practice, most of the troublesome noise is additive; thus
(given appropriate inputs) the noise stress test can simulate most of
the noisy signals of interest.  The NST DB includes noise recordings
made using standard ambulatory ECG electrodes and recorders, but with
electrodes placed on the limbs of active volunteers in configurations
in which the subject's ECG is not apparent in the recorded signals.
Given the recording technique used, it is not surprising that the
characteristics of the recorded noise closely match those of noise in
standard ambulatory ECG recordings. Although it may be argued that the
particular muscles responsible for the recorded noise might produce
different signals than those that generate the EMG present in noisy
ECGs, no such differences are apparent from comparisons of either the
signals or their power spectra.

The NST DB includes a small set of ECG records with calibrated amounts
of added noise.  EC38 specifies that performance on these records must
be reported, although no specific performance levels are required.
Program {\tt nst} can be used to generate additional records for noise
stress testing.  To do so, choose an ECG record and a noise record
(the latter may be {\tt bw}, {\tt em}, or {\tt ma} from the NST DB, or
any other available noise recording).  Run {\tt nst} and answer its
questions to generate a noisy ECG record that may then be used in the
same way as any other WFDB record.  By default, {\tt nst} adds no noise
during the first five minutes of the record, then adds noise for the
next two minutes, none for the following two minutes, and repeats this
pattern of two minutes of noise followed by two minutes of clean
signals for the remainder of the record.  The scale factors for the
noise, if determined by {\tt nst}, are adjusted such that the
signal-to-noise ratios are equal for each signal.  The durations of
the noisy periods, and the scale factors for each signal, are recorded
in a {\em protocol annotation file}, which is generated by {\tt nst}
unless an existing protocol annotation file is supplied as input.  To
change these parameters, simply edit the protocol annotation file
(using, for example, {\tt rdann} to convert it to text form, any text
editor to make the modifications, and {\tt wrann} to convert it back
to annotation file format).
\end{document}
